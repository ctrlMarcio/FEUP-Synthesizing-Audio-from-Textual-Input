\section*{Abstract Notes}

\begin{itemize}
    \item 100-300 words
    \item only one paragraph
    \item The abstract should begin with a brief but precise statement of the problem or issue, followed by a description of the research method and design, the major findings, and the conclusions reached.
\end{itemize}

abstract parts:

\begin{enumerate}
    \item introduction
    \item methods
    \item results
    \item discussion
\end{enumerate}

\textbf{Introduction}

\begin{itemize}
    \item brief context (w/o details)
    \item if the abstract has scientific terms, define them
    \item After identifying the problem, state the objective of your research. Use verbs like “investigate,” “test,” “analyze,” or “evaluate” to describe exactly what you set out to do
    \item present or past simple tense, no future
\end{itemize}

\textbf{Methods}

Next, indicate the research methods that you used to answer your question. This part should be a straightforward description of what you did in one or two sentences. It is usually written in the past simple tense, as it refers to completed actions.

Don't evaluate validity or obstacles here—the goal is not to give an account of the methodology's strengths and weaknesses, but to give the reader a quick insight into the overall approach and procedures you used.

\textbf{Results}

Depending on how long and complex your research is, you may not be able to include all results here. Try to highlight only the most important findings that will allow the reader to understand your conclusions.

\textbf{Discussion}

\begin{itemize}
    \item present simple tense
    \item if there are important limitations to research (like sample size), mention them here
    \item If your aim was to solve a practical problem, your discussion might include recommendations for implementation. If relevant, you can briefly make suggestions for further research.
\end{itemize}

\section*{Abstract}

Nowadays, audio is at the heart of most generated content. Tens of thousands of songs are published every day, and new media formats based on audio, such as podcasts, are on the rise. At the same time, audio is heavily used indirectly in videos, either in online platforms and social networks or more solidified outlets, such as film or advertisement. Usually, manual labor is behind this enormous amount of generated audio. If some creator desires a specific sound, they must research it under online databases, synthesize it, or even go outside and record it. This amount of work could be a barrier to content creation, especially if such sounds are complex or too specific. This dissertation proposes using generative AI models similar to those used in mainstream image generators but applying them instead to sound. With such models, one could insert a textual input and receive an artificial sound. This work is significant to content creators, sound engineers, and everyone interested in creating sound snippets.