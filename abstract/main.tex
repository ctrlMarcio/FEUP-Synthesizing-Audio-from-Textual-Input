\documentclass[11pt,a4paper]{article}

%\usepackage[portuguese]{babel}  % if you want Portuguese
\usepackage[utf8]{inputenc}           % 8 bits UTF8
%\usepackage[latin1]{inputenc}     %  OR 8 bits latin1
\usepackage{parskip}            % no indentation on paragraphs
\usepackage{url}                % URLs

%% margins
\RequirePackage[outer=25mm,inner=30mm,vmargin=16mm,includehead,includefoot,headheight=15pt]{geometry}

%% headers and footers
\usepackage{fancyhdr}           % page headers
\pagestyle{fancy}
\lhead{}\chead{}\rhead{PD 2022/2023 Sem 1}
% \lfoot{}\cfoot{}\rfoot{Page \thepage}
\renewcommand{\headrulewidth}{0.4pt}
\renewcommand{\footrulewidth}{0.4pt}

\pagenumbering{gobble}

%% some more macros
\newcommand{\dummy}[1]{$<$#1$>$}
\newcommand{\titles}[2]{\noindent\textbf{#1:} #2\\[2mm]}

%% LaTeX exceptions
%\hyphenation{In-fra-struc-ture}

\begin{document}

\titles{Title}{Generative Models for Audio Synthesis}
\titles{Author}{Márcio Duarte}
\titles{Supervision}{Luís Paulo Reis}
\titles{Date}{November 24, 2022}
\titles{Institution}{Faculdade de Engenharia da Universidade do Porto}

\section*{Abstract}

Nowadays, audio constitutes a central part of most generated content. Usually, manual labor is behind the terabytes of audio published daily. If some creator desires a specific sound, they must research it under online databases, synthesize it, or even record it themselves. This amount of work is a barrier to content creation, mainly if such sounds are elaborate or too specific.

Audio files present solely one dimension, \textit{i.e.}, the amplitude of its sound wave collected at a designated rate interval. In comparison, image files --- whose generator models are becoming mainstream --- present three data dimensions. Regardless, long-term dependencies convey a challenge in sound as they are more complex and intricate than those of images. For instance, there is the expectation that the timbre of a given instrument or the background noise of a given sonic environment remains over time.

This dissertation studies avant-garde generative AI models and proposes and follows the implementation of a system capable of synthesizing audio snippets with high fidelity through textual input. A generated audio snippet has high fidelity if it is challenging for humans to distinguish it from a real-world sound. The proposed system does not confine these audio snippets to a domain like music or speech but accepts any textual prompt. The dissertation focuses on the development and experimentation of various model architectures based on state-of-the-art systems, ideally resulting in an instance capable of yielding good results.

Given that meaningful work on generative models for images or text-to-speech already exists, the expectation is that a general audio synthesis model generates satisfactory outcomes. This work's results will impact the speed of the production process for content creators, sound engineers, and everyone interested in creating audio outputs.


\titles{Keywords}{Audio, Deep Learning, Generative Model, Machine Learning, Sound Synthesis}
\titles{ACM Classification}{\\Computing methodologies / Machine learning / Machine learning approaches / Neural networks \\ Applied computing / Arts and humanities / Sound and music computing}

\nocite{*}  % to include references that were not cited

%% the references using BibTeX
\bibliographystyle{unsrt}
\bibliography{abstractBib}

\end{document}
