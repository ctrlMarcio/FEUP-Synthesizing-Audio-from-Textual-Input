\chapter{Conclusions} \label{chap:concl}

\minitoc

The field of \ac{DL} and audio has witnessed the emergence of generative \ac{AI} models for audio synthesis, a challenging and fascinating research topic. This thesis has investigated the evolution and evaluation of these models, with the main objectives of understanding the current state of the art and creating a novel model using a generative approach. To achieve these goals, this thesis has followed a comprehensive methodology that included conducting a literature review and developing and evaluating a generative \ac{AI} model for audio synthesis.

This chapter concludes the thesis by presenting the main results and contributions of the research. It also reflects on the research process and methodology, and suggests future directions for further research.

The first section, Section~\ref{sec:reflection-goals}, of this chapter provides an overview of the research goals that guided the investigation into the development and evaluation of generative \ac{AI} models for audio synthesis. Section~\ref{sec:reflection-process} reflects on the research process and methodology employed throughout the research, highlighting the strengths and limitations of the chosen approach, and discussing the challenges and lessons learned during the development of the \ac{AI} models. The third section, \ref{sec:future-directions}, outlines potential areas of future work that can help advance the field of generative \ac{AI} models for audio synthesis, addressing some of the open questions and limitations identified in the research.

\section{Overview of Research Goals} \label{sec:reflection-goals}

This section presents an overview of the research goals that directed the examination into the evolution and evaluation of generative \ac{AI} models for audio synthesis.

The two research goals can be summarized as follows: to understand the current state of the art in generative \ac{DL} and audio, and to create one of these models.

\subsection{Comprehensive Study of State-of-the-Art Deep Learning Architectures and Models for Audio Synthesis}

Throughout the thesis, a comprehensive study of the current state-of-the-art deep learning architectures for audio synthesis was conducted, including \acp{GAN}, \acp{VAE}, diffusion models, and other related architectures. This study included an in-depth analysis of the strengths, limitations, and potential applications of these architectures in the context of audio synthesis.

In addition, various models that use these deep learning architectures were examined, such as VALL-E, AudioGen, AudioLM, and others. Each model was thoroughly analyzed to understand their unique approaches, techniques, and contributions to audio synthesis. The results of this study provided valuable insights into the different models and their effectiveness in producing high-quality audio.

Extensive research was also conducted to examine previous algorithms used for sound processing, including techniques for augmentation, feature extraction, and other purposes. This research involved a thorough review of existing algorithms and their applicability to audio generative models. The insights gained from this research were used in the development of practical systems and contributed to the overall understanding of sound processing techniques in the context of generative AI models.

This work culminated in a thorough state-of-the-art chapter (Chapter~\ref{chap:sota}), which provides a comprehensive overview of current advances in deep learning architectures and models for audio synthesis. The chapter presents a detailed analysis of the studied architectures and models, highlighting their strengths, limitations, and potential applications.

In addition, the findings and insights from this research have been summarized in a review paper that has been submitted to a prestigious journal for peer review. This review paper aims to provide a comprehensive and up-to-date overview of the state of the art in deep learning architectures and models for audio synthesis. It is currently awaiting approval and publication in the journal.

\subsection{Developing End-to-End Systems for Sound Synthesis and Evaluation}

The goal of developing end-to-end systems for sound synthesis from text input has been partially achieved. While initial prototypes have been created, the results have not been satisfactory due to limitations in the available datasets and the lack of hyperparameter fine-tuning in the models. However, the potential for improvement is significant, and future work proposed in the conclusion suggests new models that could yield better results.

The goal of evaluating the ability of systems to generate sound from textual input remains an ongoing challenge. Finding and testing robust evaluation functions is a complex task that requires further research and dedicated effort. Due to time constraints, this objective has not been fully completed, and only a comparison using the loss function of \ac{GAN} has been performed. 

In summary, the research objectives outlined in this dissertation have been addressed to varying degrees. A comprehensive study of \ac{DL} architectures and prior sound processing algorithms has been conducted. The development of end-to-end systems for sound synthesis from text input has shown promising progress, while the evaluation of their accuracy remains an ongoing challenge. These achievements contribute to the existing knowledge and understanding of audio generative models and pave the way for future research and development in this area.

\section{Reflection on the Research Process} \label{sec:reflection-process}

Throughout this research, a comprehensive methodology was employed to ensure the successful achievement of the research objectives. The chosen methodology is presented in Section~\ref{sec:sol-approach} and involved conducting a thorough review of the state of the art while simultaneously initiating the writing process. This iterative and agile approach allowed for continuous refinement of ideas and incorporation of the latest developments in the field.

Reflecting on the effectiveness of the chosen methodology, it can be concluded that it successfully guided the research process. The methodology provided a structured framework for conducting the research and ensured that the objectives were addressed in a systematic and efficient manner. The challenges encountered during the research process were not related to the methodology itself, but rather to the complexity of the chosen research topic.

One challenge encountered during the research process was the sometimes tedious and frustrating nature of developing \ac{AI} models. Troubleshooting problems during training required additional time and effort, often requiring extended training periods to evaluate potential solutions. However, these challenges provided valuable lessons in patience, problem solving, and the importance of careful experimentation.


\section{Future Directions} \label{sec:future-directions}

The study and development of generative \ac{AI} models for audio synthesis have shown promising results in producing realistic and diverse audio output. However, there are still several avenues for further exploration and improvement. This section outlines potential areas of future work that can help advance the field of generative \ac{AI} models for audio synthesis.

\subsection{Exploring Novel Architectures}

The objective of this section is to recommend new frameworks that expand upon the groundwork established in this thesis. Developing these suggested frameworks is deferred to future research due to limited time and resources.

The proposed theoretical methods presented in this Section are motivated by the necessity of increasing the quality, diversity, and efficacy of the produced audio.

This section outlines the objectives, design principles, and possible applications of each suggested architecture. Furthermore, it discusses the methodological considerations and potential obstacles that may arise during their development and implementation.

It is essential to note that the proposed theoretical approaches presented here are intended to serve as a basis for future research. Researchers and practitioners are encouraged to explore, refine, and contribute to the advancement of generative \ac{AI} models for audio synthesis.

This section provides detailed explanations of each proposed architecture, including insights into its design principles, implementation considerations, and potential applications. This approach aims to stimulate further exploration and innovation in the field.

\chapter{VAMOS} \label{ann:vamos}

Contained in this appendix are important insights into the architectural foundations of the VAMOS model, along with three key components that are central to its effectiveness. These components include the \textit{Text Encoder}, which uses BERT's encoder for transforming textual data, the \textit{ResNet (Audio Encoder)} designed for audio processing tasks, and the \textit{CLAP} model for creating joint audio-text embeddings. Furthermore, this text presents an implementation and analysis of the \textit{U-Net} architecture, which is known for its superior segmentation capabilities. All components of the model are accompanied by their code and academic explanation to highlight their importance in the overall framework.

\section{Text Encoder}

\begin{lstlisting}[language=Python, caption={Text encoding function utilizing BERT tokenizer and encoder for semantic representation extraction.}]
def encode(text):
    # Pass text through the BERT tokenizer
    input_ids = tokenizer(text, add_special_tokens=True)["input_ids"]
    input_tensor = torch.tensor([input_ids])

    # Pass the input tensor through the BERT encoder
    with torch.no_grad():
        encoded_output = self.model(input_tensor)[0]
        flatten_output = torch.flatten(encoded_output, start_dim=1)

    return flatten_output
\end{lstlisting}

The provided code excerpt describes the \texttt{encode} function, which plays an essential role in the developed model. This function utilizes a BERT model's encoder to convert input textual data into a structured numerical representation that encapsulates its semantic essence.

The function begins by calling a BERT tokenizer that translates human-readable text into an ordered sequence of discrete numerical identifiers. This conversion aligns with the established protocol for facilitating computational analysis and interpretation.

Afterward, the input tensor passes through the encoder module of the BERT model, which captures contextual dependencies within the text data. This process fosters a dynamic and enriched representation at the token level.

\section{ResNet (Audio Encoder)}

\begin{lstlisting}[language=Python, caption={Residual Network (ResNet) architecture for audio processing with custom residual block implementation.}]
class ResNet(nn.Module):
  
    # ... (other initialization code)

    def build_resnet(self):
        if self.useBottleneck:
            filters = [64, 256, 512, 1024, 2048]
        else:
            filters = [64, 64, 128, 256, 512]

        self.layer1 = nn.Sequential()
        self.layer1.add_module('conv2_1', resblock(
            filters[0], filters[1], downsample=False))
        for i in range(1, self.repeat[0]):
            self.layer1.add_module('conv2_%d' % (
                i+1,), resblock(filters[1], filters[1], downsample=False))

        self.layer2 = nn.Sequential()
        self.layer2.add_module('conv3_1', resblock(
            filters[1], filters[2], downsample=True))
        for i in range(1, self.repeat[1]):
            self.layer2.add_module('conv3_%d' % (
                i+1,), resblock(filters[2], filters[2], downsample=False))

        self.layer3 = nn.Sequential()
        self.layer3.add_module('conv4_1', resblock(
            filters[2], filters[3], downsample=True))
        for i in range(1, self.repeat[2]):
            self.layer3.add_module('conv2_%d' % (
                i+1,), resblock(filters[3], filters[3], downsample=False))

        self.layer4 = nn.Sequential()
        self.layer4.add_module('conv5_1', resblock(
            filters[3], filters[4], downsample=True))
        for i in range(1, self.repeat[3]):
            self.layer4.add_module('conv3_%d' % (
                i+1,), resblock(filters[4], filters[4], downsample=False))

        self.gap = torch.nn.AdaptiveAvgPool2d(1)
        self.fc = torch.nn.Linear(filters[4], outputs)

    def forward(self, input):
        input = self.layer0(input)
        input = self.layer1(input)
        input = self.layer2(input)
        input = self.layer3(input)
        input = self.layer4(input).detach()
        input = self.gap(input)
        input = torch.flatten(input, start_dim=1)
        input = self.fc(input)

        return input
  
class ResBlock(nn.Module):

    def __init__(self, in_channels, out_channels, downsample):
  
        # ... (other initialization code)

        # Define convolutional layers and batch normalization
        if downsample:
            # If downsample is True
            # apply convolution that reduces the size and apply a shortcut
            self.conv1 = nn.Conv2d(
                in_channels, out_channels, kernel_size=3, stride=2, padding=1)
            self.shortcut = nn.Sequential(
                nn.Conv2d(in_channels, out_channels, kernel_size=1, stride=2),
                nn.BatchNorm2d(out_channels)
            )
        else:
            # If downsample is False, apply convolution with stride 1 and no
            # shortcut path
            self.conv1 = nn.Conv2d(
                in_channels, out_channels, kernel_size=3, stride=1, padding=1)
            self.shortcut = nn.Identity()

        # Define additional convolutional layers and batch normalization
        self.conv2 = nn.Conv2d(out_channels, out_channels,
                               kernel_size=3, stride=1, padding=1)
        self.bn1 = nn.BatchNorm2d(out_channels)
        self.bn2 = nn.BatchNorm2d(out_channels)

    def forward(self, input_tensor):
        # Apply the shortcut path to the input
        shortcut = self.shortcut(input_tensor)
        shortcut = shortcut.detach()

        # Apply the first convolutional layer and ReLU activation, followed by
        # batch normalization
        x = nn.ReLU()(self.bn1(self.conv1(input_tensor)))
        x = x.detach()

        # Apply the second convolutional layer and ReLU activation, followed by
        # batch normalization
        x = nn.ReLU()(self.bn2(self.conv2(x)))
        x = x.detach()

        # Add the shortcut to the output of the second convolutional layer
        output_tensor = x + shortcut

        # Apply ReLU activation to the output and return it
        return nn.ReLU()(output_tensor)
\end{lstlisting}

The provided code segment introduces a customized neural network architecture based on the ResNet framework, specifically designed for audio processing. The architecture includes a variety of key components, including the use of residual blocks that are critical to constructing the ResNet layers. The explanation then provides a concise overview of the overall structure and operation of the code. 

At the core of this demonstration is the \texttt{ResNet} class. This class includes methods for assembling the architecture and managing data flow through the network.  This class includes methods for assembling the architecture and managing data flow through the network. A key method is \texttt{build\_resnet}, which creates the network by stacking residual blocks. This assembly follows the architectural configuration principles outlined by the specified guidelines.

Additionally, the \texttt{ResBlock} class serves as a fundamental construct that houses a single residual block- revered for its role as a building block within the larger ResNet architecture. Within this enclosed domain, the architectural formulation of the residual block is contextualized.

\section{CLAP}

\begin{lstlisting}[language=Python, caption={CLAP model architecture for joint audio-text embeddings with pairwise cosine similarity-based alignment.}]
class Clap(nn.Module):
    def __init__(self,
                 audio_feature_dim,
                 text_feature_dim,
                 shared_embedding_dim
                 ):
        # ... (other initialization code)
  
        # Initialize the weights that will connect the input audio and text
        # features to the shared embedding space.
        self._audio_projection = torch.nn.Linear(
            audio_feature_dim, shared_embedding_dim)
        self._text_projection = torch.nn.Linear(
            text_feature_dim, shared_embedding_dim)

        # Initialize the temperature parameter used for scaling the pairwise
        # cosine similarities between image and text embeddings.
        self._learned_temperature = torch.nn.Parameter(torch.tensor([1.0]))

    def encode_audio(self, audio_features):
        # Project the audio features into the shared embedding space and
        # normalize the resulting embedding vectors.
        audio_embeddings = F.normalize(
            self._audio_projection(audio_features), p=2, dim=1)
        return audio_embeddings

    def encode_text(self, text_features):
        # Project the text features into the shared embedding space and
        # normalize the resulting embedding vectors.
        text_embeddings = F.normalize(
            self._text_projection(text_features), p=2, dim=1)
        return text_embeddings

    def forward(self, audio_features, text_features):
        # Encode the audio and text features into their respective embedding
        # spaces.
        audio_embeddings = self.encode_audio(audio_features)
        text_embeddings = self.encode_text(text_features)

        # Compute the pairwise cosine similarities between the image and text
        # embeddings, scaled by the learned temperature parameter.
        pairwise_similarities = torch.matmul(
            audio_embeddings, text_embeddings.T) * torch.exp(
            self._learned_temperature)

        # Compute the symmetric cross-entropy loss between the predicted
        # pairwise similarities and the true pairwise similarities.
        labels = torch.arange(audio_features.size(0))
        loss_i = F.cross_entropy(
            pairwise_similarities, labels, reduction='mean')
        loss_t = F.cross_entropy(
            pairwise_similarities.T, labels, reduction='mean')
        loss = (loss_i + loss_t) / 2

        return loss
\end{lstlisting}

The presented code introduces a class called \texttt{Clap}. It is a neural network model designed to acquire shared embeddings for both audio and text input data. The model aims to align audio and text features in a shared embedding space by utilizing pairwise cosine similarities. 

The essence of the code is explained in further detail below. The initialization of the model includes parameterizing the primary dimensions: audio and text feature dimensions, coupled with the dimensionality of the embedding space for their alignment.

The model's architecture incorporates linear projection layers for audio and text input streams, as well as a temperature parameter.  The learned temperature parameter is useful in calibrating pairwise cosine similarities by incorporating a scaling factor to the magnitudes.

The \texttt{forward} method represents the procedural logic for the model's forward pass. This includes incorporating audio and text input features into their respective embedding spaces, accomplished through the use of the \texttt{encode\_audio} and \texttt{encode\_text} methods. An important step is the subsequent calculation of pairwise cosine similarities using matrix multiplication, which reflects the underlying relationships between audio and text embeddings. The use of the acquired temperature parameter adjusts the importance of these pairwise similarities. Based on this foundation, the algorithm's optimization is enhanced by calculating a symmetric cross-entropy loss that takes into account audio-to-text as well as text-to-audio relationships. The final loss metric directing the model's training trajectory is achieved by averaging these dual losses.

\section{U-Net}

\begin{lstlisting}[language=Python, caption={U-Net architecture for semantic segmentation}]
class UNet(nn.Module):

    # ... (other initialization code)

    def forward(self, x):
        x = x.to(self.device)

        # implements the forward pass with concatenations
        skip_connections = []

        # contracting path
        for block in self._contracting_path:
            # create a sequential block from the list of layers
            net = nn.Sequential(*block).to(self.device)

            # apply
            x = net(x)

            # save the skip connection
            skip_connections.append(x)

        # bottleneck
        x = self._bottleneck(x)

        # expanding path
        for layer_idx in range(self.depth - 1):
            # the first layer is a transposed convolutional layer
            transposed_conv = self._expanding_path_layers[layer_idx * 2].to(
                self.device)
            x = transposed_conv(x)

            # concatenate the skip connection
            skip_connection = skip_connections.pop()
            # make sure the shapes match
            if x.shape != skip_connection.shape:
                # resize the skip connection
                skip_connection = nn.functional.interpolate(skip_connection,
                                                            size=x.shape[2:],
                                                            mode='nearest')

            # concatenate the skip connection
            x = torch.cat((x, skip_connection), dim=1)

            # the second layer is a sequential block of convolutional layers
            conv_block = self._expanding_path_layers[layer_idx * 2 + 1].to(
                self.device)
            x = conv_block(x)

        # final convolutional layer
        x = self._expanding_path_layers[-1].to(self.device)(x)

        return x

    def _make_contracting_path(self):
        """
        Create the contracting path of the U-Net.
        """
        layers = []

        # configs the used convolutional layers
        in_channels = self.in_channels
        out_channels = 64

        # create a convolutional block for each number in depth
        for _ in range(self.depth - 1):
            # create a convolutional block
            block = []

            # create the number of convolutional layers specified by
            # conv_layers_per_block
            for _ in range(self.conv_layers_per_block):
                block.append(nn.Conv2d(in_channels=in_channels,
                             out_channels=out_channels, kernel_size=3, padding="same"))
                # batch normalization
                block.append(nn.BatchNorm2d(out_channels))
                # ReLU activation
                block.append(nn.ReLU())

                # update the in_channels
                in_channels = out_channels

            # add a max pooling layer
            block.append(nn.MaxPool2d(kernel_size=2))

            # add the block to the layers
            layers.append(block)

            # double the number of channels
            out_channels *= 2

        # create a network from the layers
        return layers

    def _make_bottleneck(self):
        """
        Create the bottleneck of the U-Net.
        """
        # build the bottleneck
        layers = []

        # config the bottleneck channels
        in_channels = 64 * (2 ** (self.depth - 2))
        out_channels = in_channels * 2

        # create the number of convolutional layers specified by
        # conv_layers_per_block
        for _ in range(self.conv_layers_per_block):
            layers.append(nn.Conv2d(in_channels=in_channels,
                                    out_channels=out_channels,
                                    kernel_size=3,
                                    padding="same"))
            # batch normalization
            layers.append(nn.BatchNorm2d(out_channels))
            # ReLU activation
            layers.append(nn.ReLU())

            # update the in_channels
            in_channels = out_channels

        # add all the layers in block in the layers list
        return nn.Sequential(*layers).to(self.device)

    def _make_expanding_path(self):
        """
        Create the expanding path of the U-Net.
        This returns a list of layers, which will be used in the forward pass.
        Some layers are transposed convolutional layers, others are sequential
            blocks of convolutional layers.
        """
        layers = []

        # configs the used convolutional layers
        # the number of in channels is the number of out channels from the last
        # block in the contracting path
        in_channels = 64 * (2 ** (self.depth - 1))
        out_channels = in_channels // 2

        # create a convolutional block for each number in depth
        for _ in range(self.depth - 1):
            # add an up conv 2x2
            layers.append(nn.ConvTranspose2d(in_channels=in_channels,
                                             out_channels=out_channels,
                                             kernel_size=2,
                                             stride=2))

            block = []

            # create the number of convolutional layers specified by
            # conv_layers_per_block
            for _ in range(self.conv_layers_per_block):
                block.append(nn.Conv2d(in_channels=in_channels,
                             out_channels=out_channels, kernel_size=3, padding="same"))
                # batch normalization
                block.append(nn.BatchNorm2d(out_channels))
                # ReLU activation
                block.append(nn.ReLU())

                # update the in_channels
                in_channels = out_channels

            # add the block to the layers
            layers.append(nn.Sequential(*block))

            # double the number of channels
            out_channels //= 2

        # final convolutional layer
        final_layer = []
        final_layer.append(nn.Conv2d(in_channels=in_channels,
                                     out_channels=self.out_channels,
                                     kernel_size=1,
                                     padding="same"))

        final_layer.append(nn.Softmax(dim=1))

        layers.append(nn.Sequential(*final_layer))

        # create a network from the layers
        return layers
\end{lstlisting}

The code provides insight into the U-Net architecture, renowned for its effectiveness in segmentation tasks. It includes a class called \texttt{UNet}, which embodies the architectural blueprint of the U-Net model. Its structure is composed of three primary components - the contracting path, the bottleneck layer, and the expanding path - which are crucial to the model's implementation. The forward pass follows the characteristic U-shaped configuration.  This architectural layout is used to create the favorable integration of skip connections, which enhances the capacity for precise segmentation.
\subsubsection{Fine-Tune Stable Diffusion for Spectrograms}

A promising avenue of research in the \ac{DL} community is the fine-tuning of existing generative models to improve their capabilities and produce high-quality audio output. This section introduces the concept of fine-tuning generative models for audio synthesis, and proposes fine-tuning the stable diffusion model to generate spectrograms.

The stable diffusion model uses the principles of diffusion processes to generate high-fidelity and diverse image samples (see Section\ \ref{sec:stable-diffusion}). One of the key advantages of the stable diffusion model is its ability to capture complex dependencies and generate realistic output.

It is worth noting that the stable diffusion model is an open source model, meaning that its code and implementation details are publicly available. This accessibility allows researchers and practitioners to study, modify, and build upon the foundations of the model. The availability of open source code for the stable diffusion model facilitates its fine-tuning for specific tasks, such as audio synthesis.

The fine-tuning of the stable diffusion model to spectrograms provides an exciting opportunity to explore the generation of high-quality audio output based on this visual representation.

One of the key advantages of the stable diffusion model is its ability to capture complex dependencies and generate realistic images. By fine-tuning the stable diffusion model on spectrograms, one can leverage its prior knowledge and adapt it to the specific characteristics of audio signals represented in the frequency and time domains.

The potential benefits of using the prior knowledge of the stable diffusion model for audio synthesis are manifold. First, the stable diffusion model has already shown impressive results in other domains, in this case image synthesis. By building on this foundation, one can exploit the model's ability to generate high-fidelity and diverse outputs, which can greatly improve the quality of the synthesized audio.

Furthermore, fine-tuning the stable diffusion model to spectrograms can provide a unique perspective on audio synthesis. By treating spectrograms as images and applying the stable diffusion model, one can explore the potential of generating audio based on this visual representation. This approach opens up new possibilities for manipulating and synthesizing audio in innovative ways.

To prove the usefulness of this method, Riffusion (see Section~\ref{sec:riffusion}), which already performs a similar task of generating audio from visual representations, has shown considerable results. By considering the insights and techniques used in Riffusion, one can build on its foundations and adapt the stable diffusion model accordingly.

It is important to note that the success of the proposed approach depends on the availability of a sufficiently large dataset for fine-tuning.

\subsubsection{Theoretical General Audio Transformer}

Transformers have become a prevalent architecture for several sequence modeling tasks in \ac{NLP}~\cite{gruetzemacher_deep_2022}. Their success has also expanded to the generative modeling of mediums such as images and videos. This section proposes adapting the transformer architecture for generative modeling and audio waveform synthesis.

A novel encoder-decoder transformer specifically designed for audio generation is introduced, referred to as the \acf{AT}. The \ac{AT} incorporates convolutional layers and custom attention mechanisms tailored to handle audio data.

The primary motivation is to enable high-quality, flexible audio generation for various applications. While transformers have proven effective in capturing long-range dependencies in sequences, the proposed \ac{AT} aims to optimize these models specifically for audio data.

The use of a transformer architecture for audio generation, propelled by textual prompts, is based on the encoder-decoder framework of the transformer, modified for audio synthesis. This configuration permits the effortless incorporation of textual data to aid in the synthesis of consistent and context-appropriate audio waveforms.

The process for generating audio using transformers includes two components: an encoder and a decoder. Each is composed of stacked layers that work together to process the textual input and create the corresponding audio output. The encoder's main function is to map the textual input to continuous representations that capture semantic information. For a deeper understanding on the encoder's functionality, as well as on transformers, in general, please refer to Section~\ref{sec:transformers}.

Upon receiving the continuous representations from the encoder, the decoder begins to generate the audio waveform. The decoder's functioning is dependent on the encoder's representations, which guarantees that the synthesized audio is aligned with the intended meaning of the input text. 

The synthesis process unfolds sequentially, with the decoder generating the output audio waveform sample-by-sample. The incorporation of multi-head attention mechanisms at the local and global levels enables the model to generate a wide range of natural-sounding audio outputs that correspond with the semantic context of the input.

The hyperparameters such as model dimensions, number of layers, attention heads, and hidden sizes can be tuned to balance performance and computational constraints. In summary, this architecture leverages the strengths of transformers for sequence modeling while optimizing the components for conditional audio generation.

\paragraph{Sound Representation}
Transformers have demonstrated power in autoregressive sequence modeling and generation. However, directly applying them to raw audio waveforms poses challenges due to the high sampling rates and lack of inherent discretization. As a potential alternative, spectrograms offer a 2D time-frequency audio representation (see section~\ref{sec:sound}). This section analyzes the trade-offs between raw audio versus spectrograms as inputs to transformer-based audio generation models.

\subparagraph{Challenges of Using Spectrograms}
Spectrograms explicitly encode frequency information, providing interpretable intermediate representations. However, transformers generate outputs one step at a time, and the notion of ``next'' becomes ambiguous on 2D spectrograms, as there are multiple values for each moment.

\subparagraph{Benefits of Raw Audio}
Raw audio waveforms allow direct modeling of the time-domain signal to be generated. The sequential structure matches the autoregressive nature of transformers without modification. Transformers can learn to model dependencies in the continuous waveform samples directly.
Raw audio represents the most natural fit for a sequential generation. Parameterizing waveform samples also avoids imposing and inverting a fixed spectrogram transformation that may discard information. Raw audio can capture nuances and exhibit fidelity beyond what prescribed spectrogram representations encode.

\subparagraph{Latent Feature Translation}
To fix these problems and bridge the gap between spectrograms and transformers, a novel approach is introduced. Instead of directly using the 2D spectrogram as input, a neural network is employed to translate each column of the spectrogram into a single continuous latent feature. This neural network, known from now on as the Latent Feature Translator, is shared across all columns and is conditioned on the corresponding timestamp of the column. The resulting continuous latent features capture the essential characteristics of the audio content while simplifying the representation for the subsequent transformer-based modeling.

This translation process effectively compresses the complex spectral information into a more manageable and informative format that still maintains temporal information, facilitating the transformer's ability to capture dependencies and generate coherent audio output.

This innovative approach leverages the strengths of both spectrogram representations and transformers, enabling efficient and effective generative audio modeling.

\paragraph{Training}

The \ac{AT} is first pretrained in an unsupervised manner to learn effective representations of audio data. During unsupervised pretraining, the model input is a segment of continuous latent features, and the target output is the next latent feature.

The model trains on audio-only data to predict the next latent feature. This relies on no text conditioning or labels. The goal is to learn generalized audio representations that capture dependencies across long sequences.

During training, audio clips are split into fixed-length chunks, 1-10 seconds. Batching multiple chunks together allows more efficient \ac{GPU} processing. Given all previous samples, the model is trained to predict the next latent feature at each step.

An error loss between the predicted and target audio samples is calculated. Over many training iterations, the model parameters are updated to minimize the loss function. Additional regularization techniques like dropout are used to prevent overfitting. The learning rate is gradually decayed for training as the loss function converges.

Validation audio clips not used in training monitor overfitting and determine early stopping. The model with the lowest validation loss is selected.

During supervised training, text conditioning is added to the pre-trained model. The loss function now optimizes conditional generation quality given input text.

This leverages the representations learned during unsupervised pretraining. Text conditioning trains the model to generate relevant audio for textual descriptions.

Pretraining provides an effective regularization technique to prime the model and prevent overfitting the paired text-audio data. This semi-supervised approach with unsupervised pretraining enables learning robust generative audio models.

\paragraph{Why Transformers}
Transformers are a natural choice for generative audio modeling compared to alternatives like \acp{GAN} or diffusion models (see Sections~\ref{sec:gan},~\ref{sec:diffusion}) due to their strength in sequential modeling. Audio signals inherently exhibit strong temporal consistency and context.

Transformers leverage a self-attention mechanism to model long-range dependencies in sequences effectively. This allows transformers to capture structures spanning longer time scales than recurrent models like \acp{LSTM} (see Section~\ref{sec:rnn-variants}). The global receptive field enables coherently modeling whole utterances, phrases, and even entire passages.

Additionally, as seen in Section~\ref{sec:transformers}, transformers have demonstrated state-of-the-art performance across various sequence transduction tasks in language, speech, and other domains. Leveraging their proven modeling capabilities for a new modality in audio is a natural progression.

The large model capacity of transformers is also advantageous, allowing sufficient expressivity to represent the complexity and nuance of audio data. With solid scalability to leverage large datasets through efficient parallel training, transformers are uniquely positioned as cutting-edge architecture for generative audio.

\paragraph{Limitations and Future Work}
While the \ac{AT} shows promise for generative audio modeling, there are several limitations and areas for improvement through future work.

First, the model requires large, diverse datasets of high-quality audio examples to train effectively. Collecting such datasets can be challenging, particularly for specialized domains like musical composition.

Data efficiency could be improved through transfer learning and unsupervised pretraining approaches. Leveraging models pretrained on other transformer tasks might provide valuable initializations for audio generation.

Training complex transformer models can also incur high computational costs and time requirements. Optimization for faster training and inference will be necessary for practical deployment. Architectural modifications to reduce model size should be explored.

Rigorously evaluating generated audio samples' coherence, naturalness, and creativity remains difficult. Developing quantitative and qualitative evaluation protocols to measure these attributes is an open research question.

There are many potential extensions to the base \ac{AT} architecture proposed here. For example, alternative conditioning mechanisms, sparser architectures optimized for audio, and adversarial training could improve results. Multi-task learning objectives that combine reconstruction, prediction, and discrimination may also help.

Multimodal integration of audio, text, and other modalities is also an exciting future direction. Jointly modeling text and audio could improve text-to-speech synthesis and transcription tasks. Exploring these multimodal representations with transformers is promising.

\paragraph{Innovative Approach}
Previous work has explored adapting transformers for raw audio generation, but these models have had limitations. Some, such as AudioGen~\ref{sec:audiogen}, have proposed transformer architectures for next-step audio sample prediction, demonstrating solid results for short-term modeling. However, these models do not allow conditional generation.

The proposed \ac{AT} builds on these predecessors to directly enable unconditional and conditional synthesis from the time-domain audio. By implementing an autoencoder, working on continuous latent features, and leveraging standard encoder-decoder transformers with pretraining, the \ac{AT} offers a novel approach to generative audio synthesis.

\subsection{Dataset Expansion}

Expanding the available datasets plays a critical role in training generative \ac{AI} models for audio synthesis. Larger and more diverse datasets provide models with a broader understanding of audio patterns, leading to more realistic and varied audio synthesis. In this subsection, we explore various strategies for dataset expansion, including the incorporation of new data augmentation techniques and the creation of new datasets through real-world recordings and crowdsourcing.

Data augmentation techniques are essential for increasing the diversity and size of the dataset. In the context of audio synthesis, several state-of-the-art data augmentation techniques have been proposed. These techniques can be seen in Section~\ref{sec:data-augmentation} and include time stretching, pitch shifting, noise injection, and others. By applying these techniques, one can artificially introduce variation into the dataset, allowing the models to learn from a wider range of audio patterns.

In future work, it is important to investigate the effectiveness of these data augmentation techniques in the context of audio synthesis. Specifically, the impact of each technique on the performance and generalization of generative models should be evaluated. This evaluation can be done by conducting systematic experiments that compare the performance of models trained with and without data augmentation. In addition, it would be valuable to investigate the combination of multiple data augmentation techniques to further increase the diversity and quality of the dataset.

Furthermore, the exploration of novel data augmentation techniques specifically tailored to audio data should be considered. This may involve exploring and adapting techniques from other domains, such as image or text data augmentation, and tailoring them to the unique characteristics of audio data. The development of domain-specific data augmentation techniques can potentially open up new possibilities for improving the realism and variety of audio synthesis.

Expanding the dataset may also involve creating new datasets that capture a broader range of audio characteristics. One approach is to include real-world recordings, such as live performances, field recordings, or professional studio recordings. These recordings provide a more realistic training environment for the models because they capture the nuances and complexities of real-world audio. Collaborating with musicians, audio engineers, or other experts in the field can ensure the acquisition of high-quality and diverse audio recordings.

Crowdsourcing provides an opportunity to expand the dataset by incorporating user-generated content from online platforms or social media. This approach allows for the inclusion of a diverse and constantly evolving dataset that reflects current trends and preferences in audio production. However, crowdsourced datasets come with their own challenges, such as ensuring data quality and addressing potential biases. Careful curation and validation processes should be implemented to maintain the integrity of the dataset.

\subsection{Evaluation Metrics}

Accurately evaluating the performance of generative \ac{AI} models' performance is challenging. Existing evaluation metrics often fail to capture the quality, variety, and realism of the generated audio. Future work should focus on developing robust evaluation metrics that align with human perception and subjective audio quality. By establishing reliable evaluation metrics, one can objectively evaluate and compare the performance of different models, facilitating advancements in the field.

As the dataset expands, it becomes necessary to develop new evaluation metrics that can assess the quality and diversity of the dataset. Existing evaluation metrics used in the field may have limitations in capturing the richness and diversity of the expanded datasets. Therefore, exploring and proposing new evaluation metrics that can effectively measure the performance and capabilities of generative models trained on the expanded datasets is important. These metrics should consider factors such as audio realism, diversity of generated outputs, and alignment with ground truth data.

By addressing these areas of future work, the scientific community can further advance the capabilities of generative \ac{AI} models for audio synthesis. Exploring novel architectures, expanding datasets, and developing improved evaluation metrics will contribute to the development of more powerful and reliable models, enabling applications in diverse domains such as music production, sound design, and interactive audio experiences.

\section{Conclusion} \label{sec:conc-conc}

In summary, this chapter has presented the conclusive results of an extensive investigation into the study and advancement of generative \ac{AI} models. The primary objectives of this research have been largely achieved, with the identification and discussion of the obstacles that prevented the full achievement of all objectives.

In addition, this study has significantly contributed to the existing body of knowledge on generative \ac{DL}. While developing end-to-end systems for sound synthesis from textual input remains arduous, particularly for researchers not affiliated with prominent technology companies, this thesis has made notable progress in this area. Initial prototypes have been developed, albeit with unsatisfactory results due to limitations in available datasets and the need for meticulous fine-tuning of hyper-parameters. Nevertheless, the potential for further improvement is recognized. Evaluating the ability of systems to generate sound from textual input also remains an ongoing challenge that warrants further investigation and refinement.

In conclusion, this research has provided invaluable insights into the study and development of generative \ac{AI}, focusing on audio synthesis. It has successfully achieved significant milestones, conducted a comprehensive investigation of \ac{DL} architectures and models, and made notable progress in creating end-to-end systems for sound synthesis.