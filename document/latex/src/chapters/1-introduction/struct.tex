\section{Dissertation Structure} \label{sec:struct}

The present dissertation commences with a comprehensive examination of the current state-of-the-art technologies pertaining to sound generation. The analysis encompasses sound generation and delves into the realm of deep learning and generative deep learning architectures, which are not limited to sound. Subsequently, the dissertation examines additional tools, such as data augmentation for sound and sound analysis. Then, the focus shifts to a more in-depth study of generative deep learning technologies specific to sound, including vocoders, end-to-end tools, and other related terms, which are thoroughly explained.

The following section of the dissertation focuses on formulating and defining the problem at hand. Subsequently, practical applications of the developed technologies are demonstrated, and the dissertation investigates the various decisions and consequences that arise in developing such a system.

The methodology and approach adopted to fulfill the thesis's objectives are outlined in the solution section. An overview of the work carried out, its results, and the work plan is presented in detail.

Finally, the dissertation concludes by assessing the extent to which the objectives proposed in the introduction have been met and by presenting a summary of the findings. To facilitate navigation and ease of reference, each chapter of the dissertation includes a table of contents.