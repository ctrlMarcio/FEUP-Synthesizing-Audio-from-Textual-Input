\section{Application of Synthesizing Soundscapes with Generative AI} \label{sec:significance}

The development of \ac{AI} technologies has enabled significant progress in sound synthesis, enabling the creation of sounds based on textual input. This technology has potential applications in various fields, including music production, film and game sound design, and therapeutic soundscapes. In this response, this Section explores possible applications for the \ac{AI} models described in this document.

One potential application of such \ac{AI} models is in the field of music composition. By generating sounds from textual descriptions, the model could assist composers in generating novel and unique sounds that match a piece's intended mood or atmosphere. For example, a composer might input the phrase ``eerie forest at night'', and the \ac{AI} model could generate a soundscape incorporating the sounds of rustling leaves, distant animal calls, and other eerie sounds one might associate with a forest at night. This technology could help composers to create soundscapes more efficiently that match their creative vision, saving time and increasing their overall output. Besides, the real-time inference characteristic of this model may help a live performer blend the timbres of instruments with, for instance, natural sounds, creating soundscapes on the fly \cite{huzaifah_deep_2021}.

Another possible application of an \ac{AI} model that generates sound from text is in the field of film and game sound design. Sound design plays a crucial role in creating immersive and engaging experiences in film and games, and the ability to generate custom soundscapes from textual input could enhance the creative potential of sound designers. For example, a sound designer might input the phrase ``a bustling city street'', and the \ac{AI} model could generate a soundscape that includes the sounds of car horns, people talking, footsteps, and other city noises. Alternatively, they may want to generate sound cues such as explosions \cite{huzaifah_deep_2021}, and a simple query would do that. This technology could help sound designers create more realistic and immersive soundscapes, improving the overall quality of the final product. 

A panoply of sounds is usually taken from expensive libraries in TV and film. One might imagine an infinite library with \ac{ML} models, and one should imagine how that will enhance the power of sound producers for such endeavors, especially in indie and low-budget productions.