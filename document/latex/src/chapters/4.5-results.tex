\chapter{Evaluation and Discussion} \label{chap:results}

\minitoc

The assessment of generative \ac{AI} models for audio is pivotal in advancing the field of audio generation and synthesis. It is imperative to comprehend the capabilities and limitations of these models for their effective development and implementation. This chapter presents the outcomes and analysis of experiments performed with the GANmix architecture, utilizing objective metrics based on loss functions. The aim of this study is to assess the efficacy of generative \ac{AI} models in producing audio, and to draw insights from their performance.

The experiments were conducted utilizing different hardware configurations, For \acp{GPU}, it consisted of Kaggle's Tesla V100, \ac{LIACC}'s GeForce GTX 1080, and \ac{LIACC}'s GPU Quadro RTX 8000. These configurations, along with the utilization of the PyTorch deep learning framework, allowed for proficient training and assessment of the generative \ac{AI} models.

In this chapter, the author describes the experimental setup in Section~\ref{sec:res-setup}, comprising both hardware and software configurations, alongside the GANmix architecture. The experiments' outcomes are presented in Section~\ref{sec:res-presentation} with line plots exhibiting the generator and discriminator's loss evolution per epoch. Additionally, the resulting spectrograms are presented to offer a graphical depiction of the produced audio.

Following the presentation of results, the author analyzes and interprets the findings in Section~\ref{sec:res-analysis}, identifying trends, patterns, and significant insights that emerged from the experiments. Comparisons are conducted among various models and variations in the GANmix framework for evaluating their respective performances.

Furthermore, in Section~\ref{sec:limitations}, the author addresses any limitations or challenges encountered during the evaluation process. These factors may include hardware limitations or other constraints that impact the performance of the models.

\section{Experimental Setup} \label{sec:res-setup}

This section outlines the experimental setup utilized to assess the audio generation capabilities of the GANmix model.

The GANmix model was trained using the \ac{BCE} loss, which was implemented through pytorch's \texttt{BCEWithLogitsLoss} module. Although \ac{BCE} is commonly employed for binary classification, it can also effectively train the generator in \acp{GAN}. In the context of GANmix, the generator's goal is to create authentic embeddings that can fool the discriminator into classifying them as authentic. To achieve this, the generator's loss is calculated using the discriminator's output for the fake embeddings generated by the generator and the true label --- either real or fake --- for a real embedding.

By using \ac{BCE}, the generator calculates its loss by measuring the difference between the discriminator's classification of genuine and false embeddings. This loss directs the generator to generate more realistic embeddings over time by minimizing this difference. Therefore, despite being a binary classification loss function, \ac{BCE} is a suitable approach for training the generator in \acp{GAN}.

Two datasets, the Audio MNIST dataset~\cite{becker_interpreting_2018} and the Clotho dataset~\cite{drossos_clotho_2019}, were utilized in these experiments. These datasets were previously described in detail in Section~\ref{sec:sol-datasets}.

The Audio MNIST dataset contains short audio clips where each clip represents a spoken digit. This dataset sets a standard for evaluating audio classification tasks.

In contrast, the Clotho dataset is more extensive and diverse. It offers a variety of audio samples including environmental sounds and speech from various sources. This dataset offers a diverse and comprehensive range of audio data, allowing the GANmix model to learn and generate audio samples that capture the complexity and diversity present in real-world audio recordings.

For a thorough understanding of the datasets, including their characteristics, preprocessing steps, and data augmentation techniques, please refer to Section~\ref{sec:sol-datasets}.

The experiments were conducted on three distinct hardware configurations, referred to as \textit{Kaggle}, \textit{\ac{LIACC} 1}, and \textit{\ac{LIACC} 2}, each selected based on the available resources during the development process.

Kaggle, the initial configuration implemented for the GANmix model development phase, utilized a Tesla V100 \ac{GPU} equipped with 12 \ac{GB} of memory, 73.1 \ac{GB} of disk space, and 13 \ac{GB} of \ac{RAM}.

As development progressed, the \ac{LIACC} 1 became available, which offered around the same computational power, \ac{LIACC} 1 used a GeForce GTX 1080 \ac{GPU} with 8 \ac{GB} of memory, 50 \ac{GB} of disk space, and 32 \ac{GB} of \ac{RAM}.

In the final stages of the project, the third configuration, \ac{LIACC} 2, was made available. \ac{LIACC} 2 employed a \ac{GPU} Quadro RTX 8000 with 48 \ac{GB} of memory, 50 \ac{GB} of disk space, and 128 \ac{GB} of \ac{RAM}, allowing for extensive training and evaluation of the GANmix model.

These hardware configurations were chosen to facilitate GANmix model training and evaluation throughout the developmental stages.

The GANmix model was implemented using the Python programming language and the PyTorch deep learning framework. There is a further discussion about this framework in Section~\ref{sec:dl-frameworks}.

Before training, a preprocessing step was applied to the audio data by randomly Random cropping was done with a set duration of 5 seconds. Random cropping is discussed further in Section~\ref{sec:findings}. This approach enabled a more diverse dataset since the majority of its samples had longer durations. Random cropping aided in capturing diverse segments of the audio and improved the model's ability to generate realistic audio samples.

The GANmix model underwent training using a set of specific hyperparameters. The batch size ranged between 1 and 32, providing varying trade-offs between computational efficiency and model convergence. The training epochs varied from a few tens to a few hundreds, depending on the dataset and model complexity. 

The learning rates utilized to train the GANmix model ranged from $1 \times 10^{-5}$ to $1 \times 10^{-2}$, chosen based on empirical observations and prior research.

For some models, the training process was halted before reaching the maximum number of epochs due to evidence of convergence. This choice was made to optimize computational resources while obtaining satisfactory outcomes.
\documentclass{beamer}

% DELETE THE NEXT LINE FOR THE PRESENTATION
\setbeameroption{show notes on second screen=right}

\usetheme{MyTheme}

\usepackage{graphicx}
\usepackage{biblatex}
\usepackage{tabularx}

\addbibresource{config/references.bib}

\title{Synthesizing Audio from Textual Input}
\subtitle{Development and Comparison of Generative AI Models}
\author{Márcio Duarte \\
    \and Luís Paulo Reis (Supervisor) \\
    \and Gilberto Bernardes (Second Supervisor)
}
\institute{Faculdade de Engenharia da Universidade do Porto}
% TODO: Change date to the date of the presentation
\date{\today}

\begin{document}

\begin{frame}
  \titlepage
\end{frame}

\begin{frame}[t]
  \frametitle{Outline}
  \begin{columns}[T]
    \begin{column}{.5\textwidth}
      \tableofcontents[sections={1-2}]
    \end{column}
    \begin{column}{.5\textwidth}
      \tableofcontents[sections={3-}]
    \end{column}
  \end{columns}
\end{frame}
\chapter{Introduction} \label{chap:intro}

\minitoc

Audio is essential for shaping human experiences and interactions in the digital age. Audio content, ranging from podcasts and music to sound effects and immersive environments, is pervasive in our lives and enhances multimedia experiences. However, generating high-quality, diverse, and contextually relevant audio is still a time-consuming and labor-intensive process. As the demand for audio content continues to increase, there is a need for innovative solutions that can simplify the process and enable creators to generate various audio content with less effort.

Recent advances in \ac{AI} and \ac{DL} are revolutionizing different domains, such as image generation and text-to-speech synthesis. These cutting-edge models exhibit exceptional abilities to generate high-quality content from basic textual inputs. This thesis is motivated by these successful models and explores the potential of \ac{AI}-driven generative models to synthesize audio snippets based on textual descriptions. This thesis is not constrained to specific domains, like music or speech, but instead studies and implements systems suitable for a wide range of audio prompts.

The motivation behind this work lies in the potential benefits that an end-to-end audio generative model can offer content creators, sound engineers, and other stakeholders in the audio production ecosystem. By reducing the time and effort required to generate unique and high-quality audio samples, this research aims to contribute to the democratization of audio content creation and facilitate new avenues for creative expression.

This chapter provides an introduction to the research study and sets the stage for the exploration of the topic. It aims to supply a comprehensive overview of the present study's context, motivation, objectives, and structure.

The context information can be found in section \ref{sec:context}. Its focus is to provide the necessary background for the research study and to explain its importance. Section \ref{sec:motivation} explores the motivation behind the study. The purpose of this section is to explain the necessity of the research and the knowledge gap it aims to address. Next, in section \ref{sec:objectives}, the research study's objectives are described. This section aims to explain the specific objectives and outcomes the research seeks to accomplish. The structure of the dissertation is presented in Section \ref{sec:struct}. This section offers an organized overview of the remaining chapters.

\section{Context} \label{sec:context}

According to the University of York \cite{university_of_york_what_nodate}, ``Computer Science is the study of computation and information''. In practice, engineers and computer scientists typically create programs that machines execute. Traditionally, these programs consist of a sequence of instructions for the computer to follow.

Currently, processing and analyzing vast datasets is imperative for most endeavors. Computer science enables the possibility of these commodities, ranging from running hospitals to creating songs for streaming platforms.

Currently, most endeavors require processing and analyzing vast datasets. These commodities are only possible with computer science, from running hospitals to creating the songs one listens to on a streaming platform.

This amount of data has given rise to a new type of application. They do not need to have these instructions wired in. Instead, algorithms learn these from data. To this end, we call this \acf{ML}.

Since computers were invented, the scientific community has wondered whether they can learn~\cite{mitchell_machine_1997}. \ac{ML} is a field devoted to understanding and building methods that let machines ``learn'' – that is, methods that leverage data to improve computer performance on some set of tasks~\cite{alpaydin_introduction_2020}. Modern \ac{AI} was born in 1958 when F. Rosenblatt \cite{rosenblatt_perceptron_1958}, in an attempt to understand the capability of higher organisms for perceptual recognition, generalization, recall, and eventually thinking, proposed three fundamental questions:

\begin{enumerate}
	\item ``How is information about the physical world sensed, or detected, by the biological system?''
	\item ``In what form is information stored, or remembered?''
	\item ``How does information contained in storage, or in memory, influence recognition and behavior?''
\end{enumerate}

With this, Rosenblatt theorized a mathematical system called the perceptron (explained in Section \ref{sec:feedforward} that, by following the supposed behavior of neurons in one's nervous system, was the central piece of a hypothetical system capable of answering these questions.

Then, during the 1960s, much work was put into convergence algorithms for the perceptron and models based on it. Both deterministic and stochastic methods were proposed \cite{fradkov_early_2020}. However, in 1969, Minksy and Papert \cite{marvin_minsky_perceptrons_1969} published a book demonstrating the limitations of perceptrons. Namely, the authors showed that perceptrons could only represent linear functions, and simple non-linear functions such as \ac{XOR} were impossible. As a result, the study of \ac{AI} was mainly halted until the 1980s. This period is usually called \textit{the first winter of \ac{AI}} \cite{fradkov_early_2020}.

During the 1980s, studies of learning under multilayer neural networks went underway \cite{fradkov_early_2020}. In 1986, Rumelhart et al. \cite{rumelhart_learning_1986} described a new learning procedure for networks of neurons. The procedure adjusts the weights of the network's connections to minimize a measure of the difference between the expected and the actual output, an error function. This method is still used nowadays and is called \textit{backpropagation}.

Fradkov et al.~\cite{fradkov_early_2020} argue that an intensive advertisement of the success of backpropagation and other computational advances produced great hope for future successes. However, real successes were not happening, and investments in \ac{ML} decreased again in the early 1990s. This period is called \textit{the second winter of \ac{AI}}~\cite{martinez_artificial_2019}.

The turn of the millennium saw a new rise in \ac{ML} technologies; this time, it was, until now, for good. According to Fradkov et al.~\cite{fradkov_early_2020}, this was due to three trends that emerged:

\begin{enumerate}
	\item The appearance of big data. Dealing with huge amounts of data is an interest not only to a small portion of scientists but to the whole market.
	\item Reduced cost of parallel computing with both software (with, for instance, Google's MapReduce \cite{dean_mapreduce_2004}) and hardware (with an investment in specialized hardware for \ac{ML} from companies such as NVidia).
	\item A newfound interest by scientists in new, more complex, \ac{ML} algorithms, denominated \textit{\acfp{DNN}}.
\end{enumerate}

The critical idea of this new \ac{ML} is that it can infer plausible models to explain the observed data. A machine can use such models to make predictions about future data and make rational decisions based on these predictions \cite{ghahramani_probabilistic_2015}. It involves training a model on a large dataset to learn patterns and relationships in the data and then make predictions or decisions based on those patterns. For instance, a computer program can learn from medical records which treatments are most effective against new diseases, or houses can learn from experience to optimize energy costs based on the particular usage patterns of their occupants.

In traditional \ac{ML}, the features input into the models were usually hand-picked by humans, which leads to errors. New models learn intermediate representations— a vector of features — from data. The model itself usually performs this feature extraction with more layers~\cite{goodfellow_deep_2016}. Hence, \acf{DL}. \Ac{DL} algorithms consist of multiple layers of interconnected nodes and are trained to learn complex patterns and relationships in the data. \Ac{DL} algorithms can automatically learn and extract features from data. They are particularly well-suited for tasks such as image and audio processing.

The use of \ac{DL} in everyday applications increased during the 2010s. Nowadays, \ac{DL} is relied upon for various computer-made tasks including text translation, recommender systems, fake news detection, spam filtering, image captioning, and even self-driving cars~\cite{dean_golden_2022}.  Generative models are a key tool for creating new data.

Generative models, which are a type of \ac{DL} model, can create synthetic data that is similar to a given training dataset. Generative models can produce new data that conforms to the distribution learned from the training data. In contrast to common learning objectives such as classification or regression tasks, which focus on labeling inputs or estimating mappings, generative models aim to replicate and capture hidden statistics in observed data~\cite{huzaifah_deep_2021}.

The scientific community turned its heads to generative models a few years ago. These models allow a variety of new applications and products. For instance, DALL-E~\cite{ramesh_zero-shot_2021} and DALL-E 2~\cite{ramesh_hierarchical_2022} allow the generation of images given any textual input (see Sections \ref{sec:dall-e} and~\ref{sec:dall-e-2}. GPT-3 is an extensive language model \cite{brown_language_2020} that powers applications such as ChatGPT, capable of generating text. Modern text-to-speech (as seen in Section~\ref{sec:tts} applications also rely on these technologies.

Because of the relevance and effectiveness of these new generative technologies, \ac{DL} has achieved mainstream status, and the general public is aware of the capabilities of these algorithms. Given the rise of \ac{AI} in the 2010s with predictions and classifications, it is plausible that the 2020s will be a decade of generative applications. Automating manual work to enhance human creativity has never been more feasible. Generative applications span diverse domains such as images, text, and audio. However, even in the context of a well-defined frame of reference and optimal individual categorization, it's important to recognize that models inevitably involve reduction to averages. This raises concerns about undesirable convergence and oversimplification of media~\cite{forero_j_are_2023}.

For the purpose of this thesis, audio is broadly classified into three main categories: music, speech, and soundscapes. Music consists of organized tones and rhythms created by humans or other living entities~\cite{oxford_english_dictionary_music_2023}. Speech encompasses all vocalizations produced by humans, which are used for communication and expression~\cite{holden_origin_2004}. Soundscapes refer to an acoustic environment that is experienced and understood by individuals in context, including natural and human-made sounds~~\cite{international_organization_for_standardization_iso_2014, schafer_tuning_1977}. These three categories encompass most of the sounds people encounter.

There are two types of soundscapes: those found in the physical world and capable of being recorded, and those that can be artificially created through methods like mathematical equations. When it comes to the latter, one can imagine sounds such as white or brown noise - which are simply repetitive mathematical patterns that can be generated with ease through computation. To the best of the author's knowledge, all remaining sounds were either recorded or generated through sampling or other creative techniques, such as capturing the behavior of vibrations~\cite{trautmann_classical_2003}.

Using neural networks, deep generative models can produce audio from parameters. These models learn hidden data patterns to create new samples that match the training data's distribution \cite{huzaifah_deep_2021}.

Despite significant advances in generative technologies, research on audio synthesis has fallen behind. Although image generation has achieved impressive levels of realism and text generation is capable of passing medical exams~\cite{strong_chatbot_2023}, differentiating when a \ac{DL} process generates a specific sound is still relatively easy. Most of the sound-related research is focused on \ac{TTS} applications, which still need further improvement. Recent developments suggest a changing landscape.

In 2023, companies have significantly increased their investments in advancing the audio synthesis capabilities for music, voice, and soundscapes. The renewed focus aims to close the gap between state-of-the-art image generation models and general sound synthesis tools. The objective is to develop advanced algorithms capable of creating highly realistic audio outputs in various domains.
\section{Motivation} \label{sec:motivation}

In recent years, there has been a significant increase in the use of \ac{ML} techniques for audio processing tasks, such as sound synthesis, audio restoration, and speech recognition. The ability of \ac{ML} algorithms to learn and extract complex patterns from large datasets has shown promising results in improving the quality and efficiency of audio processing tasks.

Furthermore, integrating \ac{ML} techniques in sound generation technologies can revolutionize how one creates and experiences sound. It can provide new avenues for artists and musicians to explore their creativity and produce unique and innovative audio content. It can also offer new possibilities for sound design in various industries, such as film, gaming, and virtual reality.

The current need for studies in sound generation technologies highlights the need for further research and development. This dissertation endeavors to establish itself as a significant study in this field. It offers high-quality resources to researchers and developers to investigate the potential and limitations of \ac{ML} techniques for sound synthesis.

In today's world, digital technologies are reshaping our relationship with music and sound by enabling innovative capabilities \cite{tahiroglu_-terity_2020}. This study strives to investigate and broaden this impact by offering a novel tool that bolsters human potential in sound creation. Additionally, the findings can be a valuable resource for audio processing researchers, developers, and practitioners. The research findings offer guidance for future research and development endeavors concerning sound generation technologies and provide valuable insights into the most effective practices and techniques for utilizing \ac{ML} in audio processing.

Overall, this study's significance and potential impact make it a worthwhile and valuable contribution to the field of audio processing and machine learning.
\section{Objectives} \label{sec:objectives}

To address the core motivations behind this research, this dissertation aims to undertake a comprehensive study of \ac{DL} and, more specifically, generative deep learning models in the context of sound synthesis.

The goals include conducting a comprehensive survey of existing \ac{DL} architectures for audio generation while analyzing their strengths and limitations. In addition, novel approaches will be proposed and developed to further advance the field.

By pursuing these revised objectives, this research aims to provide valuable insights into state-of-the-art in sound synthesis using \ac{DL} methods. Furthermore, it aims to provide practical guidance for future advances in creating high-quality audio outputs based on textual inputs.

In order to accomplish this implementation, some specific goals are set:

\begin{enumerate}
	\item Make a study of the current state-of-the-art deep learning architectures, focusing on generative ones.
	\item Examine prior algorithms that can process sound for augmentation, feature extraction, or other purposes.
	\item Make a study of the current state-of-the-art architectures used to develop sounds artificially.
	\item Develop end-to-end systems that can synthesize sound from any given text input, while accounting for hardware constraints and ensuring reliable performance.
	\item Evaluate the systems' ability to generate a sound from the given textual input accurately.
\end{enumerate}
\section{Dissertation Structure} \label{sec:struct}

The present dissertation commences with a comprehensive examination of the current state-of-the-art technologies pertaining to sound generation. The analysis encompasses sound generation and delves into the realm of deep learning and generative deep learning architectures, which are not limited to sound. Subsequently, the dissertation examines additional tools, such as data augmentation for sound and sound analysis. Then, the focus shifts to a more in-depth study of generative deep learning technologies specific to sound, including vocoders, end-to-end tools, and other related terms, which are thoroughly explained.

The following section of the dissertation focuses on formulating and defining the problem at hand. Subsequently, practical applications of the developed technologies are demonstrated, and the dissertation investigates the various decisions and consequences that arise in developing such a system.

The methodology and approach adopted to fulfill the thesis's objectives are outlined in the solution section. An overview of the work carried out, its results, and the work plan is presented in detail.

Finally, the dissertation concludes by assessing the extent to which the objectives proposed in the introduction have been met and by presenting a summary of the findings. To facilitate navigation and ease of reference, each chapter of the dissertation includes a table of contents.

\section{State of the Art}

\subsection{Introduction}
\begin{frame}
    \frametitle{State of the Art}
    \begin{itemize}
        \item There are different types of sound generation methods, depending on the level of abstraction and supervision involved
        \item Four types of sound generation methods are considered in this thesis:
              \begin{itemize}
                  \item Traditional
                  \item Unsupervised
                  \item Vocoders
                  \item End-to-end models
              \end{itemize}
    \end{itemize}
\end{frame}

\subsection{Traditional Soundscape Generation}

\begin{frame}
    \frametitle{Traditional Soundscape Generation}

    \textbf{Notable Tools}
    \begin{itemize}
        \item \textit{Scaper}~\cite{salamon_scaper_2017}: Open-source library for synthetic sound environments
        \item SEED~\cite{bernardes_seed_2016}: System for resynthesizing environmental sounds with precise control over variation
        \item Physics-Based Concatenative Sound Synthesis~\cite{magalhaes_physics-based_2020}: Creates novel auditory experiences by assembling pre-existing sound segments
    \end{itemize}
\end{frame}

\subsection{Unsupervised Sound Generation}

\begin{frame}
    \frametitle{Unsupervised Sound Generation}

    \textbf{Approach}
    \begin{itemize}
        \item Learn sound features and distributions without explicit labels
        \item Utilize unlabeled audio data for pattern capture and structure learning
        \item Valuable when labeled datasets are limited or costly
    \end{itemize}

    \textbf{Notable Models}
    \begin{itemize}
        \item WaveGAN~\cite{donahue_adversarial_2019}: Unsupervised waveform synthesis using modified GAN
        \item Generative Transformer~\cite{verma_generative_2021}: Autoregressive prediction of audio samples using transformer networks
        \item wav2vec 2.0~\cite{baevski_wav2vec_2020}: Speech generation model with convolutional feature encoder, Transformer, and quantization module
        \item SoundStream~\cite{zeghidour_soundstream_2021}: Neural audio codec for efficient audio compression
    \end{itemize}
\end{frame}


\subsection{Vocoders}

\begin{frame}
    \frametitle{Vocoders}

    \textbf{Notable Models}
    \begin{itemize}
        \item WaveNet~\cite{oord_wavenet_2016}: Generative neural network using dilated causal convolutions for raw audio waveform generation
        \item WaveNet Variants: Models like WaveRNN, FloWaveNet, and Fast WaveNet reduce complexity while maintaining effectiveness
        \item MelGAN~\cite{kumar_melgan_2019}: GAN-based model using Mel-Spectrograms for coherent audio waveform generation
        \item GANSynth~\cite{engel_gansynth_2019}: GAN using log-magnitude spectrograms and phases for waveform generation
        \item HiFi-GAN~\cite{kong_hifi-gan_2020}: GAN model combining efficiency and high-fidelity speech synthesis
    \end{itemize}
\end{frame}


\subsection{End-to-End Models}


\begin{frame}[allowframebreaks]
    \frametitle{End-to-End Audio Models Comparison}

    \begin{table}[ht]
        \centering
        \caption{A comparison of different end-to-end generative models for audio.}
        \begin{tabularx}{\textwidth}{|X|l|X|X|}
            \hline
            \textbf{Model}                           & \textbf{Type} & \textbf{Input}            & \textbf{Output}                        \\ \hline
            Char2wav~\cite{sotelo_char2wav_2017}     & Speech        & Text prompt               & Raw audio waveform                     \\ \hline
            VALL-E~\cite{wang_neural_2023}           & Speech        & Text and acoustic prompt  & Raw audio waveform                     \\ \hline
            Jukebox~\cite{dhariwal_jukebox_2020}     & Music         & Genre, artist, and lyrics & Raw audio waveform                     \\ \hline
            Riffusion~\cite{forsgren_riffusion_2022} & Music         & Text prompt               & Raw audio waveform                     \\ \hline
            MusicLM~\cite{agostinelli_musiclm_2023}  & Music         & Text prompt               & Raw audio waveform                     \\ \hline
            SampleRNN~\cite{mehri_samplernn_2017}    & General       & None                      & Raw audio waveform                     \\ \hline
            AudioLM~\cite{borsos_audiolm_2022}       & General       & Text prompt               & Raw audio waveform                     \\ \hline
            DiffSound~\cite{yang_diffsound_2022}     & General       & Text prompt               & Mel-spectrogram and raw audio waveform \\ \hline
            AudioGen~\cite{kreuk_audiogen_2023}      & General       & Text prompt               & Mel-spectrogram and raw audio waveform \\ \hline
        \end{tabularx}
        \label{tab:end-to-end-audio-models}
    \end{table}
\end{frame}

\begin{frame}
    \frametitle{Text-to-Speech (TTS)}

    \textbf{Definition}
    \begin{itemize}
        \item Convert written text into synthesized speech
        \item Use deep neural networks for direct mapping
        \item Notable TTS Models:
              \begin{itemize}
                  \item Char2wav
                  \item VALL-E
              \end{itemize}
    \end{itemize}
\end{frame}

\begin{frame}
    \frametitle{Generative Music}

    \textbf{Definition}
    \begin{itemize}
        \item Create music using generative techniques
        \item End-to-end models for composing new musical pieces
        \item Notable Generative Music Models:
              \begin{itemize}
                  \item Jukebox
                  \item Riffusion
                  \item MusicLM
              \end{itemize}
    \end{itemize}

\end{frame}

\begin{frame}
    \frametitle{General Text-to-Audio}

    \textbf{Definition}
    \begin{itemize}
        \item Convert various forms of text to corresponding audio outputs
        \item Applications: sound effects, voice transformation, environmental sound synthesis
        \item Notable Text-to-Audio Models:
              \begin{itemize}
                  \item SampleRNN
                  \item AudioLM
                  \item DiffSound
                  \item AudioGen
              \end{itemize}
    \end{itemize}

\end{frame}


\subsection{AudioLM}
\begin{frame}{AudioLM}
    \begin{itemize}
        \item A framework for high-quality audio generation with long-term consistency~\cite{borsos_audiolm_2022}.
        \item Maps input audio to a sequence of discrete tokens and treats audio generation as a language modeling task.
        \item Achieves high-quality synthesis and long-term structure through a hybrid tokenization scheme of semantic and acoustic tokens.
        \item Consists of three main components: tokenizer, language model, and detokenizer.
        \item Generates syntactically and semantically plausible speech and music continuations without any transcript or annotation.
    \end{itemize}
\end{frame}

\subsection{DiffSound}

\begin{frame}{DiffSound}
    \begin{itemize}
        \item A novel text-to-sound generation framework that uses a text encoder, a VQ-VAE, a decoder, and a vocoder~\cite{yang_diffsound_2022}.
        \item Takes text as input and outputs synthesized audio corresponding to the input text.
        \item Uses a diffusion decoder (DiffSound) that predicts and refines all Mel-Spectrogram tokens in one step, resulting in better and faster generation than an AR decoder.
        \item Produces high-quality sound synthesis for various domains such as speech, music, and environmental sounds.
    \end{itemize}
\end{frame}


\subsection{AudioGen}
\begin{frame}{AudioGen}
    \begin{itemize}
        \item An auto-regressive generative model that generates audio samples conditioned on text inputs~\cite{kreuk_audiogen_2023}.
        \item Learns a discrete representation of the raw audio using an AE method and trains a Transformer language model over the learned codes, conditioned on textual features.
        \item Uses an augmentation technique that mixes different audio samples to train the model to separate multiple sources internally.
        \item Explores the use of multi-stream modeling for faster inference, allowing the use of shorter sequences while maintaining a similar bitrate and perceptual quality.
        \item Outperforms evaluated baselines over both objective and subjective metrics and extends to conditional and unconditional audio continuation.
    \end{itemize}
\end{frame}

\section{Development}

\subsection{Datasets}

\begin{frame}
    \frametitle{Table of Datasets}

    \begin{table}[ht]
        \centering
        \caption{Comparison of datasets for soundscapes}
        \label{tab:datasets}
        \begin{tabularx}{\textwidth}{|X|X|X|X|X|}
            \hline
            \textbf{Name}                                     &
            \textbf{Type}                                     &
            \textbf{\# Samples}                               &
            \textbf{Duration}                                 &
            \textbf{Labels}                                     \\ \hline

            Acoustic Event Dataset \cite{takahashi_deep_2016} &
            Categorical labeled                               &
            5223                                              &
            Average 8.8s                                      &
            One of 28 labels                                    \\ \hline

            AudioCaps \cite{kim_audiocaps_2019}               &
            Descriptive labeled                               &
            39597                                             &
            10s each                                          &
            9 words per caption                                 \\ \hline

            AudioSet \cite{gemmeke_audio_2017}                &
            Categorical labeled                               &
            2084320                                           &
            Average 10s                                       &
            One or more of 527 labels                           \\ \hline

            Audio MNIST~\cite{becker_interpreting_2018}       &
            Categorical labeled                               &
            30000                                             &
            Average 0.6s                                      &
            One of 10 labels                                    \\ \hline
        \end{tabularx}
    \end{table}
\end{frame}

\begin{frame}
    \frametitle{Table of Datasets (contd.)}

    \begin{table}[ht]
        \centering
        \caption{Comparison of datasets for soundscapes (contd.)}
        \begin{tabularx}{\textwidth}{|X|X|X|X|X|}
            \hline
            \textbf{Name}                                           &
            \textbf{Type}                                           &
            \textbf{\# Samples}                                     &
            \textbf{Duration}                                       &
            \textbf{Labels}                                           \\ \hline

            Clotho \cite{drossos_clotho_2019}                       &
            Descriptive labeled                                     &
            4981                                                    &
            15 to 30s                                               &
            24 905 captions (5 per audio).                            \\ \hline

            FSDKaggle 2018 \cite{fonseca_general-purpose_2018}      &
            Categorical labeled                                     &
            11073                                                   &
            From 300ms to 30s                                       &
            One or more of 41 labels                                  \\ \hline

            Urban Sound 8K \cite{salamon_dataset_2014}              &
            Categorical labeled                                     &
            8732                                                    &
            Less or equal to 4s                                     &
            One of 10 labels                                          \\ \hline

            YouTube-8M Segments \cite{abu-el-haija_youtube-8m_2016} &
            Categorical labeled                                     &
            237000                                                  &
            5s                                                      &
            One or more of 1000 labels                                \\ \hline
        \end{tabularx}
    \end{table}
\end{frame}


\subsection{Exploratory Experiments}

\begin{frame}
    \frametitle{Exploratory Experiments}

    \begin{itemize}
        \item Objective: Gain insights for GANmix development
        \item Experiments: Classification, GAN, AE, VAE
        \item Findings: Foundation for audio representation, GAN effectiveness, AE/VAE capabilities
        \item Impact: Crucial for robust audio generation with GANmix
    \end{itemize}

\end{frame}

\subsection{GANmix}

\begin{frame}{GANmix}

    \begin{figure}
        \centering
        \includegraphics[height=0.8\textheight]{images/3-development/ganmix.pdf}
        \caption{GANmix architecture}
        \label{fig:ganmix}
    \end{figure}

    \note{
        \textbf{Introduction}
        \begin{itemize}
            \item GANmix: Fusion of GAN and VAE for audio generation under constraints.
            \item Addresses computational limitations for high-quality audio.
            \item Combines GAN's generative power with VAE's latent space manipulation.
        \end{itemize}

        \textbf{Model Architecture}
        \begin{itemize}
            \item Generator and discriminator operate in latent space.
            \item VAE Training: Computational challenge, requires extensive datasets.
            \item AudioLDM's High-Performance: Top model for audio generation.
            \item Accessibility of AudioLDM: Open source, accessible via Hugging Face's model hub.
        \end{itemize}

        \textbf{Early Results}
        \begin{itemize}
            \item Preliminary experiments with Audio MNIST: Promising but suboptimal.
            \item Refinements: Different optimizers, model sizes, loss functions.
            \item Clotho dataset: Significant improvement in generated audio quality.
            \item Challenges in achieving equilibrium between generator and discriminator.
        \end{itemize}

        \textbf{Final Model}
        \begin{itemize}
            \item GANmix architecture with Clotho dataset: Significant improvements.
            \item Unlike typical models using CNN, GANmix uses fully connected neural networks.
            \item Generator input: Random Gaussian noise, passes through hidden layers.
            \item Discriminator: Takes embedding as input, applies tanh activation.
            \item Loss function: BCE. Optimized with Adam. Learning rate updates every 10 epochs.
        \end{itemize}
    }
\end{frame}

\section{Results}

\begin{frame}
    \frametitle{Results}

    \begin{itemize}
        \item Setup: Overview of experimental conditions and configurations.
        \item Presentation of Results: Showcase of outcomes from GANmix experiments.
        \item Discussion: Analyzing and interpreting the obtained results.
        \item Constraints and Challenges: Addressing limitations and difficulties encountered.
    \end{itemize}
\end{frame}


\subsection{Setup}

\begin{frame}
    \frametitle{Experimental Setup}

    \begin{itemize}
        \item GANmix model trained using BCE loss in PyTorch.
        \item Utilized two datasets: Audio MNIST and Clotho for diverse training.
        \item Three hardware setups: Kaggle, LIACC 1, LIACC 2, tailored for resources.
        \item Implementation: Python, PyTorch framework.
        \item Preprocessing: Randomly cropped samples to 5 seconds for diversity.
        \item Hyperparameters adjusted for batch size, epochs, and learning rates.
        \item Stopped training based on convergence for resource efficiency.
    \end{itemize}
\end{frame}

\subsection{Presentation of Results}

\begin{frame}
    \frametitle{Presentation of Results}

    \begin{itemize}
        \item Objectives: Explore generative AI models for audio production, assess performance.
        \item Context: Each experiment designed with specific research questions and hypotheses.
        \item Methodology: Hardware, software, and architecture details provided for transparency.
        \item Evaluation: Performance assessed through evolving loss plots and spectrograms.
        \item Systematic Organization: Ensures a comprehensive understanding of procedures and results.
        \item Basis for Analysis: Provides foundation for discussing effectiveness of generative AI models.
    \end{itemize}
\end{frame}

\subsection{Experiment X: Title} \label{sec:expX}

\begin{frame}
    \frametitle{Experiment X: Title}

    \begin{itemize}
        \item Objectives: [Objectives of the experiment]
        \item Model Details: [Describe key details of the model used, e.g., parameters, loss function]
        \item Dataset: [Mention the dataset used for training and evaluation]
        \item Optimizer and Learning Rate: [Specify the optimizer and learning rate used]
        \item Training Process: [Provide essential details about the training process, e.g., convergence status]
    \end{itemize}

    [Optional: Any unique aspects or considerations for this experiment]

    % \begin{figure}[!ht]
    %     \centering
    %     \begin{subfigure}{0.45\textwidth}
    %         \includegraphics[width=\textwidth]{figures/4.5-results/expX_loss.png}
    %         \caption{Evolving losses throughout the training process for Experiment X.}
    %         \label{fig:expX_loss}
    %     \end{subfigure}
    %     \begin{subfigure}{0.45\textwidth}
    %         \includegraphics[width=\textwidth]{figures/4.5-results/expX_spectrogram.png}
    %         \caption{Spectrogram generated in Experiment X.}
    %         \label{fig:expX_spectrogram}
    %     \end{subfigure}
    %     \caption{Results of Experiment X.}
    %     \label{fig:expX_results}
    % \end{figure}

    [Optional: Any additional insights or observations from this experiment]

\end{frame}

\subsection{Discussion}

\section{Analysis and Interpretation} \label{sec:res-analysis}

\begin{frame}
    \frametitle{Analysis and Interpretation}
    
    \begin{itemize}
        \item Identifying Trends
        \item Results for Future Investigation
        \item Interpretation of Results
        \item Conclusion
    \end{itemize}
    
\end{frame}

\begin{frame}
    \frametitle{Identifying Trends}

    \begin{itemize}
        \item Inverse correlation between generator and discriminator losses
        \item Convergence tends to plateau after a certain number of epochs
        \item Impact of learning rate on convergence speed
        \item Influence of optimization algorithms (e.g., SGD, RMSprop, Adam)
        \item Benefits of regularization methods (e.g., dropout, batch normalization, Gaussian noise)
        \item Importance of dataset size
        \item Exploration of latent space
    \end{itemize}
    
\end{frame}

\begin{frame}
    \frametitle{Results for Future Investigation}

    \begin{itemize}
        \item Occurrence of performance decline and NaN losses in certain experiments
        \item Further exploration of elastic network regularization
        \item Investigation of continuously increasing generator loss
    \end{itemize}
    
\end{frame}

\begin{frame}
    \frametitle{Interpretation of Results}

    \begin{itemize}
        \item Results didn't meet initial expectations but show potential
        \item Latent space exploration as a promising strategy
        \item Limitations of small datasets, especially in audio length and quantity
        \item Need for access to comprehensive datasets
        \item Computational resource challenges
    \end{itemize}
    
\end{frame}

\begin{frame}
    \frametitle{Conclusion}

    \begin{itemize}
        \item Analysis and interpretation of trends and patterns
        \item Potential for future advances in generative AI models for audio synthesis
        \item Lack of satisfactory practical results due to dataset limitations
        \item Importance of comprehensive datasets and computational resources
    \end{itemize}
    
\end{frame}

\subsection{Constraints and Challenges}

\section{Constraints and Challenges} \label{sec:res-limitations}

\begin{frame}
    \frametitle{Constraints and Challenges}
    
    \begin{itemize}
        \item Hardware Resources
        \item Data Quality and Quantity
        \item Hyperparameter Tuning
    \end{itemize}
    
\end{frame}

\begin{frame}
    \frametitle{Hardware Resources}

    \begin{itemize}
        \item Scarcity of hardware resources for training and evaluation
        \item Challenges in accessing sufficient computing power and memory
        \item Strategies adopted to optimize hardware usage
        \item Impacts, trade-offs, and opportunities resulting from resource limitations
    \end{itemize}
    
\end{frame}

\begin{frame}
    \frametitle{Data Quality and Quantity}

    \begin{itemize}
        \item Challenges posed by the quality and quantity of available data
        \item Importance of high-quality and diverse data for generative models
        \item Strategies employed to mitigate data limitations
        \item Considerations regarding data augmentation techniques
    \end{itemize}
    
\end{frame}

\begin{frame}
    \frametitle{Hyperparameter Tuning}

    \begin{itemize}
        \item Time constraints and challenges in hyperparameter tuning
        \item Significance of hyperparameters in model performance
        \item Impact of default or arbitrary values on model potential
        \item Recommendations for future work in hyperparameter optimization
    \end{itemize}
    
\end{frame}

\begin{frame}
    \frametitle{Conclusion}

    \begin{itemize}
        \item Discussion of major limitations and challenges faced in solution development
        \item Description of strategies employed to address these issues
        \item Possible implications, trade-offs, and opportunities arising from constraints
        \item Affirmation of the proposed solution's strengths and advancements in generative AI models for audio synthesis
    \end{itemize}
    
\end{frame}

\section{Conclusions}

\subsection{Overview and Reflections}

\begin{frame}
    \frametitle{Overview and Reflections}
    \begin{itemize}
        \item Comprehensive Study of State-of-the-Art Deep Learning Architectures for Audio Synthesis
        \item Development of End-to-End Systems for Sound Synthesis and Evaluation
        \item Challenges and Lessons Learned
    \end{itemize}
\end{frame}

\subsection{Future Directions}

\begin{frame}
    \frametitle{Future Directions}
    \begin{itemize}
        \item Exploring Novel Architectures
        \item Dataset Expansion
        \item Evaluation Metrics
    \end{itemize}
\end{frame}

\subsection{Novel Architectures}


\begin{frame}
    \frametitle{Novel Architectures}
    \begin{itemize}
        \item Briefly introduce the proposed theoretical architectures
        \item Mention their objectives, design principles, and potential applications
        \item Highlight the need for future research and development in this area
    \end{itemize}
\end{frame}

\subsection{Conclusion}

\begin{frame}
    \frametitle{Conclusion}
    \begin{itemize}
        \item Summarize the main achievements and contributions of the research
        \item Emphasize the progress made in understanding and developing generative AI models for audio synthesis
        \item Acknowledge the challenges and ongoing work required for further advancements
    \end{itemize}
\end{frame}


\end{document}
\section{Analysis and Interpretation} \label{sec:res-analysis}

This section analyzes and assesses the results of experiments conducted on GANmix. The evaluation identifies trends, investigates potential future directions for research, and interprets the results in the context of the research goals and the broader field.

\subsection{Identifying Trends}

Examination of the experimental results reveals several patterns and trends. There is an inverse correlation between generator and discriminator losses; as one increases, the other decreases. However, in certain cases both losses decrease together, which is beneficial. Also, convergence tends to plateau after a certain number of epochs.

This can be seen in Figure~\ref{fig:exp1_loss}. There, it can be seen that the loss of the generator decreases as the loss of the discriminator increases, and then the opposite occurs. Eventually both losses level off and decrease slightly.

The study found that the speed of convergence was affected by the learning rate. Although higher learning rates led to a faster convergence plateau, they did not necessarily lead to better results. This was demonstrated in Experiment 2 (Section~\ref{sec:exp2}), where the learning rates were increased and a loss plot is observed that is similar to Experiment 1 (\ref{sec:exp1}), but with a significantly faster plateau.

In Experiment 4 (Section~\ref{sec:exp4}), it appears that \ac{SGD} initially outperforms RMSprop, although it learns at a significantly slower rate. The results seem to be similar to Adam. Given the limited training time, one can only make assumptions, but it seems that using \ac{SGD} as the optimization algorithm leads to better performance than alternative methods such as RMSprop and Adam, despite the slower convergence.

Regularization methods such as dropout, batch normalization, and Gaussian noise have been shown to improve results and extend convergence, as shown in Experiment 5 (\ref{sec:exp5}). Although the spectrogram was suboptimal, the loss curves showed a healthy trend, with both losses decreasing consistently over time. Although the application of the elastic net regularization presented some challenges, it showed promise. The model performed better overall and achieved faster convergence when the generator and discriminator had similar parameter sets.

It was determined that larger models resulted in more rapid and resilient convergence. However, it should be noted that achieving satisfactory results was critically impacted by the size of the dataset. Typically, larger datasets and models produced better outcomes.

The embeddings generated by AudioLDM's \ac{VAE} show similarities characterized by a normal-like distribution with a significantly low standard deviation. Figure~\ref{fig:original-latents-hist} shows a histogram of these values, which extend to hundreds on the x-axis due to the residuals. It can be seen that the distribution is predominantly concentrated in an area close to zero.

\begin{figure}[ht]
    \centering
    \includegraphics[width=0.6\textwidth]{figures/4.5-results/real_distribution.png}
    \caption{An histogram that represents the latent values created by AudioLDM's \ac{VAE}.}
    \label{fig:original-latents-hist}
\end{figure}

Experience 10 (\ref{sec:exp10}) showed an interesting trend. However, it is worth noting that although the generated spectrograms showed improvement, the generator loss increased while the discriminator loss consistently decreased after a few epochs, as can be seen in Figure~\ref{fig:exp10_loss}. The use of linear layers instead of convolutions proved to be advantageous in generating more robust spectrograms. Further investigation is needed to determine if there are problems with the optimizer.

This last experiment, which used linear layers, produced the most encouraging results and was chosen as the primary study for this dissertation.

\subsection{Results for Future Investigation}

In some experiments, a decline in performance after a few epochs was observed, resulting in losses becoming \ac{NaN}. This is seen in Experiments 8 and 9 (Sections~\ref{sec:exp8}, \ref{sec:exp9}). Further research is required to determine the underlying cause and prevent this occurrence in future experiments.

The impact of elastic network regularization on model performance requires further investigation. Despite the implementation challenges encountered, the initial results suggest that this regularization method has improved and shows potential effectiveness.

In addition, the issue of the continuously increasing generator loss in the last experiment requires further investigation in the future.

\subsection{Interpretation of Results}

The study's findings did not meet the expectations set forth in the thesis as the generated sounds are not state-of-the-art for generative models; however, they are encouraging. With sufficient data and time, the model has the potential to generate high-quality sound. This suggests that generative \ac{AI} models, especially \acp{GAN}, have significant capabilities in audio generation.

Furthermore, exploring the model's latent space is seen as a promising strategy for achieving better results. The latent space method is a simpler approach that can potentially yield more favorable outcomes for future models.

The limitations of the small datasets used in this analysis are evident. The bigger one, Clotho, comprised sounds lasting from 15 to 20 seconds, but with 5 seconds removed from each sound, certain sounds were too short to produce satisfactory results.  Additionally, the total sound count in the data set was less than 5,000, which isn't enough to properly train a generative model.

Thus, a significant concern uncovered in this study is the lack of an appropriate dataset. To achieve more precise results, it is critical to have access to comprehensive datasets. Nevertheless, it is imperative to recognize that the process of training models on large data sets involves significant computational resources that may not be readily available.

\subsection{Conclusion}

In summary, the analysis and interpretation of the data show trends and patterns observed in the experiments. The results did not meet initial expectations; however, they show potential for future advances in generative \ac{AI} models for audio synthesis. It is important to note that comparisons with state-of-the-art audio generation models are not discussed in this study due to the unsatisfactory practical results obtained. 
\section{Constraints and Challenges} \label{sec:res-limitations}

This section discusses the major constraints and challenges faced in developing the proposed solution. It also describes the strategies and solutions adopted or proposed to address these issues, and the potential impacts, tradeoffs, and opportunities that result.

\subsection{Hardware Resources}

One of the biggest challenges was the scarcity of hardware resources for training and evaluating these models. Generative models require large amounts of computing power and memory to process high-dimensional data and learn complex patterns. However, such resources are often limited or expensive to access, especially for individual researchers or small teams. This is a significant barrier to achieving state-of-the-art results in audio synthesis, as few companies and labs have the necessary hardware capabilities.

To overcome this challenge, several strategies have been employed to optimize the use of available hardware resources. First, the neurons of the models were parallelized across the available \acp{GPU} (two in the final configuration --- \ac{LIACC} 2). This allowed to distribute the workload and speed up the training process. Second, checkpoints were implemented to save and load the model state at each epoch. This allowed the training to be resumed from where it was stopped in case of interruptions or failures. Third, the audio files were dynamically read and translated into spectrograms during training. This reduced memory consumption and disk space requirements. Fourth, gradient scaling and autocast (mixed precision training) techniques were used. These techniques involve performing some computationally expensive operations in 16-bit, while performing other numerically sensitive operations, such as accumulations, in 32-bit. This improved the performance and accuracy of the models while reducing memory consumption.

By applying these strategies, the models were trained and evaluated more efficiently and effectively. However, it is also recognized that these strategies have some limitations and trade-offs. For example, parallelizing neurons across \acp{GPU} can introduce communication overhead and synchronization issues. Checkpoints may not capture the full state of the model or optimizer. Dynamic data processing can increase pipeline latency and complexity. Mixed-precision training can introduce numerical errors or instability.

It is important to note that the hardware configuration with some computational power (\ac{LIACC} 2) was only available about a month before the submission of this thesis, so most of the work was done with really scarce resources. This means that the results presented in this thesis may not reflect the full potential or optimal performance of the proposed solution, as more experiments and improvements could be done with more hardware resources.

\subsection{Data Quality and Quantity}

Another challenge was the quality and quantity of available data. Generative models require a large amount of high-quality and diverse data to learn the underlying patterns and distributions of the data domain. However, such data is often scarce or difficult to obtain, especially for audio synthesis from textual input. Existing datasets for this task are either too small, focused on a specific domain, or lack descriptive labels. This limits the generalization and robustness of the models, as they may overfit to the training data or fail to capture the variability and richness of natural language and sound.

To mitigate this challenge, some data augmentation techniques were applied to increase the size and diversity of the data. For example, random cropping was used to generate different segments of audio from the same file. This increased the number of samples and introduced some variation in the data. However, it is also recognized that these techniques are not sufficient to solve the problem of data quality and quantity. Data augmentation may not produce realistic or novel samples but may introduce noise or artifacts into the data.

\subsection{Hyperparameter Tuning}

A final challenge was the time required for the hyperparameter tuning of the generative models. Hyperparameters are parameters that are not learned by the model, but are set by the user prior to training. They include learning rate, batch size, number of layers, number of neurons, activation functions, regularization methods, etc. Hyperparameters significantly impact the performance and behavior of the model, as they determine how the model learns from the data and adapts to different situations. However, finding the optimal values for these hyperparameters is often a tedious and time-consuming process involving trial-and-error experiments with different combinations of values.

Due to time constraints and deadlines, there was insufficient time to fine-tune our hyperparameters for these generative models. All models presented in this thesis are vanilla versions with default or arbitrary values for their hyperparameters. This means they may not reach their full potential or optimal performance in audio synthesis from textual input. Therefore, it is suggested that future work should devote more time and effort to the hyperparameter tuning of our generative models using methods such as grid search, random search, Bayesian optimization, etc.

\section{Conclusion}

In this section, the author discusses the major limitations and challenges they faced in developing the proposed solution. The author has also described how they addressed these issues and the possible implications, trade-offs, and opportunities that arose. Despite these challenges and limitations, the author believes that the proposed solution has several strengths and advantages that advance the state of the art in generative \ac{AI} models for audio synthesis.