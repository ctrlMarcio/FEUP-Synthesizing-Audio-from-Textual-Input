\section{Methodology and Approach} \label{sec:sol-approach}
 
The development of this dissertation is divided into three main parts:

\begin{itemize}
    \item The state-of-the-art research;
    \item The development;
    \item The writing of the document;
\end{itemize}
 
These areas were not necessarily exclusive. That is, they overlapped. Initially, work began with general research into the state of the art in modeling and audio processing. Early development began after establishing a basic understanding of what needed to be achieved. This development took place in parallel with the ongoing research. As issues arose during development, additional research was required to address them. Writing the document was an ongoing process that led to clearer thinking and planning. Nevertheless, writing was prioritized after the bulk of the development was complete.

The interaction between these phases is both dynamic and symbiotic. The state-of-the-art research phase is the foundation, providing information for future model development decisions. Research insights directly influence the design and direction of generative models. As development progressed, new challenges and possibilities emerged, triggering further exploration in the state-of-the-art literature. Similarly, the development phase provides empirical insights that contribute to refining the writing process. Model development findings and outcomes contribute to comprehensive and informative document content. This iterative interplay ensures that each phase informs, enriches, and refines the others, leading to a coherent and impactful dissertation.

\subsection{State-of-the-Art Research}

One of the goals of this thesis is to provide a comprehensive overview of the current state-of-the-art audio synthesis through textual input. A systematic and thorough approach was taken to search and review the relevant literature.

The process started with the collection of keywords related to the topic of audio synthesis and generative \ac{AI} models. Through multiple combinations, these keywords were then used to search multiple online sources, including academic search engines. Searches were also conducted for specific papers given the author's knowledge. The results of these searches were analyzed to identify publications with significant citations and high relevance to the thesis topic.

These publications were then carefully read and analyzed, and their citations were further investigated to expand the search and deepen the understanding of the state-of-the-art in the field. If any aspect of a publication was unclear, additional research was conducted to clarify the concept and find additional relevant literature.
This process of searching, reading, and analyzing was iterated multiple times, allowing the gathering of new keywords and publications. Notes and possible citations for each publication were stored in a private database, allowing easy access and organization.

\subsection{Model Development} \label{sec:problem-model-development}

The development of audio synthesis models is a progressive approach that aims to achieve three primary objectives. Together, these goals guide the development of the models and ensure a comprehensive exploration of audio synthesis through text.

The primary objective is establishing fundamental models that facilitate an in-depth understanding of sound representation and \ac{DL} principles. To achieve this, the initial focus was on tasks like audio classification. These fundamental models are developed using smaller datasets like Audio MNIST~\cite{becker_interpreting_2018}. This step provides a foundational grasp of the models' capabilities and acts as a basis for subsequent progress.

The second goal is to develop generative models to synthesize audio from textual input. These generative models are designed to create audio content that aligns with provided text prompts by building on the insights gained from fundamental models. The progressive approach provides the ability for the generative process's iterative refinement, optimization, and enhancement.

The third goal includes proposing theoretical models that establish the foundation for more advanced generative techniques. Although not all theoretical approaches are entirely implemented, these concepts add to the broader discussion of audio synthesis research. Exploring theoretical models boosts innovation and promotes the development of new strategies for generating audio from text.

The model development approach provides an exhaustive and iterative pathway to achieve an effective and innovative audio synthesis from textual input by pursuing these interconnected objectives.

\subsection{Writing of the Dissertation}

This dissertation was composed with a focus on clarity, conciseness, and informative content. A continuous approach was utilized to achieve this objective, consisting of three key phases for each section: initial drafting, incorporation of researched references, and ongoing refinement. This iterative process required multiple re-readings and revisions. This dynamic method aligns well with the principles of agile development~\cite{dingsoyr_decade_2012}, promoting flexibility and continuous progress during the writing process.

Furthermore, this document adheres to established academic norms, encompassing writing style, citational methodologies, and the lucid and consistent presentation of results and figures. In addition to the individual author's contributions, the insights of the thesis coordinators have strengthened this project. They have diligently reviewed and enhanced the document to ensure its precision and authenticity.

This dissertation, crafted with meticulous attention to detail, linguistic technologies, and rigor and scrutiny, is the culminating embodiment of the author's resolute research pursuits and the zenith of scholarly achievement.