\subsection{Data Generators} \label{sec:data-generators}

Creating new data from existing data is known as generative modeling. This technique has numerous applications across various media types, including images, text, and video. However, each media type possesses unique characteristics, so generating them requires distinct approaches and techniques. Sound, for instance, presents a different set of challenges than other media types, and hence its generation necessitates diverse techniques.

This section aims to survey some of the state-of-the-art generators for media types that are not sound-based, primarily focusing on image generators. By doing so, one hopes to comprehensively understand the different approaches and techniques employed for generating various media types.

The discussion delves into the main ideas behind these generators, their strengths and limitations, and how they can be related to or inspired by sound generation. Through this exploration, this section highlights the diversity of methods used for generative modeling and how they can be adapted to different contexts.

\begin{table}[ht]
\centering
\caption{Comparison of Data Generators}
\begin{tabularx}{\textwidth}{|X|X|X|X|}
\hline
\textbf{Generator Name} & \textbf{Main Idea}                                                                                                                    & \textbf{Strengths}                                                                                                                                 & \textbf{Limitations}                                                                                                          \\ \hline
PixelCNN~\cite{oord_conditional_2016}                & Uses autoregressive connections to model images pixel by pixel.                                                                       & Fast and parallelizable training, high resolution and diversity, interpretable latent space.                                                       & Slow and sequential sampling, limited global coherence, difficulty in conditioning on high-level features.                    \\ \hline
DALL-E~\cite{ramesh_zero-shot_2021}                  & A transformer language model that creates images from text descriptions, using a dataset of text–image pairs.                         & Can generate novel and creative images, combine concepts in plausible ways, render text and apply transformations.                                 & Requires large amounts of data and compute, may produce harmful or biased outputs, not publicly accessible.                   \\ \hline
Stable Diffusion~\cite{rombach_high-resolution_2021}        & A latent diffusion model that creates images from text descriptions, using a dataset of text–image pairs and a pretrained CLIP model. & Can generate realistic and detailed images, is lighter than previous state of the art models.                                                      & Requires fine-tuning for specific domains, may produce artifacts or inconsistencies, sensitive to text prompts.               \\ \hline
GLIDE~\cite{nichol_glide_2021}                   & Guided diffusion-based approach for text-conditioned image generation.                                                                & High-fidelity image synthesis, classifier-free guidance preferred for photorealism and caption similarity, text-driven image editing capabilities. & Difficulty generating images for complex or unusual text prompts, slow generation speed.                                      \\ \hline
DALL-E 2~\cite{ramesh_hierarchical_2022}                & An improved version of DALL-E that generates more realistic and accurate images using a diffusion model.                              & Has better results than DALL-E and is faster.                                                                                                      & Still requires large amounts of data and compute, may still produce harmful or biased outputs, still not publicly accessible. \\ \hline
\end{tabularx}
\label{tab:data-generators}
\end{table}

\subsubsection{PixelCNN Decoders} \label{sec:pixelcnn}

Van den Oord et al. introduced \textit{PixelCNNs} as discussed in \cite{oord_conditional_2016, van_den_oord_pixel_2016}. PixelCNNs model high-dimensional discrete data, like images.

They employ the \ac{AE} architecture as described in Section \ref{sec:darn}, where pixels are generated sequently while conditioning on prior pixels.

The goal is to estimate a distribution over images that can be used to compute the likelihood of images and thus generate new ones. The network scans the image one row at a time and one pixel at a time within each row, using masked convolutions (see Section~\ref{sec:masked-conv}). For each pixel, it predicts the conditional distribution over the possible pixel values given the scanned context.

Each image $x$ is assigned a probability $p(x)$. To do this, if one considers each pixel sequentially, row by row, such that $x_k$ is the pixel number $k$, the final probability is as follows.

\begin{equation}
    p(x) = \prod_{i=1}^{h \times w} p(x_i | x_1, \dots, x_{i-1})
\end{equation}

Where $h$ is the height of the image and $w$ is the width.

For multiple colors, one can condition the colors on one another. For instance, instead of having $x_1$, $x_2$, \dots, there would be $x_1^R$, $x_1^G$, $x_1^B$, $x_2^R$, and so forth.
\subsubsection{DALL-E} \label{sec:dall-e}

In 2021, the initial version of OpenAI's DALL-E was launched, which introduced zero-shot text-to-image generation \cite{ramesh_zero-shot_2021}. Zero-shot refers to the ability to generate data from textual descriptions without any specific training on those exact combinations of text and images. In other words, DALL-E's training is based on a rich variety of image-text pairs, which enables it to generate new images from textual inputs, relying solely on this training. This issue closely relates to the one addressed by the present research.

The overall structure is easy to comprehend. It primarily comprises of two phases. In the initial phase, image representations or features are learned using a \ac{dVAE}. In this context, a \ac{dVAE} is an enhanced version of the classic \ac{VAE} (see Section~\ref{sec:vae}) that employs discrete latent variables in lieu of continuous ones. Discretizing the latent space allows for explicit control and manipulation of generated images in the synthesis process using \acp{dVAE}.

On the other hand, the second stage produces image representations from textual descriptions using a transformer model, as detailed in Section \ref{sec:transformers}.

Figure \ref{fig:dall-e} illustrates this macro-architecture.

\begin{figure}[ht]
    \centering
    \ctikzfig{figures/2-sota/dall-e}
    \caption[Dall-E macro architecture]{\textbf{Dall-E macro architecture} --- With a green background, one can see textual operations. The complex on the right is a \ac{dVAE}. Upon inference, the transformer's output is used as the latent feature values of the \ac{dVAE}.}
    \label{fig:dall-e}
\end{figure}

The \acf{dVAE} trained compresses images into a $32 \times 32$ grid of image tokens. Each token assumes one of 8192 possible values. This reduces the size of the image, improving the performance without significant degradation in visual quality.

At the second stage, the text is first encoded using \ac{BPE}. \Ac{BPE} is a compression technique for representing text using a new, single symbol to replace frequent pairs of characters or symbols. The objective of \ac{BPE} is to merge the most frequent pairs of consecutive symbols, such as letters or bytes, in the given text, until a specific number of merge operations is reached through an iterative process. As a result, this process generates a modified set of symbols that represents the original text but with a smaller vocabulary.

Applying \ac{BPE} enables effective capture of repeating patterns and combinations within the text while achieving better compression than traditional character-level encoding. This encoding step is vital in preparing textual input for subsequent stages and helps enable more efficient processing and modeling.

After the encoding, the resulting symbols are passed as input to a transformer trained to generate the tokens corresponding to the image tokens.

To generate completely new images, the process would be as follows:

\begin{enumerate}
	\item Encode the text into \acp{BPE}.
	\item Pass the encoded text to the trained transformer encoder; This outputs image tokens.
	\item Pass these tokens to the decoder of the \ac{dVAE} trained in stage 1; This outputs the image.
\end{enumerate}
\subsubsection{Fine-Tune Stable Diffusion for Spectrograms}

A promising avenue of research in the \ac{DL} community is the fine-tuning of existing generative models to improve their capabilities and produce high-quality audio output. This section introduces the concept of fine-tuning generative models for audio synthesis, and proposes fine-tuning the stable diffusion model to generate spectrograms.

The stable diffusion model uses the principles of diffusion processes to generate high-fidelity and diverse image samples (see Section\ \ref{sec:stable-diffusion}). One of the key advantages of the stable diffusion model is its ability to capture complex dependencies and generate realistic output.

It is worth noting that the stable diffusion model is an open source model, meaning that its code and implementation details are publicly available. This accessibility allows researchers and practitioners to study, modify, and build upon the foundations of the model. The availability of open source code for the stable diffusion model facilitates its fine-tuning for specific tasks, such as audio synthesis.

The fine-tuning of the stable diffusion model to spectrograms provides an exciting opportunity to explore the generation of high-quality audio output based on this visual representation.

One of the key advantages of the stable diffusion model is its ability to capture complex dependencies and generate realistic images. By fine-tuning the stable diffusion model on spectrograms, one can leverage its prior knowledge and adapt it to the specific characteristics of audio signals represented in the frequency and time domains.

The potential benefits of using the prior knowledge of the stable diffusion model for audio synthesis are manifold. First, the stable diffusion model has already shown impressive results in other domains, in this case image synthesis. By building on this foundation, one can exploit the model's ability to generate high-fidelity and diverse outputs, which can greatly improve the quality of the synthesized audio.

Furthermore, fine-tuning the stable diffusion model to spectrograms can provide a unique perspective on audio synthesis. By treating spectrograms as images and applying the stable diffusion model, one can explore the potential of generating audio based on this visual representation. This approach opens up new possibilities for manipulating and synthesizing audio in innovative ways.

To prove the usefulness of this method, Riffusion (see Section~\ref{sec:riffusion}), which already performs a similar task of generating audio from visual representations, has shown considerable results. By considering the insights and techniques used in Riffusion, one can build on its foundations and adapt the stable diffusion model accordingly.

It is important to note that the success of the proposed approach depends on the availability of a sufficiently large dataset for fine-tuning.

\subsubsection{GLIDE} \label{sec:glide}

The \acf{GLIDE} model is a state-of-the-art approach for generating high-fidelity synthetic images from free-form natural-language text prompts. The model is based on a guided diffusion-based approach, which is the first attempt by OpenAI at text-conditioned image generation using guided diffusion. The approach involves two types of guidance strategies during model training: classifier-free guidance~\cite{ho_classifier-free_2022}, which relies solely on the model's knowledge, and \ac{CLIP} guidance which uses a pre-trained \ac{CLIP} model~\cite{radford_learning_2021} to provide guidance based on caption matching.

The experimental results of the \ac{GLIDE} paper demonstrate several benefits of the proposed approach. Human evaluators preferred images generated with classifier-free guidance over \ac{CLIP} guidance regarding photorealism and caption similarity. Samples from the 3.5 billion parameter \ac{GLIDE} model were also found to outperform DALL-E (see Section~\ref{fig:dall-e}) samples according to human evaluations. Additionally, the model can perform text-driven image editing tasks beyond zero-shot image generation from text prompts. Text-driven image editing refers to editing existing images according to text prompts, such as changing attributes or objects within an image as directed by the text.

Despite its success, the \ac{GLIDE} model has some limitations. It fails to generate images for some complex or unusual text prompts. Moreover, the model's generation speed is slow, taking several seconds to generate one image on a flagship \ac{GPU}. Possible solutions to address these limitations include improving the model architecture, optimization techniques, and combining \ac{GLIDE} with faster \ac{GAN}-based methods.

\Acf{CLIP} refers to a model trained to determine if an image and text caption match. It consists of a transformer-based text encoder (see Section \ref{sec:transformers}) and a convolutional neural network-based image encoder (see Section \ref{sec:CNN}). The text encoder produces an embedding of the text, and the image encoder produces an embedding of the image. These embeddings are then compared, and during training, the model learns to produce similar embeddings for matching image-text pairs and dissimilar embeddings for mismatching pairs. This contrastive learning approach allowed CLIP to learn cross-modal understanding between text and images in an unsupervised manner. The pre-trained CLIP model can provide additional guidance to other text-to-image models, such as \ac{GLIDE}, by scoring how well-generated images match given text prompts.
\subsubsection{DALL-E 2} \label{sec:dall-e-2}

\textit{DALL-E 2} is a model proposed by researchers at OpenAI capable of generating images given a textual prompt~\cite{ramesh_hierarchical_2022}. This model can also modify given images, but this use case is not so interesting for the work present in this thesis.

The model consists of two blocks: the prior and the decoder. The prior converts captions into a lower-level representation, while the decoder turns this representation into an actual image.

They use \ac{CLIP} \cite{radford_learning_2021} and GLIDE (see Section~\ref{sec:glide}). For DALLE-2, the text is initially embedded using \ac{CLIP} embeddings. Then, the role of the prior is to translate these embeddings into embeddings related to an image and not the text itself. In other words, create an image representation with textual embeddings. For this, the researchers tried an \ac{AR} and a diffusion model. The diffusion one yielded better results (see Sections~\ref{sec:darn} and~\ref{sec:diffusion}). The decoder then takes the generated image representation and generates the image. The whole process can be seen in Figure \ref{fig:dall-e-2}.

\begin{figure}[ht]
    \centering
    \includegraphics[width=\textwidth]{figures/2-sota/dall-e-2.png}
    \caption[DALL-E 2 architecture]{\textbf{DALL-E 2 architecture} --- The image was taken from the original paper. Above the dotted line, the \ac{CLIP} training process is depicted, where given textual and image embeddings, the \ac{CLIP} learns to translate one into the other. Below the dotted line, a text-to-image generation process is represented: a text embedding is first fed to the model that produces the image embedding. Then, this embedding is used to condition the diffusion model GLIDE which produces a final image.}
    \label{fig:dall-e-2}
\end{figure}

A model is also possible without the prior by passing the textual embeddings directly to the decoder. However, while the results were okay, they were way better with the generated image embeddings.

The decoder creates $64 \times 64$ images, but another network learns to upsample images until $1024 \times 1024$. Without this, generating high-resolution images with the decoder would make the whole operation incredibly heavy.

A significant problem of this model (and others presented here proposed by big companies) is that it needs hundreds of millions of images and an incredible amount of computation power to perform well. This highlights the importance of research toward openly accessible models such as stable diffusion (see Section~\ref{sec:stable-diffusion}).