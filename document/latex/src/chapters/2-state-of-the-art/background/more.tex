\subsection{Foundations for Enhancing Generative Models for Audio} \label{sec:parallel-tasks}

To develop generative models for audio, it is necessary to address several factors that impact their performance and quality. This thesis concentrates on three main areas: data augmentation, evaluation metrics, and data embedding.

\textit{Data augmentation} is the process of applying transformations to the original data to increase its size and diversity. This can help overcome the limitations of small or imbalanced datasets and improve the generalization ability of generative models. Different types of data augmentation techniques for sound generation and their effects on the model outcomes are discussed in Sextion \ref{sec:data-augmentation}.

\textit{Evaluation metrics} are the methods used to measure the quality and diversity of the generated sounds. They provide a way to compare different generative models and assess their strengths and weaknesses. However, evaluating sound generation is not trivial, as it involves objective and subjective criteria. We review various evaluation metrics for sound generation and their advantages and disadvantages in Section \ref{sec:evaluation}

\textit{Data embedding} is the technique of converting data into numerical representations that capture its essential features and characteristics. This can facilitate the learning process of generative models and enhance their expressiveness and efficiency. We explore different data embedding methods in section \ref{sec:text-embedding}.

\subsubsection{Data Augmentation} \label{sec:data-augmentation}

Data augmentation is crucial to \ac{ML} tasks. It helps to increase the training dataset's size and improve the model's robustness. For the task at hand, there are two main types of data augmentation: acoustic and linguistic. Data augmentation can reduce overfitting and improve generalization by introducing more variations into the training set~\cite{pluscec_data_2023}.

\paragraph{Acoustic Data Augmentation}

According to Abayomi-Alli et al. \cite{abayomi-alli_data_2022}, the most commonly used data augmentation tools for audio \ac{ML} tasks are the addition of noise, time shifting, pitch shifting, \ac{GAN}-based methods (see Section \ref{sec:gan}), time stretching, and concatenation. Other techniques, such as overlapping, are also helpful. All these techniques play a critical role in increasing the size of the training dataset and providing the model with a diverse range of input data. This Section thoroughly explores these techniques and discusses their impact on the performance of sound-based \ac{ML} models.

Table \ref{tab:acoustic-data-augmentation} provides a concise summary of the advantages and limitations of each technique.

\begin{table}[hp]
\caption{Summary of Acoustic Data Augmentation Methods}
\label{tab:acoustic-data-augmentation}
\begin{tabularx}{\textwidth}{|p{2cm}|X|X|X|}
\hline
\textbf{Method}     & \textbf{Description}                                                                             & \textbf{Advantages}                                                                                  & \textbf{Limitations}                                                                                             \\ \hline
Addition of Noise   & Adding random noise to the original audio signals. & Increases dataset diversity, helps model learn to handle noisy environments.                         & Noise must be chosen to align with the problem at hand.                            \\ \hline
Time Shifting       & Altering the temporal structure of an audio signal by shifting it. & Helps model learn sound patterns invariant to temporal changes, improves generalization. & Trimming and shifting sections may introduce artifacts.                                    \\ \hline
Pitch Shifting      & Altering an audio signal's frequency. & Trains models to recognize sound patterns free from pitch variations.                                & May introduce artifacts or distortions.                                       \\ \hline
\ac{GAN} Based Methods   & Utilizing \ac{GAN} networks to generate synthetic audio signals.            & Generates high-quality data.                      & Requires significant computational resources.                     \\ \hline
Time Stretching     & Changing the time duration while preserving their spectral content.             & Generates samples with different time durations.                     & Simple methods may introduce artifacts; sophisticated methods are computationally intensive.     \\ \hline
Sound Concatenation & Joining snippets of multiple audio signals to form a new one.               & Trains models that recognize specific sound sequences.                         & Labels associated with the original audio signals may need to be modified to represent the new signal. \\ \hline
Sound Overlapping   & Merging two or more audio signals to form a new combined audio signal.                           & Helps models identify sound patterns amidst overlapping sounds.              & Amplitude adjustment and normalization are crucial to prevent overpowering or clipping in the composite signal.  \\ \hline
\end{tabularx}
\end{table}

\subparagraph{Addition of Noise}
Data augmentation for audio with the addition of noise is a common technique that involves adding random noise to the original audio signals to produce new, diverse audio samples.

The process of adding noise to audio signals involves several steps:

\begin{enumerate}
    \item Select a noise source: This can be any type of noise, generated or real \cite{novotny_analysis_2019}, such as white noise, babble noise, static noise, factory noise, jet cockpit, shouting, background noise, and others, \cite{abayomi-alli_data_2022}. The noises should be chosen according to the problem at hand.
    \item Specify the noise level: For instance, the augmented sound might be $y' = y + 0.05 \times Wn$ where $y$ is the initial sound, $y'$ the augmented sound, and $Wn$ some white noise \cite{mushtaq_environmental_2020}.
    \item Add the noise: The selected noise source is then added to the audio signal by adding the noise and audio signals element-wise.
    \item Normalize the output: Finally, the resulting audio signal with added noise is normalized to prevent clipping or overloading.
\end{enumerate}

Repeating these steps with different noise sources and noise levels makes it possible to generate multiple, diverse audio samples that can be used for data augmentation purposes.

\subparagraph{Time Shifting} \label{sec:time-shifting}
Time shifting, also known as time warping, is a data augmentation technique that involves altering the temporal structure of an audio signal.

Time shifting involves shifting the entire audio signal by a certain amount of time, either forwards or backward. This can be achieved by adding or removing samples from the audio signal or changing the existing samples' position within the signal.

One approach to implementing time shifting involves trimming the length of the audio signal and using the trimmed sections to create new, diverse audio samples. For example, consider an audio signal of size 150. If this signal is trimmed to a length of 125, up to 25 new audio samples can be generated by shifting the trimmed sections. These new samples can be labeled the same as the original audio signal.

Time shifting by trimming and shifting sections of the audio signal can significantly impact the performance of sound-based \ac{ML} models. By providing the model with diverse, time-shifted versions of the audio signal, this technique can help the model learn to identify sound patterns invariant to temporal changes, such as the presence of a particular sound event or the spoken words in an audio recording. This can lead to better generalization performance on new, unseen data and improved overall model performance.

\subparagraph{Pitch Shifting}
Pitch shifting is a technique used in audio data augmentation that involves altering an audio signal's fundamental frequency.

Pitch shifting is achieved by adjusting the pitch of an audio signal positively or negatively. For example, a plus or minus two shift can be implemented \cite{mushtaq_environmental_2020}. This process results in audio signals that have a different pitch. This can be helpful in training models to recognize sound patterns that are free from pitch variations.

\subparagraph{GAN Based Methods}
Utilizing \acp{GAN} (see Section~\ref{sec:gan}) in data augmentation for audio signals can be a powerful and effective method, albeit slower than other techniques. In this approach, a \ac{GAN} network is trained on the available audio data to learn the underlying patterns and distributions present in the data. The network then generates new, synthetic audio signals similar to the input data \cite{qian_data_2019}.

The success of the \ac{GAN}-based data augmentation method depends heavily on the quality of the \ac{GAN} training, as well as the diversity of the input data. If the \ac{GAN} is trained well and the input data is diverse, the generated data will be of high quality.

It is important to note that \acp{GAN} require considerable computational resources and training time compared to other data augmentation techniques. However, the results obtained from \acp{GAN} can be highly effective and accurate, making this approach a valuable addition to the data augmentation toolkit for audio machine learning tasks.

\subparagraph{Time Stretching}
Time stretching as a data augmentation technique for audio signals involves changing the time duration of the audio signals, typically by increasing or decreasing the time axis of the audio signals. The purpose of time stretching is to generate new audio samples from the original audio signals with different time durations.

A straightforward way of implementing time stretching is to use a stretching factor. For example, if the stretching factor is $1.2$, then the time axis of the audio signal is increased by 20\%. To achieve this, one approach is to use a naive algorithm that duplicates some of the samples in the audio signal according to the stretching factor. However, this simple method can result in undesirable artifacts, such as pitch changes, if the stretching factor is not an integer.

More sophisticated methods, such as phase vocoder-based time stretching, can produce high-quality time stretching with minimal artifacts. These methods use time and frequency domain processing techniques to stretch the audio signal while preserving its spectral content and temporal structure. The resulting audio signal has a different time duration while preserving the original pitch~\cite{akaishi_improving_2023}.

\subparagraph{Sound Concatenation}
Mixing up sounds, or sound concatenation, is a method for audio signals where multiple audio signals are joined to form a new and diverse audio signal. This technique can be achieved by taking snippets of multiple audio signals and concatenating them randomly or using cross-fade techniques to ensure a seamless transition between the different audio snippets.

This technique would be advantageous in a sound generation setting where one wants to make the network learn a prompt such as ``dog barking and then car honking''. When applying sound concatenation, one must consider that the label will also change.

\subparagraph{Sound Overlapping}

Sound overlapping, also referred to as sound mixing or audio blending, is a technical process that involves merging two or more audio signals to form a new combined audio signal. This process is currently utilized in popular data augmentation platforms~\cite{maguolo_audiogmenter_2022} to aid the model in identifying sound patterns amidst overlapping sounds, which is a frequent occurrence in real-world applications.

There are several steps involved in the process of sound overlapping. The first step involves selecting multiple audio signals, which can originate from the same or different sources depending on the desired outcome and problem at hand. The next step is to adjust the amplitude of each audio signal to achieve a balanced combination and prevent one signal from overpowering others. This step is crucial. This can be achieved by either normalizing the amplitude of each signal or scaling them based on a predetermined factor. After appropriately adjusting the amplitudes, the selected audio signals are combined by adding them element-wise. A new composite audio signal is generated using this process, which includes overlapping sounds from the original signals. It is essential to normalize the output to prevent clipping or overloading, ensuring a well-balanced and usable composite sound waveform.

Repeating these steps with different combinations of audio signals and amplitude adjustments can generate diverse composite audio samples for data augmentation purposes.

It is important to note that when using sound overlapping as a data augmentation technique, the labels associated with the original audio signals must also be considered. Sometimes, the labels may need to be combined or modified to accurately represent the new composite audio signal.
\paragraph{Linguistic Data Augmentation} \label{sec:text-augmentation}

Linguistic data transfiguration involves metamorphosing textual data to augment its diversity and quantity. This can ameliorate the performance and robustness of natural language processing models that rely on text data.   

For sonic milieu generation, linguistic data transfiguration provides an efficacious approach to creating more varied and verisimilitudinous soundscape descriptions from text. Nonetheless, various existing soundscape datasets discussed in Section \ref{sec:sol-datasets} contain categorical labels or tags instead of descriptive annotations.   

It is imperative to note that some linguistic data augmentation techniques only function when the original text is in natural language, while others can still be applied to categorical labels.

Variations in input text can facilitate the generation of soundscapes with more variability and legitimacy. The following sections cover specific linguistic data augmentation techniques and their applications for soundscape generation.

While linguistic data augmentation has several advantages, it poses some issues and challenges~\cite{shorten_text_2021}.

% Advantages
The main benefits of textual augmentation are as follows:
\begin{enumerate}
    \item It can reduce the costs of collecting and annotating textual data.
    \item It can improve the accuracy of models by increasing the training data size, alleviating data scarcity, mitigating overfitting, and creating variability in the data. 
    \item It can boost the generalizability of models by exposing them to different linguistic patterns and styles. 
    \item It can increase the robustness of our models by making them resilient against adversarial attacks that attempt to deceive them through expert alterations of the input sequences.  
\end{enumerate}

% Disadvantages
However, there are also some potential downsides:     
\begin{enumerate}   
    \item It can introduce noise or errors into the data that may impact the quality and readability of the augmented text.
    \item It can change or lose the original text's meaning, style, or complexity, mainly if the transformation is inappropriate or irrelevant to the task or domain.   
    \item It can be computationally expensive or time-consuming to generate high-quality and diverse augmented text, especially if it involves using external resources or models such as dictionaries, corpora, word embeddings, generative models, or translation models.
\end{enumerate}

Various frameworks have been developed for augmenting text data linguistically. These can be broadly grouped into symbolic and neural augmentation models.

\subparagraph{Symbolic Augmentation Models}

These methods employ rule-based transformations operating directly on the surface form of text via predetermined heuristics. They include:

\begin{itemize}     
  \item Rule-based augmentation replaces, inserts, or deletes tokens according to specified rules. An example is replacing named entities with alternatives \cite{wei_eda_2019}.   
   \item Graph-based augmentation uses graph structures to perturb the text, \textit{e.g.} swapping adjacent adjectives and nouns \cite{ahmed_text_2023}.  
  \item MixUp combines existing examples via interpolation to synthesize augmented instances, \textit{e.g.} combining ``The cat sat on the mat'' and  ``The dog lay on the rug'' to generate ``The cat lay on the mat'' \cite{guo_augmenting_2019}.
  \item Feature-based augmentation applies transformations to word embeddings, \textit{e.g.} adding noise to the embedding space \cite{cheung_modals_2021}.        
\end{itemize}   

Despite their interpretability, symbolic methods struggle with complex transformations.

\subparagraph{Neural Augmentation Models}

These techniques leverage deep neural networks and large language models. They encompass:   

\begin{itemize}    
   \item Back-translation, which translates text into another language and back, producing paraphrases, \textit{e.g.} translating ``The book was interesting.'' to French and back to English, yielding  ``The book was fascinating'' \cite{pham_meta_2021}.
  \item Generative augmentation employs generative language models to synthesize novel text, \textit{e.g.} using an \ac{LLM} such as GPT to rephrase sentences or simply fine-tune them.
\end{itemize} 

While more complex, neural methods can generate diverse and realistic augmented instances.

Both symbolic and neural augmentation aim to expose models to more variability during training, helping combat overfitting and improve performance. However, symbolic methods offer interpretability, while neural methods provide more flexibility and variation.

\begin{table}[ht]
\centering
\caption{A taxonomy of text augmentation methods for transformer language models according to their algorithmic properties and underlying approaches.}  
\begin{tabularx}{\textwidth}{|X|X|X|X|X|}   
\hline
  &\textbf{Interpretability} &\textbf{Algorithmic }   & \textbf{Capacity to} &\textbf{Paradigmatic} \\
 &\textbf{Transparency} & \textbf{Complexity}& \textbf{Leverage Labels} &\\[0.5ex] 
\hline
\textbf{Rule-based} & High &Relatively low &Incapable& Lexical substitution \\   
\hline
\textbf{Graph-based} &High & Moderate & Incapable&Knowledge graph-based\\       
\hline      
\textbf{Sample Combination} &Medium&Moderate & Capable& Embeddings summation\\
 \hline    
\textbf{Feature-based}&Medium &Moderate& Incapable &Noise injection\\     
\hline
\textbf{Generative}&Low &High &Capable &Text auto-generation\\    
\hline
\textbf{Back-translation}&Medium &High&Capable &Inverse translation\\             
\hline           
\end{tabularx}
\label{tab:textaugmethods}  
\end{table}

Table \ref{tab:textaugmethods} categorizes several common text augmentation strategies according to their transparency, complexity, dependence on labels, and paradigm.  

Regarding interpretability, rule-based and graph-based methods exhibit high transparency since they employ explicit symbolic transformations. In contrast, the stochastic nature of generative models and back-translation compromises their interpretability.

Computational complexity also differs. Rule-based and graph-based augmentation are relatively efficient since they apply explicit symbolic transformations. In comparison, training neural networks for generative modeling and back-translation requires more computational resources.

For leveraging labels, given that some datasets mentioned in Section \ref{sec:sol-datasets} contain categorical labels, and we desire descriptive annotations, it is pertinent to assess whether these techniques can transform categorical labels into text descriptions. Employing generative models or back-translation potentially enables this since the models attempt to make sense of the input and transform it into a sentence.
\subsubsection{Evaluation Metrics} \label{sec:evaluation}

The evaluation process can provide insights into the system's performance and reveal areas that need improvement. Moreover, a proper evaluation metric helps to compare the results of different models and choose the best one. The aim of this section is to provide a comprehensive overview of the available evaluation metrics for audio generation and to provide a foundation for the evaluation of the system developed in this thesis.

In this thesis, a comprehensive evaluation framework is developed that considers two different types of evaluations. The first type of evaluation focuses on metrics that can be used for training models offline, such as loss metrics, which play a crucial role in assessing the performance of a model. The second type of evaluation focuses on evaluating the broader impacts of a model, such as its environmental impact and inference time for a generation. These evaluations are essential in ensuring that the model not only performs well on its primary task but also has minimal negative impacts on other aspects of the system.

\paragraph{Loss Functions} \label{sec:loss-functions}

Evaluating generative audio systems is challenging due to the need for a standard set of metrics to capture the quality and diversity of the generated audio samples. Different studies often use different evaluation methodologies and metrics when reporting results, making a direct comparison to other systems intricate if not impossible~\cite{vinay_evaluating_2022}. Furthermore, the perceptual relevance and meaning of the reported metrics, in most cases unknown, prohibit any conclusive insights concerning practical usability and audio quality.

A review and comparison of the available evaluation metrics for audio generation is essential to provide a foundation for evaluating the system developed in this thesis. This section discusses some of the commonly used metrics for evaluating generative audio systems, such as \ac{MAE}, \ac{MSE}, \ac{KL} divergence, and \ac{ELBO}. It also discusses their advantages and limitations and how they can be applied to sound generation tasks.

\subparagraph{Mean Absolute Error} \label{sec:mae}

The \Acf{MAE} is a quantitative measure of the average magnitude of the errors between the predicted and actual values~\cite{willmott_advantages_2005}. It is computed as:

\begin{equation}
	\text{MAE} = \frac{1}{n} \sum_{i=1}^n |y_i - \hat{y}_i| 
\end{equation}

where $y_i$ denotes the true value, $\hat{y}_i$ denotes the predicted value, and $n$ denotes the number of samples. The \ac{MAE} is also called L1-norm loss or \acf{MAD}.  

The \Ac{MAE} is a relevant metric for evaluating generative models as it treats all errors equally and is arguably less sensitive to outliers compared to \ac{MSE} (see Section \ref{sec:mse}). Specifically, it indicates the average absolute difference between the predicted and actual values in the same unit as the output.

However, the \ac{MAE} has some notable limitations that should be considered. For instance, it does not directly capture the perceptual quality of the generated audio samples. The perceptual quality of audio can depend on factors such as timbre, pitch, or harmony, which are not explicitly determined by the \ac{MAE}. Therefore, to fully assess the quality of generative models, the \ac{MAE} should be complemented with other metrics and human evaluation. Additionally, the \ac{MAE} is a scale-dependent measure and cannot be used to compare predictions that use different scales.

To illustrate the difference between \ac{MAE} computed on raw audio versus spectrograms, consider the following example: For a 1D raw audio sample, the \ac{MAE} would measure the average absolute difference between the amplitude values of the audio signals. In contrast, if the audio data were represented as spectrograms, the \ac{MAE} would measure the average absolute difference between the magnitude values of the frequency bins. In this case, the spectrograms can be treated as images with a single channel, and the \ac{MAE} can be seen as a pixel-wise error metric.  

It is important to consider the limitations of the \ac{MAE} when evaluating generative models. For example, consider a scenario with two audio samples of a dog barking: one with a bark at one second and another with a bark at two seconds. Despite being similar sounds with different temporal positions, calculating their \ac{MAE} would lead to higher-than-expected error as it does not account for temporal alignment. Therefore, it is important to use other metrics and human evaluation methods along with \ac{MAE} to assess timing accuracy and perceptual quality of generated audio samples comprehensively.

\subparagraph{Mean Squared Error} \label{sec:mse}

\Acf{MSE} is a standard metric to evaluate the performance of a predictor or an estimator. It quantifies the average of the squared errors, the average squared difference between the estimated and actual values. \Ac{MSE} is always a non-negative value approaching zero as the error decreases. The smaller the \ac{MSE}, the better the predictor or estimator~\cite{hodson_mean_2021}.

\Ac{MSE} can be calculated as follows:

$$
MSE = \frac{1}{n} \sum_{i=1}^n (y_i - \hat{y}_i)^2
$$

Where $n$ is the number of data points, $y_i$ is the true value of the variable being predicted or estimated, and $\hat{y}_i$ is the predicted or estimated value.

\Ac{MSE} incorporates both the variance and the bias of the predictor or estimator. The variance measures how widely spread the estimates are from one data sample to another. The bias measures the distance of the average estimated value from the true value. For an unbiased estimator, the \ac{MSE} equals the variance.

\Ac{MSE} can compare different predictors or estimators and select the one that minimizes the \ac{MSE}. For instance, in linear regression, \ac{MSE} can be used to find the best-fitting line that minimizes the sum of squared errors. \Ac{MSE} can also evaluate the quality of a generative model that produces audio samples from textual input. In this case, \ac{MSE} can measure how similar the generated audio samples are to the target audio samples regarding their amplitude values.

Unlike \ac{MAE} (see Section \ref{sec:mae}), which assigns equal weight to all errors, \ac{MSE} penalizes larger errors more than smaller ones. This means that \ac{MSE} is more sensitive to outliers and may not reflect the overall discrepancy between the generated and target audio samples well. Moreover, \ac{MSE} does not account for perceptual aspects of audio quality, such as timbre, pitch, or loudness. Therefore, \ac{MSE} should be used with other metrics and evaluation methods, such as \ac{KL} divergence (see Section \ref{sec:kld}), subjective listening tests, or qualitative analysis.

\Ac{MSE} can be applied to sound generation tasks in different ways, depending on the representation of the audio data. Similar to \ac{MAE}, it can be applied to 1D raw audio and spectrograms.

It should be noted that \ac{MSE} is subject to the same temporal issue as \ac{MAE}. \Ac{MSE} may not be effective in identifying differences in timing accuracy. Therefore, it is crucial to employ other metrics and evaluation methods that specifically focus on timing aspects and perceptual quality when assessing audio produced by text-trained models.

\subparagraph{Cross-Entropy} \label{sec:cross-entropy}

Cross Entropy Loss is commonly used to evaluate \ac{DL} models, especially in classification or sequence generation tasks. It measures the dissimilarity between predicted and target distributions by calculating the average negative log-likelihood of predicting each class or element correctly.

Cross Entropy Loss can be applied in audio generation tasks to produce discrete elements, such as musical notes or phonemes. For instance, if one aims to create music based on textual input with particular note sequences, Cross Entropy Loss can assess the prediction accuracy of each note at every time step.

Mathematically, given data samples $x_i$ and their respective true labels $y_i$, where $i$ ranges from 1 to $n$, Cross Entropy Loss can be calculated in the following way:

\begin{equation}
    CE = -\frac{1}{n}\sum_{i=1}^{n} \sum_{j=1}^{C} y_{ij}\log(p_{ij})
\end{equation}

This is where:
The symbol $C$ represents the number of classes or elements.
The notation $y_{ij}$ indicates whether sample $x_i$ belongs to class $j$ (or has element $j$).
Additionally, $p_{ij}$ represents the predicted probability that sample $x_i$ belongs to class $j$ (or has element $j$).

The objective is to minimize the Cross Entropy Loss during training, so that the generative model can learn to predict precise and coherent distributions over classes or elements.

Nonetheless, it is important to consider some limitations when using Cross Entropy Loss to evaluate audio generation systems. First, the model assumes independence between individual predictions within one sample. However, this assumption may not hold for sequential audio data where the context and dependencies between elements are critical. Second, the Cross Entropy Loss does not directly capture perceptual aspects of audio quality, such as timbre or tonality. Therefore, it is advisable to combine Cross Entropy Loss with other evaluation metrics, such as \ac{MAE}, \ac{MSE}, or subjective listening tests to achieve a comprehensive understanding of the generative model's performance.

To summarize, the Cross Entropy Loss is commonly employed as a loss function for evaluating generative models that involve discrete element generation. Although it is applicable for audio generation tasks with categorical outputs such as music note prediction, it should be combined with other evaluation methods to gain a more comprehensive assessment of accuracy and perceptual quality.

\subparagraph{KL Divergence} \label{sec:kld}

\acf{KL} divergence, also known as relative entropy, is a non-symmetric measure of the difference between two probability distributions. It is a mathematical quantity that quantifies the distance between two probability distributions.

In simple terms, \ac{KL} divergence measures the difference between the probability distribution predicted by a model and the true underlying distribution of the data. \Ac{KL} divergence is commonly used to evaluate generative models.

\ac{KL} divergence is calculated as the expectation of the logarithmic difference between the predicted probability distribution and the actual distribution. It is a scalar value, and the smaller the \ac{KL} divergence, the closer the predicted distribution is to the real distribution.

The \ac{DL} system can be trained to produce sounds near the target sounds in terms of their probability distribution by using \ac{KL} divergence as a loss function. The concept is that the sound does not have to be alike the input, but only its distribution. Maximizing the log-likelihood between the generated output and the given input can be seen as minimizing the \ac{KL} divergence \cite{huzaifah_deep_2021}.

\subparagraph{Evidence Lower Bound (ELBO)} \label{sec:elbo}

\Acf{ELBO} is a lower bound on the log-likelihood of some observed data commonly used in variational Bayesian methods \cite{blei_variational_2017}.

The \ac{ELBO} is defined as follows:

\begin{equation}
ELBO = E_{Z \sim q}\left[\log \frac{p(X,Z; \theta)}{q(Z)} \right]
\end{equation}

where $X$ and $Z$ are random variables with joint distribution $p(X,Z; \theta)$, $\theta$ are the parameters of the model, and $q(Z)$ is an approximate posterior distribution for the latent variable $Z$. The \ac{ELBO} can be seen as the difference between two terms: the expected log joint probability of the data and the latent variables under the model and the entropy of the approximate posterior distribution.

The \ac{ELBO} has several desirable properties. First, it is a lower bound on the log-likelihood of the data, $\log p(X; \theta)$, also known as the evidence. Meaning that the \ac{ELBO} is a quantity that is always less than or equal to the log-likelihood of the data, which is the logarithm of the probability of the data given the model parameters. The log-likelihood of the data is also called the evidence because it indicates how well the model fits the data. The higher the log-likelihood, the more evidence we have that the model is suitable for the data. However, computing the log-likelihood of the data is often intractable. Therefore, the model can be optimized more easily using \ac{ELBO}.

Second, it is a tractable objective function that can be optimized concerning $\theta$ and $q(Z)$. This allows us to perform variational inference, approximating the posterior distribution $p(Z|X; \theta)$ by finding the $q(Z)$ that maximizes the \ac{ELBO}. This can be done using gradient-based methods, thus being used in machine learning systems.

Third, it can be decomposed into two significant components: the reconstruction term and the regularization term. The reconstruction term is the expected log-likelihood of the data given the latent variables under the model, $E_{Z \sim q}\left[\log p(X|Z; \theta)\right]$. It measures how well the model fits the data. The regularization term is the negative \ac{KL} divergence (see section \ref{sec:kld}) between the approximate posterior and the prior distributions, $-D_{KL}(q(Z)||p(Z))$. It measures how close the approximate posterior is to the prior. The \ac{KL} divergence is always non-negative, and it is zero if and only if $q(Z) = p(Z)$. Therefore, maximizing the \ac{ELBO} encourages data fidelity and posterior regularization.

The \ac{ELBO} can be applied to sound generation tasks using a deep generative model such as a \ac{VAE} (see section \ref{sec:vae}). This model can be trained by maximizing the \ac{ELBO} concerning its parameters and latent variables. The \ac{ELBO} can then be used to evaluate the quality and diversity of the generated sounds by comparing them to the target sounds. For instance, the \ac{ELBO} for a \ac{VAE} can be written as:

\begin{equation}
ELBO = E_{Z \sim q_\phi(Z|X)}\left[\log p_\theta(X|Z)\right] - D_{KL}(q_\phi(Z|X)||p(Z))
\end{equation}

The first term is the reconstruction term, which measures how well the decoder network reconstructs the input sound $X$ from the latent variable $Z$. The second term is the regularization term, which measures the proximity of the approximate posterior distribution to a prior distribution $p(Z)$.

By maximizing the \ac{ELBO}, a model learns to generate realistic sounds similar to the input sounds regarding their conditional distribution while ensuring that the latent variables have a smooth and regular structure that facilitates interpolation and manipulation.

\paragraph{Model Evaluation Functions}

The second type of evaluation in this thesis involves assessing the wider impacts and implications of the developed audio generation model. Although metrics used for training models offline provide insights into performance, it is equally important to evaluate how a model affects aspects beyond its primary task. This includes considerations such as the environmental impact and inference time during generation, among others. Understanding these broader impacts ensures that the model not only performs well in its intended purpose but also operates with minimal negative consequences or trade-offs in other areas of the system. Conducting evaluations encompassing these factors will provide a more comprehensive understanding of how well-rounded and sustainable our audio generation system is.

\subparagraph{Evaluating Energy Expended}

Evaluating the amount of energy expended by a deep learning model is crucial in developing and deploying these systems. With the increasing demand for machine learning applications and the complexity of deep learning models, energy efficiency has become a critical factor in designing and deploying deep learning systems.

With the growing concern for environmental sustainability, the energy footprint of deep learning models has become an essential topic in the field. Most of the recent advances produced by deep learning approaches rely on significant increases in size and complexity \cite{douwes_energy_2021}. Such improvements are backed by an increase in power consumption and carbon emissions. The high energy consumption of deep learning models during both the training and inference phases significantly impacts the environment, and it is imperative to address this issue.

Therefore, evaluating the amount of energy a deep learning model expends is essential in ensuring its practicality and scalability. This is a crucial step in ensuring that the deep learning models developed today are not only accurate, but also energy-efficient and sustainable for future deployment.

This evaluation can be done in two ways: physically measuring the energy expended by the machines on both learning and inference time or by approximating given average numbers per neuron, for instance.

A good model is a compromise between accuracy and complexity. If the model trains significantly longer to train or infer and does not provide way better results, in the context of this research, the model is not much better than a simpler counterpart.
\subsubsection{Data Embedding} \label{sec:text-embedding}

Data embedding is the technique of converting data into numerical representations that capture its essential features and characteristics. For sound generation, both audio and text embedding is important.

Data embedding is essential for generative models. This happens for two reasons.

First, it allows the models to work on smaller and lighter representations. This is, for instance, taking an audio sample with 5 seconds sampled at 16 kHz, representing 80 000 entries. Encoding that into meaningful features might reduce this number to a few thousand or even hundreds of entries. This allows for faster training.

Second, these representations are meaningful in ways where the raw input is not. For instance, take text embedding. The idea is that words such as ``pretty'' and ``beautiful'' have similar representations, helping the model generalize. If the model were to check the words, letter by letter, it would have a hard time realizing that some words, like these, have a relation between them.

So, text and audio embedding is essential for the task. However, audio embedding possesses several challenges, such as dealing with high-dimensional and sequential data, preserving temporal and spectral information, and ensuring robustness and interpretability. To handle this, both feature and learning-based methods can be applied.

Feature-based embedding methods extract predefined features from the raw audio data, such as spectral, temporal, or perceptual features. These features are then input to generative models or further processed to obtain lower-dimensional embeddings. One such example would be the application of the \ac{STFT} to build a spectrogram (see Section \ref{sec:sound}). Feature-based embedding methods have the benefit of being simple and interpretable, but they may also lose some information or introduce noise during the feature extraction process.

Using neural networks or other machine learning techniques, learning-based embedding methods learn embeddings directly from the raw audio data. These methods can automatically discover relevant features from the data without relying on predefined criteria. Learning-based embedding methods have the advantage of being flexible and adaptive, but they may also require more computational resources or suffer from overfitting or underfitting issues.

Text embedding has been a solved problem since the days of Word2Vec \cite{mikolov_efficient_2013}. This \ac{AE} (see Section \ref{sec:autoencoders}) model would be trained by getting the meaning of a word taking into account the word with whom the first appeared. This model makes possible the representation of a word through a vector of latent factors.

Nevertheless, the words cannot be merely embedded for the current issue. When a user adds a text input, embedding it is imperative. This represents the entire textual input. For example, computing the average of latent factors for each input would be a naive approximation.

Nevertheless, nowadays, this is mainly solved with transformers (see Section \ref{sec:transformers}), namely the encoder part. This part of the transformer takes a whole string and outputs a vector representation, precisely what the task needs.

After the vanilla transformer, the one that gained more prominence was BERT \cite{devlin_bert_2018} in 2018, which introduced the conditioning of the whole input for each word inputted, not only the words that appeared before, as the vanilla transformer did. Other later encoder transformers are based on BERT or tackle a different problem.

While there are many techniques for data embedding, recent advancements in the field have led to the development of specialized models for various modalities, such as CLIP \cite{radford_learning_2021} for images and MuLan for audio. Instead of embedding the text and the media separately and dealing with it afterward, media and text are embedded in the same space, meaning that a textual segment and a media sample representing the same textual segment should have similar latent factors.

\paragraph{MuLan} \label{sec:mulan}

MuLan \cite{huang_mulan_2022} is a state-of-the-art music audio embedding model that aims to link music audio directly to unconstrained natural language music descriptions.

MuLan employs a two-tower parallel encoder architecture, meaning two completely independent neural architectures, using a contrastive loss objective that elicits a shared embedding space between music audio and text.

Each MuLan model consists of two separate embedding networks for the audio and text input modalities. These networks share no weights, but each terminates in 2-normalized embedding spaces with the same dimensionality. The contrastive loss objective minimizes the distance between matching audio-text pairs while the distance between mismatched pairs is maximized. This approach enables MuLan to learn a joint representation of music audio and text that captures their semantic relationships.

MuLan is trained using 44 million music recordings (370K hours) and weakly-associated, free-form text annotations. The resulting audio-text representation subsumes existing ontologies while graduating to true zero-shot functionalities. MuLan demonstrates versatility in transfer learning, zero-shot music tagging, language understanding in the music domain, and cross-modal retrieval applications.