\section{Background} \label{sec:background}

This dissertation is situated in a vast body of research on \ac{DL}. This section positions the current dissertation within this context. This work is not intended to explain other technologies that solve similar problems, but rather to explain other technologies that are useful in addressing the problem at hand.

The dissertation aims at readers with basic \ac{ML} knowledge, but not necessarily basics on \ac{DL}. No basis on sound besides 8th-grade physics is needed, also.

The dissertation will start by explaining how sound can be processed digitally in subsection \ref{sec:sound}, keeping this objective in mind. Then, in subsection \ref{sec:deep-learning}, it will provide a comprehensive review of general deep learning architectures and techniques, emphasizing their importance in sound synthesis. In subsection \ref{sec:parallel-tasks}, the subsequent part will concentrate on crucial techniques used in developing the suggested models. Section \ref{sec:dl-frameworks} will discuss the deep learning frameworks used in this research. Subsection \ref{sec:data-generators} will explore state-of-the-art models for data generation that are unrelated to audio but can serve as references, such as image generators, to provide further context.

\subsection{Sound} \label{sec:sound}

A digital audio signal --- often called a waveform --- captures alterations in sound pressure overtime. This waveform represents how sound waves develop and spread throughout space. Distinct sounds that constitute our auditory experience can be perceived and interpreted through analyzing the changes in frequency and amplitude within the waveform. Waveforms encode the richness of sound, enabling us to capture and manipulate it for diverse purposes such as analysis, synthesis, and artistic expression.

Sound is a continuous phenomenon but can be discretized by taking samples at a specific time rate. The number of samples taken in one second is known as the "sampling rate." The standard value for this is 44,100 Hz. The sampling rate has a direct impact on the accuracy of the signal~\cite{elsea_basics_1996}.

With the discrete version of time, it becomes possible to represent sound through an array digitally. Knowledge of the array and the sampling rate enables a computer to reconstruct the sound. These arrays can become quite large. For instance, stereo sound with a sampling rate of 44,100 \ac{Hz} needs to accommodate 1,411,200 bits per second \cite{elsea_basics_1996}.

\subsubsection{Short-Time Fourier Transform} \label{sec:stft}

Even though digital sound media can be encoded as one-dimensional data~\cite{oord_wavenet_2016}, it can be converted. By using a \acf{DFT}, the array can be represented in the frequency domain~\cite{benois-pineau_deep_2021}. Since the contents of a sound sample typically vary over time, \acp{DFT} can be computed over successive time frames of the signal. This operation forms the basis of the \acf{STFT}. The equation for \ac{STFT} is as follows:

\begin{equation} \label{eq:stft}
    STFT\{x(n)\}(m, \omega) = \sum_{n=-\infty}^{\infty} x(n) w(n - mR) e^{-j\omega n}
\end{equation}

In this equation, $STFT\{x(n)\}(m, \omega)$ represents the time-frequency representation of the input signal $x(n)$ as a function of time index $m$ and frequency $\omega$. The function is a complex-valued function containing both magnitude and phase information of the signal's frequency components at different time intervals.

The summation symbol $\sum_{n=-\infty}^{\infty}$ denotes that the product of the signal, window function, and complex exponential is summed over all time indices $n$. The discrete-time input signal is represented by $x(n)$, sampled at integer time indices $n$.

The window function, represented as $w(n - mR)$, is utilized to isolate a specific time interval of the input signal. The window function is centered at the time index $mR$, where $R$ is the hop or step size between consecutive sample windows.

Finally, the complex exponential term $e^{-j\omega n}$ is utilized to analyze the signal's frequency content within the windowed interval. The variable $j$ is the imaginary unit, and $\omega$ represents the angular frequency.

In essence, the \ac{STFT} equation computes the Fourier Transform of the input signal $x(n)$ within a windowed time interval, providing a time-frequency representation of the signal. The window function isolates a specific time interval of the signal, and the complex exponential term analyzes its frequency content. This process is repeated for different time indices $m$, resulting in a time-frequency representation that allows the study of the signal's frequency components at various time intervals.

With the \ac{STFT}, one can generate spectrograms by plotting the time in the $x$ axis and the frequency in the $y$ axis. In short, a spectrogram is a graphical representation of the frequency content of a signal over time, typically displayed as a 2D image (see Figure \ref{fig:sound} for an example).

\begin{figure}[ht]
    \centering
    \ctikzfig{figures/2-sota/sound}
    \caption[Raw Wave vs. Spectrogram]{\textbf{Raw Wave vs. Spectrogram} --- A comparative analysis of a sound sample from the Audio MNIST dataset (showcased in section \ref{sec:dataset-amnist}), specifically entry number 9 of speaker Nicolas uttering the digit ``five''. The top plot illustrates the raw waveform with time (sample rate $\times$ time) on the X-axis and energy (amplitude) on the Y-axis, providing a temporal representation of the audio signal. The bottom plot presents a spectrogram generated using the \ac{STFT} method, displaying time on the X-axis and frequency on the Y-axis, offering a time-frequency representation that reveals the spectral content and evolution of the signal over time.
    }
    \label{fig:sound}
\end{figure}

\subsubsection{Meaning of Spectrograms for Machine Learning}

Representing sound as an image opens a multitude of opportunities. However, even though spectrograms can technically be processed using \acp{CNN} (see section \ref{sec:CNN}), there is a considerable difference between a spectrogram and a standard image. In a typical image, the axes represent the same concept, the spatial position. The elements of an actual image have the same meaning independent of where they are found. A sub-object of an image does not depend on the axes. At the same time, neighbor pixels are usually highly correlated.

On the other hand, the axes of the spectrograms have different meanings \cite{benois-pineau_deep_2021}. Moving a set of pixels horizontally and vertically means different things. Therefore, structures such as \ac{CNN} are not as helpful. One can still use them but should be careful about the shape of the filters and the axis along which the convolution is performed \cite{benois-pineau_deep_2021}.

\subsubsection{Soundscapes} \label{sec:soundscapes}

The digitalization of sound has allowed for multiple applications and use cases. Applications can generate speech, and movies and videos can embed audio, such as soundscapes. Soundscapes are the sonic environments or sound environments that surround environments. They are the complex and dynamic mix of sounds heard in everyday life, including sounds from nature, human-made, and cultural sounds~\cite{international_organization_for_standardization_iso_2014, schafer_tuning_1977}. In other words, a soundscape encompasses the auditory milieu characterized by a collection of naturally occurring and human-generated sounds as perceived, encountered, and comprehended within a contextual framework by individuals. It is paramount in audio content creation, augmenting the user experience across media applications by infusing emotional engagement, a greater sense of immersion, and attention~\cite{chandrasekera_virtual_2015}.

Nevertheless, for audio media generation with \ac{ML}, one usually finds models in the literature that solve music or speech generation, not soundscapes. This is no coincidence. These sounds are more straightforward and, thus, easier to generate. Speech, for instance, usually contains a single sound source (the speaker). Also, speech and music are highly structured over time and timbrically. This happens because speech is bound to grammar, and music is bound to an underlying structure. Both of them are timbrically bound to their authors. On the contrary, soundscapes have no specific structure. Hence the increased difficulty \cite{benois-pineau_deep_2021}.

\subsection{Deep Learning}\label{sec:deep-learning}

As previously mentioned, the growth of \acf{DL} began in the early 2000s as a response to the challenge of handling vast quantities of data. Fundamentally, \ac{DL} stems from the application of \ac{ML} to process large amounts of data. To be precise, \ac{DL} is a subfield of \ac{ML} that employs multiple levels of information processing and abstraction to learn and represent features, as demonstrated by Deng et al. \cite{deng_deep_2014}. Subsequently, the extracted features can be utilized for classification, regression, and other modeling techniques. In the past, such features were manually selected by humans.

This study uses \ac{DL} for sound generation because it offers several advantages over traditional sound generation techniques. By being data-driven, these models can generate new sounds based on sounds it has heard before. On traditional methods, these sounds would have to come from the inspiration of their human creator. Besides, end-to-end generation, from text to sound generation, is only possible through \ac{DL}. The model has to extract features from the text, learn features from thousands or millions of sounds, and correlate both. This highly complex task can only be achieved with \ac{DL} techniques.

This section presents traditional \ac{DL} architectures and their evolution to generative \ac{DL} architectures.

%%%%%%%%%%%%%%%%%%%%%%%%%%%%%%%%%%%%%%%%%%%

\subsubsection{Deep Learning Architectures}

Generative \ac{DL} architectures establish blueprints for developing \ac{DL} networks that synthesize diverse and novel data samples according to a learned distribution. These architectures entail creating latent data constructs and learning to emulate the fundamental statistical patterns found in observed data.

Generative deep neural models have been applied to tasks comprising image synthesis, text generation, and audio synthesis. Their popularity has recently surged owing to their remarkable ability to generate high-quality data and effectively model complex distributions. In the following sections, we outline the most ordinarily used generative \ac{DL} architectures, presented chronologically, as summarized in Table~\ref{tab:archs}.

\begin{table*}[ht]
\centering
\caption{Comparison of Generative Deep Learning Architectures}
\label{tab:archs}
\begin{tabularx}{\textwidth}{|p{20mm}|c|p{25mm}|X|c|}
\hline
\textbf{Model} & \textbf{Year} & \textbf{Type} & \textbf{Key Characteristics} & \textbf{Inference} \\ \hline
DARN & 2013 & Autoregressive & Uses a single model to predict the probability distribution of each output token conditioned on the previous tokens & Sequential \\ \hline
VAE & 2013 & Variational Autoencoder & Learns a latent representation of the input data and generates new samples by sampling from the learned latent space & Parallel \\ \hline
GAN & 2014 & Generative Adversarial Network & Consists of a generator and a discriminator that compete in a two-player minimax game to generate realistic samples & Parallel \\ \hline
Normalizing Flows & 2015 & Flow-based models & Transforms a simple probability distribution into a complex one by applying a sequence of invertible transformations & Parallel \\ \hline
Diffusion & 2015 & Flow-based models & Uses a diffusion process to model the probability distribution of the data & Parallel \\ \hline
Transformers & 2017 & Attention-based models & Uses self-attention to capture global dependencies and generate sequences & Sequential \\ \hline

VQ-VAE &
    2018 &
    Variational Autoencoder &
    Discretizes the continuous latent space by mapping each latent vector to the closest codebook vector &
    Parallel \\ \hline

MS-VQ-VAE &
    2019 &
    Variational Autoencoder &
    Is an extension of the VQ-VAE that incorporates multiple discrete latent spaces of different scales, enabling hierarchical and diverse representations with improved abstraction levels and latent space expressiveness &
    Parallel \\ \hline

\end{tabularx}
\end{table*}

This section will discuss novel architectural designs for \ac{DL}. It is important to note that there is a discrepancy between the date of their conceptualization and their widespread adoption. This trend is common in machine learning because software theories have a faster rate of development compared to hardware theories.

This section does not describe all \ac{DL} architectures that had some relevance. It, however, describes the ones used either by the models developed or by systems studied for the state-of-the-art.

\paragraph{Feedforward Neural Network (1957)} \label{sec:feedforward}

To understand \textit{feedforward neural networks} (or any neural network), it is essential to understand both Hebbian learning and the perceptron.

In 1949, inspired by the observation that neurons that fire together wire together, the psychologist Donald Hebb \cite{hebb_organization_1949} developed a rule for adjusting the strength of connections between neurons in any neural network. The rule goes by the name of Hebbian learning and states that if two neurons are activated simultaneously, their connection should be strengthened. The idea behind Hebbian learning is that learning occurs as a result of changes in the strengths of synaptic connections between neurons in the brain rather than solely due to changes in the activity of individual neurons.

The perceptron is a simple model first published in 1958 by F. Rosenblatt \cite{rosenblatt_perceptron_1958}. It consists of a single neuron with a binary threshold --- also known as bias --- and is used for binary classification tasks. A set of weights transforms the input to the perceptron. The output is determined by checking if the input is enough to trigger the bias and then applying a function to the weighted sum of inputs as shown in Figure \ref{fig:perceptron}. The output of the perceptron can be mathematically expressed as in the equation \ref{eq:perceptron-ouput}.

\begin{equation} \label{eq:perceptron-ouput}
    y = h(\sum_{x=1}^N (I_x \times W_x) - b)
\end{equation}

$y$ is the output, $N$ is the number of inputs, $I_n$ is the input number $n$, $W_n$ is the weight related to $I_n$, $b$ is the bias, and $h$ is an activation function, usually.

The perceptron is trained using an algorithm that adjusts the weights based on the error between the actual and predicted outputs. The exciting part is that Hebbian learning can be applied to perceptrons, which was one of the bases of the future backpropagation. The perceptron can only classify linear separated sets, as shown in \textit{Perceptrons} \cite{marvin_minsky_perceptrons_1969}, hence the need for more complex structures.

\begin{figure}[ht]
    \centering
    \ctikzfig{figures/2-sota/perceptron}
    \caption[Perceptron]{\textbf{Perceptron} --- The perceptron receives multiple inputs associated with a weight. It also holds a given bias. Its output is the application of a given function (\textit{e.g.} step function) to the weighted average of the inputs minus the bias.}
    \label{fig:perceptron}
\end{figure}

A feedforward neural network is no more than a set of layers of perceptron-like neurons. In a feedforward neural network, information moves in only one direction --- forward. Hebbian learning no longer applies to these networks, so more complex algorithms, such as backpropagation, are required. These networks are universal approximators \cite{cybenko_approximation_1989} if they have at least one hidden layer, meaning they can approach any function given the proper configuration. They can also learn different tasks than classification, such as regression. The first functional networks, called multi-layer perceptrons, were invented in the 1980s. An example structure of a feedforward neural network can be seen in Figure \ref{fig:feed_forward_neural_network}.

\begin{figure}[!ht]
    \centering
    \ctikzfig{figures/2-sota/feed-forward-neural-network}
    \caption[Feedforward Neural Network]{\textbf{Feedforward neural network} --- The size of the input is constant. The flow of information goes through nonlinear functions on hidden layers. Both weights between nodes on different layers and the nodes' bias are trained with backpropagation.}
    \label{fig:feed_forward_neural_network}
\end{figure}

To a certain extent, since the perceptron is the most basic feedforward neural network (consisting of only one neuron), these networks do not necessarily equate to deep learning (\ac{DL}). Smaller networks have existed and been utilized for decades. Nevertheless, the foundation for most \ac{DL} architectures comes from this, making it essential to recognize.
\paragraph{Convolutional Neural Network (CNN) (1980)} \label{sec:CNN}

During the 1950s and 1960s, Hubel and Wiesel~\cite{hubel_receptive_1959}, who were distinguished neurophysiologists, conducted pioneering research on the eyes of cats. Their research uncovered the presence of neurons with receptive fields that map various visual regions. Receptive fields refer to specific areas in the visual field that activate particular cells. Their studies revealed two fundamental categories of cells involved in visual processing: simple cells with a strong response to edges and complex cells with larger receptive fields that prioritize the organization of edges over their precise positioning.

In 1980, Kinihiko Fukushima introduced the \textit{neocognitron} which marked a significant milestone in the development of \acp{CNN} \cite{fukushima_neocognitron_1980}. It was one of the first \ac{CNN} architectures designed to mimic specific aspects of human visual perception. This laid the groundwork for future advances in deep learning for computer vision tasks.

\Acp{CNN} are a category of \ac{DL} neural networks widely used in computer vision and other sequential data types. The architecture of \acp{CNN} is specifically designed to handle inputs where data correlate with its vicinity.

The key idea behind \acp{CNN} is to learn and extract features from the input hierarchically. This is achieved through convolutional layers, where a small set of learnable filters are used to scan and transform the input image into a feature map. These networks also use pooling layers that perform down-sampling of the feature map to reduce its size while retaining important information. These operations allow the network to capture patterns and features in the input, which can then be passed to feedforward neural networks and used for classification or regression. This process is represented in Figure \ref{fig:cnn}.

\begin{figure}[ht]
    \centering
    \ctikzfig{figures/2-sota/cnn}
    \caption[Convolutional Neural Network]{\textbf{\Acf{CNN}} --- The network present in the Figure deals with 2D data, such as images. The networks' first step consists of convolutional (and pooling) layers. This first step is responsible for finding features. The deeper the convolutional layer, the bigger the receptive field; consequently, the more abstract the feature. After the features are caught, the flatten operation transforms the data into a 1D vector. And then, the network is no more than a feedforward neural network to, in this case, classify what was fed to the network.}
    \label{fig:cnn}
\end{figure}

A major benefit of \acp{CNN} compared to other kinds of networks is their ability for parallelism, where a long input sequence can be handled fast \cite{huzaifah_deep_2021}. This can greatly accelerate training since the whole output can be processed in one forward pass. The shared weights and local connectivity among neurons in the network enable efficient computation, and using pooling layers lowers the number of parameters to be learned.
\paragraph{Convolutional Operations} \label{sec:conv-layers}

This section will discuss three common layers in \acp{CNN}: convolutional layers, pooling layers, and transposed convolutions. It also introduces the concept of masked convolutions and their applications in specific tasks.

\label{sec:conv-layer}

A \textbf{convolutional layer} is a fundamental building block of \acp{CNN} (see Section \ref{sec:CNN}).

At a high level, a convolutional layer applies a set of learnable filters to the input data, extracting local features and patterns relevant to the task. The filters are typically small and slide over the input data, computing the dot product between the filter weights and the input values at each position. This operation is called convolution and can be seen in Figure \ref{fig:conv-layer}.

\begin{figure}[ht]
    \centering
    \ctikzfig{figures/2-sota/conv}
    \caption[Convolutional layer]{\textbf{Convolutional layer} --- The presented diagram depicts a 1-dimensional convolutional layer designed to process an input signal consisting of a single channel and a length of 7. The layer employs a single filter with a size of 3, resulting in an output signal of length 5 and a single channel. Each element of the output signal is produced by convolving the corresponding filter weights with a subset of the input signal. Specifically, the first output element is generated by convolving the filter with the first three elements of the input signal. The filter is then slid along the input signal with a stride of 1, such that each subsequent output element depends on a different subset of the input signal.}
    \label{fig:conv-layer}
\end{figure}

To be more precise, let us consider a 1D convolutional layer that takes an input tensor $X$ of size $C_{in} \times L_{in}$, where $C_{in}$ is the number of input channels and $L_{in}$ is the length of the input sequence. The layer also has $C_{out}$ filters, each of size $C_{in} \times k$, where $k$ is the size of the filter kernel. The output of the layer is a tensor $Y$ of size $C_{out} \times L_{out}$, where $L_{out}$ is the length of the output sequence.

The convolution operation can be expressed as follows:

\begin{equation}
	Y_{c,i} = \sum_{p=0}^{k-1}\sum_{k=0}^{C_{in}-1} X_{k,i+p} \cdot W_{c,k,p} + b_c
\end{equation}

where $c$ is the index of the output channel, $i$ is the spatial index of the output sequence, $p$ is the spatial index of the filter kernel, $k$ is the index of the input channel, $X_{k,i+p}$ is the input value at position $(k,i+p)$, $W_{c,k,p}$ is the weight of the filter at position $(c,k,p)$, and $b_c$ is the bias term for the $c$-th output channel.

The convolution operation is applied independently to each output channel and each spatial location of the output feature map, resulting in a set of feature maps that capture different aspects of the input data. The weights and biases of the filters are learned during training using backpropagation and gradient descent (see section \ref{sec:backpropagation}), optimizing a suitable loss function for the task at hand.

% alskjdaslkjdasld

\label{sec:pooling}

\textbf{Pooling} is a standard operation in \acp{CNN} that reduces the spatial dimensions of the input feature maps while retaining important information. Pooling is typically applied after convolutional layers to gradually reduce the feature maps' spatial dimensions and increase the network's receptive field. There are several types of pooling, including max pooling, average pooling, L2 pooling, and more. Of these, max pooling is one of the most commonly used.

\textbf{Max pooling} works by partitioning the input feature map into non-overlapping rectangular regions, called pooling kernels. For each pooling kernel, the maximum value is selected and used as the output value for that region. The size of the pooling kernel and the stride determine the amount of reduction in the spatial dimensions of the feature map. The stride is the number of pixels that the pooling kernel is shifted horizontally and vertically between each pooling operation. It can be seen in the Figure \ref{fig:pool}

\begin{figure}[ht]
    \centering
    \ctikzfig{figures/2-sota/pool}
    \caption[Pooling layer]{\textbf{Pooling layer} --- The Figure illustrates a $4 \times 4$ input tensor with a single channel. Each colored square in the tensor denotes a single element. The pooling layer partitions the input tensor into non-overlapping $2 \times 2$ regions and applies a pooling function to each region, resulting in a $2 \times 2$ output tensor with a single channel. The number of channels remains unchanged in this scenario, implying that the pooling function is applied separately to each channel of the input tensor, and the resulting output tensor retains the same number of channels as the input tensor. The primary objective of pooling layers is to decrease the spatial dimensions of the input tensor while maintaining the essential features. In this instance, the pooling layer decreases the spatial dimensions of the input tensor by half, resulting in a smaller output tensor. The output tensor is displayed on the right-hand side of the image, where each colored square represents a single element of the output tensor. The colors of the output tensor correspond to those of the input tensor regions from which they were derived.}
    \label{fig:pool}
\end{figure}

Max pooling has several benefits for \acp{CNN}~\cite{riesenhuber_hierarchical_1999}:
\begin{itemize}
    \item It helps to reduce the model's sensitivity to minor variations in the input. This is because the maximum value within each pooling kernel is selected, which is less sensitive to slight variations than taking the average or other statistics.
    \item It can prevent overfitting by reducing the number of parameters in the model. This is because the pooling operation reduces the spatial dimensions of the feature map, reducing the number of parameters in the subsequent layers of the network.
    \item It can increase the network's receptive field by combining the information from neighboring pixels.
\end{itemize}

There are some potential drawbacks to max pooling as well. One is that it can discard some information that may be important for the task. This is because only the maximum value within each pooling kernel is selected, and other information is discarded.

%laskdjl kajsdlkajs ldkja

\label{sec:transpoed-conv}

A \textbf{transposed convolution}, also known as a deconvolution, is a layer that performs the opposite operation of a convolutional layer. It takes an input tensor of size $C_{in} \times L_{in}$ and produces an output tensor of size $C_{out} \times L_{out}$, where $C_{in}$ and $C_{out}$ are the number of input and output channels, respectively, and $L_{in}$ and $L_{out}$ are the input and output lengths.

The transposed convolution applies a set of learnable filters to the input data. However, instead of sliding the filters over the input, it slides them over the output and computes the dot product between the filter weights and the output values at each position. This operation can be seen as an ``unfolding'' of the convolution operation. Hence the name ``transposed convolution''. An example can be seen in Figure \ref{fig:trans-conv}.

\begin{figure}[ht]
    \centering
    \ctikzfig{figures/2-sota/trans-conv}
    \caption[Transpoed convolution]{\textbf{Transposed convolution} --- This illustration shows how a transposed convolution operation works. The input array has three elements $(3, 4, 3)$ and is shown by the dark blue rectangles at the top. The filter array also has three elements $(5, 5, 2)$ and is shown by the green rectangles in the middle. The output array is obtained by sliding the filter over the input array and computing the element-wise products and sums. The stride parameter controls how much the filter moves, which is 1 in this example. The light blue rectangles at the bottom show the intermediate and final output arrays. The intermediate output arrays are $(15, 15, 6, 0, 0)$, $(0, 20, 20, 8, 0)$, and $(0, 0, 15, 15, 6)$. The final output array is the sum of the intermediate output arrays $(15, 35, 41, 23, 6)$. The gray-shaded regions indicate how the filter broadcasts each element in the input array to produce an intermediate output array.}
    \label{fig:trans-conv}
\end{figure}

The transposed convolution can be expressed as follows:

\begin{equation}
	X_{k,i} = \sum_{p=0}^{k-1}\sum_{c=0}^{C_{out}-1} Y_{c,i+p} \cdot W_{c,k,p} + b_k
\end{equation}

where $k$ is the index of the input channel, $i$ is the spatial index of the input sequence, $p$ is the spatial index of the filter kernel, $c$ is the index of the output channel, $Y_{c,i+p}$ is the output value at position $(c,i+p)$, $W_{c,k,p}$ is the weight of the filter at position $(c,k,p)$, and $b_k$ is the bias term for the $k$-th input channel.

Like convolutional layers, the weights and biases of the filters are learned during training using backpropagation and gradient descent.

\label{sec:masked-conv}

A \textbf{masked convolution} is a convolutional layer that selectively masks out specific input values based on their position in the input sequence. This masking is typically done by setting the weights of the filters to zero for certain positions in the kernel. For example, in an \ac{AR} language modeling task, where the goal is to predict the next word in a sentence given the previous words, a masked convolution can ensure that the model only has access to the previous, not the future words.

Another example where masked convolutions can be helpful is in audio generation tasks. In such tasks, the model takes as input a sequence of audio samples and generates a new sequence of samples that sound similar to the input. In some cases, generating audio that only depends on the previous time steps rather than the entire input sequence may be desirable. This can be achieved using a masked convolution that masks out the future time steps by setting the filter weights to zero for those positions in the kernel. By doing so, the model can only rely on the previous time steps to generate the following sample.

Masked convolutions can be implemented using standard convolutional layers with appropriate masking of the filter weights. Another approach is using a specialized masked convolutional layer that takes an additional mask tensor as input, indicating which input values should be masked.

One advantage of masked convolutions is that they can help prevent the model from overfitting to the training data by forcing it to rely on the input data available at each time step rather than the ground truth values that may not be available at test time. This can be particularly useful in tasks where the input data has a sequential or temporal structure, and the model needs to make predictions based on partial information.

\begin{tikzpicture}
	\begin{pgfonlayer}{nodelayer}
		\node [style=input] (0) at (-7.25, 0) {$n$};
		\node [style=none] (1) at (-11.25, 0) {};
		\node [style=none] (2) at (-3.25, 0) {};
		\node [style=none] (3) at (-12.25, 0) {$Input$};
		\node [style=none] (4) at (-2, 0) {$Output$};
		\node [style=none] (5) at (-9.25, 0.5) {$U$};
		\node [style=none] (6) at (-5.25, 0.5) {$V$};
		\node [style=none] (7) at (-7.25, 4) {$W$};
		\node [style=none] (8) at (0, 0) {$\equiv$};
		\node [style=input] (9) at (7.25, 0) {$n_{t - 1}$};
		\node [style=none] (10) at (3.25, 0) {};
		\node [style=none] (11) at (11.25, 0) {};
		\node [style=none] (12) at (1.75, 0) {$Input_{t - 1}$};
		\node [style=none] (13) at (13, 0) {$Output_{t - 1}$};
		\node [style=none] (14) at (5.25, 0.5) {$U$};
		\node [style=none] (15) at (9.25, 0.5) {$V$};
		\node [style=none] (16) at (8.25, 2.5) {$W$};
		\node [style=input] (17) at (7.25, 5) {$n_{t-2}$};
		\node [style=none] (18) at (3.25, 5) {};
		\node [style=none] (19) at (11.25, 5) {};
		\node [style=none] (20) at (1.75, 5) {$Input_{t-2}$};
		\node [style=none] (21) at (13, 5) {$Output_{t-2}$};
		\node [style=none] (22) at (5.25, 5.5) {$U$};
		\node [style=none] (23) at (9.25, 5.5) {$V$};
		\node [style=input] (24) at (7.25, -5) {$n_{t}$};
		\node [style=none] (25) at (3.25, -5) {};
		\node [style=none] (26) at (11.25, -5) {};
		\node [style=none] (27) at (2, -5) {$Input_{t}$};
		\node [style=none] (28) at (12.75, -5) {$Output_{t}$};
		\node [style=none] (29) at (5.25, -4.5) {$U$};
		\node [style=none] (30) at (9.25, -4.5) {$V$};
		\node [style=none] (31) at (8.25, -2.5) {$W$};
		\node [style=none] (32) at (7.25, 8) {...};
		\node [style=none] (34) at (7.25, -7) {};
		\node [style=none] (35) at (7.25, 10) {};
	\end{pgfonlayer}
	\begin{pgfonlayer}{edgelayer}
		\draw [style=->input] (1.center) to (0);
		\draw [style=input->neuron, in=135, out=45, loop] (0) to ();
		\draw [style=->input] (0) to (2.center);
		\draw [style=->input] (10.center) to (9);
		\draw [style=->input] (9) to (11.center);
		\draw [style=->input] (18.center) to (17);
		\draw [style=->input] (17) to (19.center);
		\draw [style=->input] (17) to (9);
		\draw [style=->input] (25.center) to (24);
		\draw [style=->input] (24) to (26.center);
		\draw [style=->input] (9) to (24);
	\end{pgfonlayer}
\end{tikzpicture}

\paragraph{RNN Variants} \label{sec:rnn-variants}

\Acp{RNN} are a class of neural networks designed to handle sequential data. They work by maintaining an internal state or memory, which allows them to process sequences of inputs and produce outputs that depend on previous inputs. This definition allows for other types of networks, different from the one shown in the previous section.

However, there are convincing reasons to investigate architectures beyond the basic approach. The typical \ac{RNN} is not without its difficulties, particularly the development of the \textit{vanishing gradient problem} as a major concern.

\Acp{RNN} may be prone to the vanishing gradient problem due to their method of updating internal states. \Acp{RNN} compute the internal state at time step $t$ by incorporating the input of time step $t$ and the previous state of time step $t-1$. This process generates a sequence of dependencies between the present and past states, which leads to multiple matrix multiplications during the backpropagation.

As the gradient keeps decreasing after each multiplication, it may ultimately shrink to such an extent that the neural network fails to learn long-term patterns present in the input sequence.

The prime example of an architecture that fights the vanishing gradient problem is the \textbf{\acf{LSTM}} \cite{hochreiter_long_1997}. \Acp{LSTM} were introduced in 1997 and have since become a popular choice for processing sequential data, especially in \ac{NLP}.

The main idea behind \ac{LSTM} is to introduce memory cells that can selectively forget or remember information from previous time steps. Each memory cell has three gates: the input gate, the forget gate, and the output gate, which control the flow of information into and out of the cell. One can see how these work in Figure \ref{fig:lstm}.

\begin{figure}[ht]
    \centering
    \includegraphics[width=\textwidth]{figures/2-sota/LSTM3-chain.png}
    \caption[Long Short-Term Memory]{\textbf{\Acf{LSTM}} --- 
    The network was taken from \cite{christopher_olah_understanding_2015}. Each green rectangle is an \ac{LSTM} cell. The blue circles represent an input at a given time, such that $X_t$ is the input at time $t$. The pink circles represent the output at a given time $h_t$. Each \ac{LSTM} cell takes three inputs: the previous cell`s memory and output, and the current input, and outputs both the real output and the current memory. In the Figure, the top exiting arrow corresponds to the memory, while the bottom corresponds to the output. The yellow rectangles have learnable parameters, weights, and biases.}
    \label{fig:lstm}
\end{figure}

The forget gate determines which information from the previous state should be forgotten or retained. It takes the current input, $X_t$, and the previous state, $H_{t-1}$, and computes the gate activation, $f_t$, a number between 0 and 1 that determines how much of the previous state should be retained or forgotten. It is calculated as

\begin{equation}
    f_t = \sigma (W_f * [H_{t-1}, X_t] + b_f)    
\end{equation}

$W_f$ is the weight matrix for the forget gate, and $b_f$ is the bias vector.

The input gate determines which information from the current input and the previous state should be allowed into the cell. It takes the current input, $X_t$, and the previous state, $H_{t-1}$, as inputs and computes the gate activation, $i_t$, which is a number between 0 and 1 that determines how much of the input and previous state should be allowed into the cell. It is calculated as

\begin{equation}
    i_t = \sigma (W_i * [H_{t-1}, X_t] + b_i)
\end{equation}

where $\sigma$ is the sigmoid activation function, $W_i$ is the weight matrix for the input gate, and $b_i$ is the bias vector.

The memory update computes the new information stored in the memory cell. It takes the current input, $X_t$, and the previous state, $H_{t-1}$, as inputs and computes the new candidate memory content, $\tilde{C_t}$. It is calculated as

\begin{equation}
    \tilde{C_t} = \tanh (W_C \times [H_{t-1}, X_t] + b_C)
\end{equation}

$W_C$ is the weight matrix for the candidate memory update, and $b_C$ is the bias vector.

The memory cell stores the current memory content, $C_t$, a combination of the previous and new candidate memory content, as determined by the input and forget gates. It is calculated as

\begin{equation}
    C_t = f_t \times C_{t-1} + i_t * \tilde{C_t}
\end{equation}

The output gate determines which information from the current memory cell should be used as output. It takes the current input, $X_t$, and the previous state, $H_{t-1}$, as inputs and computes the gate activation, $o_t$, a number between 0 and 1 that determines how much of the memory cell content should be outputted. It is calculated

\begin{equation}
    o_t = \sigma (W_o \times [H_{t-1}, X_t] + b_o)
\end{equation}

The hidden state, $H_t$, is computed by applying the output gate to the memory cell. It is calculated as

\begin{equation}
    H_t = o_t \times \tanh (C_t)
\end{equation}

By selectively forgetting or remembering information from previous time steps, \acp{LSTM} can maintain long-term dependencies in the input sequence and avoid the vanishing gradient problem that occurs in vanilla \acp{RNN}.

Other kinds of networks that handle the vanishing gradient problem have been proposed. One example is the \textbf{\acf{GRU}}~\cite{cho_learning_2014}. \Ac{GRU} is very similar to \ac{LSTM}. The main difference between \ac{GRU} and \ac{LSTM} is in their architecture and the number of gates they use to control the flow of information. While the \ac{LSTM} has the three gates mentioned above, \ac{GRU} has two gates: the reset gate and the update gate. The update gate controls how much of the previous hidden state to keep, and the reset gate determines how much of the previous hidden state to forget.

Overall, \ac{GRU} has a simpler architecture compared to \ac{LSTM}, which makes it faster to train and requires fewer parameters. However, \ac{LSTM} is generally considered more powerful and better suited for tasks that require longer-term memory, such as machine translation or speech recognition.

Another widespread helpful \ac{RNN} implementation is the \textbf{bidirectional \ac{RNN}}~\cite{schuster_bidirectional_1997}. Unlike traditional \acp{RNN}, which process input sequences in only one direction, from beginning to end, bidirectional \acp{RNN} process input sequences in both directions, from beginning to end and from end to beginning, simultaneously.

The main idea behind bidirectional \acp{RNN} is to use two separate \acp{RNN}, one that reads the input sequence in the forward direction and another that reads the sequence in the backward direction. The output of the two \acp{RNN} are then combined to produce the final output.

The benefit of using a bidirectional \ac{RNN} is that it allows the network to access both past and future context when making predictions about the current time step.
\paragraph{Autoencoder (AE)} \label{sec:autoencoders}

\Acp{AE} are a type of neural network architecture used for unsupervised learning. Their origin is difficult to precise as the literature is vast and multiple representations of this kind of network started popping up at the end of the 1980s with different names.

The primary goal of \acp{AE} is to learn an efficient representation of the input data by encoding it into a lower dimensional space, known as the bottleneck layer, and then decoding it back to the original dimensions. \Acp{AE} try to learn the identity function. The network learns to minimize the reconstruction error between the input and the reconstructed output.

The architecture of an \ac{AE} typically consists of the encoder and the decoder. The encoder maps the input to the bottleneck layer, while the decoder maps the bottleneck layer back to the original dimensions. These layers can be seen in figure \ref{fig:autoencoder}. The bottleneck layer acts as a bottleneck that restricts the amount of information that can be passed through, forcing the network to learn the most important features of the input data. For instance, if the size of the bottleneck layer were the same as the input and the output, the network would not learn the essential features, as a simple pass-through would suffice.

A simple use case for \acp{AE} is embeddings, for instance, word embeddings. If given multiple sentences, a model learns to transform a word in itself, passing it through a bottleneck layer; in the future, only the values in the bottleneck (the embedding) and the decoder are required to get the original word. Since, ideally, these embeddings are more feature rich than the word itself, these representations are beneficial for \ac{NLP} tasks.

\begin{figure}[ht]
    \centering
    \ctikzfig{figures/2-sota/autoencoder}
    \caption[Autoencoder]{\textbf{\Acf{AE}} --- This \ac{AE} receives an input of size five and tries to learn a way to transform it into itself by passing through a bottleneck of size three. In the initial part, from the input to the bottleneck, an encoder is present, while the second part displays a decoder.}
    \label{fig:autoencoder}
\end{figure}
\paragraph{U-Net (2015)}\label{sec:u-net}

U-Net is a deep learning architecture introduced by Olaf Ronneberger et al. in 2015 for biomedical image segmentation tasks \cite{ronneberger_u-net_2015}. The name \textit{U-Net} comes from the shape of the network, which resembles the letter \textit{U}.

The U-Net architecture consists of two main parts: an encoder path and a decoder path. The encoder path is a typical \ac{CNN} (Section \ref{sec:CNN}) that extracts features from the input image. On the other hand, the decoder path uses upsampling operations to recover the spatial resolution of the feature maps and generate a segmentation mask. Upsampling is the process of increasing the resolution of an image or signal by using, for instance, nearest neighbor interpolation, bilinear interpolation, or transposed convolution (see Section \ref{sec:transpoed-conv}). The U-Net architecture can be seen in Figure \ref{fig:u-net}.

\begin{figure}[ht]
    \centering
    \includegraphics[width=\textwidth]{figures/2-sota/u-net/u-net.png}
    \caption[U-Net]{\textbf{U-Net} --- This figure was taken from the original paper and follows an example for images with $572 \times 572$ pixels. Each blue box corresponds to a multi-channel feature map. The number of channels is denoted on top of the box. The x-y-size is provided at the lower left edge of the box. White boxes represent copied feature maps. The arrows denote the different operations. One can see that at its smallest size, the feature maps were $28 \times 28 \times 1024$, and the end result provides two features maps of $388 \times 388$ pixels, meaning that this specific network could be used, for instance, for segmentation of foreground vs. background. It is also essential to notice that, to keep the original image's fidelity, there is a deconvolutional step for each convolutional one. These are concatenated (represented by the white boxes).}
    \label{fig:u-net}
\end{figure}

The encoder path extracts features from the input image. These features are then compressed into a lower-dimensional representation, which the decoder path uses to generate a segmentation mask for the input image. In this sense, the U-Net architecture can be viewed as a specialized type of \ac{AE} not designed to reconstruct the input image itself.

The U-Net architecture was initially designed to classify each pixel in an image as belonging to a specific object or background: image segmentation. It combines high-level and low-level features from the input image to generate the final segmentation.
\subsubsection{Foundations of Deep Learning} \label{sec:dl-foundations}

This section explores the fundamental principles of \ac{DL}, with a specific focus on activation functions and backpropagation. Activation functions play a vital role in neural networks by introducing non-linearity and enabling the representation of complex relationships. This section covers the most commonly used activation functions. In addition, this section discusses the backpropagation algorithm, which plays a crucial role in training feedforward neural networks by calculating gradients with respect to network weights for iterative adjustments and optimal model performance. Furthermore, this section explores optimization techniques within deep learning.

\paragraph{Activation Functions} \label{sec:activation}

Activation functions are essential to neural networks, as they introduce non-linearity into the model. Nonlinearity is important because it allows the model to learn complex relationships between inputs and outputs. This section will discuss some of the most commonly used activation functions, including sigmoid, \ac{tanh}, and \ac{ReLU}.

\begin{figure}[ht]
    \centering
    \includegraphics[width=\textwidth/2]{figures/2-sota/activation/sigmoid.png}
    \caption[Sigmoid Activation Function]{One can see that the values that are output are exclusively between 0 and 1 with its value being 0.5 for $x = 0$.}
    \label{fig:sigmoid}
\end{figure}

\label{sec:sigmoid}

The \textbf{sigmoid} activation function is prevalent in neural networks. It is defined as:

\begin{equation}
	\sigma(x) = \frac{1}{1 + e^{-x}}
\end{equation}

where $x$ is the input to the function. The output of the sigmoid function is always between 0 and 1, which makes it useful for binary classification tasks. The sigmoid function is also differentiable, which is essential for backpropagation during training. Figure \ref{fig:sigmoid} visually represents this function.

However, the sigmoid function has a few drawbacks. One issue is that the function's gradient approaches zero as the input becomes very large or very small. This can cause the weights to update very slowly during training, a problem known as the \textit{vanishing gradient} problem. Additionally, the output of the sigmoid function is not zero-centered, which can make optimization more difficult. On the other hand, if the gradients become extremely large, it can cause the weights to update too much in each iteration, leading to the \textit{gradient explosion problem}. This can lead to the model diverging and failing to converge to a good solution. 

\label{sec:tanh}

The \textbf{\acf{tanh}} activation function is similar to the sigmoid function, but its output is between -1 and 1:

\begin{equation}
	\tanh(x) = \frac{e^x - e^{-x}}{e^x + e^{-x}}
\end{equation}

Tanh is differentiable and valuable for binary classification tasks like the sigmoid function. However, it has the same vanishing gradient problem as the sigmoid function. A visual representation can be seen in Figure~\ref{fig:tanh}.

\begin{figure}[ht]
    \centering
    \includegraphics[width=\textwidth/2]{figures/2-sota/activation/tanh.png}
    \caption[Tanh Activation Function]{A very similar function except that its limits are present on -1 and 1}
    \label{fig:tanh}
\end{figure}

\label{sec:relu}

The \textbf{\acf{ReLU}} activation function is a popular choice in \ac{DL}. It is defined as:

\begin{equation}
	\text{ReLU}(x) = \max(0, x)
\end{equation}

The \ac{ReLU} function is also zero-centered, which can make optimization easier. However, this function is not differentiable at $x=0$, which can cause problems during training. Variants of the \ac{ReLU} function have been proposed to address this issue. A visual representation can be seen in Figure \ref{fig:relu}.

\begin{figure}[ht]
    \centering
    \includegraphics[width=\textwidth/2]{figures/2-sota/activation/relu.png}
    \caption[Relu Activation Function]{Linear for positive values, 0 for negative values.}
    \label{fig:relu}
\end{figure}

The \ac{ReLU} activation has a potential problem known as the \textit{dying ReLU} problem. When a neuron's weights become such that its input is always negative, its gradient will always be zero. In this situation, the neuron will become inactive, or \textit{dead}, and will no longer update during training. This problem can lead to underfitting, as some neurons stop learning and contributing to the model. Variants of \ac{ReLU}, such as \textbf{Leaky \ac{ReLU}}, have been proposed to mitigate the dying \ac{ReLU} problem by assigning a slight non-zero gradient for negative input values.

The Leaky ReLU function is defined as:

\begin{equation}
\text{Leaky ReLU}(x) = \text{max}(a x , x)
\end{equation}

where $a$ is a small positive number such as $0.01$. This non-zero gradient for negative inputs can help prevent neurons from becoming inactive during training. This function can visually be seen in Figure~\ref{fig:leaky-relu}.

\begin{figure}[ht]
    \centering
    \includegraphics[width=\textwidth/2]{figures/2-sota/activation/tanh.png}
    \caption[Leaky Relu Activation Function]{Very similar to Relu but with non-zero gradient for negative inputs.}
    \label{fig:leaky-relu}
\end{figure}
\paragraph{Backpropagation Algorithm for Training Neural Networks} \label{sec:backpropagation}

\textit{Backpropagation} is an algorithm used to train feedforward neural networks by computing the gradient of the loss function concerning the network weights. This gradient is then used to update the weights in the opposite direction of the gradient, allowing the network to learn how to predict outputs given inputs accurately.

To understand backpropagation, it is crucial to define the loss function, which measures the network's performance on a given task. For instance, the cross-entropy loss is commonly used in classification tasks to quantify the difference between predicted probabilities and the correct labels. More information on loss functions can be found in Section \ref{sec:loss-functions}. Backpropagation adjusts the network weights to minimize the loss function.

The backpropagation algorithm works by computing the gradient of the loss function concerning each weight in the network. This gradient tells how much the loss function would change if one were to make a small change to the weight. This gradient is then used to update the weight in the direction that reduces the loss function.

The gradient is computed using the chain rule of calculus. Considering a simple feedforward neural network with one hidden layer. The output of the network is given by:

\begin{equation}
	y = h(\sum_{j=1}^M w_{2,j} h(\sum_{i=1}^N w_{1,i} x_i + b_1) + b_2)
\end{equation}

where $x_i$ is the $i$-th input, $w_{1, i}$ and $w_{2, j}$ are the weights connecting the input to the hidden layer and the hidden layer to the output, respectively, $b_1$ and $b_2$ are the biases of the hidden layer and the output, respectively, $h$ is the activation function, such as sigmoid, (see section \ref{sec:activation}), and $N$ and $M$ are the numbers of inputs and hidden units, respectively.

The loss function is a function of the output $y$ and the actual label $t$.

To compute the gradient of the loss function concerning a weight $w_{i,j}$, one first needs to compute the local gradient of the output for the weight. This is given by:

\begin{equation}
	\frac{\partial y}{\partial w_{i,j}} = h'(\sum_{j=1}^M w_{2,j} h(\sum_{i=1}^N w_{1,i} x_i + b_1) + b_2) h(\sum_{i=1}^N w_{1,i} x_i + b_1) w_{2,j}
\end{equation}

where $h'$ is the derivative of the activation function. One can then use the chain rule to compute the gradient of the loss function for the weight:

\begin{equation}
	\frac{\partial E}{\partial w_{i,j}} = (y - t)\frac{\partial y}{\partial w_{i,j}}
\end{equation}

Once one has computed the gradient of the loss function concerning all the weights in the network, the weights can be updated using gradient descent:

\begin{equation}
	w_{i,j} \leftarrow w_{i,j} - \alpha \frac{\partial E}{\partial w_{i,j}}
\end{equation}

where $\alpha$ is the learning rate, which determines the step size of the weight update.

Backpropagation can be extended to networks with multiple hidden layers using the chain rule to propagate the gradient backward through the network.
\paragraph{Optimization with Stochastic Gradient Descent} \label{sec:sgd}

\Acf{SGD} is a widely used optimization algorithm in \ac{DL}. It is an iterative method that minimizes the loss function by updating the model parameters in the direction of the negative gradient of the loss function. In each iteration, \ac{SGD} randomly selects a subset of the training data, called a mini-batch, and computes the gradient of the loss function concerning the parameters using the mini-batch. The parameters are then updated by subtracting the product of the gradient and a learning rate hyperparameter, which controls the step size of the update. The learning rate is typically set to a small value to ensure the stability and convergence of the algorithm.

Technically, it would be possible to use the loss of all the samples to update the model weights. However, using all the samples to update the weights, also known as batch gradient descent, can be computationally expensive and memory-intensive, especially for large datasets. In contrast, \ac{SGD} updates the weights based on a randomly selected sample mini-batch, reducing the computational and memory requirements and enabling faster convergence.

Moreover, \ac{SGD} introduces stochasticity in the optimization process, which can help the algorithm escape from local minima and explore different regions of the parameter space. This can improve the model's generalization performance and prevent overfitting the training data.

However, \ac{SGD} can also be noisier and less stable than batch gradient descent due to the mini-batches random sampling and the gradients' fluctuation. Therefore, finding an appropriate learning rate and mini-batch size is crucial for the solutions' convergence and quality in \ac{SGD}.

Formally, let $\theta$ be the vector of model parameters, $L(\theta)$ be the loss function, $D$ be the training dataset, and $B$ be a mini-batch sampled from $D$. Then, the update rule for \ac{SGD} can be written as:

\begin{equation}
	\theta_{t+1} = \theta_{t} - \alpha \nabla_{\theta} L(\theta_t; B)
\end{equation}

where $\alpha$ is the learning rate, and $\nabla_{\theta} L(\theta_t; B)$ is the gradient of the loss function concerning the parameters evaluated on the mini-batch $B$ at iteration $t$.

\ac{SGD} has several advantages, such as its simplicity and low memory requirements, which make it suitable for large-scale datasets and complex models. However, it also has some limitations, such as its sensitivity to the learning rate and the mini-batch size, which can affect the solutions' convergence and quality. Therefore, several variants of \ac{SGD}, such as Adagrad (that adapts the learning rate for each parameter based on its historical gradients, working well for sparse data) and Adam (explained in Section \ref{sec:adam}, adds a fraction of the previous update to the current update, and adapts the learning rate), have been proposed to address these issues and improve the algorithm's performance.
\paragraph{Optimization with the Adam Optimizer} \label{sec:adam}

Adam is a variant of \ac{SGD} (see Section \ref{sec:sgd}) that adapts the learning rate for each parameter based on the estimates of the first and second moments of the gradients \cite{kingma_adam_2017}.

Adam addresses some of the limitations of \ac{SGD}, such as the sensitivity to the learning rate and the mini-batch size, using a more sophisticated update rule incorporating information about the gradients and their history. Specifically, Adam computes a moving average of the gradients and their squares, which adjusts the updates' learning rate and momentum.

The estimates of the moments are computed as follows:

\begin{equation}
    m_t = \beta_1 m_{t-1} + (1-\beta_1)g_t
\end{equation}

\begin{equation}
    v_t = \beta_2 v_{t-1} + (1-\beta_2)g_t^2
\end{equation}

where $g_t$ is the gradient of the loss function with respect to the parameters at iteration $t$, $m_{t-1}$ and $v_{t-1}$ are the estimates of the moments at the previous iteration, and $\beta_1$ and $\beta_2$ are the decay rates for the moving averages of the gradients and their squares, respectively.

The update rule for Adam can then be written as follows:

\begin{equation}
    \theta_{t+1} = \theta_{t} - \frac{\alpha}{\sqrt{\hat{v}_t+\epsilon}}\hat{m}_t
\end{equation}

where $\theta_t$ is the vector of model parameters at iteration $t$, $\alpha$ is the learning rate, $\hat{m}_t$ and $\hat{v}_t$ are the biased estimates of the first and second moments of the gradients, respectively, and $\epsilon$ is a small constant to avoid division by zero.

The biased estimates of the moments are computed as follows:

\begin{equation}
    \hat{m}_t = \frac{m_t}{1-\beta_1^t}
\end{equation}

\begin{equation}
    \hat{v}_t = \frac{v_t}{1-\beta_2^t}
\end{equation}

where $t$ is the iteration number. The bias correction is necessary to account for the fact that the estimates are initialized at zero and may be biased towards zero in the early iterations.

By using the biased estimates of the moments, Adam can handle noisy or sparse gradients and converge faster than SGD on a wide range of optimization problems. However, the choice of the hyperparameters, such as the learning rate, the decay rates, and the epsilon value, can significantly affect the performance of Adam and should be carefully tuned for each problem.

Adam combines the benefits of both momentum and adaptive learning rates by using the moving average of the gradients to update the momentum and the moving average of the squared gradients to adapt the learning rate. Unlike \ac{SGD}, which uses a fixed learning rate for all parameters, Adam adapts the learning rate individually for each parameter based on the estimate of the second moment of the gradient. This can improve the solutions' convergence and quality, especially for problems with sparse or noisy gradients. Moreover, Adam can handle non-stationary objective functions and noisy gradients, which can be challenging for other optimization algorithms.

However, Adam also has some limitations, such as its sensitivity to the choice of hyperparameters and its tendency to overshoot the optimal solution. Therefore, tuning the hyperparameters carefully and monitoring the algorithm's convergence during training is important.
\subsubsection{Generative Deep Learning Architectures}

Generative \ac{DL} architectures are a subset of \ac{DL} networks designed to generate new and diverse data samples from a learned distribution.

By finding latent data structures and learning to reproduce the hidden statistics behind observed data, they do so. To achieve this, the model tries to estimate an underlying probability distribution $p_{data}$ when given a set of samples from this distribution. Thus, training a generative model involves selecting the best parameters that reduce some concept of distance/loss/error between the model and the actual distribution. As Huzaifah \cite{huzaifah_deep_2021} states: ``given training data points $X$ as samples from an empirical distribution $p_{data}(X)$, we want to learn a model $p_\theta(X)$, belonging to a model family $M$ that closely matches $p_{data}(X)$, by repeatedly changing model parameters $\theta$''. This is expressed as the problem in equation \ref{eq:generative-models-base}.

\begin{equation} \label{eq:generative-models-base}
    \min_{\theta \in M} d (p_{data}, p_\theta)
\end{equation}

Standard functions for $d$ are displayed in section \ref{sec:loss-functions}.

These models have been applied to various tasks, such as image synthesis, text generation, and audio synthesis. Generative \ac{DL} models have gained popularity recently due to their ability to produce high-quality data and model complex distributions.

This section will present the most used architectures.

\begin{tikzpicture}
	\begin{pgfonlayer}{nodelayer}
		\node [style=input] (0) at (0, 5) {};
		\node [style=input] (1) at (0, 2) {};
		\node [style=input] (2) at (0, -1) {};
		\node [style=input] (3) at (0, -4) {};
		\node [style=input] (4) at (0, 8) {};
		\node [style=input] (5) at (0, -7) {};
		\node [style=neuron] (6) at (4, -7) {};
		\node [style=neuron] (7) at (4, -4) {};
		\node [style=neuron] (8) at (4, -1) {};
		\node [style=neuron] (9) at (4, 2) {};
		\node [style=neuron] (10) at (4, 5) {};
		\node [style=neuron] (11) at (8, -7) {};
		\node [style=neuron] (12) at (8, -4) {};
		\node [style=neuron] (13) at (8, -1) {};
		\node [style=neuron] (14) at (8, 2) {};
		\node [style=neuron] (15) at (12, -7) {};
		\node [style=neuron] (16) at (12, -4) {};
		\node [style=neuron] (17) at (12, -1) {};
		\node [style=output] (18) at (16, -7) {};
		\node [style=output] (19) at (16, -4) {};
		\node [style=none] (20) at (8, -9) {};
		\node [style=none] (21) at (7, 12) {};
		\node [style=none] (22) at (0, 11) {Input Layer};
		\node [style=none] (23) at (9, 4) {Hidden Layers};
		\node [style=none] (24) at (16, -1) {Output Layer};
	\end{pgfonlayer}
	\begin{pgfonlayer}{edgelayer}
		\draw [style=input->neuron] (4) to (10);
		\draw [style=input->neuron] (0) to (10);
		\draw [style=input->neuron] (0) to (9);
		\draw [style=input->neuron] (1) to (9);
		\draw [style=input->neuron] (1) to (8);
		\draw [style=input->neuron] (2) to (8);
		\draw [style=input->neuron] (2) to (7);
		\draw [style=input->neuron] (3) to (7);
		\draw [style=input->neuron] (3) to (6);
		\draw [style=input->neuron] (5) to (6);
		\draw [style=neuron->neuron] (10) to (14);
		\draw [style=neuron->neuron] (9) to (14);
		\draw [style=neuron->neuron] (9) to (13);
		\draw [style=neuron->neuron] (8) to (13);
		\draw [style=neuron->neuron] (8) to (12);
		\draw [style=neuron->neuron] (7) to (12);
		\draw [style=neuron->neuron] (7) to (11);
		\draw [style=neuron->neuron] (6) to (11);
		\draw [style=neuron->neuron] (14) to (17);
		\draw [style=neuron->neuron] (13) to (17);
		\draw [style=neuron->neuron] (13) to (16);
		\draw [style=neuron->neuron] (12) to (16);
		\draw [style=neuron->neuron] (12) to (15);
		\draw [style=neuron->neuron] (11) to (15);
		\draw [style=neuron->output] (17) to (19);
		\draw [style=neuron->output] (16) to (19);
		\draw [style=neuron->output] (16) to (18);
		\draw [style=neuron->output] (15) to (18);
	\end{pgfonlayer}
\end{tikzpicture}

\paragraph{Variational Autoencoder (VAE)} \label{sec:vae}

Kingma and Welling proposed the concept of \acfp{VAE} in 2013 \cite{kingma_auto-encoding_2022}. The authors proposed a new approach to traditional \acp{AE} that utilizes a variational inference method to model complex distributions in high-dimensional spaces. Thus allowing for the generation of new, unseen data similar to the initial training.

In the realm of \acp{AE}, traditional approaches, as discussed in Section \ref{sec:autoencoders}, aim to map the identity function through the utilization of encoder and decoder networks. The encoder takes the input and transforms it into a compressed vector representation, while the decoder reconstructs the input from this compressed representation. However, Variational Autoencoders (\acp{VAE}) introduce a distinct variation in their approach. Instead of modeling the input as a deterministic vector, the encoder in \acp{VAE} characterizes the input data as a probability distribution across potential representations. This distribution is typically represented by two sets of latent values: one corresponding to the mean and the other to the variance. These latent sets are often modeled using fully connected layers within the architecture.

Consequently, the \ac{VAE} framework imposes a constraint on the embedded vector by confining it to a specified number of points in a hyperplane. Interestingly, this enables us to discard the encoder entirely. By working with a continuous latent space, the decoder of the \ac{VAE} can generate novel and diverse samples akin to the training data. Figure \ref{fig:vae} provides a visual depiction of this concept, illustrating how the decoder operates based on samples from the latent distribution to generate new output.

\begin{figure}[ht]
    \centering
    \ctikzfig{figures/2-sota/vae}
    \caption[Variational autoencoder]{\textbf{\Acf{VAE}} --- The \ac{VAE} encoder operates in a manner comparable to the traditional \ac{AE}, but with a notable distinction. Instead of directly mapping the input to a single latent representation, the \ac{VAE} encoder translates the input into two sets of latent features: the normals and the variances. These latent features are represented in the Figure by the two sets of nodes within the bottleneck of the encoder architecture.}
    \label{fig:vae}
\end{figure}

This was encouraging, as distributions near each other would produce similar outputs. This means, it created smooth changes between data points. The explanation for this is that \acp{VAE} discover low-dimensional parameterized representations of the data \cite{huzaifah_deep_2021}.

For training, these networks use an objective function that aims to minimize the loss between input and output and ensure that the learned distribution is similar to a prior distribution, such as a Gaussian.

However, sometimes during training, the \ac{VAE} can learn to ignore the latent variable and instead rely solely on the decoder network to generate the output. This means that the encoder network outputs the same distribution over the latent space for all input data points, resulting in a collapsed posterior distribution. In other words, the encoder fails to capture the variability in the input data, and the decoder generates outputs that are not diverse.

This phenomenon is known as \textit{posterior collapse}, and it can occur due to various reasons, such as a high reconstruction loss weight or a small latent space size. Posterior collapse can severely impact the performance of the \ac{VAE} and result in poor-quality generated samples.

These models can be used for any generative task, such as computer vision, natural language processing, and sound generation.

\begin{tikzpicture}

	\node[] (z) {$\vec{z}$};
	\node[style=input, right=3em of z] (x) {$G(\vec{z})$};
	\draw[-stealth, thick] (z) -- (x);
	\node[left=of z] (i) {};
	\node[below=of x] (xt) {$X$};
	\node[left=5em of xt] (it) {};
	\node[style=output, right=5em of x, yshift=-2.5em] (D) {$D(x)$};
	\node[right=3em of D] (out) {real?};
	\draw[-stealth, thick] (D) -- (out);
			
	\node[right=2.5em of x, circle, fill, inner sep=0.15em] (pt1) {};
	\node[right=2.5em of xt, circle, fill, inner sep=0.15em] (pt2) {};
			
	\draw[dashed, thick] (pt2) edge[bend left] (pt1);
			
	\node[circle, draw, thick, fill=white, inner sep=0.15em] at ([xshift=-0.9em, yshift=-3em]pt1.north) (pt3) {};
			
	\draw[-stealth, thick] (x) --  node[above] {$\hat{X}$} (pt1);
	\draw[-stealth, thick] (xt) -- (pt2);
	\draw[-stealth, thick] (pt3) -- (D);

\end{tikzpicture}
\paragraph{Normalizing Flow Models} \label{sec:flow-model}

Normalizing flow models provide a flexible and robust framework for generative modeling and were first introduced in 2015 \cite{rezende_variational_2016}.

The main idea is to use the change of variables in the probability distributions technique to convert simple distributions into more complicated ones. This technique requires applying a transformation to a distribution that changes it into another, more intricate, distribution. The entire idea begins with a simple distribution (for instance, Gaussian) for a set of hidden variables $z$. The goal is to change this distribution into a complex one that corresponds to an output $X$. A single transformation is provided by a smooth and reversible function $f$ that can relate $z$ and $X$, such that $X = f (z)$ and $z = f^{- 1}(X)$. Considering the complexity of $X$, one of these transformations might not produce a sufficiently complex distribution. Hence, multiple reversible transformations are combined sequentially, forming a ``flow''. Neural network layers determine each mapping function in the flow \cite{huzaifah_deep_2021}. Figure \ref{fig:normalizing-flows} illustrates this process.

\begin{figure}[ht]
    \centering
    \ctikzfig{figures/2-sota/normalizing-flows}
    \caption[Normalizing flows network]{\textbf{Normalizing flows network} --- This illustration was based on \cite{weng_flow-based_2018} and shows the application of multiple invertible functions $f_k$ composed one after the other in order to build the complex output $z_K = x$ from a simple Gaussian distribution.}
    \label{fig:normalizing-flows}
\end{figure}

Accurately, let $z_0$ be a multivariate random variable with a distribution $p_0(z_0)$ where $p_0$ is, for example, a Gaussian distribution. Then, for $i = 1, ..., K$ where $K$ is the number of flow operations, let $z_i = f_i(z_{i - 1})$ be a sequence of random multivariate variables. $f_i^{-1}$ should exist for training to occur. The final output $z_K$ models the target distribution.

Normalizing flow models are flexible, meaning they can model various distributions by stacking multiple normalizing flows to form a deep network. This allows it to capture complex relationships between variables in the data.

These models have been proven effective for the generative modeling of high-dimensional data.

In the generative scene, these models are distinguished from the previously mentioned ones because they can speed up the generations and modeling processes \cite{huzaifah_deep_2021}.
\paragraph{Diffusion Models} \label{sec:diffusion}

Until the proliferation of diffusion models, the architecture most used for data generation was the \ac{GAN} (Section \ref{sec:gan}). The problem is that \acp{GAN} are hard to train. For instance, mode collapse can happen. In mode collapse, the generator always generates the same data that fools the discriminator.

\textit{Diffusion models} \cite{sohl-dickstein_deep_2015} simplify this generation process into more intuitive small steps where the work of the network is lighter and is run multiple times. This is done by taking inspiration from non-equilibrium thermodynamics. These models define a Markov chain of diffusion steps to slowly add random noise to data and then learn to reverse the diffusion process to construct desired data samples from the noise.

Practically, diffusion models use a Markov chain to gradually convert one distribution into another. This chain starts from a simple known distribution (\textit{e.g.} a Gaussian) into a target distribution using a diffusion process. Learning in this framework involves estimating small perturbations to a diffusion process, using a network such as the U-Net (Section \ref{sec:u-net}). Estimating small perturbations is more tractable than explicitly describing the whole distribution with a single, non-analytically-normalizable potential function. This process can be seen in Figure \ref{fig:diffusion}

\begin{figure}[ht]
    \centering
    \ctikzfig{figures/2-sota/diffusion}
    \caption[Diffusion model]{\textbf{Diffusion model} --- This illustration was based on \cite{ho_denoising_2020} and shows the process of applying Gaussian noise to an image sample through multiple steps $q(X_t|X_{t-1})$. The model will then learn the operation $p$ that transforms $X_t$ into $X_{t-1}$ with $p(X_{t-1}|X_t)$ and so on until $X_0$. At this point, the model has generated a new data sample.}
    \label{fig:diffusion}
\end{figure}

The ultimate goal is to define a forward (or inference) diffusion process which converts any complex data distribution into a simple, tractable distribution and then learn a finite-time reversal of this diffusion process which defines the generative model distribution \cite{sohl-dickstein_deep_2015}.

One problem is that one needs to decide how much noise one wants to increment per iteration. For instance, if one decides to train a network that directly learns to denoise full Gaussian to a real image, then one is simply training a \ac{GAN} generator. It is easier to remove a small amount of noise per iteration. The amount of noise added per iteration is a hyperparameter called a scheduler. For instance, one can add the same amount of noise per iteration, called the \textit{linear schedule}. Multiple schedules may have different impacts.

For instance, given a linear scheduler, one can define that for $t = x$, the sample would be the original one with $k = x \times 10$ random data points with Gaussian noise. This allows data generation in different timestamps without running through all timestamps. For instance, generating a data sample with $t = 5$ would be as easy as noising $k = 50$ random data points.

To train these networks, one would give pairs of the original data sample $X$ plus a data sample at a random timestamp $X_t$ plus the random step $t$, $X_t = X + N(t)$ where $N$ is a noising function. The network would learn to get the noise from the data given a timestamp. This means that the network would learn to predict $N(t)$ using image segmentation. This will not always be perfect, so the network learns to predict $\tilde{N(t)}$. Then, theoretically, by applying $X_t - \tilde{N(t)}$, one gets $\tilde{X}$, which should be as close as possible to $X$. This process for $t = 50$ is challenging, as most of the data is Gaussian noise. However, applying the process for, for instance, $t = 1$, should be quite easy.

For inference, one gets noisy data $X_t$ and a given timestamp $t$. Applying the network returns $\tilde{N(t)}$ as explained previously. By doing $\tilde{X} = X_t - \tilde{N(t)}$, one generates a bad data sample. But then, the algorithm takes $\tilde{X}$ and applies $N(t - 1)$. This results in another noisy data sample with less noise. This process loops until $t = 0$. By then, a new data sample is generated.
\paragraph{Transformers} \label{sec:transformers}

In 2017, the introduction of the ``Attention is All You Need''~\cite{vaswani_attention_2017} paper marked a significant milestone in \ac{DL}. Although initially introduced for \ac{NLP}, the transformer architecture has proven helpful in various data generation tasks, including audio synthesis as shown in Section~\ref{sec:related-work}. This marked a paradigm shift from the conventional \ac{RNN}-based models, which were earlier widely used, with some incorporating a rudimentary form of the attention mechanism.

In transformers, attention is a key component that allows the model to focus on relevant parts of the input sequence when making predictions. Mathematically, attention can be defined as a weighted sum of values based on their importance or relevance.

Let's break down the mathematical formulation of attention in transformers:

Query \(Q\), Key \(K\), and Value \(V\): These are three linear transformations applied to the input sequence. The query represents the element for which we want to compute attention weights, while keys and values represent all elements in the sequence.

To calculate how much each value contributes to the output for a given query, we compute dot products between the query and all keys:

\begin{equation}
    \text{{scores}} = Q \cdot K^T
\end{equation}

Here, \(\cdot\) represents matrix multiplication, \(K^T\) denotes transpose of matrix \(K\).

The next step is to normalize these scores using a softmax function along dimension 1 (rows) to obtain attention weights that sum up to 1:

\begin{equation}
    \text{{weights}} = \text{{softmax}}(\text{{scores}})
\end{equation}

Finally, we take a weighted sum of values using these normalized attention weights:

\begin{equation}
\begin{split}
    &\text{{attended\_values}} = V \cdot {\text{{weights}}} \\
    &\text{{output}} = {\sum{( {\text{attended\_values} } })}
\end{split}
\end{equation}

Here, $\text{softmax}$ computes exponentiated values scaled by their row-wise sums, while $\text{sum}$ performs summation across rows.

The resulting output represents an attended representation obtained by giving higher weightage/importance to more relevant parts of the input sequence based on similarity with respect to the query. This mathematical formulation allows transformer models to capture long-range dependencies effectively by attending to the pertinent information in the source sequence when generating predictions.

The attention mechanism allows the model to give different importance to different input parts. For instance, let us imagine a translation task English-Portuguese. If naively translated, the sentence ``She is a doctor'' could be translated to ``Ela é um doutor''. However, if, when generating the last word, the model gives some importance to the word ``she'', it might guess that the correct word is ``doutora''.

The key idea behind transformers is to completely disregard the recurrent architecture and only use attention by using self-attention. Self-attention is a mechanism that allows the calculation of the importance of each input element concerning all other elements in the input sequence. This allows the model to dynamically focus on the most relevant information at each step of the calculation instead of relying on fixed relationships between elements in the input as in traditional recurrent neural networks.


%% HERE
The transformer architecture, shown in Figure \ref{fig:transformer}, serves as the basis for advanced transformers utilized in a variety of applications, such as audio synthesis. This architecture includes an encoder and decoder, both with stacked layers, which are essential to processing textual input data and producing consistent audio waveforms.

The encoder, consisting of $N$ identical layers, aims to convert the input text into continuous representations that encapsulate crucial semantic information. Each layer features two sub-layers that enable this conversion:

\begin{enumerate}
    \item Multi-Head Self-Attention: As explained above, this mechanism allows the encoder to model input text sequence dependencies objectively. Multiple scaled dot-product attention heads attend to various positions of the text sequence simultaneously. This process captures intricate relationships within the text in parallel, enriching the representation.
    \item Position-wise Feedforward Network: Composed of two linear transformations with \ac{ReLU} activation, this neural network introduces non-linear interactions within the embedded space. These interactions bolster the model's capability to comprehend intricate semantic nuances that exist within the text.
\end{enumerate}

Residual skip connections and layer normalization surround each sub-layer, enhancing training stability. By mapping input text to continuous representations, the encoder empowers subsequent stages to generate audio that aligns with the input's semantics.

The decoder, which includes $M$ stacked layers, generates the output audio waveform while conditioned on the continuous representations from the encoder. This conditioning guarantees that the synthesized audio aligns with the intended semantics of the input text. Each decoder layer comprises three sub-layers.

\begin{enumerate}
    \item Masked Multi-Head Self-Attention over Previous Outputs: This sublayer enables modeling of dependencies between previously generated audio samples, facilitating the creation of coherent output samples. It blocks leftward information flow, ensuring a causal relationship between created samples.

    \item Multi-Head Attention over Encoder Outputs: By attending over the final encoder representations, this sublayer allows the decoder to incorporate the input text semantics into the generation process. This attention mechanism guarantees that the synthesized sample remains aligned with the intended meaning of the input.

    \item The Position-wise Feedforward Network consists of two linear transformations with \ac{ReLU} activation, similar to the encoder. It improves the decoder's ability to comprehend complex relationships between different samples.
\end{enumerate}

Like the encoder, the decoder employs residual connections and layer normalization for stability during training. The decoder generates the output sequentially, predicting one sample at a time. The integration of multi-head attention mechanisms at both the local and global levels enables the synthesis of diverse and natural outputs. This autoregressive process, conditioned on powerful semantic representations, produces high-fidelity outputs aligned with the input.

This architectural change significantly accelerates the training and inference processes by allowing the use of larger data sets while greatly improving the generation results - thus revolutionizing the progress made in the field.

\begin{figure}[ht]
    \centering
    \includegraphics[width=0.5\textwidth, scale=0.8]{figures/2-sota/transformer.png}
    \caption[Transformer]{\textbf{Transformer} --- This illustration was taken from \cite{vaswani_attention_2017} and shows the general architecture of the base transformer. One can see that the constituents are pretty simple. These are simple word embeddings (which are not covered in this study), self-attention, and feedforward layers. The left side of the structure is called the encoder, while the right side is called the decoder.}
    \label{fig:transformer}
\end{figure}
\paragraph{Vector Quantised Variational AutoEncoder (VQ-VAE) (2018)} \label{sec:vq-vae}

The \acf{VQ-VAE}, introduced in 2018, model distinguishes itself from traditional \acp{VAE} in two main aspects: the encoder network outputs discrete codes instead of continuous ones, and the prior is learned rather than static. While continuous feature learning has been the focus of many previous works, this model, introduced by \cite{oord_neural_2018}, concentrates on discrete representations, a natural fit for complex reasoning, planning, and predictive learning.

The \ac{VQ-VAE} model combines the \ac{VAE} framework with discrete latent representations through a parameterization of the posterior distribution of (discrete) latents given an observation. Based on vector quantization, this model is simple to train, does not suffer from significant variance, and avoids the ``posterior collapse''. As illustrated in Fig~\ref{fig:vq-vae}, the \ac{VQ-VAE} architecture consists of an encoder, a discrete latent space, and a decoder.

\begin{figure*}[ht]
    \centering
    \includegraphics[width=\textwidth]{figures/2-sota/vq-vae.png}
    \caption[VQ-VAE]{\textbf{VQ-VAE} --- Taken from the original paper, this Figure presents two distinct illustrations. On the left side, a detailed diagram of the \ac{VQ-VAE} architecture is provided, showcasing the flow of information through the encoder, the discrete latent space, and the decoder. On the right side, a visualization of the embedding space is displayed, where the encoder output $z(X)$ is mapped to its nearest embedding point $e_2$. The red arrow represents the gradient $\nabla_z L$, influencing the encoder's output adjustment. This adjustment may result in a different configuration during the subsequent forward pass, highlighting the dynamic nature of the learning process within the \ac{VQ-VAE} model.}
    \label{fig:vq-vae}
\end{figure*}

The \ac{VQ-VAE} defines a latent embedding space $e \in R^{N \times D}$, where $N$ is the size of the discrete latent space (i.e., a $N$-way categorical), and $D$ is the dimensionality of each latent embedding vector $e_n$. There are $N$ embedding vectors $e_n \in R^D, n \in 1, 2, ..., N$. The model takes an input $X$, passed through an encoder producing output $z_e(X)$. The discrete latent variables $z$ are then calculated by the nearest neighbor look-up using the shared embedding space $e$. The input to the decoder is the corresponding embedding vector $e_n$. This forward computation pipeline is a regular autoencoder with a non-linearity that maps the latents to 1-of-N embedding vectors.

The posterior categorical distribution $q(z|X)$ probabilities are defined as one-hot (Eq. \ref{eq:vq-vae-posterior}):

\begin{equation} \label{eq:vq-vae-posterior}
  q(z = n|X) =
  \begin{cases}
    1 & \text{for } n = \text{argmin}_j ||z_e(X)-e_j||_2, \\
    0 & \text{otherwise}.
  \end{cases}
\end{equation}

where $z = e_n$ is the closest embedding vector to the encoder output $z_e(X)$. During forward computation, the nearest embedding $z_q(X)$ is passed to the decoder, and during the backward pass, the gradient $\nabla_z L$ is passed unaltered to the encoder. The overall loss function has three components to train different parts of the \ac{VQ-VAE}: the reconstruction loss, the \ac{VQ} objective, and the commitment loss. The total training objective becomes:

\setlength{\arraycolsep}{0.0em}
\begin{eqnarray}
\label{eq:vq-vae-loss}
\label{eq:reconstruction-loss}L&{}={}&\log p(X|z_q(X))\\
\label{eq:vq-objective}&&{+}\:||\text{sg}[z_e(X)] - e||_2^2\\
\label{eq:commitment-loss}&&{+}\:\beta||z_e(X) - \text{sg}[e]||_2^2
\end{eqnarray}
\setlength{\arraycolsep}{5pt}

This equation combines the three following terms:
\begin{enumerate}
	\item \textbf{Reconstruction loss} (Equation \ref{eq:reconstruction-loss}): This term represents the log probability of the input data $X$ given the latent variable  $z_q(X)$. It measures how well the model can reconstruct the input data using $z_q(X)$ as a representation. Maximizing this term would lead to a better reconstruction of the input data.
	\item \textbf{\Ac{VQ}} (Equation \ref{eq:vq-objective}): The second term measures the difference between the stop-gradient of the encoder output $z_e(X)$ and the embedding vector $e$. The stop-gradient operator, denoted as $\text{sg}$, acts as the identity during the forward pass but has zero partial derivatives during the backward pass. This term encourages the model to use the embeddings effectively by minimizing the distance between the encoder output and the closest embedding vector.
	\item \textbf{Commitment loss} (Equation \ref{eq:commitment-loss}): This term acts as a regularization term that measures the difference between the encoder output $z_e(X)$ and the stop-gradient of the embedding vector $e$. The  $\beta$ parameter controls the strength of this regularization. Minimizing this term would make $z_e(X)$  closer to the straight-through estimator of $e$.
\end{enumerate}

VQ-VAE has emerged as a vital component in generative artificial intelligence, spanning domains such as image~\cite{ramesh_zero-shot_2021} and sound generation~\cite{yang_diffsound_2022}.
\paragraph{Multi-Scale Vector Quantised Variational AutoEncoder (MS-VQ-VAE) (2019)} \label{sec:ms-vq-vae}

The \acf{MS-VQ-VAE} model is a generalization of the \ac{VQ-VAE} model (see Section \ref{sec:vq-vae}) that employs multiple discrete latent spaces with different scales and dimensions. Tjandra et al. proposed this model \cite{tjandra_vqvae_2019} to learn unsupervised hierarchical and discrete representations of complex data. The \ac{MS-VQ-VAE} architecture comprises an encoder, a multiscale discrete latent space, and a decoder.

The key difference between the \ac{MS-VQ-VAE} and the \ac{VQ-VAE} is that the former defines a collection of latent embedding spaces $e^s \in R^{K_s \times D_s}$, where $s$ denotes the scale index, $K_s$ denotes the cardinality of the discrete latent space at scale $s$, and $D_s$ denotes the dimensionality of each latent embedding vector $e^s_i$. There are $K_s$ embedding vectors $e^s_i \in R^{D_s}, i \in 1, 2, ..., K_s$. The model takes an input $x$, encoded into outputs $z_e^s(x)$ at different scales. The discrete latent variables $z^s$ are then obtained by the nearest neighbor look-up using the shared embedding space $e^s$. The decoder input is the corresponding embedding vector $e^s_k$. This forward computation pipeline resembles a regular \ac{AE} (see section \ref{sec:autoencoders}) with a non-linearity that maps the latents to 1-of-$K_s$ embedding vectors.

The posterior categorical distribution $q(z^s|x)$ probabilities are defined as one-hot (Eq. \ref{eq:ms-vq-vae-posterior}), analogous to Eq. \ref{eq:vq-vae-posterior} in section \ref{sec:vq-vae}, but with an additional scale index:

\begin{equation} \label{eq:ms-vq-vae-posterior}
 q(z^s = k|x) = \begin{cases}
 1 & \text{for } k = \text{argmin}_j ||z_e^s(x)-e^s_j||_2, \\
 0 & \text{otherwise}.
 \end{cases}
\end{equation}

The overall loss function consists of three components for each scale: the reconstruction loss, the VQ objective, and the commitment loss. The total training objective becomes (Eq. \ref{eq:ms-vq-vae-loss}), analogous to Eq. \ref{eq:vq-vae-loss} in section \ref{sec:vq-vae}, but with a summation over scales:

\begin{equation} \label{eq:ms-vq-vae-loss}
 L = \sum_{s=1}^{S} (\log p(x|z_q^s(x)) + ||\text{sg}[z_e^s(x)] - e^s||_2^2 + \beta||z_e^s(x) - \text{sg}[e^s]||_2^2)
\end{equation}

The benefit of using multiple codebooks and scales is that it enables the model to capture different levels of abstraction and granularity in audio signals. For instance, lower scales can encode phonetic information in speech, while higher scales can encode prosodic information. Furthermore, using multiple codebooks can enhance the diversity and expressiveness of the latent space by allowing more combinations of discrete codes.
\subsection{Foundations for Enhancing Generative Models for Audio} \label{sec:parallel-tasks}

To develop generative models for audio, it is necessary to address several factors that impact their performance and quality. This thesis concentrates on three main areas: data augmentation, evaluation metrics, and data embedding.

\textit{Data augmentation} is the process of applying transformations to the original data to increase its size and diversity. This can help overcome the limitations of small or imbalanced datasets and improve the generalization ability of generative models. Different types of data augmentation techniques for sound generation and their effects on the model outcomes are discussed in Sextion \ref{sec:data-augmentation}.

\textit{Evaluation metrics} are the methods used to measure the quality and diversity of the generated sounds. They provide a way to compare different generative models and assess their strengths and weaknesses. However, evaluating sound generation is not trivial, as it involves objective and subjective criteria. We review various evaluation metrics for sound generation and their advantages and disadvantages in Section \ref{sec:evaluation}

\textit{Data embedding} is the technique of converting data into numerical representations that capture its essential features and characteristics. This can facilitate the learning process of generative models and enhance their expressiveness and efficiency. We explore different data embedding methods in section \ref{sec:text-embedding}.

\subsubsection{Data Augmentation} \label{sec:data-augmentation}

Data augmentation is crucial to \ac{ML} tasks. It helps to increase the training dataset's size and improve the model's robustness. For the task at hand, there are two main types of data augmentation: acoustic and linguistic. Data augmentation can reduce overfitting and improve generalization by introducing more variations into the training set~\cite{pluscec_data_2023}.

\paragraph{Acoustic Data Augmentation}

According to Abayomi-Alli et al. \cite{abayomi-alli_data_2022}, the most commonly used data augmentation tools for audio \ac{ML} tasks are the addition of noise, time shifting, pitch shifting, \ac{GAN}-based methods (see Section \ref{sec:gan}), time stretching, and concatenation. Other techniques, such as overlapping, are also helpful. All these techniques play a critical role in increasing the size of the training dataset and providing the model with a diverse range of input data. This Section thoroughly explores these techniques and discusses their impact on the performance of sound-based \ac{ML} models.

Table \ref{tab:acoustic-data-augmentation} provides a concise summary of the advantages and limitations of each technique.

\begin{table}[hp]
\caption{Summary of Acoustic Data Augmentation Methods}
\label{tab:acoustic-data-augmentation}
\begin{tabularx}{\textwidth}{|p{2cm}|X|X|X|}
\hline
\textbf{Method}     & \textbf{Description}                                                                             & \textbf{Advantages}                                                                                  & \textbf{Limitations}                                                                                             \\ \hline
Addition of Noise   & Adding random noise to the original audio signals. & Increases dataset diversity, helps model learn to handle noisy environments.                         & Noise must be chosen to align with the problem at hand.                            \\ \hline
Time Shifting       & Altering the temporal structure of an audio signal by shifting it. & Helps model learn sound patterns invariant to temporal changes, improves generalization. & Trimming and shifting sections may introduce artifacts.                                    \\ \hline
Pitch Shifting      & Altering an audio signal's frequency. & Trains models to recognize sound patterns free from pitch variations.                                & May introduce artifacts or distortions.                                       \\ \hline
\ac{GAN} Based Methods   & Utilizing \ac{GAN} networks to generate synthetic audio signals.            & Generates high-quality data.                      & Requires significant computational resources.                     \\ \hline
Time Stretching     & Changing the time duration while preserving their spectral content.             & Generates samples with different time durations.                     & Simple methods may introduce artifacts; sophisticated methods are computationally intensive.     \\ \hline
Sound Concatenation & Joining snippets of multiple audio signals to form a new one.               & Trains models that recognize specific sound sequences.                         & Labels associated with the original audio signals may need to be modified to represent the new signal. \\ \hline
Sound Overlapping   & Merging two or more audio signals to form a new combined audio signal.                           & Helps models identify sound patterns amidst overlapping sounds.              & Amplitude adjustment and normalization are crucial to prevent overpowering or clipping in the composite signal.  \\ \hline
\end{tabularx}
\end{table}

\subparagraph{Addition of Noise}
Data augmentation for audio with the addition of noise is a common technique that involves adding random noise to the original audio signals to produce new, diverse audio samples.

The process of adding noise to audio signals involves several steps:

\begin{enumerate}
    \item Select a noise source: This can be any type of noise, generated or real \cite{novotny_analysis_2019}, such as white noise, babble noise, static noise, factory noise, jet cockpit, shouting, background noise, and others, \cite{abayomi-alli_data_2022}. The noises should be chosen according to the problem at hand.
    \item Specify the noise level: For instance, the augmented sound might be $y' = y + 0.05 \times Wn$ where $y$ is the initial sound, $y'$ the augmented sound, and $Wn$ some white noise \cite{mushtaq_environmental_2020}.
    \item Add the noise: The selected noise source is then added to the audio signal by adding the noise and audio signals element-wise.
    \item Normalize the output: Finally, the resulting audio signal with added noise is normalized to prevent clipping or overloading.
\end{enumerate}

Repeating these steps with different noise sources and noise levels makes it possible to generate multiple, diverse audio samples that can be used for data augmentation purposes.

\subparagraph{Time Shifting} \label{sec:time-shifting}
Time shifting, also known as time warping, is a data augmentation technique that involves altering the temporal structure of an audio signal.

Time shifting involves shifting the entire audio signal by a certain amount of time, either forwards or backward. This can be achieved by adding or removing samples from the audio signal or changing the existing samples' position within the signal.

One approach to implementing time shifting involves trimming the length of the audio signal and using the trimmed sections to create new, diverse audio samples. For example, consider an audio signal of size 150. If this signal is trimmed to a length of 125, up to 25 new audio samples can be generated by shifting the trimmed sections. These new samples can be labeled the same as the original audio signal.

Time shifting by trimming and shifting sections of the audio signal can significantly impact the performance of sound-based \ac{ML} models. By providing the model with diverse, time-shifted versions of the audio signal, this technique can help the model learn to identify sound patterns invariant to temporal changes, such as the presence of a particular sound event or the spoken words in an audio recording. This can lead to better generalization performance on new, unseen data and improved overall model performance.

\subparagraph{Pitch Shifting}
Pitch shifting is a technique used in audio data augmentation that involves altering an audio signal's fundamental frequency.

Pitch shifting is achieved by adjusting the pitch of an audio signal positively or negatively. For example, a plus or minus two shift can be implemented \cite{mushtaq_environmental_2020}. This process results in audio signals that have a different pitch. This can be helpful in training models to recognize sound patterns that are free from pitch variations.

\subparagraph{GAN Based Methods}
Utilizing \acp{GAN} (see Section~\ref{sec:gan}) in data augmentation for audio signals can be a powerful and effective method, albeit slower than other techniques. In this approach, a \ac{GAN} network is trained on the available audio data to learn the underlying patterns and distributions present in the data. The network then generates new, synthetic audio signals similar to the input data \cite{qian_data_2019}.

The success of the \ac{GAN}-based data augmentation method depends heavily on the quality of the \ac{GAN} training, as well as the diversity of the input data. If the \ac{GAN} is trained well and the input data is diverse, the generated data will be of high quality.

It is important to note that \acp{GAN} require considerable computational resources and training time compared to other data augmentation techniques. However, the results obtained from \acp{GAN} can be highly effective and accurate, making this approach a valuable addition to the data augmentation toolkit for audio machine learning tasks.

\subparagraph{Time Stretching}
Time stretching as a data augmentation technique for audio signals involves changing the time duration of the audio signals, typically by increasing or decreasing the time axis of the audio signals. The purpose of time stretching is to generate new audio samples from the original audio signals with different time durations.

A straightforward way of implementing time stretching is to use a stretching factor. For example, if the stretching factor is $1.2$, then the time axis of the audio signal is increased by 20\%. To achieve this, one approach is to use a naive algorithm that duplicates some of the samples in the audio signal according to the stretching factor. However, this simple method can result in undesirable artifacts, such as pitch changes, if the stretching factor is not an integer.

More sophisticated methods, such as phase vocoder-based time stretching, can produce high-quality time stretching with minimal artifacts. These methods use time and frequency domain processing techniques to stretch the audio signal while preserving its spectral content and temporal structure. The resulting audio signal has a different time duration while preserving the original pitch~\cite{akaishi_improving_2023}.

\subparagraph{Sound Concatenation}
Mixing up sounds, or sound concatenation, is a method for audio signals where multiple audio signals are joined to form a new and diverse audio signal. This technique can be achieved by taking snippets of multiple audio signals and concatenating them randomly or using cross-fade techniques to ensure a seamless transition between the different audio snippets.

This technique would be advantageous in a sound generation setting where one wants to make the network learn a prompt such as ``dog barking and then car honking''. When applying sound concatenation, one must consider that the label will also change.

\subparagraph{Sound Overlapping}

Sound overlapping, also referred to as sound mixing or audio blending, is a technical process that involves merging two or more audio signals to form a new combined audio signal. This process is currently utilized in popular data augmentation platforms~\cite{maguolo_audiogmenter_2022} to aid the model in identifying sound patterns amidst overlapping sounds, which is a frequent occurrence in real-world applications.

There are several steps involved in the process of sound overlapping. The first step involves selecting multiple audio signals, which can originate from the same or different sources depending on the desired outcome and problem at hand. The next step is to adjust the amplitude of each audio signal to achieve a balanced combination and prevent one signal from overpowering others. This step is crucial. This can be achieved by either normalizing the amplitude of each signal or scaling them based on a predetermined factor. After appropriately adjusting the amplitudes, the selected audio signals are combined by adding them element-wise. A new composite audio signal is generated using this process, which includes overlapping sounds from the original signals. It is essential to normalize the output to prevent clipping or overloading, ensuring a well-balanced and usable composite sound waveform.

Repeating these steps with different combinations of audio signals and amplitude adjustments can generate diverse composite audio samples for data augmentation purposes.

It is important to note that when using sound overlapping as a data augmentation technique, the labels associated with the original audio signals must also be considered. Sometimes, the labels may need to be combined or modified to accurately represent the new composite audio signal.
\paragraph{Linguistic Data Augmentation} \label{sec:text-augmentation}

Linguistic data transfiguration involves metamorphosing textual data to augment its diversity and quantity. This can ameliorate the performance and robustness of natural language processing models that rely on text data.   

For sonic milieu generation, linguistic data transfiguration provides an efficacious approach to creating more varied and verisimilitudinous soundscape descriptions from text. Nonetheless, various existing soundscape datasets discussed in Section \ref{sec:sol-datasets} contain categorical labels or tags instead of descriptive annotations.   

It is imperative to note that some linguistic data augmentation techniques only function when the original text is in natural language, while others can still be applied to categorical labels.

Variations in input text can facilitate the generation of soundscapes with more variability and legitimacy. The following sections cover specific linguistic data augmentation techniques and their applications for soundscape generation.

While linguistic data augmentation has several advantages, it poses some issues and challenges~\cite{shorten_text_2021}.

% Advantages
The main benefits of textual augmentation are as follows:
\begin{enumerate}
    \item It can reduce the costs of collecting and annotating textual data.
    \item It can improve the accuracy of models by increasing the training data size, alleviating data scarcity, mitigating overfitting, and creating variability in the data. 
    \item It can boost the generalizability of models by exposing them to different linguistic patterns and styles. 
    \item It can increase the robustness of our models by making them resilient against adversarial attacks that attempt to deceive them through expert alterations of the input sequences.  
\end{enumerate}

% Disadvantages
However, there are also some potential downsides:     
\begin{enumerate}   
    \item It can introduce noise or errors into the data that may impact the quality and readability of the augmented text.
    \item It can change or lose the original text's meaning, style, or complexity, mainly if the transformation is inappropriate or irrelevant to the task or domain.   
    \item It can be computationally expensive or time-consuming to generate high-quality and diverse augmented text, especially if it involves using external resources or models such as dictionaries, corpora, word embeddings, generative models, or translation models.
\end{enumerate}

Various frameworks have been developed for augmenting text data linguistically. These can be broadly grouped into symbolic and neural augmentation models.

\subparagraph{Symbolic Augmentation Models}

These methods employ rule-based transformations operating directly on the surface form of text via predetermined heuristics. They include:

\begin{itemize}     
  \item Rule-based augmentation replaces, inserts, or deletes tokens according to specified rules. An example is replacing named entities with alternatives \cite{wei_eda_2019}.   
   \item Graph-based augmentation uses graph structures to perturb the text, \textit{e.g.} swapping adjacent adjectives and nouns \cite{ahmed_text_2023}.  
  \item MixUp combines existing examples via interpolation to synthesize augmented instances, \textit{e.g.} combining ``The cat sat on the mat'' and  ``The dog lay on the rug'' to generate ``The cat lay on the mat'' \cite{guo_augmenting_2019}.
  \item Feature-based augmentation applies transformations to word embeddings, \textit{e.g.} adding noise to the embedding space \cite{cheung_modals_2021}.        
\end{itemize}   

Despite their interpretability, symbolic methods struggle with complex transformations.

\subparagraph{Neural Augmentation Models}

These techniques leverage deep neural networks and large language models. They encompass:   

\begin{itemize}    
   \item Back-translation, which translates text into another language and back, producing paraphrases, \textit{e.g.} translating ``The book was interesting.'' to French and back to English, yielding  ``The book was fascinating'' \cite{pham_meta_2021}.
  \item Generative augmentation employs generative language models to synthesize novel text, \textit{e.g.} using an \ac{LLM} such as GPT to rephrase sentences or simply fine-tune them.
\end{itemize} 

While more complex, neural methods can generate diverse and realistic augmented instances.

Both symbolic and neural augmentation aim to expose models to more variability during training, helping combat overfitting and improve performance. However, symbolic methods offer interpretability, while neural methods provide more flexibility and variation.

\begin{table}[ht]
\centering
\caption{A taxonomy of text augmentation methods for transformer language models according to their algorithmic properties and underlying approaches.}  
\begin{tabularx}{\textwidth}{|X|X|X|X|X|}   
\hline
  &\textbf{Interpretability} &\textbf{Algorithmic }   & \textbf{Capacity to} &\textbf{Paradigmatic} \\
 &\textbf{Transparency} & \textbf{Complexity}& \textbf{Leverage Labels} &\\[0.5ex] 
\hline
\textbf{Rule-based} & High &Relatively low &Incapable& Lexical substitution \\   
\hline
\textbf{Graph-based} &High & Moderate & Incapable&Knowledge graph-based\\       
\hline      
\textbf{Sample Combination} &Medium&Moderate & Capable& Embeddings summation\\
 \hline    
\textbf{Feature-based}&Medium &Moderate& Incapable &Noise injection\\     
\hline
\textbf{Generative}&Low &High &Capable &Text auto-generation\\    
\hline
\textbf{Back-translation}&Medium &High&Capable &Inverse translation\\             
\hline           
\end{tabularx}
\label{tab:textaugmethods}  
\end{table}

Table \ref{tab:textaugmethods} categorizes several common text augmentation strategies according to their transparency, complexity, dependence on labels, and paradigm.  

Regarding interpretability, rule-based and graph-based methods exhibit high transparency since they employ explicit symbolic transformations. In contrast, the stochastic nature of generative models and back-translation compromises their interpretability.

Computational complexity also differs. Rule-based and graph-based augmentation are relatively efficient since they apply explicit symbolic transformations. In comparison, training neural networks for generative modeling and back-translation requires more computational resources.

For leveraging labels, given that some datasets mentioned in Section \ref{sec:sol-datasets} contain categorical labels, and we desire descriptive annotations, it is pertinent to assess whether these techniques can transform categorical labels into text descriptions. Employing generative models or back-translation potentially enables this since the models attempt to make sense of the input and transform it into a sentence.
\subsubsection{Evaluation Metrics} \label{sec:evaluation}

The evaluation process can provide insights into the system's performance and reveal areas that need improvement. Moreover, a proper evaluation metric helps to compare the results of different models and choose the best one. The aim of this section is to provide a comprehensive overview of the available evaluation metrics for audio generation and to provide a foundation for the evaluation of the system developed in this thesis.

In this thesis, a comprehensive evaluation framework is developed that considers two different types of evaluations. The first type of evaluation focuses on metrics that can be used for training models offline, such as loss metrics, which play a crucial role in assessing the performance of a model. The second type of evaluation focuses on evaluating the broader impacts of a model, such as its environmental impact and inference time for a generation. These evaluations are essential in ensuring that the model not only performs well on its primary task but also has minimal negative impacts on other aspects of the system.

\paragraph{Loss Functions} \label{sec:loss-functions}

Evaluating generative audio systems is challenging due to the need for a standard set of metrics to capture the quality and diversity of the generated audio samples. Different studies often use different evaluation methodologies and metrics when reporting results, making a direct comparison to other systems intricate if not impossible~\cite{vinay_evaluating_2022}. Furthermore, the perceptual relevance and meaning of the reported metrics, in most cases unknown, prohibit any conclusive insights concerning practical usability and audio quality.

A review and comparison of the available evaluation metrics for audio generation is essential to provide a foundation for evaluating the system developed in this thesis. This section discusses some of the commonly used metrics for evaluating generative audio systems, such as \ac{MAE}, \ac{MSE}, \ac{KL} divergence, and \ac{ELBO}. It also discusses their advantages and limitations and how they can be applied to sound generation tasks.

\subparagraph{Mean Absolute Error} \label{sec:mae}

The \Acf{MAE} is a quantitative measure of the average magnitude of the errors between the predicted and actual values~\cite{willmott_advantages_2005}. It is computed as:

\begin{equation}
	\text{MAE} = \frac{1}{n} \sum_{i=1}^n |y_i - \hat{y}_i| 
\end{equation}

where $y_i$ denotes the true value, $\hat{y}_i$ denotes the predicted value, and $n$ denotes the number of samples. The \ac{MAE} is also called L1-norm loss or \acf{MAD}.  

The \Ac{MAE} is a relevant metric for evaluating generative models as it treats all errors equally and is arguably less sensitive to outliers compared to \ac{MSE} (see Section \ref{sec:mse}). Specifically, it indicates the average absolute difference between the predicted and actual values in the same unit as the output.

However, the \ac{MAE} has some notable limitations that should be considered. For instance, it does not directly capture the perceptual quality of the generated audio samples. The perceptual quality of audio can depend on factors such as timbre, pitch, or harmony, which are not explicitly determined by the \ac{MAE}. Therefore, to fully assess the quality of generative models, the \ac{MAE} should be complemented with other metrics and human evaluation. Additionally, the \ac{MAE} is a scale-dependent measure and cannot be used to compare predictions that use different scales.

To illustrate the difference between \ac{MAE} computed on raw audio versus spectrograms, consider the following example: For a 1D raw audio sample, the \ac{MAE} would measure the average absolute difference between the amplitude values of the audio signals. In contrast, if the audio data were represented as spectrograms, the \ac{MAE} would measure the average absolute difference between the magnitude values of the frequency bins. In this case, the spectrograms can be treated as images with a single channel, and the \ac{MAE} can be seen as a pixel-wise error metric.  

It is important to consider the limitations of the \ac{MAE} when evaluating generative models. For example, consider a scenario with two audio samples of a dog barking: one with a bark at one second and another with a bark at two seconds. Despite being similar sounds with different temporal positions, calculating their \ac{MAE} would lead to higher-than-expected error as it does not account for temporal alignment. Therefore, it is important to use other metrics and human evaluation methods along with \ac{MAE} to assess timing accuracy and perceptual quality of generated audio samples comprehensively.

\subparagraph{Mean Squared Error} \label{sec:mse}

\Acf{MSE} is a standard metric to evaluate the performance of a predictor or an estimator. It quantifies the average of the squared errors, the average squared difference between the estimated and actual values. \Ac{MSE} is always a non-negative value approaching zero as the error decreases. The smaller the \ac{MSE}, the better the predictor or estimator~\cite{hodson_mean_2021}.

\Ac{MSE} can be calculated as follows:

$$
MSE = \frac{1}{n} \sum_{i=1}^n (y_i - \hat{y}_i)^2
$$

Where $n$ is the number of data points, $y_i$ is the true value of the variable being predicted or estimated, and $\hat{y}_i$ is the predicted or estimated value.

\Ac{MSE} incorporates both the variance and the bias of the predictor or estimator. The variance measures how widely spread the estimates are from one data sample to another. The bias measures the distance of the average estimated value from the true value. For an unbiased estimator, the \ac{MSE} equals the variance.

\Ac{MSE} can compare different predictors or estimators and select the one that minimizes the \ac{MSE}. For instance, in linear regression, \ac{MSE} can be used to find the best-fitting line that minimizes the sum of squared errors. \Ac{MSE} can also evaluate the quality of a generative model that produces audio samples from textual input. In this case, \ac{MSE} can measure how similar the generated audio samples are to the target audio samples regarding their amplitude values.

Unlike \ac{MAE} (see Section \ref{sec:mae}), which assigns equal weight to all errors, \ac{MSE} penalizes larger errors more than smaller ones. This means that \ac{MSE} is more sensitive to outliers and may not reflect the overall discrepancy between the generated and target audio samples well. Moreover, \ac{MSE} does not account for perceptual aspects of audio quality, such as timbre, pitch, or loudness. Therefore, \ac{MSE} should be used with other metrics and evaluation methods, such as \ac{KL} divergence (see Section \ref{sec:kld}), subjective listening tests, or qualitative analysis.

\Ac{MSE} can be applied to sound generation tasks in different ways, depending on the representation of the audio data. Similar to \ac{MAE}, it can be applied to 1D raw audio and spectrograms.

It should be noted that \ac{MSE} is subject to the same temporal issue as \ac{MAE}. \Ac{MSE} may not be effective in identifying differences in timing accuracy. Therefore, it is crucial to employ other metrics and evaluation methods that specifically focus on timing aspects and perceptual quality when assessing audio produced by text-trained models.

\subparagraph{Cross-Entropy} \label{sec:cross-entropy}

Cross Entropy Loss is commonly used to evaluate \ac{DL} models, especially in classification or sequence generation tasks. It measures the dissimilarity between predicted and target distributions by calculating the average negative log-likelihood of predicting each class or element correctly.

Cross Entropy Loss can be applied in audio generation tasks to produce discrete elements, such as musical notes or phonemes. For instance, if one aims to create music based on textual input with particular note sequences, Cross Entropy Loss can assess the prediction accuracy of each note at every time step.

Mathematically, given data samples $x_i$ and their respective true labels $y_i$, where $i$ ranges from 1 to $n$, Cross Entropy Loss can be calculated in the following way:

\begin{equation}
    CE = -\frac{1}{n}\sum_{i=1}^{n} \sum_{j=1}^{C} y_{ij}\log(p_{ij})
\end{equation}

This is where:
The symbol $C$ represents the number of classes or elements.
The notation $y_{ij}$ indicates whether sample $x_i$ belongs to class $j$ (or has element $j$).
Additionally, $p_{ij}$ represents the predicted probability that sample $x_i$ belongs to class $j$ (or has element $j$).

The objective is to minimize the Cross Entropy Loss during training, so that the generative model can learn to predict precise and coherent distributions over classes or elements.

Nonetheless, it is important to consider some limitations when using Cross Entropy Loss to evaluate audio generation systems. First, the model assumes independence between individual predictions within one sample. However, this assumption may not hold for sequential audio data where the context and dependencies between elements are critical. Second, the Cross Entropy Loss does not directly capture perceptual aspects of audio quality, such as timbre or tonality. Therefore, it is advisable to combine Cross Entropy Loss with other evaluation metrics, such as \ac{MAE}, \ac{MSE}, or subjective listening tests to achieve a comprehensive understanding of the generative model's performance.

To summarize, the Cross Entropy Loss is commonly employed as a loss function for evaluating generative models that involve discrete element generation. Although it is applicable for audio generation tasks with categorical outputs such as music note prediction, it should be combined with other evaluation methods to gain a more comprehensive assessment of accuracy and perceptual quality.

\subparagraph{KL Divergence} \label{sec:kld}

\acf{KL} divergence, also known as relative entropy, is a non-symmetric measure of the difference between two probability distributions. It is a mathematical quantity that quantifies the distance between two probability distributions.

In simple terms, \ac{KL} divergence measures the difference between the probability distribution predicted by a model and the true underlying distribution of the data. \Ac{KL} divergence is commonly used to evaluate generative models.

\ac{KL} divergence is calculated as the expectation of the logarithmic difference between the predicted probability distribution and the actual distribution. It is a scalar value, and the smaller the \ac{KL} divergence, the closer the predicted distribution is to the real distribution.

The \ac{DL} system can be trained to produce sounds near the target sounds in terms of their probability distribution by using \ac{KL} divergence as a loss function. The concept is that the sound does not have to be alike the input, but only its distribution. Maximizing the log-likelihood between the generated output and the given input can be seen as minimizing the \ac{KL} divergence \cite{huzaifah_deep_2021}.

\subparagraph{Evidence Lower Bound (ELBO)} \label{sec:elbo}

\Acf{ELBO} is a lower bound on the log-likelihood of some observed data commonly used in variational Bayesian methods \cite{blei_variational_2017}.

The \ac{ELBO} is defined as follows:

\begin{equation}
ELBO = E_{Z \sim q}\left[\log \frac{p(X,Z; \theta)}{q(Z)} \right]
\end{equation}

where $X$ and $Z$ are random variables with joint distribution $p(X,Z; \theta)$, $\theta$ are the parameters of the model, and $q(Z)$ is an approximate posterior distribution for the latent variable $Z$. The \ac{ELBO} can be seen as the difference between two terms: the expected log joint probability of the data and the latent variables under the model and the entropy of the approximate posterior distribution.

The \ac{ELBO} has several desirable properties. First, it is a lower bound on the log-likelihood of the data, $\log p(X; \theta)$, also known as the evidence. Meaning that the \ac{ELBO} is a quantity that is always less than or equal to the log-likelihood of the data, which is the logarithm of the probability of the data given the model parameters. The log-likelihood of the data is also called the evidence because it indicates how well the model fits the data. The higher the log-likelihood, the more evidence we have that the model is suitable for the data. However, computing the log-likelihood of the data is often intractable. Therefore, the model can be optimized more easily using \ac{ELBO}.

Second, it is a tractable objective function that can be optimized concerning $\theta$ and $q(Z)$. This allows us to perform variational inference, approximating the posterior distribution $p(Z|X; \theta)$ by finding the $q(Z)$ that maximizes the \ac{ELBO}. This can be done using gradient-based methods, thus being used in machine learning systems.

Third, it can be decomposed into two significant components: the reconstruction term and the regularization term. The reconstruction term is the expected log-likelihood of the data given the latent variables under the model, $E_{Z \sim q}\left[\log p(X|Z; \theta)\right]$. It measures how well the model fits the data. The regularization term is the negative \ac{KL} divergence (see section \ref{sec:kld}) between the approximate posterior and the prior distributions, $-D_{KL}(q(Z)||p(Z))$. It measures how close the approximate posterior is to the prior. The \ac{KL} divergence is always non-negative, and it is zero if and only if $q(Z) = p(Z)$. Therefore, maximizing the \ac{ELBO} encourages data fidelity and posterior regularization.

The \ac{ELBO} can be applied to sound generation tasks using a deep generative model such as a \ac{VAE} (see section \ref{sec:vae}). This model can be trained by maximizing the \ac{ELBO} concerning its parameters and latent variables. The \ac{ELBO} can then be used to evaluate the quality and diversity of the generated sounds by comparing them to the target sounds. For instance, the \ac{ELBO} for a \ac{VAE} can be written as:

\begin{equation}
ELBO = E_{Z \sim q_\phi(Z|X)}\left[\log p_\theta(X|Z)\right] - D_{KL}(q_\phi(Z|X)||p(Z))
\end{equation}

The first term is the reconstruction term, which measures how well the decoder network reconstructs the input sound $X$ from the latent variable $Z$. The second term is the regularization term, which measures the proximity of the approximate posterior distribution to a prior distribution $p(Z)$.

By maximizing the \ac{ELBO}, a model learns to generate realistic sounds similar to the input sounds regarding their conditional distribution while ensuring that the latent variables have a smooth and regular structure that facilitates interpolation and manipulation.

\paragraph{Model Evaluation Functions}

The second type of evaluation in this thesis involves assessing the wider impacts and implications of the developed audio generation model. Although metrics used for training models offline provide insights into performance, it is equally important to evaluate how a model affects aspects beyond its primary task. This includes considerations such as the environmental impact and inference time during generation, among others. Understanding these broader impacts ensures that the model not only performs well in its intended purpose but also operates with minimal negative consequences or trade-offs in other areas of the system. Conducting evaluations encompassing these factors will provide a more comprehensive understanding of how well-rounded and sustainable our audio generation system is.

\subparagraph{Evaluating Energy Expended}

Evaluating the amount of energy expended by a deep learning model is crucial in developing and deploying these systems. With the increasing demand for machine learning applications and the complexity of deep learning models, energy efficiency has become a critical factor in designing and deploying deep learning systems.

With the growing concern for environmental sustainability, the energy footprint of deep learning models has become an essential topic in the field. Most of the recent advances produced by deep learning approaches rely on significant increases in size and complexity \cite{douwes_energy_2021}. Such improvements are backed by an increase in power consumption and carbon emissions. The high energy consumption of deep learning models during both the training and inference phases significantly impacts the environment, and it is imperative to address this issue.

Therefore, evaluating the amount of energy a deep learning model expends is essential in ensuring its practicality and scalability. This is a crucial step in ensuring that the deep learning models developed today are not only accurate, but also energy-efficient and sustainable for future deployment.

This evaluation can be done in two ways: physically measuring the energy expended by the machines on both learning and inference time or by approximating given average numbers per neuron, for instance.

A good model is a compromise between accuracy and complexity. If the model trains significantly longer to train or infer and does not provide way better results, in the context of this research, the model is not much better than a simpler counterpart.
\subsubsection{Data Embedding} \label{sec:text-embedding}

Data embedding is the technique of converting data into numerical representations that capture its essential features and characteristics. For sound generation, both audio and text embedding is important.

Data embedding is essential for generative models. This happens for two reasons.

First, it allows the models to work on smaller and lighter representations. This is, for instance, taking an audio sample with 5 seconds sampled at 16 kHz, representing 80 000 entries. Encoding that into meaningful features might reduce this number to a few thousand or even hundreds of entries. This allows for faster training.

Second, these representations are meaningful in ways where the raw input is not. For instance, take text embedding. The idea is that words such as ``pretty'' and ``beautiful'' have similar representations, helping the model generalize. If the model were to check the words, letter by letter, it would have a hard time realizing that some words, like these, have a relation between them.

So, text and audio embedding is essential for the task. However, audio embedding possesses several challenges, such as dealing with high-dimensional and sequential data, preserving temporal and spectral information, and ensuring robustness and interpretability. To handle this, both feature and learning-based methods can be applied.

Feature-based embedding methods extract predefined features from the raw audio data, such as spectral, temporal, or perceptual features. These features are then input to generative models or further processed to obtain lower-dimensional embeddings. One such example would be the application of the \ac{STFT} to build a spectrogram (see Section \ref{sec:sound}). Feature-based embedding methods have the benefit of being simple and interpretable, but they may also lose some information or introduce noise during the feature extraction process.

Using neural networks or other machine learning techniques, learning-based embedding methods learn embeddings directly from the raw audio data. These methods can automatically discover relevant features from the data without relying on predefined criteria. Learning-based embedding methods have the advantage of being flexible and adaptive, but they may also require more computational resources or suffer from overfitting or underfitting issues.

Text embedding has been a solved problem since the days of Word2Vec \cite{mikolov_efficient_2013}. This \ac{AE} (see Section \ref{sec:autoencoders}) model would be trained by getting the meaning of a word taking into account the word with whom the first appeared. This model makes possible the representation of a word through a vector of latent factors.

Nevertheless, the words cannot be merely embedded for the current issue. When a user adds a text input, embedding it is imperative. This represents the entire textual input. For example, computing the average of latent factors for each input would be a naive approximation.

Nevertheless, nowadays, this is mainly solved with transformers (see Section \ref{sec:transformers}), namely the encoder part. This part of the transformer takes a whole string and outputs a vector representation, precisely what the task needs.

After the vanilla transformer, the one that gained more prominence was BERT \cite{devlin_bert_2018} in 2018, which introduced the conditioning of the whole input for each word inputted, not only the words that appeared before, as the vanilla transformer did. Other later encoder transformers are based on BERT or tackle a different problem.

While there are many techniques for data embedding, recent advancements in the field have led to the development of specialized models for various modalities, such as CLIP \cite{radford_learning_2021} for images and MuLan for audio. Instead of embedding the text and the media separately and dealing with it afterward, media and text are embedded in the same space, meaning that a textual segment and a media sample representing the same textual segment should have similar latent factors.

\paragraph{MuLan} \label{sec:mulan}

MuLan \cite{huang_mulan_2022} is a state-of-the-art music audio embedding model that aims to link music audio directly to unconstrained natural language music descriptions.

MuLan employs a two-tower parallel encoder architecture, meaning two completely independent neural architectures, using a contrastive loss objective that elicits a shared embedding space between music audio and text.

Each MuLan model consists of two separate embedding networks for the audio and text input modalities. These networks share no weights, but each terminates in 2-normalized embedding spaces with the same dimensionality. The contrastive loss objective minimizes the distance between matching audio-text pairs while the distance between mismatched pairs is maximized. This approach enables MuLan to learn a joint representation of music audio and text that captures their semantic relationships.

MuLan is trained using 44 million music recordings (370K hours) and weakly-associated, free-form text annotations. The resulting audio-text representation subsumes existing ontologies while graduating to true zero-shot functionalities. MuLan demonstrates versatility in transfer learning, zero-shot music tagging, language understanding in the music domain, and cross-modal retrieval applications.
 \subsection{Deep Learning Frameworks} \label{sec:dl-frameworks}

\Ac{DL} frameworks have revolutionized the field of \ac{AI}, enabling researchers and practitioners to efficiently develop and deploy complex neural networks for the multiple \ac{DL} tasks. These frameworks provide a wide range of tools and techniques for building, training, and evaluating deep neural networks and have significantly accelerated the pace of progress in the field. This section explores some of the most popular \ac{DL} frameworks, their key features and capabilities, and how they have been used to develop state-of-the-art generative \ac{AI} models for audio synthesis from textual input.

Several \ac{DL} frameworks are available, and they differ in several ways, including their programming languages, ease of use, and performance. However, to the best of the author's knowledge, there is no recent study on the performance of today's \ac{DL} networks. The most recent is from 2017 \cite{parvat_survey_2017}.	

\subsubsection{TensorFlow} \label{sec:tensorflow}

Developed by Google, TensorFlow \cite{martin_abadi_tensorflow_2015} is one of the most widely used \ac{DL}. It supports both \ac{CPU} and \ac{GPU} computations and provides a variety of \acp{API} for building different types of neural networks. TensorFlow is written in Python, but its core functionality is implemented in C++ for optimal performance.

TensorFlow is an interface for expressing \ac{ML} algorithms and an implementation for executing them. It allows computations to be executed with little or no change on various systems, from mobile devices to large-scale distributed systems. The system is flexible and can express many different algorithms, including training and inference algorithms for deep neural network models.

TensorFlow can be used with various programming languages, including Python, JavaScript, C++, and Java. Python is the recommended language for TensorFlow, but other languages' \acp{API} may offer some performance advantages. Other languages like Julia, R, Haskell, and others have bindings.

\subsubsection{PyTorch} \label{sec:pytorch}

PyTorch is an open-source \ac{DL} framework developed by Facebook \cite{paszke_pytorch_2019}. It has gained popularity due to its ease of use and dynamic computation graph, which allows for more flexible and intuitive programming. PyTorch also supports \ac{CPU} and \ac{GPU} computations, and although it supports Python and C++, it has a Python-first approach, making it easy to integrate with other Python libraries.

PyTorch is a popular deep-learning framework that is easy to use and learn. It has a simple and intuitive \ac{API} that makes it easy to learn and use. PyTorch is also flexible and can be used for various applications.

\subsubsection{Keras} \label{sec:keras}

Keras is a Python library for building and training neural network models at a high-level. It offers a user-friendly interface, and it is commonly used for \ac{DL} purposes~\cite{chollet_keras_2015}. Keras is built on top of lower-level libraries such as TensorFlow (see Section~\ref{sec:tensorflow}). It simplifies the creation and training of neural networks by abstracting away many low-level details.
Keras enables fast neural network model creation by assembling pre-built building block layers. These building blocks consist of input layers, \ac{CNN} layers, \ac{RNN} layers, fully connected layers, activation functions, and other components.

Keras is often regarded as more user-friendly than TensorFlow, as it offers a high-level interface that hides many low-level details.

Keras provides a simplified \ac{API} for constructing and training deep learning models. Keras includes a variety of utilities for manipulating data, such as data preprocessing, data augmentation, and data visualization, facilitating data manipulation.

\subsubsection{Conclusions on Deep Learning Frameworks} \label{sec:dl-frameworks-conclusions}

Selecting a \ac{DL} framework is critical in developing a \ac{DL} project. It is vital for the development of this thesis that the best framework that can offer flexibility, ease of use, and optimal performance of the \ac{DL} models is selected.

PyTorch uses a dynamic computation graph, which is created for each iteration in an epoch. In each iteration, the code executes the forward pass, computes the derivatives of output \textit{w.r.t} to the network parameters, and updates the parameters to fit the given examples. After doing the backward pass, the graph is freed to save memory. A dynamic graph can be changed on the fly, allowing for more freedom, easier debugging, and easier experimentation with architectures and hyperparameters during the model development process. Tensorflow, on the other hand, uses a static graph. There, the library creates and connects all the variables at the beginning and initializes them into a static (unchanging) session. This session and graph persist and are reused: it is not rebuilt after each iteration of training, making it more efficient and restrictive.

Furthermore, PyTorch offers a more straightforward and intuitive \ac{API} compared to TensorFlow. This feature makes it easier for developers to write and debug code. Additionally, PyTorch offers more flexibility when creating custom layers and functions, which is impossible in Keras. This flexibility enables developers to create more complex and innovative models, which can lead to better performance. Keras offers a straightforward but too simple network, while low-level Tensorflow offers a complicated and convoluted \ac{API} making it challenging to focus on the real problem. PyTorch offers a good balance between simplicity and features.

One of the significant drawbacks of Keras is the need for more flexibility when customizing \ac{DL} models. Keras offers a limited set of pre-defined layers, making it difficult for developers to create complex custom layers and functions. This lack of flexibility limits the ability to make changes to the architecture of a network during the development process, which can hinder the performance of the final model. When one needs more custom layers, one has to resort to the Tensorflow jungle, making the development a hassle.

Both PyTorch and TensorFlow have good hardware support. However, PyTorch has a feature that distinguishes it from TensorFlow: data parallelism. PyTorch optimizes performance by using native support for asynchronous execution from Python. In TensorFlow, one has to manually code and fine-tune every operation to be run on a specific device to allow distributed training. Running on top of TensorFlow, Keras presents the same hardware problems as the latter.

These conclusions are in table \ref{tab:dl-frameworks}. Given all of this, and considering that ease of use, debugging, and high customization are essential for this work, the clear choice is PyTorch. One should consider its use when this work uses practical terms.

\begin{table}[h]
\centering
\caption{Comparison of PyTorch, Raw TensorFlow, and Keras}
\begin{tabular}{|c|c|c|c|}
\hline
\textbf{Feature} & \textbf{PyTorch} & \textbf{Raw TensorFlow} & \textbf{Keras} \\
\hline
Graph computation & Dynamic & Static & Static \\
Ease of use & Moderate & Difficult & Easy \\
Debugging & Good & Difficult & Moderate \\
Customization & High & High & Moderate \\
Hardware support & Good & Moderate & Moderate \\
\hline
\end{tabular}
\label{tab:dl-frameworks}
\end{table}

\subsection{Data Generators} \label{sec:data-generators}

Creating new data from existing data is known as generative modeling. This technique has numerous applications across various media types, including images, text, and video. However, each media type possesses unique characteristics, so generating them requires distinct approaches and techniques. Sound, for instance, presents a different set of challenges than other media types, and hence its generation necessitates diverse techniques.

This section aims to survey some of the state-of-the-art generators for media types that are not sound-based, primarily focusing on image generators. By doing so, one hopes to comprehensively understand the different approaches and techniques employed for generating various media types.

The discussion delves into the main ideas behind these generators, their strengths and limitations, and how they can be related to or inspired by sound generation. Through this exploration, this section highlights the diversity of methods used for generative modeling and how they can be adapted to different contexts.

\begin{table}[ht]
\centering
\caption{Comparison of Data Generators}
\begin{tabularx}{\textwidth}{|X|X|X|X|}
\hline
\textbf{Generator Name} & \textbf{Main Idea}                                                                                                                    & \textbf{Strengths}                                                                                                                                 & \textbf{Limitations}                                                                                                          \\ \hline
PixelCNN~\cite{oord_conditional_2016}                & Uses autoregressive connections to model images pixel by pixel.                                                                       & Fast and parallelizable training, high resolution and diversity, interpretable latent space.                                                       & Slow and sequential sampling, limited global coherence, difficulty in conditioning on high-level features.                    \\ \hline
DALL-E~\cite{ramesh_zero-shot_2021}                  & A transformer language model that creates images from text descriptions, using a dataset of text–image pairs.                         & Can generate novel and creative images, combine concepts in plausible ways, render text and apply transformations.                                 & Requires large amounts of data and compute, may produce harmful or biased outputs, not publicly accessible.                   \\ \hline
Stable Diffusion~\cite{rombach_high-resolution_2021}        & A latent diffusion model that creates images from text descriptions, using a dataset of text–image pairs and a pretrained CLIP model. & Can generate realistic and detailed images, is lighter than previous state of the art models.                                                      & Requires fine-tuning for specific domains, may produce artifacts or inconsistencies, sensitive to text prompts.               \\ \hline
GLIDE~\cite{nichol_glide_2021}                   & Guided diffusion-based approach for text-conditioned image generation.                                                                & High-fidelity image synthesis, classifier-free guidance preferred for photorealism and caption similarity, text-driven image editing capabilities. & Difficulty generating images for complex or unusual text prompts, slow generation speed.                                      \\ \hline
DALL-E 2~\cite{ramesh_hierarchical_2022}                & An improved version of DALL-E that generates more realistic and accurate images using a diffusion model.                              & Has better results than DALL-E and is faster.                                                                                                      & Still requires large amounts of data and compute, may still produce harmful or biased outputs, still not publicly accessible. \\ \hline
\end{tabularx}
\label{tab:data-generators}
\end{table}

\subsubsection{PixelCNN Decoders} \label{sec:pixelcnn}

Van den Oord et al. introduced \textit{PixelCNNs} as discussed in \cite{oord_conditional_2016, van_den_oord_pixel_2016}. PixelCNNs model high-dimensional discrete data, like images.

They employ the \ac{AE} architecture as described in Section \ref{sec:darn}, where pixels are generated sequently while conditioning on prior pixels.

The goal is to estimate a distribution over images that can be used to compute the likelihood of images and thus generate new ones. The network scans the image one row at a time and one pixel at a time within each row, using masked convolutions (see Section~\ref{sec:masked-conv}). For each pixel, it predicts the conditional distribution over the possible pixel values given the scanned context.

Each image $x$ is assigned a probability $p(x)$. To do this, if one considers each pixel sequentially, row by row, such that $x_k$ is the pixel number $k$, the final probability is as follows.

\begin{equation}
    p(x) = \prod_{i=1}^{h \times w} p(x_i | x_1, \dots, x_{i-1})
\end{equation}

Where $h$ is the height of the image and $w$ is the width.

For multiple colors, one can condition the colors on one another. For instance, instead of having $x_1$, $x_2$, \dots, there would be $x_1^R$, $x_1^G$, $x_1^B$, $x_2^R$, and so forth.
\subsubsection{DALL-E} \label{sec:dall-e}

In 2021, the initial version of OpenAI's DALL-E was launched, which introduced zero-shot text-to-image generation \cite{ramesh_zero-shot_2021}. Zero-shot refers to the ability to generate data from textual descriptions without any specific training on those exact combinations of text and images. In other words, DALL-E's training is based on a rich variety of image-text pairs, which enables it to generate new images from textual inputs, relying solely on this training. This issue closely relates to the one addressed by the present research.

The overall structure is easy to comprehend. It primarily comprises of two phases. In the initial phase, image representations or features are learned using a \ac{dVAE}. In this context, a \ac{dVAE} is an enhanced version of the classic \ac{VAE} (see Section~\ref{sec:vae}) that employs discrete latent variables in lieu of continuous ones. Discretizing the latent space allows for explicit control and manipulation of generated images in the synthesis process using \acp{dVAE}.

On the other hand, the second stage produces image representations from textual descriptions using a transformer model, as detailed in Section \ref{sec:transformers}.

Figure \ref{fig:dall-e} illustrates this macro-architecture.

\begin{figure}[ht]
    \centering
    \ctikzfig{figures/2-sota/dall-e}
    \caption[Dall-E macro architecture]{\textbf{Dall-E macro architecture} --- With a green background, one can see textual operations. The complex on the right is a \ac{dVAE}. Upon inference, the transformer's output is used as the latent feature values of the \ac{dVAE}.}
    \label{fig:dall-e}
\end{figure}

The \acf{dVAE} trained compresses images into a $32 \times 32$ grid of image tokens. Each token assumes one of 8192 possible values. This reduces the size of the image, improving the performance without significant degradation in visual quality.

At the second stage, the text is first encoded using \ac{BPE}. \Ac{BPE} is a compression technique for representing text using a new, single symbol to replace frequent pairs of characters or symbols. The objective of \ac{BPE} is to merge the most frequent pairs of consecutive symbols, such as letters or bytes, in the given text, until a specific number of merge operations is reached through an iterative process. As a result, this process generates a modified set of symbols that represents the original text but with a smaller vocabulary.

Applying \ac{BPE} enables effective capture of repeating patterns and combinations within the text while achieving better compression than traditional character-level encoding. This encoding step is vital in preparing textual input for subsequent stages and helps enable more efficient processing and modeling.

After the encoding, the resulting symbols are passed as input to a transformer trained to generate the tokens corresponding to the image tokens.

To generate completely new images, the process would be as follows:

\begin{enumerate}
	\item Encode the text into \acp{BPE}.
	\item Pass the encoded text to the trained transformer encoder; This outputs image tokens.
	\item Pass these tokens to the decoder of the \ac{dVAE} trained in stage 1; This outputs the image.
\end{enumerate}
\subsubsection{Fine-Tune Stable Diffusion for Spectrograms}

A promising avenue of research in the \ac{DL} community is the fine-tuning of existing generative models to improve their capabilities and produce high-quality audio output. This section introduces the concept of fine-tuning generative models for audio synthesis, and proposes fine-tuning the stable diffusion model to generate spectrograms.

The stable diffusion model uses the principles of diffusion processes to generate high-fidelity and diverse image samples (see Section\ \ref{sec:stable-diffusion}). One of the key advantages of the stable diffusion model is its ability to capture complex dependencies and generate realistic output.

It is worth noting that the stable diffusion model is an open source model, meaning that its code and implementation details are publicly available. This accessibility allows researchers and practitioners to study, modify, and build upon the foundations of the model. The availability of open source code for the stable diffusion model facilitates its fine-tuning for specific tasks, such as audio synthesis.

The fine-tuning of the stable diffusion model to spectrograms provides an exciting opportunity to explore the generation of high-quality audio output based on this visual representation.

One of the key advantages of the stable diffusion model is its ability to capture complex dependencies and generate realistic images. By fine-tuning the stable diffusion model on spectrograms, one can leverage its prior knowledge and adapt it to the specific characteristics of audio signals represented in the frequency and time domains.

The potential benefits of using the prior knowledge of the stable diffusion model for audio synthesis are manifold. First, the stable diffusion model has already shown impressive results in other domains, in this case image synthesis. By building on this foundation, one can exploit the model's ability to generate high-fidelity and diverse outputs, which can greatly improve the quality of the synthesized audio.

Furthermore, fine-tuning the stable diffusion model to spectrograms can provide a unique perspective on audio synthesis. By treating spectrograms as images and applying the stable diffusion model, one can explore the potential of generating audio based on this visual representation. This approach opens up new possibilities for manipulating and synthesizing audio in innovative ways.

To prove the usefulness of this method, Riffusion (see Section~\ref{sec:riffusion}), which already performs a similar task of generating audio from visual representations, has shown considerable results. By considering the insights and techniques used in Riffusion, one can build on its foundations and adapt the stable diffusion model accordingly.

It is important to note that the success of the proposed approach depends on the availability of a sufficiently large dataset for fine-tuning.

\subsubsection{GLIDE} \label{sec:glide}

The \acf{GLIDE} model is a state-of-the-art approach for generating high-fidelity synthetic images from free-form natural-language text prompts. The model is based on a guided diffusion-based approach, which is the first attempt by OpenAI at text-conditioned image generation using guided diffusion. The approach involves two types of guidance strategies during model training: classifier-free guidance~\cite{ho_classifier-free_2022}, which relies solely on the model's knowledge, and \ac{CLIP} guidance which uses a pre-trained \ac{CLIP} model~\cite{radford_learning_2021} to provide guidance based on caption matching.

The experimental results of the \ac{GLIDE} paper demonstrate several benefits of the proposed approach. Human evaluators preferred images generated with classifier-free guidance over \ac{CLIP} guidance regarding photorealism and caption similarity. Samples from the 3.5 billion parameter \ac{GLIDE} model were also found to outperform DALL-E (see Section~\ref{fig:dall-e}) samples according to human evaluations. Additionally, the model can perform text-driven image editing tasks beyond zero-shot image generation from text prompts. Text-driven image editing refers to editing existing images according to text prompts, such as changing attributes or objects within an image as directed by the text.

Despite its success, the \ac{GLIDE} model has some limitations. It fails to generate images for some complex or unusual text prompts. Moreover, the model's generation speed is slow, taking several seconds to generate one image on a flagship \ac{GPU}. Possible solutions to address these limitations include improving the model architecture, optimization techniques, and combining \ac{GLIDE} with faster \ac{GAN}-based methods.

\Acf{CLIP} refers to a model trained to determine if an image and text caption match. It consists of a transformer-based text encoder (see Section \ref{sec:transformers}) and a convolutional neural network-based image encoder (see Section \ref{sec:CNN}). The text encoder produces an embedding of the text, and the image encoder produces an embedding of the image. These embeddings are then compared, and during training, the model learns to produce similar embeddings for matching image-text pairs and dissimilar embeddings for mismatching pairs. This contrastive learning approach allowed CLIP to learn cross-modal understanding between text and images in an unsupervised manner. The pre-trained CLIP model can provide additional guidance to other text-to-image models, such as \ac{GLIDE}, by scoring how well-generated images match given text prompts.
\subsubsection{DALL-E 2} \label{sec:dall-e-2}

\textit{DALL-E 2} is a model proposed by researchers at OpenAI capable of generating images given a textual prompt~\cite{ramesh_hierarchical_2022}. This model can also modify given images, but this use case is not so interesting for the work present in this thesis.

The model consists of two blocks: the prior and the decoder. The prior converts captions into a lower-level representation, while the decoder turns this representation into an actual image.

They use \ac{CLIP} \cite{radford_learning_2021} and GLIDE (see Section~\ref{sec:glide}). For DALLE-2, the text is initially embedded using \ac{CLIP} embeddings. Then, the role of the prior is to translate these embeddings into embeddings related to an image and not the text itself. In other words, create an image representation with textual embeddings. For this, the researchers tried an \ac{AR} and a diffusion model. The diffusion one yielded better results (see Sections~\ref{sec:darn} and~\ref{sec:diffusion}). The decoder then takes the generated image representation and generates the image. The whole process can be seen in Figure \ref{fig:dall-e-2}.

\begin{figure}[ht]
    \centering
    \includegraphics[width=\textwidth]{figures/2-sota/dall-e-2.png}
    \caption[DALL-E 2 architecture]{\textbf{DALL-E 2 architecture} --- The image was taken from the original paper. Above the dotted line, the \ac{CLIP} training process is depicted, where given textual and image embeddings, the \ac{CLIP} learns to translate one into the other. Below the dotted line, a text-to-image generation process is represented: a text embedding is first fed to the model that produces the image embedding. Then, this embedding is used to condition the diffusion model GLIDE which produces a final image.}
    \label{fig:dall-e-2}
\end{figure}

A model is also possible without the prior by passing the textual embeddings directly to the decoder. However, while the results were okay, they were way better with the generated image embeddings.

The decoder creates $64 \times 64$ images, but another network learns to upsample images until $1024 \times 1024$. Without this, generating high-resolution images with the decoder would make the whole operation incredibly heavy.

A significant problem of this model (and others presented here proposed by big companies) is that it needs hundreds of millions of images and an incredible amount of computation power to perform well. This highlights the importance of research toward openly accessible models such as stable diffusion (see Section~\ref{sec:stable-diffusion}).