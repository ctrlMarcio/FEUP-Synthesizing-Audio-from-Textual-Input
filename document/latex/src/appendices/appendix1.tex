\chapter{Classification Model} \label{ann:classification}


This appendix presents the architecture and training process of the developed audio classification model.

\section{Model Architecture}

The ConvClassifier class defines the architecture of the one-dimensional \ac{CNN}. The following code snippet provides an overview of the ConvClassifier class:

\begin{lstlisting}[language=Python, caption={ConvClassifier class for sound classification}]
class ConvClassifier(nn.Module):
    def __init__(self):
        super().__init__()

        self.conv1 = nn.Conv1d(
            in_channels=1, out_channels=32, kernel_size=9, stride=1, padding=1)
        self.pool1 = nn.MaxPool1d(2, stride=2)

        self.conv2 = nn.Conv1d(32, 64, 9, stride=1, padding=1)
        self.pool2 = nn.MaxPool1d(2, stride=2)

        self.conv3 = nn.Conv1d(64, 128, 9, stride=1, padding=1)
        self.pool3 = nn.MaxPool1d(2, stride=2)

        self.conv4 = nn.Conv1d(128, 256, 9, stride=1, padding=1)
        self.pool4 = nn.MaxPool1d(2, stride=2)

        self.conv5 = nn.Conv1d(256, 512, 9, stride=1, padding=1)
        self.pool5 = nn.MaxPool1d(2, stride=2)

        self.gap = nn.AdaptiveAvgPool1d(1)

        self.linear1 = nn.Linear(512, 256)
        self.linear2 = nn.Linear(256, 128)
        self.linear3 = nn.Linear(128, 10)

        self.relu = nn.ReLU()

    def forward(self, x):
        # pass to float
        x = x.float()

        x = self.pool1(self.relu(self.conv1(x)))
        x = self.pool2(self.relu(self.conv2(x)))
        x = self.pool3(self.relu(self.conv3(x)))
        x = self.pool4(self.relu(self.conv4(x)))
        x = self.pool5(self.relu(self.conv5(x)))

        x = self.gap(x)

        x = x.view(-1, 512)

        x = self.relu(self.linear1(x))
        x = self.relu(self.linear2(x))
        x = self.linear3(x)

        return x
\end{lstlisting}

\section{Training Process}

The model is trained using the provided \texttt{train} function, which implements \ac{SGD} optimization and batch-wise backpropagation. The following code snippet outlines the training process:

\begin{lstlisting}[language=Python, caption={Training process for sound classification}]
def train(data_loader, model, loss_fn, optimizer):
    model.train()

    for batch, (sample, label, _) in enumerate(data_loader):
        sample, label = sample.to(device), label.to(device)

        # Compute prediction error
        pred = model(sample)
        loss = loss_fn(pred, label)

        # Backpropagation
        optimizer.zero_grad()
        loss.backward()
        optimizer.step()

        if batch % 100 == 0:
            loss, current = loss.item(), batch * len(sample)
            print(f"loss: {loss:>7f}  [{current:>5d}/{len(data_loader.dataset):>5d}]")

def test(data_loader, model, loss_fn):
    model.eval()

    test_loss, correct = 0, 0

    with torch.no_grad():
        for sample, label, _ in data_loader:
            sample, label = sample.to(device), label.to(device)

            pred = model(sample)
            test_loss += loss_fn(pred, label).item()

            correct += (pred.argmax(1) == label).type(
                torch.float).sum().item()
            
    test_loss /= len(data_loader)
    correct /= len(data_loader.dataset)

    print(f"Test Error: \n Accuracy: {(100*correct):>0.1f}%, Avg loss: {test_loss:>8f} \n")

epochs = 20

for epoch in range(epochs):
    print(f"Epoch {epoch + 1}\n-------------------------------")
    
    train(dataloader_train, model, loss_fn, optimizer)
    test(dataloader_test, model, loss_fn)
    print()

print("Done!")
\end{lstlisting}