% Define the introduction
\section{Introduction}

\subsection{Context}

\begin{frame}
    \frametitle{Overview of Computer Science and its Evolution}
    \begin{itemize}
        \item ``Computer Science is the study of computation and information.''~\cite{university_of_york_what_nodate}
        \item Evolution of Computer Science: From traditional programming to advanced Machine Learning (ML) and Deep Learning (DL) techniques.
        \item Importance of Big Data and Parallel Computing: Catalysts for advancements in ML and DL.
        \item Role of Deep Learning: Automated feature extraction, particularly effective in tasks like image and audio processing.
        \item Significance of Generative Models: Creating synthetic data for various applications.
    \end{itemize}
\end{frame}

\begin{frame}
    \frametitle{Timeline of Key Events in AI and Machine Learning}
    \begin{enumerate}
        \item \textbf{1958 - Birth of Modern AI:}
              \begin{itemize}
                  \item F. Rosenblatt proposes three fundamental questions leading to the development of the perceptron.
              \end{itemize}

        \item \textbf{1960s - Perceptron Convergence:}
              \begin{itemize}
                  \item Intensive work on convergence algorithms for the perceptron.
              \end{itemize}

        \item \textbf{1969 - Limitations of Perceptrons:}
              \begin{itemize}
                  \item Minksy and Papert demonstrate the limitations of perceptrons, leading to a slowdown in AI research.
              \end{itemize}

        \item \textbf{1980s - Emergence of Multilayer Neural Networks:}
              \begin{itemize}
                  \item Studies on learning under multilayer neural networks.
              \end{itemize}

        \item \textbf{1986 - Backpropagation:}
              \begin{itemize}
                  \item Rumelhart et al. describe backpropagation, a key learning procedure for neural networks.
              \end{itemize}

        \item \textbf{1990s - Second Winter of AI:}
              \begin{itemize}
                  \item Decreased investments in ML due to lack of real successes.
              \end{itemize}

        \item \textbf{Turn of the Millennium - Resurgence of ML:}
              \begin{itemize}
                  \item Emergence of three trends: Big Data, reduced cost of parallel computing, and interest in Deep Neural Networks (DNN).
              \end{itemize}

        \item \textbf{2010s - DL in Everyday Applications:}
              \begin{itemize}
                  \item DL becomes integral for various computer-made tasks, including text translation, recommender systems, and more.
              \end{itemize}

        \item \textbf{Generative Models and Their Impact:}
              \begin{itemize}
                  \item Introduction of generative models and their applications in image and text generation.
              \end{itemize}

        \item \textbf{2020s - Decade of Generative Applications:}
              \begin{itemize}
                  \item Rise of generative
              \end{itemize}
    \end{enumerate}
\end{frame}


\subsection{Background}

\begin{frame}
    \frametitle{Introduction to Background}
    % Briefly introduce the vast body of research on Deep Learning (DL).
\end{frame}

\begin{frame}
    \frametitle{Digital Audio Processing}
    % Explain how audio is represented digitally and how it can be processed.
\end{frame}

\begin{frame}
    \frametitle{Foundations for Enhancing Generative Models for Audio}
\end{frame}

\begin{frame}
    \frametitle{Deep Learning Frameworks}
\end{frame}

\begin{frame}
    \frametitle{State of the Art Generative Models}
\end{frame}

\subsection{Motivation}
% Explain what motivated you to conduct this research. What personal or professional experiences inspired your work?
% You can also use this slide to discuss your research goals. What did you hope to achieve by conducting this research?

\begin{frame}{Motivation}
    \begin{enumerate}
        \item \textbf{ML in Audio Processing}
              \begin{itemize}
                  \item ML techniques enhance sound synthesis, restoration, and speech recognition.
                  \item Learn complex patterns, improving quality and efficiency.
              \end{itemize}
        \item \textbf{Revolutionizing Sound Generation}
              \begin{itemize}
                  \item Integration of ML transforms sound creation and experience.
                  \item Opens creative avenues for artists, impacts industries like film, gaming, VR.
              \end{itemize}
        \item \textbf{Need for Further Research}
              \begin{itemize}
                  \item Urgency for studies in sound generation technologies.
                  \item This dissertation contributes significantly, offering resources for exploration.
              \end{itemize}
        \item \textbf{Impact and Contribution}
              \begin{itemize}
                  \item Reshaping human potential in sound creation through digital technologies.
                  \item Valuable resource for audio processing professionals, guiding future endeavors.
              \end{itemize}
        \item \textbf{Overall Significance}
              \begin{itemize}
                  \item Valuable contribution to audio processing and machine learning field.
              \end{itemize}
    \end{enumerate}
\end{frame}

\subsection{Research objectives}

\begin{frame}{Research Objectives}
    \begin{enumerate}
        \item Make a study of the current state-of-the-art deep learning architectures, focusing on generative ones.
        \item Examine prior algorithms that can process sound for augmentation, feature extraction, or other purposes.
        \item Make a study of the current state-of-the-art architectures used to develop sounds artificially.
        \item Develop end-to-end systems that can synthesize sound from any given text input, while accounting for hardware constraints and ensuring reliable performance.
        \item Evaluate the systems’ ability to generate a sound from the given textual input accurately.
    \end{enumerate}
\end{frame}