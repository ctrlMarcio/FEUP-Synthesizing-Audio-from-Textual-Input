\section{Conclusions}

\subsection{Overview and Reflections}

\begin{frame}
    \frametitle{Overview and Reflections}
    \begin{itemize}
        \item Comprehensive Study of State-of-the-Art Deep Learning Architectures for Audio Synthesis
        \item Development of End-to-End Systems for Sound Synthesis and Evaluation
        \item Challenges and Lessons Learned
    \end{itemize}
\end{frame}

\subsection{Future Directions}

\begin{frame}
    \frametitle{Future Directions}
    \begin{itemize}
        \item Exploring Novel Architectures
        \item Dataset Expansion
        \item Evaluation Metrics
    \end{itemize}
\end{frame}

\subsection{Novel Architectures}


\begin{frame}
    \frametitle{Fine-Tuning Stable Diffusion}

    \begin{itemize}
        \item Explore fine-tuning stable diffusion model for audio synthesis
        \item It is open source and accessible
        \item Adapt model to characteristics of audio signals in frequency-time domains
        \item Treat spectrograms as images
        \item Draw inspiration from successful Riffusion techniques for similar tasks
    \end{itemize}

    \note{
        \begin{itemize}
            \item Explore fine-tuning stable diffusion model for audio synthesis.
            \item Leverage its ability to capture complex dependencies and generate realistic output.
            \item Accessible open source code facilitates adaptation to specific tasks like audio synthesis.
            \item Adapt model to characteristics of audio signals in frequency-time domains.
            \item Potential to significantly enhance audio quality through high-fidelity, diverse output.
            \item Treat spectrograms as images, tapping into visual representation for innovative audio synthesis.
            \item Draw inspiration from successful Riffusion techniques for similar tasks.
            \item Note: Success hinges on availability of sufficiently large fine-tuning dataset.
        \end{itemize}
    }
\end{frame}

\begin{frame}
    \frametitle{Theoretical General Audio Transformer}

    \begin{itemize}
        \item Proposed audio transformer for generative audio synthesis.
        \item Optimized transformer architecture for audio data.
        \item Encoder-decoder framework with custom attention mechanisms.
        \item Incorporates textual prompts for context-appropriate audio synthesis.
        \item Decoder generates audio waveform sample-by-sample.
        \item Utilizes multi-head attention for natural-sounding outputs.
        \item Hyperparameters adjustable for performance and constraints.
    \end{itemize}

    \note{
        \begin{itemize}
            \item AT designed for high-quality, flexible audio generation.
            \item Encoder-decoder structure tailored for audio data.
            \item Encoder maps input to semantic representations.
            \item Decoder aligns audio with input text's meaning.
            \item Multi-head attention enables natural audio synthesis.
            \item Hyperparameters tunable for performance.
            \item Use of spectrograms discussed for input flexibility.
            \item Challenges and benefits of raw audio vs. spectrograms explained.
            \item Innovative Latent Feature Translator introduced.
            \item Training process outlined for unsupervised and supervised stages.
            \item Choice of transformers justified for audio modeling.
            \item AT limitations and avenues for future work discussed.
            \item Proposed AT distinguished from previous models.
        \end{itemize}
    }
\end{frame}

\begin{frame}
    \frametitle{VAMOS - Variable Audio Model for Sound Synthesis}

    \begin{itemize}
        \item VAMOS (Variable Audio Model for Sound Synthesis) extends DALL-E 2 to generate audio from text.
        \item Four interconnected models - CLAP, Text Encoder, Audio Encoder, VamGen.
        \item CLAP aligns text and audio in a common space.
        \item Text Encoder (BERT-based) provides semantic context.
        \item Audio Encoder (ResNet-based) extracts informative audio embeddings.
        \item VamGen utilizes diffusion process for audio synthesis.
        \item U-net architecture for faithful reconstruction during diffusion.
        \item VAMOS serves as a foundation for future advancements.
    \end{itemize}

    \note{
        \begin{itemize}
            \item VAMOS extends DALL-E 2 to audio synthesis from text.
            \item Four models: CLAP, Text Encoder, Audio Encoder, VamGen.
            \item CLAP aligns text and audio in a common space.
            \item Text Encoder (BERT-based) provides semantic context.
            \item Audio Encoder (ResNet-based) extracts audio embeddings.
            \item VamGen uses diffusion process for audio synthesis.
            \item U-net preserves audio fidelity during diffusion.
            \item VAMOS is a prototype, sets foundation for future work.
        \end{itemize}
    }
\end{frame}



\subsection{Conclusion}

\begin{frame}
    \frametitle{Overview of Research Goals}

\end{frame}

\begin{frame}
    \frametitle{Developing End-to-End Systems}

    \begin{itemize}
        \item Partial achievement in creating end-to-end systems for sound synthesis.
        \item Challenges due to limited datasets and model fine-tuning.
        \item Ongoing challenge in evaluating generated sound from text input.
    \end{itemize}

    \note{
        \begin{itemize}
            \item Partial achievement in creating end-to-end systems.
            \item Challenges: Limited datasets, model fine-tuning.
            \item Ongoing challenge in evaluating generated sound.
        \end{itemize}
    }
\end{frame}

\begin{frame}
    \frametitle{Reflection on the Research Process}

    \begin{itemize}
        \item Comprehensive methodology with literature review and iterative writing.
        \item Structured framework for research process.
        \item Challenges in developing AI models, lessons in patience and problem-solving.
    \end{itemize}

    \note{
        \begin{itemize}
            \item Comprehensive methodology with iterative writing.
            \item Structured framework for research.
            \item Challenges in developing AI models, lessons learned.
        \end{itemize}
    }
\end{frame}

\begin{frame}
    \frametitle{Conclusion}

    \begin{itemize}
        \item Primary objectives largely achieved with identified obstacles.
        \item Significant contribution to generative DL knowledge.
        \item Progress in creating end-to-end systems for sound synthesis.
    \end{itemize}

    \note{
        \begin{itemize}
            \item Objectives achieved with identified obstacles.
            \item Significant contribution to generative DL knowledge.
            \item Progress in creating end-to-end systems.
        \end{itemize}
    }
\end{frame}

