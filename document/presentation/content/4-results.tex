\section{Results}

\begin{frame}
    \frametitle{Results}

    \begin{itemize}
        \item Setup: Overview of experimental conditions and configurations.
        \item Presentation of Results: Showcase of outcomes from GANmix experiments.
        \item Discussion: Analyzing and interpreting the obtained results.
        \item Constraints and Challenges: Addressing limitations and difficulties encountered.
    \end{itemize}
\end{frame}


\subsection{Setup}

\begin{frame}
    \frametitle{Experimental Setup}

    \begin{itemize}
        \item GANmix model trained using BCE loss in PyTorch.
        \item Utilized two datasets: Audio MNIST and Clotho for diverse training.
        \item Three hardware setups: Kaggle, LIACC 1, LIACC 2, tailored for resources.
        \item Implementation: Python, PyTorch framework.
        \item Preprocessing: Randomly cropped samples to 5 seconds for diversity.
        \item Hyperparameters adjusted for batch size, epochs, and learning rates.
        \item Stopped training based on convergence for resource efficiency.
    \end{itemize}
\end{frame}

\subsection{Presentation of Results}

\begin{frame}
    \frametitle{Presentation of Results}

    \begin{itemize}
        \item Objectives: Explore generative AI models for audio production, assess performance.
        \item Context: Each experiment designed with specific research questions and hypotheses.
        \item Methodology: Hardware, software, and architecture details provided for transparency.
        \item Evaluation: Performance assessed through evolving loss plots and spectrograms.
        \item Systematic Organization: Ensures a comprehensive understanding of procedures and results.
        \item Basis for Analysis: Provides foundation for discussing effectiveness of generative AI models.
    \end{itemize}
\end{frame}

\subsection{Experiment X: Title} \label{sec:expX}

\begin{frame}
    \frametitle{Experiment X: Title}

    \begin{itemize}
        \item Objectives: [Objectives of the experiment]
        \item Model Details: [Describe key details of the model used, e.g., parameters, loss function]
        \item Dataset: [Mention the dataset used for training and evaluation]
        \item Optimizer and Learning Rate: [Specify the optimizer and learning rate used]
        \item Training Process: [Provide essential details about the training process, e.g., convergence status]
    \end{itemize}

    [Optional: Any unique aspects or considerations for this experiment]

    % \begin{figure}[!ht]
    %     \centering
    %     \begin{subfigure}{0.45\textwidth}
    %         \includegraphics[width=\textwidth]{figures/4.5-results/expX_loss.png}
    %         \caption{Evolving losses throughout the training process for Experiment X.}
    %         \label{fig:expX_loss}
    %     \end{subfigure}
    %     \begin{subfigure}{0.45\textwidth}
    %         \includegraphics[width=\textwidth]{figures/4.5-results/expX_spectrogram.png}
    %         \caption{Spectrogram generated in Experiment X.}
    %         \label{fig:expX_spectrogram}
    %     \end{subfigure}
    %     \caption{Results of Experiment X.}
    %     \label{fig:expX_results}
    % \end{figure}

    [Optional: Any additional insights or observations from this experiment]

\end{frame}

\subsection{Discussion}

\section{Analysis and Interpretation} \label{sec:res-analysis}

\begin{frame}
    \frametitle{Analysis and Interpretation}
    
    \begin{itemize}
        \item Identifying Trends
        \item Results for Future Investigation
        \item Interpretation of Results
        \item Conclusion
    \end{itemize}
    
\end{frame}

\begin{frame}
    \frametitle{Identifying Trends}

    \begin{itemize}
        \item Inverse correlation between generator and discriminator losses
        \item Convergence tends to plateau after a certain number of epochs
        \item Impact of learning rate on convergence speed
        \item Influence of optimization algorithms (e.g., SGD, RMSprop, Adam)
        \item Benefits of regularization methods (e.g., dropout, batch normalization, Gaussian noise)
        \item Importance of dataset size
        \item Exploration of latent space
    \end{itemize}
    
\end{frame}

\begin{frame}
    \frametitle{Results for Future Investigation}

    \begin{itemize}
        \item Occurrence of performance decline and NaN losses in certain experiments
        \item Further exploration of elastic network regularization
        \item Investigation of continuously increasing generator loss
    \end{itemize}
    
\end{frame}

\begin{frame}
    \frametitle{Interpretation of Results}

    \begin{itemize}
        \item Results didn't meet initial expectations but show potential
        \item Latent space exploration as a promising strategy
        \item Limitations of small datasets, especially in audio length and quantity
        \item Need for access to comprehensive datasets
        \item Computational resource challenges
    \end{itemize}
    
\end{frame}

\begin{frame}
    \frametitle{Conclusion}

    \begin{itemize}
        \item Analysis and interpretation of trends and patterns
        \item Potential for future advances in generative AI models for audio synthesis
        \item Lack of satisfactory practical results due to dataset limitations
        \item Importance of comprehensive datasets and computational resources
    \end{itemize}
    
\end{frame}

\subsection{Constraints and Challenges}

\section{Constraints and Challenges} \label{sec:res-limitations}

\begin{frame}
    \frametitle{Constraints and Challenges}
    
    \begin{itemize}
        \item Hardware Resources
        \item Data Quality and Quantity
        \item Hyperparameter Tuning
    \end{itemize}
    
\end{frame}

\begin{frame}
    \frametitle{Hardware Resources}

    \begin{itemize}
        \item Scarcity of hardware resources for training and evaluation
        \item Challenges in accessing sufficient computing power and memory
        \item Strategies adopted to optimize hardware usage
        \item Impacts, trade-offs, and opportunities resulting from resource limitations
    \end{itemize}
    
\end{frame}

\begin{frame}
    \frametitle{Data Quality and Quantity}

    \begin{itemize}
        \item Challenges posed by the quality and quantity of available data
        \item Importance of high-quality and diverse data for generative models
        \item Strategies employed to mitigate data limitations
        \item Considerations regarding data augmentation techniques
    \end{itemize}
    
\end{frame}

\begin{frame}
    \frametitle{Hyperparameter Tuning}

    \begin{itemize}
        \item Time constraints and challenges in hyperparameter tuning
        \item Significance of hyperparameters in model performance
        \item Impact of default or arbitrary values on model potential
        \item Recommendations for future work in hyperparameter optimization
    \end{itemize}
    
\end{frame}

\begin{frame}
    \frametitle{Conclusion}

    \begin{itemize}
        \item Discussion of major limitations and challenges faced in solution development
        \item Description of strategies employed to address these issues
        \item Possible implications, trade-offs, and opportunities arising from constraints
        \item Affirmation of the proposed solution's strengths and advancements in generative AI models for audio synthesis
    \end{itemize}
    
\end{frame}
