%% FEUP THESIS STYLE for LaTeX2e
%% how to use feupteses (English version)
%%
%% FEUP, JCL & JCF, 31 July 2012
%%
%% PLEASE send improvements to jlopes at fe.up.pt and to jcf at fe.up.pt
%%

%%========================================
%% Commands: pdflatex tese
%%           bibtex tese
%%           makeindex tese (only if creating an index)
%%           pdflatex tese
%% Alternative:
%%          latexmk -pdf tese.tex
%%========================================

%% 2021-07-20: One-sided output by default
\documentclass[11pt,a4paper]{report}
%% For two-sided printing (for dead-tree output) comment previous line
%% and uncomment the next line
%% \documentclass[11pt,a4paper,twoside,openright]{report}

%% For iso-8859-1 (latin1), comment next line and uncomment the second line
\usepackage[utf8]{inputenc}
%\usepackage[latin1]{inputenc}
% \usepackage[applemac]{inputenc} % para mac

%% English version

%% MEIC options
\usepackage[meic, backrefs]{src/extra/feupteses}
%\usepackage[meic,juri]{feupteses}
%\usepackage[meic,final]{feupteses}
%\usepackage[meic,final,onpaper]{feupteses}

%% Additional options for feupteses.sty:
%% - onpaper: links are not shown (for paper versions)
%% - backrefs: include back references from bibliography to citation place

%% Uncomment the next lines if side by side graphics used
% \usepackage[lofdepth,lotdepth]{subfig}
\usepackage{graphicx}
\usepackage{float}

% Listings
\definecolor{cloudwhite}{cmyk}{0,0,0,0.025}  % color

%% Include source-code listings package
\usepackage{listings}
\lstset{ %
 language=Python,                        % choose the language of the code
 basicstyle=\footnotesize\ttfamily,
 keywordstyle=\bfseries,
 numbers=left,                      % where to put the line-numbers
 numberstyle=\scriptsize\texttt,    % the size of the fonts that are used for the line-numbers
 stepnumber=1,                      % the step between two line-numbers. If it's 1 each line will be numbered
 numbersep=8pt,                     % how far the line-numbers are from the code
 frame=tb,
 float=htb,
 aboveskip=8mm,
 belowskip=4mm,
 backgroundcolor=\color{cloudwhite},
 showspaces=false,                  % show spaces adding particular underscores
 showstringspaces=false,            % underline spaces within strings
 showtabs=false,                    % show tabs within strings adding particular underscores
 tabsize=2,                         % sets default tabsize to 2 spaces
 captionpos=b,                      % sets the caption-position to bottom
 breaklines=true,                   % sets automatic line breaking
 breakatwhitespace=false,           % sets if automatic breaks should only happen at whitespace
 escapeinside={\%*}{*)},            % if you want to add a comment within your code
 morekeywords={*,var,template,new}  % if you want to add more keywords to the set
}

\definecolor{salmon}{HTML}{ff8c69}
\definecolor{input}{HTML}{086faf}
\definecolor{output}{HTML}{0fabff}
\definecolor{neuron}{HTML}{07b785}

%% Imported packages
\usepackage{comment} % to comment things
\usepackage{parskip}
%\usepackage[color=salmon, textcolor=white, bordercolor=salmon, textsize=small, disable]{todonotes}
\usepackage{minitoc}
\setcounter{minitocdepth}{6}
\usepackage[]{acronym}
\usepackage{pgfgantt}
\usepackage{tikz}
\usetikzlibrary{backgrounds}
\usetikzlibrary{arrows}
\usetikzlibrary{shapes,shapes.geometric,shapes.misc}
\usetikzlibrary{fadings}
\usetikzlibrary{3d,decorations.text,shapes.arrows,positioning,fit}

% this style is applied by default to any tikzpicture included via \tikzfig
\tikzstyle{tikzfig}=[baseline=-0.25em,scale=0.5]

% standard layers used in .tikz files
\pgfdeclarelayer{edgelayer}
\pgfdeclarelayer{nodelayer}
\pgfsetlayers{background,edgelayer,nodelayer,main}

% style for blank nodes
\tikzstyle{none}=[inner sep=0mm]

% include a .tikz file
\newcommand{\tikzfig}[1]{%
{\tikzstyle{every picture}=[tikzfig]
\IfFileExists{#1.tikz}
  {\input{#1.tikz}}
  {%
    \IfFileExists{./figures/#1.tikz}
      {\input{./figures/#1.tikz}}
      {\tikz[baseline=-0.5em]{\node[draw=red,font=\color{red},fill=red!10!white] {\textit{#1}};}}%
  }}%
}

% the same as \tikzfig, but in a {center} environment
\newcommand{\ctikzfig}[1]{%
\begin{center}\rm
  \tikzfig{#1}
\end{center}}

% fix strange self-loops, which are PGF/TikZ default
\tikzstyle{every loop}=[]

% TiKZ style file generated by TikZiT. You may edit this file manually,
% but some things (e.g. comments) may be overwritten. To be readable in
% TikZiT, the only non-comment lines must be of the form:
% \tikzstyle{NAME}=[PROPERTY LIST]

% Node styles
\tikzstyle{input}=[draw={rgb,255: red,8; green,111; blue,175}, ultra thick, shape=circle, minimum size=1cm]
\tikzstyle{neuron}=[ultra thick, shape=circle, minimum size=1cm, draw={rgb,255: red,7; green,183; blue,133}]
\tikzstyle{output}=[draw={rgb,255: red,15; green,171; blue,255}, ultra thick, shape=circle, minimum size=1cm]
\tikzstyle{etc}=[fill=black, draw=none, shape=circle, scale=0.2, text height=0.333cm]
\tikzstyle{main rect}=[fill={rgb,255: red,197; green,255; blue,190}, draw=none, shape=rectangle]
\tikzstyle{blue square}=[fill={rgb,255: red,152; green,197; blue,255}, draw={rgb,255: red,84; green,110; blue,141}, shape=rectangle, minimum width=1cm, minimum height=1cm]
\tikzstyle{paint gray}=[fill={rgb,255: red,7; green,183; blue,133}, fill opacity=0.1, draw=black]
\tikzstyle{transformer rect}=[fill={rgb,255: red,244; green,244; blue,244}, draw=none, shape=rectangle, minimum width=3cm, minimum height=6cm, rounded corners=10]

% Edge styles
\tikzstyle{->input}=[draw={rgb,255: red,8; green,111; blue,175}, ->, thick]
\tikzstyle{input->neuron}=[draw={rgb,255: red,8; green,111; blue,175}, ->, thick]
\tikzstyle{neuron->neuron}=[draw={rgb,255: red,7; green,183; blue,133}, ->, thick]
\tikzstyle{neuron->output}=[draw={rgb,255: red,15; green,171; blue,255}, ->, thick]
\tikzstyle{new edge style 0}=[-, draw={rgb,255: red,114; green,67; blue,255}]
\tikzstyle{arrow}=[->, thin]
\tikzstyle{neuron-neuron}=[-, draw={rgb,255: red,7; green,183; blue,133}, thick]
\tikzstyle{new edge style 1}=[-, fill={rgb,255: red,213; green,213; blue,213}, draw=none]



% For the CNN

\tikzset{pics/fake box/.style args={% #1=color, #2=x dimension, #3=y dimension, #4=z dimension
#1 with dimensions #2 and #3 and #4}{
code={
\draw[gray,ultra thin,fill=#1]  (0,0,0) coordinate(-front-bottom-left) to
++ (0,#3,0) coordinate(-front-top-right) --++
(#2,0,0) coordinate(-front-top-right) --++ (0,-#3,0) 
coordinate(-front-bottom-right) -- cycle;
\draw[gray,ultra thin,fill=#1] (0,#3,0)  --++ 
 (0,0,#4) coordinate(-back-top-left) --++ (#2,0,0) 
 coordinate(-back-top-right) --++ (0,0,-#4)  -- cycle;
\draw[gray,ultra thin,fill=#1!80!black] (#2,0,0) --++ (0,0,#4) coordinate(-back-bottom-right)
--++ (0,#3,0) --++ (0,0,-#4) -- cycle;
\path[gray,decorate,decoration={text effects along path,text={CNN}}] (#2/2,{2+(#3-2)/2},0) -- (#2/2,0,0);
}
}}
% from https://tex.stackexchange.com/a/52856/121799
\tikzset{circle dotted/.style={dash pattern=on .05mm off 2mm,
                                         line cap=round}}

\usepackage{amsmath} % for equations with multiple conditions
\usepackage{nicematrix} % for matrices
\usepackage{algorithm}
\usepackage{algpseudocode}
\usepackage{tabularx}
\usepackage{caption}
\usepackage{subcaption}
\usepackage[page,toc,titletoc,title]{appendix}
%\usepackage{tocloft}
\usepackage{blindtext}


%% Uncomment to create an index (at the end of the document)
\makeindex

%% Path to the figures directory
%% TIP: use folder ``figures'' to keep all your figures
\graphicspath{{figures/}}

%%========================================
%% Start of document
%%========================================
\begin{document}

%%----------------------------------------
%% TIP: if you want to define more macros, use an external file to keep them
\input{src/extra/mymacros}
%%----------------------------------------

%%----------------------------------------
%% Information about the work
%%----------------------------------------
\title{Synthesizing Audio from Textual Input \\
    \Large Development and Comparison of Generative AI Models}
\author{Márcio Duarte}

%% Comment next line if not necessary for degree
%\degree{Programa Doutoral em Engenharia Informática}

%% Uncomment next line for date of submission
\thesisdate{September 1st, 2023}

%% Comment next line copyright text if not used
\copyrightnotice{Márcio Duarte, 2023}

\supervisor{Supervisor}{Luís Paulo Reis}
%% Uncomment next line if necessary
\supervisor{Second Supervisor}{Gilberto Bernardes}

%% Uncomment committee stuff in the final version if used
%\committeetext{Approved by \ldots:}
%\committeemember{President}{Name of the President}
%\committeemember{Referee}{Name of the Referee}
%\committeemember{Referee}{Name of the Referee}

%% Uncomment signature line in the final on paper version if used
%\signature

%% Specify cover logo (in folder ``figures'')
\logo{extras/uporto-feup.pdf}

%% Uncomment next line for additional text below the author's name (front page)
%\additionalfronttext{Preparação da Dissertação}

%\maketitle

%%----------------------------------------
%% Preliminary materials
%%----------------------------------------

\begin{Prolog}
    \dominitoc
    
    \chapter*{Resumo}
%\addcontentsline{toc}{chapter}{Resumo}

Hoje em dia, o áudio é um elemento essencial na maioria do conteúdo produzido online. Normalmente, é trabalho manual que está por detrás dos terabytes de áudio publicados diariamente. Se algum produtor desejar um som específico, tem de o pesquisar em bases de dados online, sintetizá-lo ou até gravá-lo. Essa quantidade de trabalho é uma restrição à criação de conteúdo, principalmente se esses sons forem elaborados ou muito específicos.

Os ficheiros de áudio apresentam apenas uma dimensão, \textit{i.e.}, a amplitude da sua onda sonora recolhida num determinado intervalo de tempo. Em comparação, os ficheiros de imagem apresentam três dimensões de dados. Independentemente, as dependências temporais constituem um desafio no som, uma vez que são mais complexas e intrincadas do que as das imagens. Por exemplo, existe a expetativa de que o timbre de um determinado instrumento ou o ruído de fundo de um determinado ambiente sonoro se mantenha ao longo do tempo.

Esta dissertação apresenta uma investigação exaustiva de modelos avançados de IA generativa. Sugere alguns modelos e executa outros. Redirecciona a sua atenção para um modelo baseado em GANs que opera no espaço latente. Ao contrário de sistemas anteriores que limitam a síntese de áudio a domínios específicos, como a música ou a fala, o sistema proposto ultrapassa estas restrições, gerando áudio a partir de qualquer pedido textual. Outro foco desta dissertação é o desenvolvimento de diversas estruturas de modelos de deep learning, todas baseadas em abordagens modernas, bem como os desafios que surgem quando se trabalha com hardware limitado.

Dado que já existe trabalho significativo com modelos que criam imagens ou realizam a conversão de texto em fala, a expetativa é que um modelo generativo de áudio de ponta produza resultados satisfatórios. Os resultados deste trabalho terão impacto na velocidade do processo de produção para criadores de conteúdo, engenheiros de som e todos os interessados na criação de produtos sonoros.

\chapter*{Abstract}
%\addcontentsline{toc}{chapter}{Abstract}

Nowadays, audio is a fundamental aspect of most online content. Generating terabytes of audio content daily relies heavily on manual labor. Content creators often find themselves tackling arduous tasks such as researching sounds from online databases, synthesizing complex audio, or even resorting to self-recording. This substantial workload poses a significant hurdle to content creation, particularly when dealing with intricate or highly specific sounds.

Unlike image files, which present three-dimensional data dimensions and are utilized by mainstream generative models, audio files are restricted to a single dimension, representing the amplitude of sound waves at specific time intervals. However, the challenge of long-term dependencies in audio is more complex than in images. Instrument timbre and persistent background noise are factors that further complicate the audio synthesis process.

This dissertation presents an extensive investigation of advanced generative AI models. It suggests certain models and carries out others. It redirects its attention to a model based on GANs that operates within the latent space. Unlike previous systems that limit audio synthesis to specific domains such as music or speech, the proposed system overcomes these constraints by generating audio from any textual prompt. Another focus of this dissertation is the development of varied deep learning model structures, all based on cutting-edge methods, as well as the challenges that arise when working with limited hardware.

The main contribution of this work focuses on investigating cutting-edge generative AI models, as well as developing and enhancing generative AI frameworks customized for audio synthesis. This involves thoroughly exploring model components, training strategies, and performance benchmarks. This research bridges a critical gap in the current AI landscape by allowing the creation of diverse and contextually rich audio content from textual cues. It not only demonstrates the capabilities of AI-powered audio synthesis, but also offers a useful resource for researchers and professionals exploring this growing field.

 % the abstract
    \chapter*{Acknowledgements}
%\addcontentsline{toc}{chapter}{Agradecimentos}

In embarking on this thesis development journey, I am grateful for the support and guidance provided by many individuals and institutions. Their contributions played pivotal roles in shaping the trajectory of this work.

I am grateful to my distinguished thesis supervisors, Luís Paulo Reis and Gilberto Bernardes. Their dedicated mentorship and guidance have been pivotal in guiding this research. Their expertise and encouragement have enlightened every step of this journey, and for that, I am profoundly appreciative.

The Artificial Intelligence and Computer Science Laboratory (LIACC) deserves thanks for the necessary server infrastructure for my project's development. 

The academic atmosphere at the Department of Informatics and FEUP (Faculdade de Engenharia da Universidade do Porto) contributed to my intellectual curiosity and academic growth. The institution's commitment to excellence supported my development as a student.

I am grateful for the support of my family, as well as the contributions of individuals, institutions, and communities involved in this endeavor. To my mother, father, and brother, Fábio, I appreciate their support. I pay tribute to my beloved cat, Kikka, whose presence has brought joy to my life and serves as an inspiration for my work. I am grateful to my close friends for their enduring support throughout this journey.

I humbly acknowledge the larger AI community as a critical driving force behind technological advancement. The collaborative effort in this community has been the primary driving force behind the advancements that will propel us into the future.

I appreciate the music world, as it has been a crucial aspect of my identity, in addition to my pursuit of computer science. Music stimulates emotions, demonstrates complex patterns, and shapes our perception of the world, which has motivated me during my journey. Being a musician, I find parallels between harmonies in melodies and algorithms. Furthermore, I express gratitude to the institutions and communities that uphold musical expression. Their dedication to the art of sound has enriched my own creative ventures and surely influenced how I explore sound in this thesis.

This journey demonstrates the effectiveness of collaboration, shared contributions, and the synergistic relationship between disciplines. This work testifies to our joint pursuit of knowledge and innovation, and I am grateful to have contributed to this exceptional and harmonious effort.

\vspace{10mm}
\flushleft{Márcio Duarte}
  % the acknowledgments
    \cleardoublepage
\thispagestyle{plain}

\vspace*{8cm}

\begin{flushright}
   \textsl{``To go wrong in one's own way\\
   is better than to go right in someone else’s.''} \\
\vspace*{1cm}
           Fyodor Mikhailovich Dostoyevsky
\end{flushright}
    % initial quotation if desired
    \cleardoublepage

    \pdfbookmark[0]{Table of Contents}{contents}
    \tableofcontents
    \cleardoublepage
    
    \pdfbookmark[0]{List of Figures}{figures}
    \listoffigures
    \cleardoublepage
    
    \pdfbookmark[0]{List of Tables}{tables}
    \listoftables 
    \chapter*{Abbreviations and Symbols}
%\addcontentsline{toc}{chapter}{Abbreviations}
\chaptermark{ABBREVIATIONS AND SYMBOLS}

\begin{acronym}
    \acro{AE}{autoencoder}
    \acro{AI}{artificial intelligence}
    \acro{API}{Application Programming Interface}
    \acro{AR}{autoregressive}
    \acro{AT}{Audio Transformer}
    \acro{BCE}{binary cross-entropy}
    \acro{BPE}{byte pair encoding}
    \acro{CLIP}{Contrastive Language–Image Pre-training}
    \acro{CNN}{convolutional neural network}
    \acro{CPU}{Central Processing Unit}
    \acro{CSS}{concatenative sound synthesis}
    \acro{DARN}{deep autoregressive network}
    \acro{DCGAN}{deep convolutional generative adversarial network}
    %\acro{DCNN}{deconvolutional neural network}
    \acro{DFT}{Discrete-Fourier-Transform}
    \acro{DL}{deep learning}
    \acro{DNN}{deep neural network}
    \acro{dVAE}{discrete variational autoencoder}
    \acro{ELBO}{evidence lower bound}
    \acro{GAN}{generative adversarial network}
    \acro{GB}{gigabyte}
    \acro{GLIDE}{Guided Language to Image Diffusion for Generation and Editing}
    \acro{GPU}{graphics processing unit}
    \acro{GRU}{gated recurrent unit}
    \acro{Hz}{Hertz}
    \acro{KL}{Kullback–Leibler}
    \acro{LIACC}{Artificial Intelligence and Computer Science Laboratory}
    \acro{LLM}{Large Language Model}
    \acro{LSTM}{long short-term memory}
    \acro{MAD}{mean absolute deviation}
    \acro{MAE}{mean absolute error}
    \acro{ML}{machine learning}
    \acro{MOS}{mean opinion score}
    \acro{MPD}{multi-period discrimintor}
    \acro{MS-VQ-VAE}{multi-scale vector quantised variational autoencoder}
    \acro{MSD}{multi-scale discriminator}
    \acro{MSE}{mean squared error}
    \acro{NaN}{not a number}
    \acro{NLP}{natural language processing}
    \acro{RAM}{Random Access Memory}
    \acro{ReLU}{rectified linear unit}
    \acro{RNN}{recurrent neural network}
    \acro{RVQ}{residual vector quantizer}
    \acro{SGD}{stochastic gradient descent}
    \acro{STFT}{Short-Time Fourier Transform}
    \acro{tanh}{hyperbolic tangent}
    \acro{TTS}{text-to-speech}
    \acro{VAE}{variational autoencoder}
    \acro{VQ}{vector quantized}
    \acro{VQ-VAE}{vector quantized variational autoencoder}
    \acro{XOR}{exclusive OR}
\end{acronym}

  % the list of abbreviations used 

    %\listoftodos
\end{Prolog}

%%----------------------------------------
%% Body
%%----------------------------------------
\StartBody

\chapter{Introduction} \label{chap:intro}

\minitoc

Audio is essential for shaping human experiences and interactions in the digital age. Audio content, ranging from podcasts and music to sound effects and immersive environments, is pervasive in our lives and enhances multimedia experiences. However, generating high-quality, diverse, and contextually relevant audio is still a time-consuming and labor-intensive process. As the demand for audio content continues to increase, there is a need for innovative solutions that can simplify the process and enable creators to generate various audio content with less effort.

Recent advances in \ac{AI} and \ac{DL} are revolutionizing different domains, such as image generation and text-to-speech synthesis. These cutting-edge models exhibit exceptional abilities to generate high-quality content from basic textual inputs. This thesis is motivated by these successful models and explores the potential of \ac{AI}-driven generative models to synthesize audio snippets based on textual descriptions. This thesis is not constrained to specific domains, like music or speech, but instead studies and implements systems suitable for a wide range of audio prompts.

The motivation behind this work lies in the potential benefits that an end-to-end audio generative model can offer content creators, sound engineers, and other stakeholders in the audio production ecosystem. By reducing the time and effort required to generate unique and high-quality audio samples, this research aims to contribute to the democratization of audio content creation and facilitate new avenues for creative expression.

This chapter provides an introduction to the research study and sets the stage for the exploration of the topic. It aims to supply a comprehensive overview of the present study's context, motivation, objectives, and structure.

The context information can be found in section \ref{sec:context}. Its focus is to provide the necessary background for the research study and to explain its importance. Section \ref{sec:motivation} explores the motivation behind the study. The purpose of this section is to explain the necessity of the research and the knowledge gap it aims to address. Next, in section \ref{sec:objectives}, the research study's objectives are described. This section aims to explain the specific objectives and outcomes the research seeks to accomplish. The structure of the dissertation is presented in Section \ref{sec:struct}. This section offers an organized overview of the remaining chapters.

\section{Context} \label{sec:context}

According to the University of York \cite{university_of_york_what_nodate}, ``Computer Science is the study of computation and information''. In practice, engineers and computer scientists typically create programs that machines execute. Traditionally, these programs consist of a sequence of instructions for the computer to follow.

Currently, processing and analyzing vast datasets is imperative for most endeavors. Computer science enables the possibility of these commodities, ranging from running hospitals to creating songs for streaming platforms.

Currently, most endeavors require processing and analyzing vast datasets. These commodities are only possible with computer science, from running hospitals to creating the songs one listens to on a streaming platform.

This amount of data has given rise to a new type of application. They do not need to have these instructions wired in. Instead, algorithms learn these from data. To this end, we call this \acf{ML}.

Since computers were invented, the scientific community has wondered whether they can learn~\cite{mitchell_machine_1997}. \ac{ML} is a field devoted to understanding and building methods that let machines ``learn'' – that is, methods that leverage data to improve computer performance on some set of tasks~\cite{alpaydin_introduction_2020}. Modern \ac{AI} was born in 1958 when F. Rosenblatt \cite{rosenblatt_perceptron_1958}, in an attempt to understand the capability of higher organisms for perceptual recognition, generalization, recall, and eventually thinking, proposed three fundamental questions:

\begin{enumerate}
	\item ``How is information about the physical world sensed, or detected, by the biological system?''
	\item ``In what form is information stored, or remembered?''
	\item ``How does information contained in storage, or in memory, influence recognition and behavior?''
\end{enumerate}

With this, Rosenblatt theorized a mathematical system called the perceptron (explained in Section \ref{sec:feedforward} that, by following the supposed behavior of neurons in one's nervous system, was the central piece of a hypothetical system capable of answering these questions.

Then, during the 1960s, much work was put into convergence algorithms for the perceptron and models based on it. Both deterministic and stochastic methods were proposed \cite{fradkov_early_2020}. However, in 1969, Minksy and Papert \cite{marvin_minsky_perceptrons_1969} published a book demonstrating the limitations of perceptrons. Namely, the authors showed that perceptrons could only represent linear functions, and simple non-linear functions such as \ac{XOR} were impossible. As a result, the study of \ac{AI} was mainly halted until the 1980s. This period is usually called \textit{the first winter of \ac{AI}} \cite{fradkov_early_2020}.

During the 1980s, studies of learning under multilayer neural networks went underway \cite{fradkov_early_2020}. In 1986, Rumelhart et al. \cite{rumelhart_learning_1986} described a new learning procedure for networks of neurons. The procedure adjusts the weights of the network's connections to minimize a measure of the difference between the expected and the actual output, an error function. This method is still used nowadays and is called \textit{backpropagation}.

Fradkov et al.~\cite{fradkov_early_2020} argue that an intensive advertisement of the success of backpropagation and other computational advances produced great hope for future successes. However, real successes were not happening, and investments in \ac{ML} decreased again in the early 1990s. This period is called \textit{the second winter of \ac{AI}}~\cite{martinez_artificial_2019}.

The turn of the millennium saw a new rise in \ac{ML} technologies; this time, it was, until now, for good. According to Fradkov et al.~\cite{fradkov_early_2020}, this was due to three trends that emerged:

\begin{enumerate}
	\item The appearance of big data. Dealing with huge amounts of data is an interest not only to a small portion of scientists but to the whole market.
	\item Reduced cost of parallel computing with both software (with, for instance, Google's MapReduce \cite{dean_mapreduce_2004}) and hardware (with an investment in specialized hardware for \ac{ML} from companies such as NVidia).
	\item A newfound interest by scientists in new, more complex, \ac{ML} algorithms, denominated \textit{\acfp{DNN}}.
\end{enumerate}

The critical idea of this new \ac{ML} is that it can infer plausible models to explain the observed data. A machine can use such models to make predictions about future data and make rational decisions based on these predictions \cite{ghahramani_probabilistic_2015}. It involves training a model on a large dataset to learn patterns and relationships in the data and then make predictions or decisions based on those patterns. For instance, a computer program can learn from medical records which treatments are most effective against new diseases, or houses can learn from experience to optimize energy costs based on the particular usage patterns of their occupants.

In traditional \ac{ML}, the features input into the models were usually hand-picked by humans, which leads to errors. New models learn intermediate representations— a vector of features — from data. The model itself usually performs this feature extraction with more layers~\cite{goodfellow_deep_2016}. Hence, \acf{DL}. \Ac{DL} algorithms consist of multiple layers of interconnected nodes and are trained to learn complex patterns and relationships in the data. \Ac{DL} algorithms can automatically learn and extract features from data. They are particularly well-suited for tasks such as image and audio processing.

The use of \ac{DL} in everyday applications increased during the 2010s. Nowadays, \ac{DL} is relied upon for various computer-made tasks including text translation, recommender systems, fake news detection, spam filtering, image captioning, and even self-driving cars~\cite{dean_golden_2022}.  Generative models are a key tool for creating new data.

Generative models, which are a type of \ac{DL} model, can create synthetic data that is similar to a given training dataset. Generative models can produce new data that conforms to the distribution learned from the training data. In contrast to common learning objectives such as classification or regression tasks, which focus on labeling inputs or estimating mappings, generative models aim to replicate and capture hidden statistics in observed data~\cite{huzaifah_deep_2021}.

The scientific community turned its heads to generative models a few years ago. These models allow a variety of new applications and products. For instance, DALL-E~\cite{ramesh_zero-shot_2021} and DALL-E 2~\cite{ramesh_hierarchical_2022} allow the generation of images given any textual input (see Sections \ref{sec:dall-e} and~\ref{sec:dall-e-2}. GPT-3 is an extensive language model \cite{brown_language_2020} that powers applications such as ChatGPT, capable of generating text. Modern text-to-speech (as seen in Section~\ref{sec:tts} applications also rely on these technologies.

Because of the relevance and effectiveness of these new generative technologies, \ac{DL} has achieved mainstream status, and the general public is aware of the capabilities of these algorithms. Given the rise of \ac{AI} in the 2010s with predictions and classifications, it is plausible that the 2020s will be a decade of generative applications. Automating manual work to enhance human creativity has never been more feasible. Generative applications span diverse domains such as images, text, and audio. However, even in the context of a well-defined frame of reference and optimal individual categorization, it's important to recognize that models inevitably involve reduction to averages. This raises concerns about undesirable convergence and oversimplification of media~\cite{forero_j_are_2023}.

For the purpose of this thesis, audio is broadly classified into three main categories: music, speech, and soundscapes. Music consists of organized tones and rhythms created by humans or other living entities~\cite{oxford_english_dictionary_music_2023}. Speech encompasses all vocalizations produced by humans, which are used for communication and expression~\cite{holden_origin_2004}. Soundscapes refer to an acoustic environment that is experienced and understood by individuals in context, including natural and human-made sounds~~\cite{international_organization_for_standardization_iso_2014, schafer_tuning_1977}. These three categories encompass most of the sounds people encounter.

There are two types of soundscapes: those found in the physical world and capable of being recorded, and those that can be artificially created through methods like mathematical equations. When it comes to the latter, one can imagine sounds such as white or brown noise - which are simply repetitive mathematical patterns that can be generated with ease through computation. To the best of the author's knowledge, all remaining sounds were either recorded or generated through sampling or other creative techniques, such as capturing the behavior of vibrations~\cite{trautmann_classical_2003}.

Using neural networks, deep generative models can produce audio from parameters. These models learn hidden data patterns to create new samples that match the training data's distribution \cite{huzaifah_deep_2021}.

Despite significant advances in generative technologies, research on audio synthesis has fallen behind. Although image generation has achieved impressive levels of realism and text generation is capable of passing medical exams~\cite{strong_chatbot_2023}, differentiating when a \ac{DL} process generates a specific sound is still relatively easy. Most of the sound-related research is focused on \ac{TTS} applications, which still need further improvement. Recent developments suggest a changing landscape.

In 2023, companies have significantly increased their investments in advancing the audio synthesis capabilities for music, voice, and soundscapes. The renewed focus aims to close the gap between state-of-the-art image generation models and general sound synthesis tools. The objective is to develop advanced algorithms capable of creating highly realistic audio outputs in various domains.
\section{Motivation} \label{sec:motivation}

In recent years, there has been a significant increase in the use of \ac{ML} techniques for audio processing tasks, such as sound synthesis, audio restoration, and speech recognition. The ability of \ac{ML} algorithms to learn and extract complex patterns from large datasets has shown promising results in improving the quality and efficiency of audio processing tasks.

Furthermore, integrating \ac{ML} techniques in sound generation technologies can revolutionize how one creates and experiences sound. It can provide new avenues for artists and musicians to explore their creativity and produce unique and innovative audio content. It can also offer new possibilities for sound design in various industries, such as film, gaming, and virtual reality.

The current need for studies in sound generation technologies highlights the need for further research and development. This dissertation endeavors to establish itself as a significant study in this field. It offers high-quality resources to researchers and developers to investigate the potential and limitations of \ac{ML} techniques for sound synthesis.

In today's world, digital technologies are reshaping our relationship with music and sound by enabling innovative capabilities \cite{tahiroglu_-terity_2020}. This study strives to investigate and broaden this impact by offering a novel tool that bolsters human potential in sound creation. Additionally, the findings can be a valuable resource for audio processing researchers, developers, and practitioners. The research findings offer guidance for future research and development endeavors concerning sound generation technologies and provide valuable insights into the most effective practices and techniques for utilizing \ac{ML} in audio processing.

Overall, this study's significance and potential impact make it a worthwhile and valuable contribution to the field of audio processing and machine learning.
\section{Objectives} \label{sec:objectives}

To address the core motivations behind this research, this dissertation aims to undertake a comprehensive study of \ac{DL} and, more specifically, generative deep learning models in the context of sound synthesis.

The goals include conducting a comprehensive survey of existing \ac{DL} architectures for audio generation while analyzing their strengths and limitations. In addition, novel approaches will be proposed and developed to further advance the field.

By pursuing these revised objectives, this research aims to provide valuable insights into state-of-the-art in sound synthesis using \ac{DL} methods. Furthermore, it aims to provide practical guidance for future advances in creating high-quality audio outputs based on textual inputs.

In order to accomplish this implementation, some specific goals are set:

\begin{enumerate}
	\item Make a study of the current state-of-the-art deep learning architectures, focusing on generative ones.
	\item Examine prior algorithms that can process sound for augmentation, feature extraction, or other purposes.
	\item Make a study of the current state-of-the-art architectures used to develop sounds artificially.
	\item Develop end-to-end systems that can synthesize sound from any given text input, while accounting for hardware constraints and ensuring reliable performance.
	\item Evaluate the systems' ability to generate a sound from the given textual input accurately.
\end{enumerate}
\section{Dissertation Structure} \label{sec:struct}

The present dissertation commences with a comprehensive examination of the current state-of-the-art technologies pertaining to sound generation. The analysis encompasses sound generation and delves into the realm of deep learning and generative deep learning architectures, which are not limited to sound. Subsequently, the dissertation examines additional tools, such as data augmentation for sound and sound analysis. Then, the focus shifts to a more in-depth study of generative deep learning technologies specific to sound, including vocoders, end-to-end tools, and other related terms, which are thoroughly explained.

The following section of the dissertation focuses on formulating and defining the problem at hand. Subsequently, practical applications of the developed technologies are demonstrated, and the dissertation investigates the various decisions and consequences that arise in developing such a system.

The methodology and approach adopted to fulfill the thesis's objectives are outlined in the solution section. An overview of the work carried out, its results, and the work plan is presented in detail.

Finally, the dissertation concludes by assessing the extent to which the objectives proposed in the introduction have been met and by presenting a summary of the findings. To facilitate navigation and ease of reference, each chapter of the dissertation includes a table of contents.

\chapter{Overview of the State-of-the-Art Techniques} \label{chap:sota}

\minitoc

This chapter provides an objective review of the pertinent literature on \ac{DL} and audio processing. Its goal is to offer readers a comprehensive understanding of the current state of knowledge in the field and its evolution. Recent advancements in tools for sound comprehension and generation are enhancing existing models, enabling the modeling of new sounds more efficiently and rapidly~\cite{tahiroglu_-terity_2020}. Specifically, this chapter plays the following essential roles:

\begin{enumerate}
    \item To establish the context for the research problem: By reviewing the existing literature, the chapter sets the stage for the research problem and provides a basis for understanding its significance and importance.
    \item To identify gaps in the literature: The chapter helps to identify areas where further research is needed, as well as potential opportunities for contribution.
    \item To provide a foundation for the research design: The chapter helps to inform the design of the research study by highlighting previous research and its limitations.
    \item To demonstrate the originality of the research: By reviewing the existing literature, the chapter helps demonstrate the originality of the research problem and the thesis's contribution to the field.
    \item To position the thesis within the larger context of the field: The chapter helps to position the thesis within the larger context of the field, demonstrating its relevance and significance.
\end{enumerate}

The chapter is divided into two main sections: Background and Related Work. The Background section, present in \ref{sec:background}, discusses previous work that is important for understanding the context of the research problem, even though it does not address the same problem as the thesis. On the other hand, the Related Work section, present in \ref{sec:related-work}, focuses on issues that are similar or closely related to the research problem addressed in the thesis. This chapter serves as a foundation for the research problem and contributes to the overall contribution of the thesis.

\section{Background} \label{sec:background}

This dissertation is situated in a vast body of research on \ac{DL}. This section positions the current dissertation within this context. This work is not intended to explain other technologies that solve similar problems, but rather to explain other technologies that are useful in addressing the problem at hand.

The dissertation aims at readers with basic \ac{ML} knowledge, but not necessarily basics on \ac{DL}. No basis on sound besides 8th-grade physics is needed, also.

The dissertation will start by explaining how sound can be processed digitally in subsection \ref{sec:sound}, keeping this objective in mind. Then, in subsection \ref{sec:deep-learning}, it will provide a comprehensive review of general deep learning architectures and techniques, emphasizing their importance in sound synthesis. In subsection \ref{sec:parallel-tasks}, the subsequent part will concentrate on crucial techniques used in developing the suggested models. Section \ref{sec:dl-frameworks} will discuss the deep learning frameworks used in this research. Subsection \ref{sec:data-generators} will explore state-of-the-art models for data generation that are unrelated to audio but can serve as references, such as image generators, to provide further context.

\subsection{Sound} \label{sec:sound}

A digital audio signal --- often called a waveform --- captures alterations in sound pressure overtime. This waveform represents how sound waves develop and spread throughout space. Distinct sounds that constitute our auditory experience can be perceived and interpreted through analyzing the changes in frequency and amplitude within the waveform. Waveforms encode the richness of sound, enabling us to capture and manipulate it for diverse purposes such as analysis, synthesis, and artistic expression.

Sound is a continuous phenomenon but can be discretized by taking samples at a specific time rate. The number of samples taken in one second is known as the "sampling rate." The standard value for this is 44,100 Hz. The sampling rate has a direct impact on the accuracy of the signal~\cite{elsea_basics_1996}.

With the discrete version of time, it becomes possible to represent sound through an array digitally. Knowledge of the array and the sampling rate enables a computer to reconstruct the sound. These arrays can become quite large. For instance, stereo sound with a sampling rate of 44,100 \ac{Hz} needs to accommodate 1,411,200 bits per second \cite{elsea_basics_1996}.

\subsubsection{Short-Time Fourier Transform} \label{sec:stft}

Even though digital sound media can be encoded as one-dimensional data~\cite{oord_wavenet_2016}, it can be converted. By using a \acf{DFT}, the array can be represented in the frequency domain~\cite{benois-pineau_deep_2021}. Since the contents of a sound sample typically vary over time, \acp{DFT} can be computed over successive time frames of the signal. This operation forms the basis of the \acf{STFT}. The equation for \ac{STFT} is as follows:

\begin{equation} \label{eq:stft}
    STFT\{x(n)\}(m, \omega) = \sum_{n=-\infty}^{\infty} x(n) w(n - mR) e^{-j\omega n}
\end{equation}

In this equation, $STFT\{x(n)\}(m, \omega)$ represents the time-frequency representation of the input signal $x(n)$ as a function of time index $m$ and frequency $\omega$. The function is a complex-valued function containing both magnitude and phase information of the signal's frequency components at different time intervals.

The summation symbol $\sum_{n=-\infty}^{\infty}$ denotes that the product of the signal, window function, and complex exponential is summed over all time indices $n$. The discrete-time input signal is represented by $x(n)$, sampled at integer time indices $n$.

The window function, represented as $w(n - mR)$, is utilized to isolate a specific time interval of the input signal. The window function is centered at the time index $mR$, where $R$ is the hop or step size between consecutive sample windows.

Finally, the complex exponential term $e^{-j\omega n}$ is utilized to analyze the signal's frequency content within the windowed interval. The variable $j$ is the imaginary unit, and $\omega$ represents the angular frequency.

In essence, the \ac{STFT} equation computes the Fourier Transform of the input signal $x(n)$ within a windowed time interval, providing a time-frequency representation of the signal. The window function isolates a specific time interval of the signal, and the complex exponential term analyzes its frequency content. This process is repeated for different time indices $m$, resulting in a time-frequency representation that allows the study of the signal's frequency components at various time intervals.

With the \ac{STFT}, one can generate spectrograms by plotting the time in the $x$ axis and the frequency in the $y$ axis. In short, a spectrogram is a graphical representation of the frequency content of a signal over time, typically displayed as a 2D image (see Figure \ref{fig:sound} for an example).

\begin{figure}[ht]
    \centering
    \ctikzfig{figures/2-sota/sound}
    \caption[Raw Wave vs. Spectrogram]{\textbf{Raw Wave vs. Spectrogram} --- A comparative analysis of a sound sample from the Audio MNIST dataset (showcased in section \ref{sec:dataset-amnist}), specifically entry number 9 of speaker Nicolas uttering the digit ``five''. The top plot illustrates the raw waveform with time (sample rate $\times$ time) on the X-axis and energy (amplitude) on the Y-axis, providing a temporal representation of the audio signal. The bottom plot presents a spectrogram generated using the \ac{STFT} method, displaying time on the X-axis and frequency on the Y-axis, offering a time-frequency representation that reveals the spectral content and evolution of the signal over time.
    }
    \label{fig:sound}
\end{figure}

\subsubsection{Meaning of Spectrograms for Machine Learning}

Representing sound as an image opens a multitude of opportunities. However, even though spectrograms can technically be processed using \acp{CNN} (see section \ref{sec:CNN}), there is a considerable difference between a spectrogram and a standard image. In a typical image, the axes represent the same concept, the spatial position. The elements of an actual image have the same meaning independent of where they are found. A sub-object of an image does not depend on the axes. At the same time, neighbor pixels are usually highly correlated.

On the other hand, the axes of the spectrograms have different meanings \cite{benois-pineau_deep_2021}. Moving a set of pixels horizontally and vertically means different things. Therefore, structures such as \ac{CNN} are not as helpful. One can still use them but should be careful about the shape of the filters and the axis along which the convolution is performed \cite{benois-pineau_deep_2021}.

\subsubsection{Soundscapes} \label{sec:soundscapes}

The digitalization of sound has allowed for multiple applications and use cases. Applications can generate speech, and movies and videos can embed audio, such as soundscapes. Soundscapes are the sonic environments or sound environments that surround environments. They are the complex and dynamic mix of sounds heard in everyday life, including sounds from nature, human-made, and cultural sounds~\cite{international_organization_for_standardization_iso_2014, schafer_tuning_1977}. In other words, a soundscape encompasses the auditory milieu characterized by a collection of naturally occurring and human-generated sounds as perceived, encountered, and comprehended within a contextual framework by individuals. It is paramount in audio content creation, augmenting the user experience across media applications by infusing emotional engagement, a greater sense of immersion, and attention~\cite{chandrasekera_virtual_2015}.

Nevertheless, for audio media generation with \ac{ML}, one usually finds models in the literature that solve music or speech generation, not soundscapes. This is no coincidence. These sounds are more straightforward and, thus, easier to generate. Speech, for instance, usually contains a single sound source (the speaker). Also, speech and music are highly structured over time and timbrically. This happens because speech is bound to grammar, and music is bound to an underlying structure. Both of them are timbrically bound to their authors. On the contrary, soundscapes have no specific structure. Hence the increased difficulty \cite{benois-pineau_deep_2021}.

\subsection{Deep Learning}\label{sec:deep-learning}

As previously mentioned, the growth of \acf{DL} began in the early 2000s as a response to the challenge of handling vast quantities of data. Fundamentally, \ac{DL} stems from the application of \ac{ML} to process large amounts of data. To be precise, \ac{DL} is a subfield of \ac{ML} that employs multiple levels of information processing and abstraction to learn and represent features, as demonstrated by Deng et al. \cite{deng_deep_2014}. Subsequently, the extracted features can be utilized for classification, regression, and other modeling techniques. In the past, such features were manually selected by humans.

This study uses \ac{DL} for sound generation because it offers several advantages over traditional sound generation techniques. By being data-driven, these models can generate new sounds based on sounds it has heard before. On traditional methods, these sounds would have to come from the inspiration of their human creator. Besides, end-to-end generation, from text to sound generation, is only possible through \ac{DL}. The model has to extract features from the text, learn features from thousands or millions of sounds, and correlate both. This highly complex task can only be achieved with \ac{DL} techniques.

This section presents traditional \ac{DL} architectures and their evolution to generative \ac{DL} architectures.

%%%%%%%%%%%%%%%%%%%%%%%%%%%%%%%%%%%%%%%%%%%

\input{src/chapters/2-state-of-the-art/background/deep-learning-models/deep-learning-architectures.tex}
\input{src/chapters/2-state-of-the-art/background/deep-learning-models/dl-basics}
\input{src/chapters/2-state-of-the-art/background/deep-learning-models/generative-deep-learning-architectures.tex}
\subsection{Foundations for Enhancing Generative Models for Audio} \label{sec:parallel-tasks}

To develop generative models for audio, it is necessary to address several factors that impact their performance and quality. This thesis concentrates on three main areas: data augmentation, evaluation metrics, and data embedding.

\textit{Data augmentation} is the process of applying transformations to the original data to increase its size and diversity. This can help overcome the limitations of small or imbalanced datasets and improve the generalization ability of generative models. Different types of data augmentation techniques for sound generation and their effects on the model outcomes are discussed in Sextion \ref{sec:data-augmentation}.

\textit{Evaluation metrics} are the methods used to measure the quality and diversity of the generated sounds. They provide a way to compare different generative models and assess their strengths and weaknesses. However, evaluating sound generation is not trivial, as it involves objective and subjective criteria. We review various evaluation metrics for sound generation and their advantages and disadvantages in Section \ref{sec:evaluation}

\textit{Data embedding} is the technique of converting data into numerical representations that capture its essential features and characteristics. This can facilitate the learning process of generative models and enhance their expressiveness and efficiency. We explore different data embedding methods in section \ref{sec:text-embedding}.

\input{src/chapters/2-state-of-the-art/background/more/data-augmentation}
\input{src/chapters/2-state-of-the-art/background/more/evaluation}
\input{src/chapters/2-state-of-the-art/background/more/embedding}
 \subsection{Deep Learning Frameworks} \label{sec:dl-frameworks}

\Ac{DL} frameworks have revolutionized the field of \ac{AI}, enabling researchers and practitioners to efficiently develop and deploy complex neural networks for the multiple \ac{DL} tasks. These frameworks provide a wide range of tools and techniques for building, training, and evaluating deep neural networks and have significantly accelerated the pace of progress in the field. This section explores some of the most popular \ac{DL} frameworks, their key features and capabilities, and how they have been used to develop state-of-the-art generative \ac{AI} models for audio synthesis from textual input.

Several \ac{DL} frameworks are available, and they differ in several ways, including their programming languages, ease of use, and performance. However, to the best of the author's knowledge, there is no recent study on the performance of today's \ac{DL} networks. The most recent is from 2017 \cite{parvat_survey_2017}.	

\subsubsection{TensorFlow} \label{sec:tensorflow}

Developed by Google, TensorFlow \cite{martin_abadi_tensorflow_2015} is one of the most widely used \ac{DL}. It supports both \ac{CPU} and \ac{GPU} computations and provides a variety of \acp{API} for building different types of neural networks. TensorFlow is written in Python, but its core functionality is implemented in C++ for optimal performance.

TensorFlow is an interface for expressing \ac{ML} algorithms and an implementation for executing them. It allows computations to be executed with little or no change on various systems, from mobile devices to large-scale distributed systems. The system is flexible and can express many different algorithms, including training and inference algorithms for deep neural network models.

TensorFlow can be used with various programming languages, including Python, JavaScript, C++, and Java. Python is the recommended language for TensorFlow, but other languages' \acp{API} may offer some performance advantages. Other languages like Julia, R, Haskell, and others have bindings.

\subsubsection{PyTorch} \label{sec:pytorch}

PyTorch is an open-source \ac{DL} framework developed by Facebook \cite{paszke_pytorch_2019}. It has gained popularity due to its ease of use and dynamic computation graph, which allows for more flexible and intuitive programming. PyTorch also supports \ac{CPU} and \ac{GPU} computations, and although it supports Python and C++, it has a Python-first approach, making it easy to integrate with other Python libraries.

PyTorch is a popular deep-learning framework that is easy to use and learn. It has a simple and intuitive \ac{API} that makes it easy to learn and use. PyTorch is also flexible and can be used for various applications.

\subsubsection{Keras} \label{sec:keras}

Keras is a Python library for building and training neural network models at a high-level. It offers a user-friendly interface, and it is commonly used for \ac{DL} purposes~\cite{chollet_keras_2015}. Keras is built on top of lower-level libraries such as TensorFlow (see Section~\ref{sec:tensorflow}). It simplifies the creation and training of neural networks by abstracting away many low-level details.
Keras enables fast neural network model creation by assembling pre-built building block layers. These building blocks consist of input layers, \ac{CNN} layers, \ac{RNN} layers, fully connected layers, activation functions, and other components.

Keras is often regarded as more user-friendly than TensorFlow, as it offers a high-level interface that hides many low-level details.

Keras provides a simplified \ac{API} for constructing and training deep learning models. Keras includes a variety of utilities for manipulating data, such as data preprocessing, data augmentation, and data visualization, facilitating data manipulation.

\subsubsection{Conclusions on Deep Learning Frameworks} \label{sec:dl-frameworks-conclusions}

Selecting a \ac{DL} framework is critical in developing a \ac{DL} project. It is vital for the development of this thesis that the best framework that can offer flexibility, ease of use, and optimal performance of the \ac{DL} models is selected.

PyTorch uses a dynamic computation graph, which is created for each iteration in an epoch. In each iteration, the code executes the forward pass, computes the derivatives of output \textit{w.r.t} to the network parameters, and updates the parameters to fit the given examples. After doing the backward pass, the graph is freed to save memory. A dynamic graph can be changed on the fly, allowing for more freedom, easier debugging, and easier experimentation with architectures and hyperparameters during the model development process. Tensorflow, on the other hand, uses a static graph. There, the library creates and connects all the variables at the beginning and initializes them into a static (unchanging) session. This session and graph persist and are reused: it is not rebuilt after each iteration of training, making it more efficient and restrictive.

Furthermore, PyTorch offers a more straightforward and intuitive \ac{API} compared to TensorFlow. This feature makes it easier for developers to write and debug code. Additionally, PyTorch offers more flexibility when creating custom layers and functions, which is impossible in Keras. This flexibility enables developers to create more complex and innovative models, which can lead to better performance. Keras offers a straightforward but too simple network, while low-level Tensorflow offers a complicated and convoluted \ac{API} making it challenging to focus on the real problem. PyTorch offers a good balance between simplicity and features.

One of the significant drawbacks of Keras is the need for more flexibility when customizing \ac{DL} models. Keras offers a limited set of pre-defined layers, making it difficult for developers to create complex custom layers and functions. This lack of flexibility limits the ability to make changes to the architecture of a network during the development process, which can hinder the performance of the final model. When one needs more custom layers, one has to resort to the Tensorflow jungle, making the development a hassle.

Both PyTorch and TensorFlow have good hardware support. However, PyTorch has a feature that distinguishes it from TensorFlow: data parallelism. PyTorch optimizes performance by using native support for asynchronous execution from Python. In TensorFlow, one has to manually code and fine-tune every operation to be run on a specific device to allow distributed training. Running on top of TensorFlow, Keras presents the same hardware problems as the latter.

These conclusions are in table \ref{tab:dl-frameworks}. Given all of this, and considering that ease of use, debugging, and high customization are essential for this work, the clear choice is PyTorch. One should consider its use when this work uses practical terms.

\begin{table}[h]
\centering
\caption{Comparison of PyTorch, Raw TensorFlow, and Keras}
\begin{tabular}{|c|c|c|c|}
\hline
\textbf{Feature} & \textbf{PyTorch} & \textbf{Raw TensorFlow} & \textbf{Keras} \\
\hline
Graph computation & Dynamic & Static & Static \\
Ease of use & Moderate & Difficult & Easy \\
Debugging & Good & Difficult & Moderate \\
Customization & High & High & Moderate \\
Hardware support & Good & Moderate & Moderate \\
\hline
\end{tabular}
\label{tab:dl-frameworks}
\end{table}

\subsection{Data Generators} \label{sec:data-generators}

Creating new data from existing data is known as generative modeling. This technique has numerous applications across various media types, including images, text, and video. However, each media type possesses unique characteristics, so generating them requires distinct approaches and techniques. Sound, for instance, presents a different set of challenges than other media types, and hence its generation necessitates diverse techniques.

This section aims to survey some of the state-of-the-art generators for media types that are not sound-based, primarily focusing on image generators. By doing so, one hopes to comprehensively understand the different approaches and techniques employed for generating various media types.

The discussion delves into the main ideas behind these generators, their strengths and limitations, and how they can be related to or inspired by sound generation. Through this exploration, this section highlights the diversity of methods used for generative modeling and how they can be adapted to different contexts.

\begin{table}[ht]
\centering
\caption{Comparison of Data Generators}
\begin{tabularx}{\textwidth}{|X|X|X|X|}
\hline
\textbf{Generator Name} & \textbf{Main Idea}                                                                                                                    & \textbf{Strengths}                                                                                                                                 & \textbf{Limitations}                                                                                                          \\ \hline
PixelCNN~\cite{oord_conditional_2016}                & Uses autoregressive connections to model images pixel by pixel.                                                                       & Fast and parallelizable training, high resolution and diversity, interpretable latent space.                                                       & Slow and sequential sampling, limited global coherence, difficulty in conditioning on high-level features.                    \\ \hline
DALL-E~\cite{ramesh_zero-shot_2021}                  & A transformer language model that creates images from text descriptions, using a dataset of text–image pairs.                         & Can generate novel and creative images, combine concepts in plausible ways, render text and apply transformations.                                 & Requires large amounts of data and compute, may produce harmful or biased outputs, not publicly accessible.                   \\ \hline
Stable Diffusion~\cite{rombach_high-resolution_2021}        & A latent diffusion model that creates images from text descriptions, using a dataset of text–image pairs and a pretrained CLIP model. & Can generate realistic and detailed images, is lighter than previous state of the art models.                                                      & Requires fine-tuning for specific domains, may produce artifacts or inconsistencies, sensitive to text prompts.               \\ \hline
GLIDE~\cite{nichol_glide_2021}                   & Guided diffusion-based approach for text-conditioned image generation.                                                                & High-fidelity image synthesis, classifier-free guidance preferred for photorealism and caption similarity, text-driven image editing capabilities. & Difficulty generating images for complex or unusual text prompts, slow generation speed.                                      \\ \hline
DALL-E 2~\cite{ramesh_hierarchical_2022}                & An improved version of DALL-E that generates more realistic and accurate images using a diffusion model.                              & Has better results than DALL-E and is faster.                                                                                                      & Still requires large amounts of data and compute, may still produce harmful or biased outputs, still not publicly accessible. \\ \hline
\end{tabularx}
\label{tab:data-generators}
\end{table}

\input{src/chapters/2-state-of-the-art/background/data-generators/pixelcnn}
\input{src/chapters/2-state-of-the-art/background/data-generators/dall-e}
\input{src/chapters/2-state-of-the-art/background/data-generators/stable-diffusion}
\input{src/chapters/2-state-of-the-art/background/data-generators/glide}
\input{src/chapters/2-state-of-the-art/background/data-generators/dall-e-2}
\section{Related Work} \label{sec:related-work}

Traditionally, sound designers relied on manual labor to create audio, which involved recording and editing real-world sounds, mixing, and adding sound effects~\cite{sonnenschein_sound_2001}. Creating high-quality sounds is challenging, costly, and time-consuming, requiring specialized skills and resources. Hence, it engenders a notable impediment to the creation of soundscapes or any type of sound at scale~\cite{bernardes_seed_2016, strobl_sound_2006}, namely in light of its growing popularity and consumption within podcasts, movies, and video games.

Consumer data reported in 2021 showed compelling evidence regarding the listening habits of individuals within the United States of America. The findings indicate a substantial growth in podcast listenership over the past decade, with 41\% of Americans aged 12 or older having engaged with podcasts in the preceding month and 28\% within the last week. Moreover, at the beginning of the same year, a notable 68\% of Americans aged 12 and above had indulged in online audio consumption within the previous month, while 62\% had done so within the preceding week~\cite{research_infinite_2021}.

To overcome the aforementioned limitations, algorithmic audio generation has emerged as a promising solution that streamlines its creation altogether. Focusing on soundscapes, preceding 2018, prevailing models for their generation primarily revolved around statistical methods, featuring prominent employment of \ac{ML} techniques with feature engineering. For a comprehensive overview of techniques employed before the era of \ac{DL}, reference can be made to the review papers by Alias et al.\cite{alias_review_2016} and Kalonaris et al.\cite{kalonaris_computational_2018}. Noteworthy efforts at the feature engineering level are exemplified by Fernandez et al.~\cite{fernandez_ai_2013}, who represent sounds as high-level features, such as musical sheets, as an approach to generating musical sounds.

\Ac{DL} models for audio generation aim to produce high-quality audio signals by learning from existing audio data. Typically, these models consist of three main components: translation of the sound signal into a compressed representation, generation of a new representation from previous data, and translation back into an audio signal.

The first component of the model involves transforming the original sound signal into a Mel-spectrogram (or other) representation (see Section~\ref{sec:sound}), which is more compact and more accessible to process than the raw audio signal.

The second component involves generating new low-resolution representations from previous representations, such as spectrograms or feature vectors \cite{kong_hifi-gan_2020}. This is typically done using deep generative architectures. These models are trained on existing sound data to learn the target representation distribution and generate new, high-quality representations.

The final component of the model involves translating these representations back into an audio signal. These algorithms are called \textit{vocoders} (for more, see Section~\ref{sec:vocoders}). This component aims to produce high-quality audio signals that closely resemble the original sound data used to train the model.

The purpose of this section is to review the related work in this area. First, it examines the methods for generating traditional soundscapes, as described in Section~\ref{sec:trad-soundscape}. It then examines sound generation using machine learning. State-of-the-art models focus primarily on either using unsupervised sound generation techniques, as discussed in Section~\ref{sec:unsupervised-generation}, or generating sounds from internal representations, known as vocoders and discussed in Section~\ref{sec:vocoders}, or creating an end-to-end system, as discussed in Section~\ref{sec:end-to-end}.

\subsection{Traditional Soundscape Generation} \label{sec:trad-soundscape}

Feature engineering methods for soundscape generation typically adopt a threefold strategy to resynthesize (and extend) a short soundscape recording provided by the user:

\begin{enumerate}
    \item Segmentation,
    \item Feature extraction and modeling, and
    \item resynthesis of a given environmental sound.
\end{enumerate}

Statistical models adopting stochastic processes or pattern recognition methods were commonly applied to model and recreate a given soundscape recording with a degree of variation while maintaining its structure. Generated soundscapes relied on the similarity among audio segments to create smooth transitions~\cite{hoskinson_manipulation_2001}.

\subsubsection{Scaper}

Searching through the academic search engines, one finds that the most cited software for soundscape generation is \textit{Scaper} \cite{salamon_scaper_2017}.

Scaper is an open-source software library for soundscape generation designed to facilitate the creation of synthetic sound environments. It is a tool that allows users to simulate complex soundscapes, including urban, natural, and interior spaces, and investigate how various sound sources interact in these environments.

Scaper implements a modular soundscape generation framework based on basic sound-generating objects or ``sound sources''. These sound sources can represent simple sounds such as bird songs, human speech, or car horns, or more complex sounds like those produced by a crowd of people or a construction site. The user can specify the attributes of each sound source, such as its location, volume, and duration, and can adjust these parameters in real-time to create a dynamic soundscape.

One of the key features of Scaper is its ability to generate synthetic soundscapes that are diverse and statistically representative of real-world environments. To achieve this, the library implements various sound-generating algorithms that can be used to create sounds that are randomized yet realistic. For example, the library can generate sounds similar to real-world sources but with variations in volume, pitch, and timbre to avoid repetition and create a more diverse soundscape.

\subsubsection{SEED}

SEED is a system that addresses the formidable task of resynthesizing environmental sounds, such as city ambiances or nature scenes~\cite{bernardes_seed_2016}. SEED aims to provide a solution that not only extends the duration of environmental sounds but also provides precise control over the degree of variation in the output. This control over variation is critical in applications where maintaining the authenticity and coherence of the audio environment is essential.

SEED is built on a tri-partite architecture consisting of three main modules: segmentation, analysis, and generation. 

In the segmentation module, SEED performs the task of dividing the input audio into segments. This segmentation process is based on detecting spectral stability between frames. Spectral stability is a measure of how similar the frequency spectrum is between consecutive frames. When this stability falls below a certain threshold, it signals a change in the underlying sound source or event, prompting the placement of a segment boundary. This approach ensures that the resynthesized audio remains cohesive and retains its natural flow.

The Analysis module has two main processes. First, it extracts several audio features that capture both the sonic and temporal characteristics of the segments. These features are then clustered into a discrete ``dictionary'' of audio classes, effectively reducing the feature space to a finite set. At the same time, the module builds a transition table that records the sequences of these audio classes. This table is used to determine the probability that one class follows another.

In addition, the analysis module computes a concatenation cost matrix that quantifies how smoothly two segments can transition from one to the other. This matrix is computed by comparing the features at the segment boundaries. A lower cost indicates a smoother transition, while a higher cost indicates a more abrupt change.

In the Generation module, SEED generates new audio by searching for segment sequences that meet certain criteria. To achieve this, SEED references the transition table to determine viable next classes based on the current class in the audio sequence. It then assembles segments belonging to these classes and selects the one with the lowest concatenation cost. Notably, SEED applies a temporary cost penalty to recently selected segments to encourage diversity in the generated audio.

\subsubsection{Physics-Based Concatenative Sound Synthesis}

In the current development of virtual environments, the generation of audio content has been the subject of extensive research. One prominent approach in this area is \acf{CSS}, a method that creates novel auditory experiences by assembling segments of pre-existing sounds from a given database, often referred to as ``audio units''.

A recent scientific paper by Magalhães et al. presents an innovative \ac{CSS} framework based on physics-based principles for virtual reality~\cite{magalhaes_physics-based_2020}. This framework consists of two main components, namely the ``Capture Component'' and the ``Synthesis Component''.

The capture component of the framework is responsible for capturing essential data during interactions with virtual objects. This includes physics simulation data, haptic feedback data, position sensor data, and audio data. In particular, the physics data includes critical information such as collision points, velocities, impulses, and normals, among other parameters. The haptic and audio data are derived from real-world interactions with a variety of materials. This capture process culminates in the creation of a multimodal corpus of annotated audio units, which serves as the foundational resource for subsequent synthesis efforts.

The synthesis component of the framework uses the captured data to orchestrate the synthesis of auditory and haptic feedback by concatenating audio units extracted from the corpus. This unique mapping between physics data and audio units ensures congruence between user interactions and the resulting sensory feedback. For example, when a user applies a certain force and angle to interact with a virtual metal object, the synthesis component selects an audio unit recorded from a similar interaction with a real-world metal object.

At runtime, the framework relies on the target physics vectors to guide the selection of audio units, thus generating congruent auditory and haptic experiences. An overlap-add phase vocoder is used to concatenate the audio segments, while temporal repetition penalties are incorporated to ensure smooth transitions between these segments.
\subsection{Unsupervised Sound Generation} \label{sec:unsupervised-generation}

This Section focuses on models that address unsupervised or self-supervised training through learning sound features and their distributions without relying on explicit labels or annotations. In unsupervised sound generation, models learn from unlabeled audio data to capture underlying patterns and structures. This enables the generation of novel sound samples and the representation of latent features. This approach is particularly valuable when labeled datasets are scarce or expensive to acquire. Next, this discussion covers notable models in this area, selected based on their suitability for generating audio.

\subsubsection{WaveGAN} \label{sec:wavegan}

\Acp{GAN} have a notable impact on generating coherent images at the local and global levels, as discussed in Section~\ref{sec:gan}. A model based on \acp{GAN} called WaveGAN \cite{donahue_adversarial_2019} was proposed in 2019, to synthesize waveforms in an unsupervised manner. The model modifies the transposed convolution operation, used in \acp{DCGAN} for image generation, to capture waveform structure at different timescales.

WaveGAN modifies the transposed convolution operation in \acp{DCGAN}, expanding conventional \acp{GAN} to encompass image generation tasks and precisely capture the structure of audio signals of varying timescales. This model uses lengthier, one-dimensional filters of 25 units in place of two-dimensional filters with dimensions of $5 \times 5$. The model also upsamples each layer by a factor of 4, as is done in traditional \acp{DCGAN}. Despite these modifications, WaveGAN has the same number of parameters, numerical operations, and output dimensionality as \acp{DCGAN} have.

The experiments conducted on WaveGAN show that it can synthesize one-second slices of audio waveforms with global coherence, which is suitable for sound effect generation. The model also learns to produce intelligible words when trained on a small-vocabulary speech dataset without labels.

The success of WaveGAN in generating coherent audio signals demonstrates that \acp{GAN} can generate high-quality sounds. This work opens up new possibilities for unsupervised synthesis of raw-waveform audio, such as music and speech. It also suggests that \acp{GAN} can learn to capture the structure of signals across various timescales, which is crucial for generating realistic audio.

\subsubsection{Generative Transformer for Audio Synthesis}

In this work, Verma and Chafe~\cite{verma_generative_2021}, proposed in 2022, explore an alternative architecture using transformer networks (see Section~\ref{sec:transformers}), which have shown great success in sequential modeling tasks such as language translation.

The authors develop a generative transformer model for raw audio waveforms. The model is trained to autoregressively predict the next audio sample by attending over previous context samples. Specifically, the input waveform is split into overlapping frames and embedded into a latent space. A series of multi-headed causal self-attention layers then learn to focus on relevant parts of the input context to predict the subsequent sample distribution.

To retain information about the relative sample positions, positional encodings are added. Training deeper models is facilitated by layer normalization and residual connections. In a neural network, residual connections (also known as skip connections) enable the direct flow of information from one layer to subsequent layers. The use of residual connections helps to mitigate the vanishing gradient problem and allows for more effective gradient propagation during training. The inclusion of these connections enables the model to learn new representations at each layer and retain useful features from previous layers, resulting in improved performance and faster convergence. Self-attention provides the model with flexibility and frees it from the fixed topology of convolutions in other models such as WaveNet (Section~\ref{sec:wavenet}).

During training, previous samples are fed as input to the model to predict the next sample, optimized with cross-entropy loss (see Section~\ref{sec:cross-entropy}). The authors quantitatively evaluate next-step sample accuracy and find that the transformer architecture can outperform WaveNet baselines substantially.

\subsubsection{wav2vec 2.0}

The wav2vec 2.0 model comprises three essential components: a convolutional feature encoder, a Transformer network, and a quantization module. This model was originally introduced in the 2020 paper by Baevski et al. \cite{baevski_wav2vec_2020} and designed for speech generation tasks. For further details on the convolutional layer and the Transformer network, refer to Sections~\ref{sec:conv-layer} and \ref{sec:transformers}.

Quantization is the process of discretizing continuous values into a finite set of discrete symbols or codes, particularly in the context of generative models. This method is comparable to the technique used in the \ac{VQ-VAE} (refer to Section~\ref{sec:vq-vae}) In this technique, the input data is mapped to a limited number of discrete codebook entries.

For convolution, the feature encoder takes the raw audio waveform as input and generates a sequence of speech representations underlying it. This consists of several convolutional blocks, with each block including 1D temporal convolution and layer normalization. Wide kernels (\textit{e.g.,} 10ms) are used in the convolutions and progressively reduce the resolution of the input to extract hierarchical features.

The output of the feature encoder is fed into a transformer network to build contextualized representations. For encoding positional information specific to speech generation tasks, we use a convolutional layer instead of absolute positional embeddings. The self-attention mechanism enables each time step to consider all other time steps, thus capturing long-range dependencies in the sequence. Several Transformer layers extract higher levels of contextual abstraction.

A quantization module is applied to the output of the feature encoder. It discretizes the continuous latent representations into a finite inventory of speech units. Multiple codebooks are maintained, and concatenating selections from each codebook construct discrete units.

After pre-training on unlabeled speech, the model is fine-tuned on transcribed speech for speech recognition by adding a randomly initialized output layer. Various augmentations are used during fine-tuning to improve robustness.

The key innovations are the joint training of discrete speech units and contextualized representations in a completely self-supervised fashion. Experiments demonstrate strong performance even with just minutes of labeled data, highlighting the benefits of pre-training on large unlabeled corpora.

\subsubsection{SoundStream} \label{sec:soundstream}

SoundStream is a neural audio codec proposed in 2021 \cite{zeghidour_soundstream_2021} that can efficiently compress speech, music, and general audio. A codec is software or hardware that compresses and decompresses audio signals. The model architecture consists of a fully convolutional encoder/decoder network and a residual vector quantizer.

The fully convolutional encoder receives a time-domain waveform as input. It produces a sequence of embeddings at a lower sampling rate, which is then quantized by the \ac{RVQ}. The fully convolutional decoder then receives the quantized embeddings and reconstructs an approximation of the original waveform. Both the encoder and decoder use only causal convolutions, so the overall architectural latency of the model is determined solely by the temporal resampling ratio between the original time-domain waveform and the embeddings.

While there are similarities between SoundStream and a standard \ac{AE} (see Section \ref{sec:autoencoders}) in terms of the encoder-decoder architecture, SoundStream includes additional components such as the \ac{RVQ} and the use of structured dropout for variable bitrate compression.

A \acf{RVQ} is a vector quantization method. It is a variant of the traditional vector quantization method present, for instance, in \acp{VQ-VAE} (see Section \ref{sec:vq-vae}). In an \ac{RVQ}, the input data is first transformed into a lower-dimensional space using a neural network encoder. The resulting embeddings are then quantized using a codebook of fixed-size vectors, where each input embedding is assigned to the nearest codebook vector. However, instead of encoding the input embedding directly as the index of the assigned codebook vector, an \ac{RVQ} computes the difference between the input embedding and the assigned codebook vector, known as the residual. The residual is then quantized using a second codebook, and the indices of both codebook vectors are transmitted as the compressed representation.

Using residual vectors in \acp{RVQ} allows for better compression performance than traditional vector quantization methods. It captures the fine details of the input data that may be lost during quantization. In SoundStream, the \ac{RVQ} is used to quantify the embeddings produced by the fully convolutional encoder, enabling efficient audio compression at low bitrates while maintaining high audio quality.
\subsection{Vocoders} \label{sec:vocoders}

\Ac{DL} vocoders are neural network models that have the ability to generate artificial audio~\cite{mehrish_review_2023}. These models employ deep learning networks to learn the mapping between the input and waveform data directly. There is no reliance on any predefined model or feature extraction method. This approach has the ability to capture complex nonlinear relationships between input and output representations that are difficult to be modeled analytically.

There are different types of \ac{DL} vocoders, depending on the input and output representations they use. Some use Mel-spectrum features as conditioning inputs, while others do not require explicit features and directly generate raw waveform samples.

These models can achieve high quality and naturalness of audio synthesis, but they also face some challenges that limit their applicability. One challenge is the high computational cost of generating raw waveform samples at high sampling rates, which requires many computation and memory resources. This limits the scalability and efficiency of these models for real-time applications. Another challenge is the need for high-quality audio data with consistent annotations. This makes training these models with sufficient data diversity and coverage difficult. A third challenge is the generalization problem of these models, which tend to overfit the training data.

\subsubsection{WaveNet} \label{sec:wavenet}

\textit{WaveNet} is a generative neural network developed by DeepMind in 2016. It uses a unique architecture based on dilated causal convolutions to generate raw audio waveforms \cite{oord_wavenet_2016}. It implements the PixelCNN (see Section \ref{sec:pixelcnn}) model for sound and follows an \ac{AR} architecture (see Section \ref{sec:darn}) with the predictive distribution for each audio sample being conditioned on a window of previous ones.

WaveNet's structure allows it to process input sequences in parallel, enabling it to model long context dependencies, even with thousands of timesteps. It uses a series of dilated convolutional layers, where the dilation rate is increased with each layer, which effectively increases the receptive field of the network without increasing the number of parameters.

A dilated convolution happens when the filter is applied over an area larger than its length by skipping input values with a specific step \cite{oord_wavenet_2016}. This architecture can be seen in Figure \ref{fig:dilated-convolution}.

\begin{figure}[ht]
    \centering
    \includegraphics[width=\textwidth]{figures/2-sota/dilated-convolution.png}
    \caption[WaveNet]{\textbf{WaveNet} --- This illustration was taken from \cite{oord_wavenet_2016}. It shows the idea behind WaveNet, applying dilated convolutions to \ac{AR} models.}
    \label{fig:dilated-convolution}
\end{figure}

This structure enables WaveNet to capture long-range dependencies in the input sequence, which is crucial for generating high-quality audio. If an \ac{RNN} (see Section \ref{sec:rnn}) sees only one input sample at each time step, WaveNet has direct access to multiple input samples \cite{huzaifah_deep_2021}. For example, in speech generation, WaveNet can use its sizeable receptive field to model the relationship between a word spoken early in a sentence and its pronunciation later in the sentence.

WaveNet uses a softmax activation function at each output node to produce a probability distribution over the possible values at each time step. During training, the network is fed sequences of input data and their corresponding ground truth values. The model's parameters are adjusted so that its outputs match the ground truth as closely as possible.

WaveNet can use its trained parameters to generate new sequences by sampling from its output probability distribution during generation. This allows it to generate diverse and high-quality outputs, such as realistic human speech or written text, by combining its learned representations of the underlying data distribution with a small amount of randomness.

The input of WaveNet is usually a Mel-Spectrogram (or other representations), and the output is the sound signal.

WaveNet can be conditioned on, for instance, text for \ac{TTS} settings by feeding extra information about the text itself (\textit{e.g.} embeddings). If a model is not conditioned on text, it generates random sounds without any global structure behind it.

The results were astonishing. ``A single WaveNet can capture the characteristics of many different speakers with equal fidelity, and can switch between them by conditioning on the speaker identity. When trained to model music, we find that it generates novel and often highly realistic musical fragments.'' \cite{oord_wavenet_2016}.

Even though this model is good at learning the characteristics of sounds over brief periods, it struggles with global latent structure. They are also very slow for training and inferring \cite{tahiroglu_-terity_2020}.

%%%%%%%%%%%%%%%%%%%%%%%%%%%%%%%%%

\subsubsection{WaveNet Variants} \label{sec:wavenet-variants}

The WaveNet model has emerged as a powerful tool for generating high-quality audio waveforms, particularly for speech and music applications. However, its architecture, which employs dilated convolutions and deep residual networks, can be computationally intensive and challenging to train. To address these limitations, several WaveNet variants have been proposed in recent years that aim to reduce the complexity of the model while maintaining its effectiveness.

One such variant is \textbf{WaveRNN} \cite{kalchbrenner_efficient_2018}, which employs a single \ac{RNN} (see Section \ref{sec:rnn}) to approximate the dilated convolutions in WaveNet. This approach significantly speeds up training time while maintaining the quality of the generated audio. Another variant, FloWaveNet \cite{kim_flowavenet_2018}, employs a flow-based generative model (Section \ref{sec:flow-model}) that allows for efficient training with only one training stage while producing high-quality audio. Additionally, Fast WaveNet  \cite{paine_fast_2016} employs a caching mechanism to reduce the computational cost of the model while maintaining an \ac{AR} structure.

These WaveNet variants are unique in their architectures and training procedures but share the goal of making audio generation more efficient and accessible. While these models are primarily focused on speech and music generation, they can be adapted to other types of audio data. Ongoing research in this area may explore further optimization of these models, integration with other models, and application to new domains.

\subsubsection{MelGAN} \label{sec:melgan}

According to Kumar et al. in 2019, in their \textit{MelGAN} paper~\cite{kumar_melgan_2019}, audio generation with \acp{GAN} is possible although a challenging task (see Section \ref{sec:gan}). Previous studies in this field have encountered difficulties generating coherent raw audio waveforms using \acp{GAN}. Nonetheless, Kumar et al. demonstrate in their \textit{MelGAN} paper that introducing certain architectural changes makes it feasible to train \acp{GAN} to generate high-quality and coherent audio waveforms reliably.

The generator of MelGAN is a fully convolutional feed-forward network that takes a Mel-Spectrogram as input and generates a raw waveform as output. This approach allows for efficient and parallelized processing of audio data.

The decoder takes the waveform and decides whether it is a realistic sound. The decoder is not a single neural network but a multi-scale architecture with three discriminators (D1, D2, D3). These discriminators have identical network structures but operate on different audio scales. D1 operates on the scale of raw audio, while D2 and D3 operate on raw audio downsampled by a factor of 2 and 4, respectively. The use of multiple discriminators at different scales is motivated by the fact that audio has structure at different levels.

MelGAN proved itself way faster than other architectures such as WaveNet (see Section \ref{sec:wavenet}) with comparable results (for inference, roughly thirty-six thousand times faster than WaveNet), given its reduced number of parameters.


%%%%%%%%%%%%%%%%%%%%%%%%%%%%%%%%%

\subsubsection{GANSynth} \label{sec:gansynth}

\textit{GanSynth}, presented in 2019, \cite{engel_gansynth_2019} is a \ac{GAN} (see Section \ref{sec:gan}) that uses log-magnitude spectrograms and phases to generate coherent waveforms. Compared to directly generating waveforms with stridden convolutions, the use of spectrograms and phases has been shown to produce better results.

The study focuses on the NSynth \cite{engel_neural_2017} dataset, a collection of 300 000 musical notes from 1 000 different instruments.

The model first samples a random vector $z$ from a spherical Gaussian distribution. This vector is passed through a stack of transposed convolutions, which upsample and generate output data $x = G(z)$. This generated data is then fed into a discriminator network, which uses downsampling convolutions to estimate a divergence measure between the real and generated distributions.

The architecture of the discriminator network mirrors that of the generator, which allows for a more efficient training process. Optimizing the divergence measure allows the generator to produce spectrograms and phases that resemble actual musical notes more closely.

The study results demonstrate that \acp{GAN} outperform WaveNet (see Section \ref{sec:wavenet}) baselines on automated and human evaluation metrics and can efficiently generate several audio orders of magnitude faster than their \ac{AR} counterparts.

%%%%%%%%%%%%%%%%%%%%%%%%%%%%%%%%%

\subsubsection{HiFi-GAN}

Proposed in 2020, \textit{HiFi-GAN} \cite{kong_hifi-gan_2020} is a \ac{GAN} (see Section \ref{sec:gan}) model that combines efficiency and high-fidelity speech synthesis. HiFi-GAN achieves this by leveraging the periodic patterns inherent in speech audio, demonstrating that modeling these patterns is crucial for enhancing sample quality. The model includes a generator and two discriminators, trained adversarially, and two additional losses for improving training stability and model performance.

The generator is a fully \ac{CNN} (see Section~\ref{sec:CNN}) that takes Mel-Spectrograms as input and upsamples them through transposed convolutions, matching the temporal resolution of raw waveforms. The discriminators are the \ac{MSD} and a \ac{MPD}. \Ac{MSD} evaluates the audio sequence on different scales using a mixture of three convolutional sub-discriminators with different average pools. At the same time, \ac{MPD} consists of small sub-discriminators that capture different implicit structures of input audio by looking at different parts, accepting only equally spaced samples of input audio with different periods. 

HiFi-GAN's performance is evaluated using a subjective human evaluation (\ac{MOS}) on a single speaker dataset, which shows that the proposed method exhibits similarity to human quality. The model achieves a higher \ac{MOS} score than WaveNet (see Section \ref{sec:wavenet}).

Importantly, HiFi-GAN achieves this high-quality synthesis efficiently. Specifically, the model generates 22.05 kHz high-fidelity audio 167.9 times faster than real-time on a single V100 \ac{GPU}, demonstrating superior computational efficiency compared to AR and flow-based models. Moreover, a small-footprint version of HiFi-GAN generates samples 13.4 times faster than real-time on \ac{CPU} with comparable quality to an \ac{AR} counterpart.

\subsection{End-to-End Models} \label{sec:end-to-end}

Audio synthesis is the task of producing artificial audio from text or other kinds of data. Traditionally, audio synthesis systems consist of multiple stages, such as a data analysis frontend, a sound model, and an audio synthesis module. Building these components requires extensive domain expertise and may contain brittle design choices. Moreover, these components are usually trained separately on different objectives and datasets, which may introduce errors and inconsistencies in the final output. To overcome these limitations, end-to-end models have been proposed that directly learn the mapping between text (or other kinds of data) and audio waveform using deep neural networks. These models are presented in this Section.

Existing research establishes two main frameworks for end-to-end models: specialized models designed for a specific domain, and universal models aimed at broader applications. The table \ref{tab:end-to-end-audio-models} shows some examples of these models based on their type, input, output, model architecture, and type of conditioning. Specialized models target either speech or music synthesis, such as Char2wav and Jukebox. Researchers have developed different subsets of technologies within speech and music synthesis models, such as neural codec speech models and discrete diffusion models. Universal models, such as SampleRNN and AudioGen, can generate audio from various inputs and domains, such as text or raw audio seeds.

\begin{table}[ht]
\centering
\caption{A comparison of different end-to-end generative models for audio.}
\begin{tabularx}{\textwidth}{|l|l|X|X|X|}
\hline
\textbf{Model} & \textbf{Type} & \textbf{Input}            & \textbf{Output}                        & \textbf{Model Architecture}                                                      \\ \hline
Char2wav~\cite{sotelo_char2wav_2017}       & Speech        & Text prompt               & Raw audio waveform                     & Encoder-decoder with attention and neural vocoder                                \\ \hline
VALL-E~\cite{wang_neural_2023}         & Speech        & Text and acoustic prompt  & Raw audio waveform                     & Neural codec language model and neural vocoder                                   \\ \hline
Jukebox~\cite{dhariwal_jukebox_2020}        & Music         & Genre, artist, and lyrics & Raw audio waveform                     & Hierarchical VQ-VAE and autoregressive Transformer                               \\ \hline
Riffusion~\cite{forsgren_riffusion_2022}      & Music         & Text prompt               & Raw audio waveform                     & Neural codec language model based on discrete diffusion model and neural vocoder \\ \hline
MusicLM~\cite{agostinelli_musiclm_2023}        & Music         & Text prompt               & Raw audio waveform                     & Neural codec language model and neural vocoder                                   \\ \hline
SampleRNN~\cite{mehri_samplernn_2017}      & General       & None                      & Raw audio waveform                     & Hierarchical RNN and neural vocoder                                              \\ \hline
AudioLM~\cite{borsos_audiolm_2022}        & General       & Text prompt               & Raw audio waveform                     & Hybrid tokenization scheme with Transformer models and neural vocoder            \\ \hline
DiffSound~\cite{yang_diffsound_2022}      & General       & Text prompt               & Mel-spectrogram and raw audio waveform & VQ-VAE, discrete diffusion model, and neural vocoder                             \\ \hline
AudioGen~\cite{kreuk_audiogen_2023}       & General       & Text prompt               & Mel-spectrogram and raw audio waveform & Transformer-based generative model and neural vocoder                            \\ \hline
\end{tabularx}
\label{tab:end-to-end-audio-models}
\end{table}

%%%%%%%%%%%%%%%%%%%%%%%%%%%%%%%%%

\subsubsection{Text-to-Speech} \label{sec:tts}

\Acf{TTS} models are designed to convert written text into synthesized speech. These models use deep neural networks to directly learn the mapping between written text and audio waveform. Leveraging developments in \ac{NLP} and speech synthesis techniques, \ac{TTS} models have made significant progress in generating high-quality, human-like speech from text input. This Section examines some of the notable \ac{TTS} models that have been developed recently.

\input{src/chapters/2-state-of-the-art/related-work/end-to-end/char2wav}
\input{src/chapters/2-state-of-the-art/related-work/end-to-end/vall-e}

\subsubsection{Generative Music}

Generative music is created using generative techniques. End-to-end generative music models enable the production of new musical compositions without using predefined templates or samples, directly from textual or other data inputs. These models use \ac{DL} architectures to capture patterns and structures within different genres or styles of music, and can produce original pieces based on given prompts. This Section explores remarkable generative music models that demonstrate their ability to compose novel musical arrangements.

\input{src/chapters/2-state-of-the-art/related-work/end-to-end/jukebox}
\input{src/chapters/2-state-of-the-art/related-work/end-to-end/riffusion}
\input{src/chapters/2-state-of-the-art/related-work/end-to-end/musiclm}

\subsubsection{General Text-to-Audio}

Text-to-audio systems have a wide range of applications beyond speech synthesis or generative music tasks. End-to-end models convert different forms of textual input into corresponding audio outputs. The outputs have diverse purposes, including sound effects generation, voice transformation, and environmental sound synthesis. These models provide flexible solutions for transforming text into realistic auditory experiences by training on large-scale datasets containing paired text-audio examples across various domains. This Section presents several text-to-audio approaches that demonstrate innovative methods of audio synthesis based on specific textual cues.

\input{src/chapters/2-state-of-the-art/related-work/end-to-end/samplernn}
\input{src/chapters/2-state-of-the-art/related-work/end-to-end/audiolm}
\input{src/chapters/2-state-of-the-art/related-work/end-to-end/diffsound}
\input{src/chapters/2-state-of-the-art/related-work/end-to-end/audiogen}



\chapter{The Synthesis Problem}\label{chap:problem}

\minitoc

This chapter explores the intricate domain of sound synthesis through advanced generative \ac{AI} models that operate on textual input. The exploration seeks to unravel the complexities and nuances inherent in the fusion of \ac{AI} and sound engineering, resulting in exceptional fidelity and realism in auditory creations.

Sound synthesis, a venerable discipline in the aural arts, stands at the intersection of traditional craftsmanship and contemporary technology. The digital age is ushering in a profound transformation in which the marriage of textual prompts and generative \ac{AI} creates a symphony of possibilities for creating immersive auditory experiences.

Generative \ac{AI} orchestrates a paradigm shift in the landscape of sound synthesis. This chapter seeks to illuminate the nascent symphony of generative \ac{AI}'s role in creating intricate soundscapes, shedding light on its potential to revolutionize content creation, music composition, and multimedia production.

This chapter delves into the synthesis problem through several key sections. The first Section (\ref{sec:problem-definition}) defines the core challenges. Then, Section~\ref{sec:significance} highlights the broader implications. Section~\ref{sec:prob-datasets} examines the underlying data. Section~\ref{sec:limitations} outlines the limitations of the research. Model fine-tuning is explored in Section~\ref{sec:parametric-control}, while architectural subtleties are revealed in Section~\ref{sec:model}.


\section{Problem Definition} \label{sec:problem-definition}

The rapid advancement of artificial intelligence and machine learning techniques has paved the way for transformative developments in content creation across multiple disciplines. However, a significant gap remains in the field of audio synthesis, especially in the context of generative \ac{AI} models. The challenge at hand is to bridge this gap and enable the transformation of textual input into high-fidelity and immersive soundscapes, thereby creating a new form of creative expression.

Recently, the landscape of audio generation has been undergoing a paradigm shift, catalyzed by significant investments and efforts from prominent technology conglomerates. As of 2023, major industry players are placing significant bets on the potential of \ac{AI}-driven audio synthesis. This noteworthy development underscores the growing recognition of the transformative impact that sophisticated audio generation can have across sectors, including entertainment, education, virtual reality, and more.

\subsection{Gap in the Literature}

The landscape of audio synthesis through generative \ac{AI} models presents a spectrum of challenges and limitations that impact various applications, including content creation, sound design, music production, and interactive media. These challenges include computational constraints, model architectural intricacies, quality/realism trade-offs, data-related hurdles, interpretability issues, and scalability concerns. Addressing these limitations has the potential to significantly improve the quality and utility of audio synthesis methods.

Significant hurdles in the use of generative \ac{AI} models for audio synthesis are the computational cost and the training time. PixelCNN Decoders, Jukebox, and Hi-Fi GAN, which exemplify these models, require extensive computational resources and long training times. These limitations render these models inaccessible to researchers and developers and limit their applicability in real-world situations where efficient generation is critical. Developing more efficient training algorithms or model architectures could democratize access and broaden the reach of audio synthesis methods by overcoming these challenges.

Producing \textbf{high-quality and realistic audio} synthesis is a complex task. The lack of ability to capture minute details, as seen in models such as WaveNet and MelGAN, causes audio outputs to be deficient in delicate nuances and textures. Limitations such as mode collapse, GANSynth, and Jukebox can cause a lack of variety, leading to repetitive and monotonous audio outputs. Additionally, maintaining coherence throughout extended audio sequences - as observed in PixelCNN, WaveNet, Jukebox, and Riffusion - is a challenge that can affect the final output.

The versatility of synthesis models is impacted by data efficiency and generalization capabilities. Models such as DALL-E, VALL-E, and MusicLM have suboptimal data efficiency, which requires large datasets for effective training. It is crucial to address the challenges in handling unusual concepts, as observed in \ac{GLIDE}, to broaden the creative potential of these models. Moreover, the practical utility and user-friendliness of models such as DALL-E and VALL-E are constrained due to their limited control and interpretability.

The challenges and limitations within audio synthesis through generative AI models align with the key objectives of this thesis. The aim to develop advanced systems that handle sound, classification, and end-to-end generation while accounting for hardware limitations aligns with the need to overcome computational constraints highlighted in the literature. The thesis aims to address the computational hurdles that hinder the accessibility and practicality of audio synthesis models by devising more efficient training algorithms or model architectures. This directly contributes to the broader objective of providing valuable contributions to deep learning and its applications.

The goal to create end-to-end systems that generate sound from any textual input is intricately connected to the challenges of quality, realism, and coherence in audio synthesis. The thesis aims to bridge the gap between textual input and high-fidelity, immersive soundscapes by exploring the intricacies of model architectural design and refining the generation process. Evaluating the accuracy of generated audio aligns with the broader pursuit of overcoming limitations in capturing fine-grained details, mode collapse, and long-term coherence. This pursuit ultimately enhances the authenticity and richness of the synthesized audio.

The ambitious mission to contribute to \ac{DL} and its applications aligns well with the goal of addressing data-related issues, such as biases and limitations arising from training data. This thesis aims to enhance the adaptability of generative \ac{AI} models by curating diverse and representative datasets and implementing techniques for bias detection and mitigation, to make them more responsive to various tasks and domains. This alignment highlights the research's importance in advancing audio synthesis and establishing a wider framework for responsible and inclusive development of \ac{AI}.

To summarize, the literature's identified gap, marked by challenges ranging from computational constraints and quality/realism to data efficiency and interpretability issues, is intricately linked with this thesis's objectives. Advancing end-to-end audio synthesis systems while addressing these challenges not only deepens the field's scientific comprehension but also profoundly aligns with the broader goals of your research endeavor.

\subsection{Formal Problem Definition}

The goal of this study is to address the synthesis of audio based on textual prompts. The audio is not limited to a specific domain, such as speech or music. The audio produced by the model must be dependent on the inserted textual input. The generated audio must be produced in real-time, with minimal latency after the textual prompt is inserted into the model. This study considers a model that resolves this issue and refers to it as an end-to-end model. Furthermore, it is considered that the terms end-to-end and text-to-sound are interchangeable in this context.

These models are trained using sounds labeled by humans from openly available datasets. Furthermore, the evaluation involves comparing the generated sounds with their corresponding real sounds.

To formally define this problem, the following elements must be specified:

\begin{itemize}
    \item input data
    \item output data
    \item objective function
    \item optimization algorithm
    \item evaluation metrics
\end{itemize}

\subsubsection{Input Data}

The synthesis of audio from textual prompts introduces a captivating interplay between audio samples and textual labels, infusing the model's learning process with a symphony of complexity and richness. However, synthesizing audio from textual prompts entails several challenges, especially when dealing with the diverse formats present in sound samples and text labels. As these formats converge, they add layers of intricacy that require innovative solutions and thoughtful consideration.

In the field of sound samples, dealing with diverse formats manifests itself in a medley of sonic expressions, ranging from musical melodies and spoken dialogues to environmental noises and abstract compositions. These formats encompass a spectrum of acoustic textures and temporal dynamics, demanding a unique processing and analysis approach for each. The model must navigate through this sonic kaleidoscope, effectively capturing and learning from the essence of each format without resorting to a one-size-fits-all approach.

Text labels also introduce a variety of challenges as they take various forms from different sources. The range of text labels spans from brief categorical tags that capture high-level features to verbose textual descriptions that explore intricate details. This variation poses the model with the task of decoding and utilizing the diverse meanings, tones, and contexts embedded within these labels. Achieving balance between the weight of these different formats while incorporating them into the generative fabric poses its own challenge.

The intersection of different audio formats and textual labeling structures presents a challenge for the model to generate coherent and harmonious audio outputs. Each format has its implicit dimensions, nuances, and potential biases associated with it. Ensuring that the model avoids being biased towards a particular format or ignores others is a challenge that needs precise calibration. Additionally, the model must handle the complex task of cross-modal learning, deciphering the interplay between sound and text to produce outputs that match seamlessly with textual prompts.

When embarking on the complex process of audio synthesis from text, it is important to acknowledge the intricacies that arise from combining sound samples and text labels. The difficulties that come with handling various formats serve as milestones of progress and discovery. These challenges put the model's true capabilities to the test as it navigates through the maze of formats, becoming a creative force capable of orchestrating sonic symphonies that blend seamlessly with the tapestry of human expression.

\subsubsection{Output Data}

The output produced by these models is in a particular audio format, resulting from the complex interplay between textual cues and sonic expression. An essential factor in this synthesis is real-time generation, which is critical for both user experience and practical applications.

Real-time generation has the potential to provide immediate results, smoothly converting textual input into audio output. In practical contexts, such as interactive media, virtual environments, or assistive technologies, the capability to respond promptly to user inputs with sonic output enhances the immersive quality of the experience.

The pursuit of real-time generation comes with its share of difficulties and trade-offs. The requirement for swift responsiveness can create limitations that may affect the precision and complexity of the produced audio. The delicate balance between speed and sonic complexity requires careful consideration.

Real-time generation plays a critical role in generative audio synthesis. Achieving real-time outputs requires a balance between immediacy and auditory finesse, guided by studies on human perception. During this pursuit, the generative model attunes itself not only to the cadence of textual cues, but also to the rhythm of human interaction and appreciation.

\subsubsection{Objective Function}

The development of an objective function is a crucial step in this work, where challenges, strategies, and insights come together to shape the core of creative synthesis.

The purpose of generative models is to uncover and encapsulate the complex distributions that form the basis of the data. This journey of comprehension accompanies the aspiration to create new data, which evokes familiarity with the original while testing the limits of artistic innovation. This essence, charming as it may be, emerges within a paradox that challenges us to bridge the gap between accuracy and realism by channeling the learned distributions into outputs that reflect authenticity.

The challenge in defining an objective function lies in extracting the core components of accuracy and realism, and quantifying them into a measurable metric. One approach that aligns with the generative pursuit involves reducing the divergence between the original and synthesized data.

Within the realm of generative literature, numerous strategies intertwine to inspire the creation of objective functions. Approaches presented in Section~\ref{sec:evaluation} should be taken into account.

The objective function emerges as a guiding principle in this interplay of creativity and computation. The objective function embodies the urge for realism and accuracy, reflecting the very essence of generative models in its mathematical grasp. The definition of the objective function is intertwined with the fabric of the generative narrative, uniting aspirations and strategies into a symphony that echoes across the domains of imagination and reality.

\subsubsection{Optimization Algorithm}

The optimization algorithm aims to shape the generative landscape based on the objectives defined within the objective function.

Selecting an optimization algorithm is a crucial decision in this field that needs careful consideration. Each generative model has its own intrinsic characteristics, strengths, and idiosyncrasies, and it is these traits that determine the choice of optimization algorithm. The algorithm guides the process towards optimal solutions and shapes the generative narrative with each iterative step.

Selecting an optimization algorithm is an arbitrated process where the properties of the generative model are tailored to the unique contours of problem landscape. It entails a delicate calibration to adapt the nuances of established algorithms, harmonizing them with the demands of the generative journey. Fine-tuning hyperparameters, sculpting convergence criteria, and managing trade-offs are integral facets of optimization.

The optimization landscape within the explored generative models consists of various algorithms that align with the specific requirements of each model.

Exploring optimization algorithms reveals a landscape that mirrors the fusion of disciplines within generative fields. Gaining insights from seminal works across various optimization methodologies, the optimization algorithm acts as a creative tool, ready to enhance the generative canvas with precision and convergence.

\subsubsection{Evaluation Metrics}

Evaluation metrics are necessary to guide exploration into the realm of generative models.

However, evaluating generative models involves navigating uncharted territories, especially when considering the intricate realm of subjective attributes. The concepts of audio realism, coherence, and emotive resonance are subjective, making it challenging to measure them objectively. Achieving sound realism requires balancing the timbral fidelity and temporal authenticity, which creates a challenge requiring innovative solutions. Similarly, coherence in generative outputs, determined by the harmony among sequential elements, necessitates metrics that can reveal the intricate patterns.

Specific evaluation metrics emerge in this pursuit as guiding stars, each one poised to illuminate a unique facet of generative prowess. Perceptual audio quality metrics, which are rooted in the perceptual space of human hearing, transcend the realm of raw signal processing. They reflect the intricate tapestry of auditory perception. Subjective metrics such as \ac{MOS} are used. Research delves into the visceral nature of human judgment and provides a medium on which subjective impressions are recorded.

Among the many metrics, their relevance, appropriateness, and coherence with the generative narrative are crucial. Metrics anchor the ephemeral expanse of creativity to the solid ground of quantitative analysis.
\section{Application of Synthesizing Soundscapes with Generative AI} \label{sec:significance}

The development of \ac{AI} technologies has enabled significant progress in sound synthesis, enabling the creation of sounds based on textual input. This technology has potential applications in various fields, including music production, film and game sound design, and therapeutic soundscapes. In this response, this Section explores possible applications for the \ac{AI} models described in this document.

One potential application of such \ac{AI} models is in the field of music composition. By generating sounds from textual descriptions, the model could assist composers in generating novel and unique sounds that match a piece's intended mood or atmosphere. For example, a composer might input the phrase ``eerie forest at night'', and the \ac{AI} model could generate a soundscape incorporating the sounds of rustling leaves, distant animal calls, and other eerie sounds one might associate with a forest at night. This technology could help composers to create soundscapes more efficiently that match their creative vision, saving time and increasing their overall output. Besides, the real-time inference characteristic of this model may help a live performer blend the timbres of instruments with, for instance, natural sounds, creating soundscapes on the fly \cite{huzaifah_deep_2021}.

Another possible application of an \ac{AI} model that generates sound from text is in the field of film and game sound design. Sound design plays a crucial role in creating immersive and engaging experiences in film and games, and the ability to generate custom soundscapes from textual input could enhance the creative potential of sound designers. For example, a sound designer might input the phrase ``a bustling city street'', and the \ac{AI} model could generate a soundscape that includes the sounds of car horns, people talking, footsteps, and other city noises. Alternatively, they may want to generate sound cues such as explosions \cite{huzaifah_deep_2021}, and a simple query would do that. This technology could help sound designers create more realistic and immersive soundscapes, improving the overall quality of the final product. 

A panoply of sounds is usually taken from expensive libraries in TV and film. One might imagine an infinite library with \ac{ML} models, and one should imagine how that will enhance the power of sound producers for such endeavors, especially in indie and low-budget productions.
\section{Datasets} \label{sec:prob-datasets}

For the proposed problem, ideal datasets would have entries of sound samples under multiple conditions with labels written in a \ac{NLP} form.

With multiple conditions, one may think, for instance, of multiple sounds of a dog barking. An ideal dataset would have a dog barking in different places, under different atmospheric conditions, and with other possible variants. For example, this ideal dataset would have ``A dog barking in a city'', ``A dog barking in the countryside where it is raining'', and others.

Multiple sounds over or followed by each other would exist in the ideal dataset. For example, ``A dog barking followed by a truck honking'' or ``A traffic jam while a woman sings'' would be examples of the entries in the perfect dataset.

The perfect dataset would also have characteristics of the sounds themselves. For instance, not only ``A dog barking'', but rather ``A dog barking aggressively'', or ``A dog barking with joy'' would be examples of entries in the perfect dataset.

These datasets do not exist in the real world. Thus, some measures must be taken. This discussion is proposed in Section~\ref{sec:sol-datasets}.
\section{Scope, Limitations, and Technical Constraints} \label{sec:limitations}

This study aims to advance the capabilities of audio generation from textual input, at the intersection of audio synthesis and generative \ac{AI}. This research focuses on developing and refining end-to-end generative models that seamlessly translate textual prompts into immersive soundscapes, in the landscape of \ac{AI}-driven audio synthesis.

\subsection{Technical Constraints}

This research in algorithmic audio generation is shaped by a series of pragmatic technical constraints that influence our study's course. These constraints include hardware and computational considerations that are intrinsic to the development and training of sophisticated AI models for audio synthesis.

Contemporary AI research requires significant resource allocation, profoundly impacting our efforts. Studies, such as the development of DALL-E 2 (see Section~\ref{sec:dall-e-2}), vividly illustrate this resource-intensive trajectory.

The training process for this model requires an investment of 100,000 to 200,000 GPU hours, based on the architecture presented in Annex C of the paper. Assuming V100 GPUs are utilized at \$3 per hour (in AWS~\cite{amazon_web_services_inc_amazon_nodate}), the minimum financial commitment amounts to \$300,000. This significant investment highlights the need for substantial resources in the ongoing pursuit of AI advancements. This substantial figure reflects not only the computational demands of AI but also the economic dimensions that drive progress within these domains.

When focusing on audio synthesis, a parallel relationship becomes evident. Generating audio snippets from textual cues poses similar challenges to those encountered in computer vision. Effective audio synthesis relies on robust generative \ac{AI} models for which the training process demands a substantial share of computational resources. These computational resources are not available to individual scientists working alone. As larger models tend to perform better in the generative domain, it is increasingly challenging for scientists and developers to compete against big tech companies.

\subsection{Ethical Considerations}

Ethical considerations play a crucial role in the domain of audio synthesis driven by generative \ac{AI} models. As the possibilities for creative expression continue to grow, it is important to address the ethical aspects that come with the transformative potential of these technologies. This discussion explores the potential ethical implications and highlights the utmost significance of responsible deployment in the field of generative audio synthesis.

One major concern arises in the potential creation of harmful or deceitful content. Generative models' ability to create various audio outputs from text inputs presents opportunities for innovation but also poses inherent risks of producing malicious or deceitful content. The ethical tightrope that creators must balance is underscored by the dual nature of generative prowess.

For example, if an ill-intentioned person utilizes generative audio synthesis to create authentic voice recordings of people, it may result in spreading false information or fabricating forged audio evidence. The remarkable precision of generative models in imitating voices could be misused to commit identity fraud, cause disharmony, or construct digital impersonations that deceive and manipulate unsuspecting listeners.

Moreover, ethical concerns apply to the various sources of textual inputs. Textual clues extracted from public texts, personal submissions, or online repositories are used as raw material for generative processes. These sources require rigorous methods to refine the generative process itself, ensuring that audio outputs are true and precise. Moreover, the responsibility also includes careful curation of data inputs to prevent the perpetuation of biases or prejudiced content. The handling of text inputs may perpetuate ethical implications based on cultural, social, or ideological connotations, which calls for robust mechanisms to prevent the unintended amplification of unethical content.

In studies like this, the researcher does not collect the dataset, but they are responsible for ensuring that the dataset used is unbiased and free of prejudices.

Considering these ethical dimensions, it is clear that the use of generative audio synthesis carries significant ethical responsibilities. While one explores this unexplored territory, it is essential that ethical considerations serve as steadfast sentinels, guiding the responsible use of generative \ac{AI} models in audio synthesis.
\section{Parametric Control} \label{sec:parametric-control}

If one creates a generator model that generates sounds without any input, one would only generate random sounds without any meaning behind them. Without conditioning, WaveNet generates ``babble'' \cite{huzaifah_deep_2021}.

An end-to-end model must be able to take textual input and generate a sound from it. For this, the textual input must be somehow passed to the model during training and inference.

This is called parametric control, sound modeling, or model conditioning. ``Sound modelling is [...] developing algorithms that generate sound under parametric control'' \cite{huzaifah_deep_2021}. This is the ability to manipulate and control the characteristics of the generated audio by adjusting the model parameters. For example, a generative model for musical audio synthesis may allow the user to control the pitch, timbre, or duration of the generated audio by adjusting the specific parameters of the model. The goal of parametric control in generative audio models is to allow the user to fine-tune the characteristics of the generated audio and achieve the desired results. Parametric control is essential in this work because the output must be tailored specifically for the user's input prompt.

This is a crucial job for these types of models and also one of the toughest obstacles in creating digital audio \cite{huzaifah_deep_2021}.

Luckily, this is a solved problem in the data generation realm. Given the quick advances in generative deep learning technologies, every generator production model relies on model conditioning. For instance, transformers receive an input text vector natively. Also, \acp{GAN} can be conditioned on a specific range of inputs. These examples illustrate that it is more than possible to incorporate this kind of technology in an end-to-end model for sound generation.

Modern models for multimodal learning, which involve processing information from different media types such as images, sound, and text, have been based on a variation of \acp{VAE} as seen in sections \ref{sec:data-generators} and \ref{sec:related-work}. One approach to conditioning \acp{VAE} on text involves training different media types, such as images and text, to share a common latent space. This is achieved by optimizing a joint objective function that balances the reconstruction loss of each modality with the alignment of the respective latent spaces.

The text input is first encoded into a latent representation using an encoder network during inference. The image decoder network then decodes this latent representation to generate the output image that corresponds to the given text input.

This is lightly debated in Section~\ref{sec:model}, and more deeply argued in Section~\ref{sec:sol-models}.
\section{Model Requirements and Design} \label{sec:model}

A model that solves the end-to-end problem aims to transform a textual prompt into a sound.

A model to solve this problem must be able to extract a representation of latent features of the given prompt. Then, it must condition a given generator on this representation. This generator must create the sound itself or create a lower-level sound representation. If the latter occurs, another piece must transform this representation into a raw sound, a vocoder.

The generator model needs to be trained with the text embeddings first. So, let us follow the example of training for a single sample: a pair of a sound and the respective natural language label. First, one must translate the sound into a lower-level feature representation. Second, one must have a model that translates the label into a latent features representation, this is, into a text embedding vector. Then, a model that generates a sound representation must be conditioned on these embeddings. Finally, the results of this model must be compared with the lower-level representation of the training sample. This process would train the generator of sound representations. A second process of training the vocoder could be done separately. Other architectures are possible. This simple architectural example would scale well because the two most difficult parts, the generator, and the vocoder, can be trained in parallel and do not depend on one another. Further discussion on this topic is presented in section \ref{sec:sol-models}.

For this architecture to be successful, at least three models need to be trained: the text embedding, the lower-level sound generation, and the vocoder. Vocoders already exist and are a field of relatively intense development, as shown in \ref{sec:vocoders}. Also, text embedding can be already done very well, as \ref{sec:text-embedding} shows. This simplifies the problem as, theoretically, only the generator needs to be developed and trained. The possibility of focusing on this generator is an asset because it eases the workload. However, for best results, one should develop all the models or at least fine-tune them to the specific data that one is working with.
\chapter{Development and Implementation} \label{chap:solution}

\minitoc

This thesis's development and implementation chapter corresponds to the crucial task of developing the solution. This chapter comprehensively explores the methodology, dataset analysis, generative model development, and research plan. This chapter aims to provide a detailed account of the steps taken to achieve the research objectives and contribute to the field of generative AI models for audio.

In Section~\ref{sec:sol-approach}, the author outlines the approach taken in developing this thesis. Section~\ref{sec:sol-datasets} details the various datasets considered in this research. A thorough description and analysis of these datasets is provided, highlighting their characteristics and potential challenges. 

Section~\ref{sec:sol-models} forms the core of this research. There, the author presents the development of several generative models, including the central model, GANmix. The model architecture, training process, and evaluation are discussed, demonstrating their contributions to the field and their potential for generating audio samples.

In addition, Section~\ref{sec:work-plan} outlines the timeline for the development of this thesis. A roadmap of key milestones and stages is provided.

Finally, practical information and resources, such as the source code, are available at \url{https://ctrlmarcio.github.io/audio-gen-ai/}. This allows interested readers to access and explore the codebase, facilitating further research and collaboration.

\section{Methodology and Approach} \label{sec:sol-approach}
 
The development of this dissertation is divided into three main parts:

\begin{itemize}
    \item The state-of-the-art research;
    \item The development;
    \item The writing of the document;
\end{itemize}
 
These areas were not necessarily exclusive. That is, they overlapped. Initially, work began with general research into the state of the art in modeling and audio processing. Early development began after establishing a basic understanding of what needed to be achieved. This development took place in parallel with the ongoing research. As issues arose during development, additional research was required to address them. Writing the document was an ongoing process that led to clearer thinking and planning. Nevertheless, writing was prioritized after the bulk of the development was complete.

The interaction between these phases is both dynamic and symbiotic. The state-of-the-art research phase is the foundation, providing information for future model development decisions. Research insights directly influence the design and direction of generative models. As development progressed, new challenges and possibilities emerged, triggering further exploration in the state-of-the-art literature. Similarly, the development phase provides empirical insights that contribute to refining the writing process. Model development findings and outcomes contribute to comprehensive and informative document content. This iterative interplay ensures that each phase informs, enriches, and refines the others, leading to a coherent and impactful dissertation.

\subsection{State-of-the-Art Research}

One of the goals of this thesis is to provide a comprehensive overview of the current state-of-the-art audio synthesis through textual input. A systematic and thorough approach was taken to search and review the relevant literature.

The process started with the collection of keywords related to the topic of audio synthesis and generative \ac{AI} models. Through multiple combinations, these keywords were then used to search multiple online sources, including academic search engines such as Google Scholar and the Universidade do Porto's. Searches were also conducted for specific papers given the author's knowledge. The results of these searches were analyzed to identify publications with significant citations and high relevance to the thesis topic.

These publications were then carefully read and analyzed, and their citations were further investigated to expand the search and deepen the understanding of the state-of-the-art in the field. If any aspect of a publication was unclear, additional research was conducted to clarify the concept and find additional relevant literature.
This process of searching, reading, and analyzing was iterated multiple times, allowing the gathering of new keywords and publications. Notes and possible citations for each publication were stored in a private database, allowing easy access and organization.

\subsection{Model Development} \label{sec:problem-model-development}

The development of audio synthesis models is a progressive approach that aims to achieve three primary objectives. Together, these goals guide the development of the models and ensure a comprehensive exploration of audio synthesis through text.

The primary objective is establishing fundamental models that facilitate an in-depth understanding of sound representation and \ac{DL} principles. To achieve this, the initial focus was on tasks like audio classification. These fundamental models are developed using smaller datasets like Audio MNIST~\cite{becker_interpreting_2018}. This step provides a foundational grasp of the models' capabilities and acts as a basis for subsequent progress.

The second goal is to develop generative models to synthesize audio from textual input. These generative models are designed to create audio content that aligns with provided text prompts by building on the insights gained from fundamental models. The progressive approach provides the ability for the generative process's iterative refinement, optimization, and enhancement.

The third goal includes proposing theoretical models that establish the foundation for more advanced generative techniques. Although not all theoretical approaches are entirely implemented, these concepts add to the broader discussion of audio synthesis research. Exploring theoretical models boosts innovation and promotes the development of new strategies for generating audio from text.

The model development approach provides an exhaustive and iterative pathway to achieve an effective and innovative audio synthesis from textual input by pursuing these interconnected objectives.

\subsection{Writing of the Dissertation}

This dissertation was composed with a focus on clarity, conciseness, and informative content. A continuous approach was utilized to achieve this objective, consisting of three key phases for each section: initial drafting, incorporation of researched references, and ongoing refinement. This iterative process required multiple re-readings and revisions. This dynamic method aligns well with the principles of agile development~\cite{dingsoyr_decade_2012}, promoting flexibility and continuous progress during the writing process.

Furthermore, this document adheres to established academic norms, encompassing writing style, citational methodologies, and the lucid and consistent presentation of results and figures. In addition to the individual author's contributions, the insights of the thesis coordinators have strengthened this project. They have diligently reviewed and enhanced the document to ensure its precision and authenticity.

This dissertation, crafted with meticulous attention to detail, linguistic technologies, and rigor and scrutiny, is the culminating embodiment of the author's resolute research pursuits and the zenith of scholarly achievement.
\section{Datasets} \label{sec:prob-datasets}

For the proposed problem, ideal datasets would have entries of sound samples under multiple conditions with labels written in a \ac{NLP} form.

With multiple conditions, one may think, for instance, of multiple sounds of a dog barking. An ideal dataset would have a dog barking in different places, under different atmospheric conditions, and with other possible variants. For example, this ideal dataset would have ``A dog barking in a city'', ``A dog barking in the countryside where it is raining'', and others.

Multiple sounds over or followed by each other would exist in the ideal dataset. For example, ``A dog barking followed by a truck honking'' or ``A traffic jam while a woman sings'' would be examples of the entries in the perfect dataset.

The perfect dataset would also have characteristics of the sounds themselves. For instance, not only ``A dog barking'', but rather ``A dog barking aggressively'', or ``A dog barking with joy'' would be examples of entries in the perfect dataset.

These datasets do not exist in the real world. Thus, some measures must be taken. This discussion is proposed in Section~\ref{sec:sol-datasets}.
\section{Generative Model Development} \label{sec:sol-models}

Audio synthesis is the process of creating artificial sounds from scratch. It is a challenging task that requires a deep understanding of the nature and structure of sound, as well as the ability to generate realistic and diverse audio samples. Generative models have been widely used in image synthesis, but their application to audio synthesis is relatively new and less explored.

This section presents the development of a novel generative model for audio synthesis, called GANmix, which combines elements of both a \ac{GAN} and a \ac{VAE}. GANmix exploits the strengths of both models to produce high-quality and diverse audio samples in a computationally efficient manner. GANmix also allows for latent space manipulation, which can be used to control various aspects of the generated audio, such as pitch, timbre, and style.

Before introducing GANmix, this section also describes a series of exploratory experiments that provide a valuable learning phase for the development of the main model. The experiments cover different aspects of audio synthesis, such as audio representation, model architecture, training process, dataset, and evaluation metrics. The experiments aim to provide insight into the performance, limitations, and potential of different generative audio models.

The section is organized as follows: Section~\ref{sec:findings} presents the exploratory experiments that inform the development of GANmix. Section~\ref{sec:ganmix} introduces GANmix as a novel technique for audio synthesis. 

\subsection{Exploratory Experiments} \label{sec:findings}

In the development of generative \ac{AI} models for audio synthesis, it is crucial to conduct extensive experiments that explore various aspects of model architecture, training processes, datasets, and evaluation metrics. This section presents a series of experiments that provide a valuable learning phase before developing the primary model, GANmix. The experiments aim to gain insight into the performance, limitations, and potential of different generative audio models. The author analyzes empirical findings to inform the development of GANmix, aiming for a robust and efficient solution.

The initial experiment focuses on a classification model to uncover the fundamental principles of sound. Training the model on labeled audio data, the author aims to grasp the connections among different audio features and their corresponding categories. This experiment lays the foundation for further research on audio representation.

Based on the results of the classification experiment, the author proceeds to develop a \ac{GAN} for audio synthesis. The aim of this study is to assess the effectiveness of \acp{GAN} in generating high-quality and realistic audio samples.

Furthermore, the author explores the efficacy of \acp{AE} and \acp{VAE} in audio generation. The experiments investigate the reconstruction and generation capabilities of these models, respectively.

The purpose of these experiments is to achieve a comprehensive understanding of both audio and different generative models for audio synthesis. These empirical findings provide a crucial basis for developing GANmix, ensuring a robust and effective solution for audio generation.

\include{src/chapters/4-solution/models/empirical/1-classification}
\include{src/chapters/4-solution/models/empirical/2-gan}
\include{src/chapters/4-solution/models/empirical/3-autoencoder}
\include{src/chapters/4-solution/models/empirical/4-vae}
\subsection{GANmix} \label{sec:ganmix}

This section introduces GANmix as a novel technique for audio generation that fuses components of both a \ac{GAN} and a \ac{VAE}. GANmix addresses the problem of generating high-quality audio under resource constraints.

Audio generation is computationally expensive, especially when constrained by hardware capabilities. GANmix offers a potential solution by combining the strengths of \acp{GAN} and \acp{VAE} to improve audio synthesis in limited computational environments through latent space manipulation.

\paragraph{Model Architecture Plan}

The GANmix architecture consists of a generator and a discriminator that operate in latent space rather than in the space of the sound or soundscape. This approach was inspired by stable diffusion (see Section~\ref{sec:stable-diffusion}), which generates samples through its latent space instead of working on the sample space itself.

Traditionally, in \acp{GAN}, the generator produces samples to be evaluated by the discriminator. However, GANmix takes a different approach: its generator produces values in an embedding space defined by the pre-trained AudioLDM \ac{VAE} encoder~\cite{liu_audioldm_2023}. The discriminator, on the other hand, objectively verifies latent features by comparing the generated features with those obtained by the \ac{VAE} encoder. Samples are generated posteriorly by passing the generated features through the decoder of the \ac{VAE}.

The decision to incorporate the AudioLDM \ac{VAE} model within the GANmix architecture was based on thoughtful reasons.

\begin{enumerate}
    \item \textbf{\ac{VAE} Training:} Training \acp{VAE} poses a challenge since it typically requires significant computational resources and extensive datasets to achieve effective convergence. Previous experiences with model development (in Section~\ref{sec:training-vae}) for this thesis highlighted the intricacies involved in \ac{VAE} training.
    \item \textbf{AudioLDM's High-Performance:} At the time of GANmix's development, the AudioLDM model was one of the top models for audio generation.
    \item \textbf{Accessibility and Open Source Nature of AudioLDM:} One key benefit of integrating the AudioLDM \ac{VAE} model into GANmix architecture was its openness and accessibility through platforms like Hugging Face's model hub through \url{https://huggingface.co/cvssp/audioldm}. This accessibility streamlined the incorporation of the high-performing AudioLDM.
\end{enumerate}

The generator is designed to convert random Gaussian noise vectors into latent features that resemble the encodings of the AudioLDM \ac{VAE} model. Initially, a convolutional model was designed. It used four convolutional transpose 2D layer blocks, with Leaky \ac{ReLU} activation functions following three of them. Each block comprised several convolutional layers. Furthermore, after the deconvolutional layer, two convolutional layers are employed to preserve the data's shape. The final block, responsible for producing the transformed audio samples, did not use an activation function. This design decision enabled the generator to generate unrestricted values, ensuring the accuracy of the produced audio samples.

The decision to utilize 2D convolutions arised from the innate characteristics of the latent features generated by the AudioLDM \ac{VAE} model. These latent features have various dimensions, and integrating 2D convolutional layers enabled GANmix to efficiently grasp complicated patterns across these dimensions. By utilizing 2D convolutions, the architecture could more effectively utilize the multidimensional information in the latent representations, resulting in an improved quality of the generated audio output.

The discriminator architecture mimicked that of the generator with three blocks of convolutional 2D layers using leaky \ac{ReLU} activation. Each block contained multiple convolutional layers (three in the proposed architecture). The final layer of the discriminator applied a sigmoid activation function to generate a probability score indicating the authenticity of the input audio sample.

\paragraph{Experimental Results}

Preliminary experiments with GANmix and the Audio MNIST dataset produced promising but flawed results. Objective analysis reveals that the generator's performance lagged behind the discriminator's, resulting in suboptimal sample quality. Therefore, the training process requires further refinement.

In order to bridge the performance gap between generator and discriminator, a range of refinements were explored. These efforts included exploring different optimizers, such as Adam, RMSProp, and SGD (see Section~\ref{sec:dl-foundations}), each with varying hyperparameters. In addition, the author adjusted model sizes, experimented with diverse loss functions, and introduced noise to training samples. However, the speed of training and comprehensive experimentation were still hindered by hardware resource limitations.

More information regarding these explorations can be found in Chapter~\ref{chap:results}.

With access to a more powerful computing environment, GANmix was further developed using the Clotho dataset (see Section~\ref{sec:clotho}), leading to a significant improvement in the quality of generated audio samples. Nonetheless, there were still issues with achieving equilibrium between the generator and discriminator, even with the upgraded dataset.

\paragraph{Final Model}

After conducting a series of experiments, the GANmix model reached its culmination. This architecture incorporates the Clotho dataset and includes significant improvements to both the generator and discriminator.

The GANmix model is the final model used in this thesis, corresponding to the model used in Experiment 10. Unlike most state-of-the-art generative models that use \acp{CNN}, the GANmix model uses fully connected neural networks for both the generator and the discriminator. The rationale for this was empirical and is given in section~\ref{sec:exp10}.

The generator takes as input a random Gaussian noise vector of size $NZ$ and passes it through a fully connected hidden layer of $NH$ neurons. The output of the hidden layer is then fully connected to another layer of size $EW \times EH \times ED$, where $EW$, $EH$, and $ED$ are the width, height, and number of dimensions of the embeddings generated by the \ac{VAE}, respectively. The output layer is reshaped to match the shape of the \ac{VAE} embeddings.

The discriminator takes as input an embedding of shape $EW \times EH \times ED$ and flattens it to a vector. The vector is then passed through a fully connected hidden layer of $NH$ neurons, and then fully connected to a single output neuron that applies a tanh activation function.

The \ac{VAE} encoder and decoder are pre-trained by a state-of-the-art network, AudioLDM. However, the GANmix model can work with any \ac{VAE} network, as long as the size of the last layer of the generator (and the first layer of the discriminator) is adjusted accordingly.

The loss function used to train the GANmix model is \ac{BCE}. The generator is optimized using Adam with a learning rate of $1 \times 10^{-3}$, while the discriminator is optimized using Adam with a learning rate of $1 \times 10^{-4}$. A scheduler updates the learning rate every 10 epochs.

The implementation of GANmix can be found in Appendix~\ref{ann:ganmix-implementation}. More about the configuration and parameters of the GANmix model can be found in Appendix~\ref{ann:ganmix-conf}. The model's architecture is shown in Figure~\ref{fig:ganmix-architecture}.

\begin{figure}[ht]
    \centering
    \includegraphics[width=\textwidth]{figures/4-solution/ganmix.pdf}
    \caption[GANmix architecture]{\textbf{GANmix architecture} --- On the left is the GANmix generator with three layers: the initial noise layer with $NZ$ neurons, the hidden layer with $NH$ neurons, and the layer corresponding to the embedding space with $EW \times EH \times ED$ neurons. The outputs of this layer are transformed into the form of the outputs of the \ac{VAE}. On the right is the discriminator that mirrors the generator. The last layer has a single neuron with a sigmoid activation function. At the top is a spectrogram that passes through the pre-trained \ac{VAE} generator to generate the latent space. At the bottom, the opposite happens: embeddings in the latent space go through the \ac{VAE} decoder to generate the audio sample in the form of a spectrogram.}
    \label{fig:ganmix-architecture}
\end{figure}

\section{Research Plan} \label{sec:work-plan}

% The work plan refers to the specific tasks and activities that will be carried out in order to execute the approach

% The work plan, on the other hand, is a more detailed plan that outlines the specific tasks and activities that will be carried out in order to implement the approach. This might include tasks such as data collection, data analysis, and the development of any necessary software or tools. The work plan should also include a timeline for completing these tasks and any milestones that need to be achieved along the way.

% the work plan should be described in more detail in the later stages of the project, as the research progresses and the specific tasks and activities become clearer.

\begin{figure}[ht]
    \centering
    \begin{ganttchart}[
            time slot format=isodate,
            expand chart=\textwidth,
            title/.style={draw=none, fill=none},
            bar label node/.append style={align=right},
        ]{2022-09-01}{2023-08-31}
        \gantttitlecalendar{year, month} \\
        
        \ganttbar[bar/.append style={fill=input, draw=none}]{State-of-the-Art}{2022-09-01}{2023-02-15} \\
                    
        \ganttbar[bar/.append style={fill=output, draw=none}]{Introduction}{2022-10-01}{2022-12-31} \\
                    
        \ganttbar[bar/.append style={fill=input, draw=none}]{Gather of Dataset}{2022-11-01}{2023-05-31} \\
        
        \ganttbar[bar/.append style={fill=output, draw=none}]{Defining the Problem}{2022-11-15}{2022-12-15} \\
        
        \ganttbar[bar/.append style={fill=neuron, draw=none}]{Basic Classification}{2023-03-01}{2023-03-31} \\
                    
        \ganttlinkedbar[bar/.append style={fill=neuron, draw=none}]{Basic Generation}{2023-04-01}{2023-05-31} \\
                    
        \ganttlinkedbar[bar/.append style={fill=neuron, draw=none}]{Apply Large Datasets}{2023-06-01}{2023-06-30} \\

        \ganttlinkedbar[bar/.append style={fill=neuron, draw=none}]{Collect Results}{2023-07-01}{2023-08-14} \\

        \ganttbar[bar/.append style={fill=output, draw=none}]{Write Survey Paper}{2023-07-01}{2023-07-31} \\
                    
        \ganttbar[bar/.append style={fill=output, draw=none}]{Dissertation Writing}{2023-05-01}{2023-08-31}
    \end{ganttchart}
    \caption[Work Plan Gantt Chart]{\textbf{Work Plan Gantt Chart} where the dark blue represents researching, light blue represents writing, and green represents developing.}
    \label{fig:gantt}
\end{figure}

This section presents a brief overview of the research tasks, complete with their corresponding timelines and milestones. One should be aware that as the activities and tasks become more defined, the research plan may be altered and enhanced over time.

The primary phase of this research entails conducting an exhaustive examination of existing generative \ac{AI} models for audio. This task requires acquiring a fundamental understanding of the topic, exploring the history of \ac{AI}, and gaining basic knowledge of sound. Furthermore, an extensive study was conducted on the latest advancements in generative models, specifically for audio. The findings of this analysis are presented in the Chapter~\ref{chap:sota}. The interim checkpoint in February 2023, ending of the Dissertation Planning course, mandated the completion of this task.

After conducting a thorough literature review, the next step is to craft the introduction section of the thesis, which involves understanding the research objectives and scope and providing a comprehensive overview. The introduction sets the context, motivation, and objectives of the study. The deadline for completing the introduction is February 2023. 

To aid in the experimentation and development of generative \ac{AI} models for audio, a comprehensive exploration and analysis of prevalent audio datasets was conducted. This included identifying relevant datasets that align with the research objectives and gaining permission and access to these datasets.

Following the assembly of the datasets, the subsequent step involves outlining the problem to be addressed in the thesis. This involves clearly stating the research questions and goals that will direct the development and assessment of these models.

In anticipation of future assignments, a rudimentary dataset to generate a straightforward classification model was utilized. This entailed constructing and training the model, evaluating its effectiveness, and recording the outcomes.

Having established the classification model, the author proceeded to build a foundational audio generation model. This task involved implementing and training the model on the same basic dataset, evaluating its performance, and documenting the findings.

To improve the developed generative models, significant datasets were employed. The models were modified and updated to account for the limitations and intricacies of these larger datasets. The model's performance was assessed, and the outcomes were analyzed and documented.

During this phase, a range of extensive experiments were conducted to evaluate the effectiveness of the developed models. Different techniques were examined, and the models were improved based on the findings. The outcomes of these experiments were carefully examined and recorded.

Subsequently, after concluding the research and experimentation phases, the focus shifted to drafting the thesis. This involved consolidating the research results, analyzing their significance, and producing an organized paper that neutrally presents the research methodology, findings, and outcomes.

In addition to the dissertation, a survey paper was composed that outlines the latest developments in generative \ac{AI} models for soundscapes. The paper provides an extensive synopsis of advancements in the domain and contributes significantly to the scholarly audience.

It is important to note that the deadline for completing these tasks has been extended until September 1st due to the challenges in obtaining permission to access the \ac{LIACC} servers. As the research progressed, the schedule was reviewed and updated to ensure accuracy and importance.
\chapter{Evaluation and Discussion} \label{chap:results}

\minitoc

The assessment of generative \ac{AI} models for audio is pivotal in advancing the field of audio generation and synthesis. It is imperative to comprehend the capabilities and limitations of these models for their effective development and implementation. This chapter presents the outcomes and analysis of experiments performed with the GANmix architecture, utilizing objective metrics based on loss functions. The aim of this study is to assess the efficacy of generative \ac{AI} models in producing audio, and to draw insights from their performance.

The experiments were conducted utilizing different hardware configurations, For \acp{GPU}, it consisted of Kaggle's Tesla V100, \ac{LIACC}'s GeForce GTX 1080, and \ac{LIACC}'s GPU Quadro RTX 8000. These configurations, along with the utilization of the PyTorch deep learning framework, allowed for proficient training and assessment of the generative \ac{AI} models.

In this chapter, the author describes the experimental setup in Section~\ref{sec:res-setup}, comprising both hardware and software configurations, alongside the GANmix architecture. The experiments' outcomes are presented in Section~\ref{sec:res-presentation} with line plots exhibiting the generator and discriminator's loss evolution per epoch. Additionally, the resulting spectrograms are presented to offer a graphical depiction of the produced audio.

Following the presentation of results, the author analyzes and interprets the findings in Section~\ref{sec:res-analysis}, identifying trends, patterns, and significant insights that emerged from the experiments. Comparisons are conducted among various models and variations in the GANmix framework for evaluating their respective performances.

Furthermore, in Section~\ref{sec:limitations}, the author addresses any limitations or challenges encountered during the evaluation process. These factors may include hardware limitations or other constraints that impact the performance of the models.

\section{Experimental Setup} \label{sec:res-setup}

This section outlines the experimental setup utilized to assess the audio generation capabilities of the GANmix model.

The GANmix model was trained using the \ac{BCE} loss, which was implemented through pytorch's \texttt{BCEWithLogitsLoss} module. Although \ac{BCE} is commonly employed for binary classification, it can also effectively train the generator in \acp{GAN}. In the context of GANmix, the generator's goal is to create authentic embeddings that can fool the discriminator into classifying them as authentic. To achieve this, the generator's loss is calculated using the discriminator's output for the fake embeddings generated by the generator and the true label --- either real or fake --- for a real embedding.

By using \ac{BCE}, the generator calculates its loss by measuring the difference between the discriminator's classification of genuine and false embeddings. This loss directs the generator to generate more realistic embeddings over time by minimizing this difference. Therefore, despite being a binary classification loss function, \ac{BCE} is a suitable approach for training the generator in \acp{GAN}.

Two datasets, the Audio MNIST dataset~\cite{becker_interpreting_2018} and the Clotho dataset~\cite{drossos_clotho_2019}, were utilized in these experiments. These datasets were previously described in detail in Section~\ref{sec:sol-datasets}.

The Audio MNIST dataset contains short audio clips where each clip represents a spoken digit. This dataset sets a standard for evaluating audio classification tasks.

In contrast, the Clotho dataset is more extensive and diverse. It offers a variety of audio samples including environmental sounds and speech from various sources. This dataset offers a diverse and comprehensive range of audio data, allowing the GANmix model to learn and generate audio samples that capture the complexity and diversity present in real-world audio recordings.

For a thorough understanding of the datasets, including their characteristics, preprocessing steps, and data augmentation techniques, please refer to Section~\ref{sec:sol-datasets}.

The experiments were conducted on three distinct hardware configurations, referred to as \textit{Kaggle}, \textit{\ac{LIACC} 1}, and \textit{\ac{LIACC} 2}, each selected based on the available resources during the development process.

Kaggle, the initial configuration implemented for the GANmix model development phase, utilized a Tesla V100 \ac{GPU} equipped with 12 \ac{GB} of memory, 73.1 \ac{GB} of disk space, and 13 \ac{GB} of \ac{RAM}.

As development progressed, the \ac{LIACC} 1 became available, which offered around the same computational power, \ac{LIACC} 1 used a GeForce GTX 1080 \ac{GPU} with 8 \ac{GB} of memory, 50 \ac{GB} of disk space, and 32 \ac{GB} of \ac{RAM}.

In the final stages of the project, the third configuration, \ac{LIACC} 2, was made available. \ac{LIACC} 2 employed a \ac{GPU} Quadro RTX 8000 with 48 \ac{GB} of memory, 50 \ac{GB} of disk space, and 128 \ac{GB} of \ac{RAM}, allowing for extensive training and evaluation of the GANmix model.

These hardware configurations were chosen to facilitate GANmix model training and evaluation throughout the developmental stages.

The GANmix model was implemented using the Python programming language and the PyTorch deep learning framework. There is a further discussion about this framework in Section~\ref{sec:dl-frameworks}.

Before training, a preprocessing step was applied to the audio data by randomly Random cropping was done with a set duration of 5 seconds. Random cropping is discussed further in Section~\ref{sec:findings}. This approach enabled a more diverse dataset since the majority of its samples had longer durations. Random cropping aided in capturing diverse segments of the audio and improved the model's ability to generate realistic audio samples.

The GANmix model underwent training using a set of specific hyperparameters. The batch size ranged between 1 and 32, providing varying trade-offs between computational efficiency and model convergence. The training epochs varied from a few tens to a few hundreds, depending on the dataset and model complexity. 

The learning rates utilized to train the GANmix model ranged from $1 \times 10^{-5}$ to $1 \times 10^{-2}$, chosen based on empirical observations and prior research.

For some models, the training process was halted before reaching the maximum number of epochs due to evidence of convergence. This choice was made to optimize computational resources while obtaining satisfactory outcomes.
\documentclass{beamer}

% DELETE THE NEXT LINE FOR THE PRESENTATION
\setbeameroption{show notes on second screen=right}

\usetheme{MyTheme}

\input{config/packages.tex}

\addbibresource{config/references.bib}

\input{config/meta.tex}

\begin{document}

\begin{frame}
  \titlepage
\end{frame}

\begin{frame}[t]
  \frametitle{Outline}
  \begin{columns}[T]
    \begin{column}{.5\textwidth}
      \tableofcontents[sections={1-2}]
    \end{column}
    \begin{column}{.5\textwidth}
      \tableofcontents[sections={3-}]
    \end{column}
  \end{columns}
\end{frame}
\chapter{Introduction} \label{chap:intro}

\minitoc

Audio is essential for shaping human experiences and interactions in the digital age. Audio content, ranging from podcasts and music to sound effects and immersive environments, is pervasive in our lives and enhances multimedia experiences. However, generating high-quality, diverse, and contextually relevant audio is still a time-consuming and labor-intensive process. As the demand for audio content continues to increase, there is a need for innovative solutions that can simplify the process and enable creators to generate various audio content with less effort.

Recent advances in \ac{AI} and \ac{DL} are revolutionizing different domains, such as image generation and text-to-speech synthesis. These cutting-edge models exhibit exceptional abilities to generate high-quality content from basic textual inputs. This thesis is motivated by these successful models and explores the potential of \ac{AI}-driven generative models to synthesize audio snippets based on textual descriptions. This thesis is not constrained to specific domains, like music or speech, but instead studies and implements systems suitable for a wide range of audio prompts.

The motivation behind this work lies in the potential benefits that an end-to-end audio generative model can offer content creators, sound engineers, and other stakeholders in the audio production ecosystem. By reducing the time and effort required to generate unique and high-quality audio samples, this research aims to contribute to the democratization of audio content creation and facilitate new avenues for creative expression.

This chapter provides an introduction to the research study and sets the stage for the exploration of the topic. It aims to supply a comprehensive overview of the present study's context, motivation, objectives, and structure.

The context information can be found in section \ref{sec:context}. Its focus is to provide the necessary background for the research study and to explain its importance. Section \ref{sec:motivation} explores the motivation behind the study. The purpose of this section is to explain the necessity of the research and the knowledge gap it aims to address. Next, in section \ref{sec:objectives}, the research study's objectives are described. This section aims to explain the specific objectives and outcomes the research seeks to accomplish. The structure of the dissertation is presented in Section \ref{sec:struct}. This section offers an organized overview of the remaining chapters.

\input{src/chapters/1-introduction/context}
\input{src/chapters/1-introduction/motivation}
\input{src/chapters/1-introduction/objectives}
\input{src/chapters/1-introduction/struct}

\section{State of the Art}

\subsection{Introduction}
\begin{frame}
    \frametitle{State of the Art}
    \begin{itemize}
        \item There are different types of sound generation methods, depending on the level of abstraction and supervision involved
        \item Four types of sound generation methods are considered in this thesis:
              \begin{itemize}
                  \item Traditional
                  \item Unsupervised
                  \item Vocoders
                  \item End-to-end models
              \end{itemize}
    \end{itemize}
\end{frame}

\subsection{Traditional Soundscape Generation}

\begin{frame}
    \frametitle{Traditional Soundscape Generation}

    \textbf{Notable Tools}
    \begin{itemize}
        \item \textit{Scaper}~\cite{salamon_scaper_2017}: Open-source library for synthetic sound environments
        \item SEED~\cite{bernardes_seed_2016}: System for resynthesizing environmental sounds with precise control over variation
        \item Physics-Based Concatenative Sound Synthesis~\cite{magalhaes_physics-based_2020}: Creates novel auditory experiences by assembling pre-existing sound segments
    \end{itemize}
\end{frame}

\subsection{Unsupervised Sound Generation}

\begin{frame}
    \frametitle{Unsupervised Sound Generation}

    \textbf{Approach}
    \begin{itemize}
        \item Learn sound features and distributions without explicit labels
        \item Utilize unlabeled audio data for pattern capture and structure learning
        \item Valuable when labeled datasets are limited or costly
    \end{itemize}

    \textbf{Notable Models}
    \begin{itemize}
        \item WaveGAN~\cite{donahue_adversarial_2019}: Unsupervised waveform synthesis using modified GAN
        \item Generative Transformer~\cite{verma_generative_2021}: Autoregressive prediction of audio samples using transformer networks
        \item wav2vec 2.0~\cite{baevski_wav2vec_2020}: Speech generation model with convolutional feature encoder, Transformer, and quantization module
        \item SoundStream~\cite{zeghidour_soundstream_2021}: Neural audio codec for efficient audio compression
    \end{itemize}
\end{frame}


\subsection{Vocoders}

\begin{frame}
    \frametitle{Vocoders}

    \textbf{Notable Models}
    \begin{itemize}
        \item WaveNet~\cite{oord_wavenet_2016}: Generative neural network using dilated causal convolutions for raw audio waveform generation
        \item WaveNet Variants: Models like WaveRNN, FloWaveNet, and Fast WaveNet reduce complexity while maintaining effectiveness
        \item MelGAN~\cite{kumar_melgan_2019}: GAN-based model using Mel-Spectrograms for coherent audio waveform generation
        \item GANSynth~\cite{engel_gansynth_2019}: GAN using log-magnitude spectrograms and phases for waveform generation
        \item HiFi-GAN~\cite{kong_hifi-gan_2020}: GAN model combining efficiency and high-fidelity speech synthesis
    \end{itemize}
\end{frame}


\subsection{End-to-End Models}


\begin{frame}[allowframebreaks]
    \frametitle{End-to-End Audio Models Comparison}

    \begin{table}[ht]
        \centering
        \caption{A comparison of different end-to-end generative models for audio.}
        \begin{tabularx}{\textwidth}{|X|l|X|X|}
            \hline
            \textbf{Model}                           & \textbf{Type} & \textbf{Input}            & \textbf{Output}                        \\ \hline
            Char2wav~\cite{sotelo_char2wav_2017}     & Speech        & Text prompt               & Raw audio waveform                     \\ \hline
            VALL-E~\cite{wang_neural_2023}           & Speech        & Text and acoustic prompt  & Raw audio waveform                     \\ \hline
            Jukebox~\cite{dhariwal_jukebox_2020}     & Music         & Genre, artist, and lyrics & Raw audio waveform                     \\ \hline
            Riffusion~\cite{forsgren_riffusion_2022} & Music         & Text prompt               & Raw audio waveform                     \\ \hline
            MusicLM~\cite{agostinelli_musiclm_2023}  & Music         & Text prompt               & Raw audio waveform                     \\ \hline
            SampleRNN~\cite{mehri_samplernn_2017}    & General       & None                      & Raw audio waveform                     \\ \hline
            AudioLM~\cite{borsos_audiolm_2022}       & General       & Text prompt               & Raw audio waveform                     \\ \hline
            DiffSound~\cite{yang_diffsound_2022}     & General       & Text prompt               & Mel-spectrogram and raw audio waveform \\ \hline
            AudioGen~\cite{kreuk_audiogen_2023}      & General       & Text prompt               & Mel-spectrogram and raw audio waveform \\ \hline
        \end{tabularx}
        \label{tab:end-to-end-audio-models}
    \end{table}
\end{frame}

\begin{frame}
    \frametitle{Text-to-Speech (TTS)}

    \textbf{Definition}
    \begin{itemize}
        \item Convert written text into synthesized speech
        \item Use deep neural networks for direct mapping
        \item Notable TTS Models:
              \begin{itemize}
                  \item Char2wav
                  \item VALL-E
              \end{itemize}
    \end{itemize}
\end{frame}

\begin{frame}
    \frametitle{Generative Music}

    \textbf{Definition}
    \begin{itemize}
        \item Create music using generative techniques
        \item End-to-end models for composing new musical pieces
        \item Notable Generative Music Models:
              \begin{itemize}
                  \item Jukebox
                  \item Riffusion
                  \item MusicLM
              \end{itemize}
    \end{itemize}

\end{frame}

\begin{frame}
    \frametitle{General Text-to-Audio}

    \textbf{Definition}
    \begin{itemize}
        \item Convert various forms of text to corresponding audio outputs
        \item Applications: sound effects, voice transformation, environmental sound synthesis
        \item Notable Text-to-Audio Models:
              \begin{itemize}
                  \item SampleRNN
                  \item AudioLM
                  \item DiffSound
                  \item AudioGen
              \end{itemize}
    \end{itemize}

\end{frame}


\subsection{AudioLM}
\begin{frame}{AudioLM}
    \begin{itemize}
        \item A framework for high-quality audio generation with long-term consistency~\cite{borsos_audiolm_2022}.
        \item Maps input audio to a sequence of discrete tokens and treats audio generation as a language modeling task.
        \item Achieves high-quality synthesis and long-term structure through a hybrid tokenization scheme of semantic and acoustic tokens.
        \item Consists of three main components: tokenizer, language model, and detokenizer.
        \item Generates syntactically and semantically plausible speech and music continuations without any transcript or annotation.
    \end{itemize}
\end{frame}

\subsection{DiffSound}

\begin{frame}{DiffSound}
    \begin{itemize}
        \item A novel text-to-sound generation framework that uses a text encoder, a VQ-VAE, a decoder, and a vocoder~\cite{yang_diffsound_2022}.
        \item Takes text as input and outputs synthesized audio corresponding to the input text.
        \item Uses a diffusion decoder (DiffSound) that predicts and refines all Mel-Spectrogram tokens in one step, resulting in better and faster generation than an AR decoder.
        \item Produces high-quality sound synthesis for various domains such as speech, music, and environmental sounds.
    \end{itemize}
\end{frame}


\subsection{AudioGen}
\begin{frame}{AudioGen}
    \begin{itemize}
        \item An auto-regressive generative model that generates audio samples conditioned on text inputs~\cite{kreuk_audiogen_2023}.
        \item Learns a discrete representation of the raw audio using an AE method and trains a Transformer language model over the learned codes, conditioned on textual features.
        \item Uses an augmentation technique that mixes different audio samples to train the model to separate multiple sources internally.
        \item Explores the use of multi-stream modeling for faster inference, allowing the use of shorter sequences while maintaining a similar bitrate and perceptual quality.
        \item Outperforms evaluated baselines over both objective and subjective metrics and extends to conditional and unconditional audio continuation.
    \end{itemize}
\end{frame}

\section{Development}

\subsection{Datasets}

\begin{frame}
    \frametitle{Table of Datasets}

    \begin{table}[ht]
        \centering
        \caption{Comparison of datasets for soundscapes}
        \label{tab:datasets}
        \begin{tabularx}{\textwidth}{|X|X|X|X|X|}
            \hline
            \textbf{Name}                                     &
            \textbf{Type}                                     &
            \textbf{\# Samples}                               &
            \textbf{Duration}                                 &
            \textbf{Labels}                                     \\ \hline

            Acoustic Event Dataset \cite{takahashi_deep_2016} &
            Categorical labeled                               &
            5223                                              &
            Average 8.8s                                      &
            One of 28 labels                                    \\ \hline

            AudioCaps \cite{kim_audiocaps_2019}               &
            Descriptive labeled                               &
            39597                                             &
            10s each                                          &
            9 words per caption                                 \\ \hline

            AudioSet \cite{gemmeke_audio_2017}                &
            Categorical labeled                               &
            2084320                                           &
            Average 10s                                       &
            One or more of 527 labels                           \\ \hline

            Audio MNIST~\cite{becker_interpreting_2018}       &
            Categorical labeled                               &
            30000                                             &
            Average 0.6s                                      &
            One of 10 labels                                    \\ \hline
        \end{tabularx}
    \end{table}
\end{frame}

\begin{frame}
    \frametitle{Table of Datasets (contd.)}

    \begin{table}[ht]
        \centering
        \caption{Comparison of datasets for soundscapes (contd.)}
        \begin{tabularx}{\textwidth}{|X|X|X|X|X|}
            \hline
            \textbf{Name}                                           &
            \textbf{Type}                                           &
            \textbf{\# Samples}                                     &
            \textbf{Duration}                                       &
            \textbf{Labels}                                           \\ \hline

            Clotho \cite{drossos_clotho_2019}                       &
            Descriptive labeled                                     &
            4981                                                    &
            15 to 30s                                               &
            24 905 captions (5 per audio).                            \\ \hline

            FSDKaggle 2018 \cite{fonseca_general-purpose_2018}      &
            Categorical labeled                                     &
            11073                                                   &
            From 300ms to 30s                                       &
            One or more of 41 labels                                  \\ \hline

            Urban Sound 8K \cite{salamon_dataset_2014}              &
            Categorical labeled                                     &
            8732                                                    &
            Less or equal to 4s                                     &
            One of 10 labels                                          \\ \hline

            YouTube-8M Segments \cite{abu-el-haija_youtube-8m_2016} &
            Categorical labeled                                     &
            237000                                                  &
            5s                                                      &
            One or more of 1000 labels                                \\ \hline
        \end{tabularx}
    \end{table}
\end{frame}


\subsection{Exploratory Experiments}

\begin{frame}
    \frametitle{Exploratory Experiments}

    \begin{itemize}
        \item Objective: Gain insights for GANmix development
        \item Experiments: Classification, GAN, AE, VAE
        \item Findings: Foundation for audio representation, GAN effectiveness, AE/VAE capabilities
        \item Impact: Crucial for robust audio generation with GANmix
    \end{itemize}

\end{frame}

\subsection{GANmix}

\begin{frame}{GANmix}

    \begin{figure}
        \centering
        \includegraphics[height=0.8\textheight]{images/3-development/ganmix.pdf}
        \caption{GANmix architecture}
        \label{fig:ganmix}
    \end{figure}

    \note{
        \textbf{Introduction}
        \begin{itemize}
            \item GANmix: Fusion of GAN and VAE for audio generation under constraints.
            \item Addresses computational limitations for high-quality audio.
            \item Combines GAN's generative power with VAE's latent space manipulation.
        \end{itemize}

        \textbf{Model Architecture}
        \begin{itemize}
            \item Generator and discriminator operate in latent space.
            \item VAE Training: Computational challenge, requires extensive datasets.
            \item AudioLDM's High-Performance: Top model for audio generation.
            \item Accessibility of AudioLDM: Open source, accessible via Hugging Face's model hub.
        \end{itemize}

        \textbf{Early Results}
        \begin{itemize}
            \item Preliminary experiments with Audio MNIST: Promising but suboptimal.
            \item Refinements: Different optimizers, model sizes, loss functions.
            \item Clotho dataset: Significant improvement in generated audio quality.
            \item Challenges in achieving equilibrium between generator and discriminator.
        \end{itemize}

        \textbf{Final Model}
        \begin{itemize}
            \item GANmix architecture with Clotho dataset: Significant improvements.
            \item Unlike typical models using CNN, GANmix uses fully connected neural networks.
            \item Generator input: Random Gaussian noise, passes through hidden layers.
            \item Discriminator: Takes embedding as input, applies tanh activation.
            \item Loss function: BCE. Optimized with Adam. Learning rate updates every 10 epochs.
        \end{itemize}
    }
\end{frame}

\section{Results}

\begin{frame}
    \frametitle{Results}

    \begin{itemize}
        \item Setup: Overview of experimental conditions and configurations.
        \item Presentation of Results: Showcase of outcomes from GANmix experiments.
        \item Discussion: Analyzing and interpreting the obtained results.
        \item Constraints and Challenges: Addressing limitations and difficulties encountered.
    \end{itemize}
\end{frame}


\subsection{Setup}

\begin{frame}
    \frametitle{Experimental Setup}

    \begin{itemize}
        \item GANmix model trained using BCE loss in PyTorch.
        \item Utilized two datasets: Audio MNIST and Clotho for diverse training.
        \item Three hardware setups: Kaggle, LIACC 1, LIACC 2, tailored for resources.
        \item Implementation: Python, PyTorch framework.
        \item Preprocessing: Randomly cropped samples to 5 seconds for diversity.
        \item Hyperparameters adjusted for batch size, epochs, and learning rates.
        \item Stopped training based on convergence for resource efficiency.
    \end{itemize}
\end{frame}

\subsection{Presentation of Results}

\begin{frame}
    \frametitle{Presentation of Results}

    \begin{itemize}
        \item Objectives: Explore generative AI models for audio production, assess performance.
        \item Context: Each experiment designed with specific research questions and hypotheses.
        \item Methodology: Hardware, software, and architecture details provided for transparency.
        \item Evaluation: Performance assessed through evolving loss plots and spectrograms.
        \item Systematic Organization: Ensures a comprehensive understanding of procedures and results.
        \item Basis for Analysis: Provides foundation for discussing effectiveness of generative AI models.
    \end{itemize}
\end{frame}

\subsection{Experiment X: Title} \label{sec:expX}

\begin{frame}
    \frametitle{Experiment X: Title}

    \begin{itemize}
        \item Objectives: [Objectives of the experiment]
        \item Model Details: [Describe key details of the model used, e.g., parameters, loss function]
        \item Dataset: [Mention the dataset used for training and evaluation]
        \item Optimizer and Learning Rate: [Specify the optimizer and learning rate used]
        \item Training Process: [Provide essential details about the training process, e.g., convergence status]
    \end{itemize}

    [Optional: Any unique aspects or considerations for this experiment]

    % \begin{figure}[!ht]
    %     \centering
    %     \begin{subfigure}{0.45\textwidth}
    %         \includegraphics[width=\textwidth]{figures/4.5-results/expX_loss.png}
    %         \caption{Evolving losses throughout the training process for Experiment X.}
    %         \label{fig:expX_loss}
    %     \end{subfigure}
    %     \begin{subfigure}{0.45\textwidth}
    %         \includegraphics[width=\textwidth]{figures/4.5-results/expX_spectrogram.png}
    %         \caption{Spectrogram generated in Experiment X.}
    %         \label{fig:expX_spectrogram}
    %     \end{subfigure}
    %     \caption{Results of Experiment X.}
    %     \label{fig:expX_results}
    % \end{figure}

    [Optional: Any additional insights or observations from this experiment]

\end{frame}

\subsection{Discussion}

\section{Analysis and Interpretation} \label{sec:res-analysis}

\begin{frame}
    \frametitle{Analysis and Interpretation}
    
    \begin{itemize}
        \item Identifying Trends
        \item Results for Future Investigation
        \item Interpretation of Results
        \item Conclusion
    \end{itemize}
    
\end{frame}

\begin{frame}
    \frametitle{Identifying Trends}

    \begin{itemize}
        \item Inverse correlation between generator and discriminator losses
        \item Convergence tends to plateau after a certain number of epochs
        \item Impact of learning rate on convergence speed
        \item Influence of optimization algorithms (e.g., SGD, RMSprop, Adam)
        \item Benefits of regularization methods (e.g., dropout, batch normalization, Gaussian noise)
        \item Importance of dataset size
        \item Exploration of latent space
    \end{itemize}
    
\end{frame}

\begin{frame}
    \frametitle{Results for Future Investigation}

    \begin{itemize}
        \item Occurrence of performance decline and NaN losses in certain experiments
        \item Further exploration of elastic network regularization
        \item Investigation of continuously increasing generator loss
    \end{itemize}
    
\end{frame}

\begin{frame}
    \frametitle{Interpretation of Results}

    \begin{itemize}
        \item Results didn't meet initial expectations but show potential
        \item Latent space exploration as a promising strategy
        \item Limitations of small datasets, especially in audio length and quantity
        \item Need for access to comprehensive datasets
        \item Computational resource challenges
    \end{itemize}
    
\end{frame}

\begin{frame}
    \frametitle{Conclusion}

    \begin{itemize}
        \item Analysis and interpretation of trends and patterns
        \item Potential for future advances in generative AI models for audio synthesis
        \item Lack of satisfactory practical results due to dataset limitations
        \item Importance of comprehensive datasets and computational resources
    \end{itemize}
    
\end{frame}

\subsection{Constraints and Challenges}

\section{Constraints and Challenges} \label{sec:res-limitations}

\begin{frame}
    \frametitle{Constraints and Challenges}
    
    \begin{itemize}
        \item Hardware Resources
        \item Data Quality and Quantity
        \item Hyperparameter Tuning
    \end{itemize}
    
\end{frame}

\begin{frame}
    \frametitle{Hardware Resources}

    \begin{itemize}
        \item Scarcity of hardware resources for training and evaluation
        \item Challenges in accessing sufficient computing power and memory
        \item Strategies adopted to optimize hardware usage
        \item Impacts, trade-offs, and opportunities resulting from resource limitations
    \end{itemize}
    
\end{frame}

\begin{frame}
    \frametitle{Data Quality and Quantity}

    \begin{itemize}
        \item Challenges posed by the quality and quantity of available data
        \item Importance of high-quality and diverse data for generative models
        \item Strategies employed to mitigate data limitations
        \item Considerations regarding data augmentation techniques
    \end{itemize}
    
\end{frame}

\begin{frame}
    \frametitle{Hyperparameter Tuning}

    \begin{itemize}
        \item Time constraints and challenges in hyperparameter tuning
        \item Significance of hyperparameters in model performance
        \item Impact of default or arbitrary values on model potential
        \item Recommendations for future work in hyperparameter optimization
    \end{itemize}
    
\end{frame}

\begin{frame}
    \frametitle{Conclusion}

    \begin{itemize}
        \item Discussion of major limitations and challenges faced in solution development
        \item Description of strategies employed to address these issues
        \item Possible implications, trade-offs, and opportunities arising from constraints
        \item Affirmation of the proposed solution's strengths and advancements in generative AI models for audio synthesis
    \end{itemize}
    
\end{frame}

\section{Conclusions}

\subsection{Overview and Reflections}

\begin{frame}
    \frametitle{Overview and Reflections}
    \begin{itemize}
        \item Comprehensive Study of State-of-the-Art Deep Learning Architectures for Audio Synthesis
        \item Development of End-to-End Systems for Sound Synthesis and Evaluation
        \item Challenges and Lessons Learned
    \end{itemize}
\end{frame}

\subsection{Future Directions}

\begin{frame}
    \frametitle{Future Directions}
    \begin{itemize}
        \item Exploring Novel Architectures
        \item Dataset Expansion
        \item Evaluation Metrics
    \end{itemize}
\end{frame}

\subsection{Novel Architectures}


\begin{frame}
    \frametitle{Novel Architectures}
    \begin{itemize}
        \item Briefly introduce the proposed theoretical architectures
        \item Mention their objectives, design principles, and potential applications
        \item Highlight the need for future research and development in this area
    \end{itemize}
\end{frame}

\subsection{Conclusion}

\begin{frame}
    \frametitle{Conclusion}
    \begin{itemize}
        \item Summarize the main achievements and contributions of the research
        \item Emphasize the progress made in understanding and developing generative AI models for audio synthesis
        \item Acknowledge the challenges and ongoing work required for further advancements
    \end{itemize}
\end{frame}


\end{document}
\section{Analysis and Interpretation} \label{sec:res-analysis}

This section analyzes and assesses the results of experiments conducted on GANmix. The evaluation identifies trends, investigates potential future directions for research, and interprets the results in the context of the research goals and the broader field.

\subsection{Identifying Trends}

Examination of the experimental results reveals several patterns and trends. There is an inverse correlation between generator and discriminator losses; as one increases, the other decreases. However, in certain cases both losses decrease together, which is beneficial. Also, convergence tends to plateau after a certain number of epochs.

This can be seen in Figure~\ref{fig:exp1_loss}. There, it can be seen that the loss of the generator decreases as the loss of the discriminator increases, and then the opposite occurs. Eventually both losses level off and decrease slightly.

The study found that the speed of convergence was affected by the learning rate. Although higher learning rates led to a faster convergence plateau, they did not necessarily lead to better results. This was demonstrated in Experiment 2 (Section~\ref{sec:exp2}), where the learning rates were increased and a loss plot is observed that is similar to Experiment 1 (\ref{sec:exp1}), but with a significantly faster plateau.

In Experiment 4 (Section~\ref{sec:exp4}), it appears that \ac{SGD} initially outperforms RMSprop, although it learns at a significantly slower rate. The results seem to be similar to Adam. Given the limited training time, one can only make assumptions, but it seems that using \ac{SGD} as the optimization algorithm leads to better performance than alternative methods such as RMSprop and Adam, despite the slower convergence.

Regularization methods such as dropout, batch normalization, and Gaussian noise have been shown to improve results and extend convergence, as shown in Experiment 5 (\ref{sec:exp5}). Although the spectrogram was suboptimal, the loss curves showed a healthy trend, with both losses decreasing consistently over time. Although the application of the elastic net regularization presented some challenges, it showed promise. The model performed better overall and achieved faster convergence when the generator and discriminator had similar parameter sets.

It was determined that larger models resulted in more rapid and resilient convergence. However, it should be noted that achieving satisfactory results was critically impacted by the size of the dataset. Typically, larger datasets and models produced better outcomes.

The embeddings generated by AudioLDM's \ac{VAE} show similarities characterized by a normal-like distribution with a significantly low standard deviation. Figure~\ref{fig:original-latents-hist} shows a histogram of these values, which extend to hundreds on the x-axis due to the residuals. It can be seen that the distribution is predominantly concentrated in an area close to zero.

\begin{figure}[ht]
    \centering
    \includegraphics[width=0.6\textwidth]{figures/4.5-results/real_distribution.png}
    \caption{An histogram that represents the latent values created by AudioLDM's \ac{VAE}.}
    \label{fig:original-latents-hist}
\end{figure}

Experience 10 (\ref{sec:exp10}) showed an interesting trend. However, it is worth noting that although the generated spectrograms showed improvement, the generator loss increased while the discriminator loss consistently decreased after a few epochs, as can be seen in Figure~\ref{fig:exp10_loss}. The use of linear layers instead of convolutions proved to be advantageous in generating more robust spectrograms. Further investigation is needed to determine if there are problems with the optimizer.

This last experiment, which used linear layers, produced the most encouraging results and was chosen as the primary study for this dissertation.

\subsection{Results for Future Investigation}

In some experiments, a decline in performance after a few epochs was observed, resulting in losses becoming \ac{NaN}. This is seen in Experiments 8 and 9 (Sections~\ref{sec:exp8}, \ref{sec:exp9}). Further research is required to determine the underlying cause and prevent this occurrence in future experiments.

The impact of elastic network regularization on model performance requires further investigation. Despite the implementation challenges encountered, the initial results suggest that this regularization method has improved and shows potential effectiveness.

In addition, the issue of the continuously increasing generator loss in the last experiment requires further investigation in the future.

\subsection{Interpretation of Results}

The study's findings did not meet the expectations set forth in the thesis as the generated sounds are not state-of-the-art for generative models; however, they are encouraging. With sufficient data and time, the model has the potential to generate high-quality sound. This suggests that generative \ac{AI} models, especially \acp{GAN}, have significant capabilities in audio generation.

Furthermore, exploring the model's latent space is seen as a promising strategy for achieving better results. The latent space method is a simpler approach that can potentially yield more favorable outcomes for future models.

The limitations of the small datasets used in this analysis are evident. The bigger one, Clotho, comprised sounds lasting from 15 to 20 seconds, but with 5 seconds removed from each sound, certain sounds were too short to produce satisfactory results.  Additionally, the total sound count in the data set was less than 5,000, which isn't enough to properly train a generative model.

Thus, a significant concern uncovered in this study is the lack of an appropriate dataset. To achieve more precise results, it is critical to have access to comprehensive datasets. Nevertheless, it is imperative to recognize that the process of training models on large data sets involves significant computational resources that may not be readily available.

\subsection{Conclusion}

In summary, the analysis and interpretation of the data show trends and patterns observed in the experiments. The results did not meet initial expectations; however, they show potential for future advances in generative \ac{AI} models for audio synthesis. It is important to note that comparisons with state-of-the-art audio generation models are not discussed in this study due to the unsatisfactory practical results obtained. 
\section{Constraints and Challenges} \label{sec:res-limitations}

This section discusses the major constraints and challenges faced in developing the proposed solution. It also describes the strategies and solutions adopted or proposed to address these issues, and the potential impacts, tradeoffs, and opportunities that result.

\subsection{Hardware Resources}

One of the biggest challenges was the scarcity of hardware resources for training and evaluating these models. Generative models require large amounts of computing power and memory to process high-dimensional data and learn complex patterns. However, such resources are often limited or expensive to access, especially for individual researchers or small teams. This is a significant barrier to achieving state-of-the-art results in audio synthesis, as few companies and labs have the necessary hardware capabilities.

To overcome this challenge, several strategies have been employed to optimize the use of available hardware resources. First, the neurons of the models were parallelized across the available \acp{GPU} (two in the final configuration --- \ac{LIACC} 2). This allowed to distribute the workload and speed up the training process. Second, checkpoints were implemented to save and load the model state at each epoch. This allowed the training to be resumed from where it was stopped in case of interruptions or failures. Third, the audio files were dynamically read and translated into spectrograms during training. This reduced memory consumption and disk space requirements. Fourth, gradient scaling and autocast (mixed precision training) techniques were used. These techniques involve performing some computationally expensive operations in 16-bit, while performing other numerically sensitive operations, such as accumulations, in 32-bit. This improved the performance and accuracy of the models while reducing memory consumption.

By applying these strategies, the models were trained and evaluated more efficiently and effectively. However, it is also recognized that these strategies have some limitations and trade-offs. For example, parallelizing neurons across \acp{GPU} can introduce communication overhead and synchronization issues. Checkpoints may not capture the full state of the model or optimizer. Dynamic data processing can increase pipeline latency and complexity. Mixed-precision training can introduce numerical errors or instability.

It is important to note that the hardware configuration with some computational power (\ac{LIACC} 2) was only available about a month before the submission of this thesis, so most of the work was done with really scarce resources. This means that the results presented in this thesis may not reflect the full potential or optimal performance of the proposed solution, as more experiments and improvements could be done with more hardware resources.

\subsection{Data Quality and Quantity}

Another challenge was the quality and quantity of available data. Generative models require a large amount of high-quality and diverse data to learn the underlying patterns and distributions of the data domain. However, such data is often scarce or difficult to obtain, especially for audio synthesis from textual input. Existing datasets for this task are either too small, focused on a specific domain, or lack descriptive labels. This limits the generalization and robustness of the models, as they may overfit to the training data or fail to capture the variability and richness of natural language and sound.

To mitigate this challenge, some data augmentation techniques were applied to increase the size and diversity of the data. For example, random cropping was used to generate different segments of audio from the same file. This increased the number of samples and introduced some variation in the data. However, it is also recognized that these techniques are not sufficient to solve the problem of data quality and quantity. Data augmentation may not produce realistic or novel samples but may introduce noise or artifacts into the data.

\subsection{Hyperparameter Tuning}

A final challenge was the time required for the hyperparameter tuning of the generative models. Hyperparameters are parameters that are not learned by the model, but are set by the user prior to training. They include learning rate, batch size, number of layers, number of neurons, activation functions, regularization methods, etc. Hyperparameters significantly impact the performance and behavior of the model, as they determine how the model learns from the data and adapts to different situations. However, finding the optimal values for these hyperparameters is often a tedious and time-consuming process involving trial-and-error experiments with different combinations of values.

Due to time constraints and deadlines, there was insufficient time to fine-tune our hyperparameters for these generative models. All models presented in this thesis are vanilla versions with default or arbitrary values for their hyperparameters. This means they may not reach their full potential or optimal performance in audio synthesis from textual input. Therefore, it is suggested that future work should devote more time and effort to the hyperparameter tuning of our generative models using methods such as grid search, random search, Bayesian optimization, etc.

\section{Conclusion}

In this section, the author discusses the major limitations and challenges they faced in developing the proposed solution. The author has also described how they addressed these issues and the possible implications, trade-offs, and opportunities that arose. Despite these challenges and limitations, the author believes that the proposed solution has several strengths and advantages that advance the state of the art in generative \ac{AI} models for audio synthesis.
\section{Conclusions}

\subsection{Overview and Reflections}

\begin{frame}
    \frametitle{Overview and Reflections}
    \begin{itemize}
        \item Comprehensive Study of State-of-the-Art Deep Learning Architectures for Audio Synthesis
        \item Development of End-to-End Systems for Sound Synthesis and Evaluation
        \item Challenges and Lessons Learned
    \end{itemize}
\end{frame}

\subsection{Future Directions}

\begin{frame}
    \frametitle{Future Directions}
    \begin{itemize}
        \item Exploring Novel Architectures
        \item Dataset Expansion
        \item Evaluation Metrics
    \end{itemize}
\end{frame}

\subsection{Novel Architectures}


\begin{frame}
    \frametitle{Novel Architectures}
    \begin{itemize}
        \item Briefly introduce the proposed theoretical architectures
        \item Mention their objectives, design principles, and potential applications
        \item Highlight the need for future research and development in this area
    \end{itemize}
\end{frame}

\subsection{Conclusion}

\begin{frame}
    \frametitle{Conclusion}
    \begin{itemize}
        \item Summarize the main achievements and contributions of the research
        \item Emphasize the progress made in understanding and developing generative AI models for audio synthesis
        \item Acknowledge the challenges and ongoing work required for further advancements
    \end{itemize}
\end{frame}


%%----------------------------------------
%% Final materials
%%----------------------------------------

%% Bibliography
%% Comment the next command if BibTeX file not used
%% bibliography is in ``myrefs.bib''
\PrintBib{myrefs}

%% 2021-07-20: change
%% comment next 2 commands if numbered appendices are not used
\begin{appendices}
    \chapter{Classification Model} \label{ann:classification}


This appendix presents the architecture and training process of the developed audio classification model.

\section{Model Architecture}

The ConvClassifier class defines the architecture of the one-dimensional \ac{CNN}. The following code snippet provides an overview of the ConvClassifier class:

\begin{lstlisting}[language=Python, caption={ConvClassifier class for sound classification}]
class ConvClassifier(nn.Module):
    def __init__(self):
        super().__init__()

        self.conv1 = nn.Conv1d(
            in_channels=1, out_channels=32, kernel_size=9, stride=1, padding=1)
        self.pool1 = nn.MaxPool1d(2, stride=2)

        self.conv2 = nn.Conv1d(32, 64, 9, stride=1, padding=1)
        self.pool2 = nn.MaxPool1d(2, stride=2)

        self.conv3 = nn.Conv1d(64, 128, 9, stride=1, padding=1)
        self.pool3 = nn.MaxPool1d(2, stride=2)

        self.conv4 = nn.Conv1d(128, 256, 9, stride=1, padding=1)
        self.pool4 = nn.MaxPool1d(2, stride=2)

        self.conv5 = nn.Conv1d(256, 512, 9, stride=1, padding=1)
        self.pool5 = nn.MaxPool1d(2, stride=2)

        self.gap = nn.AdaptiveAvgPool1d(1)

        self.linear1 = nn.Linear(512, 256)
        self.linear2 = nn.Linear(256, 128)
        self.linear3 = nn.Linear(128, 10)

        self.relu = nn.ReLU()

    def forward(self, x):
        # pass to float
        x = x.float()

        x = self.pool1(self.relu(self.conv1(x)))
        x = self.pool2(self.relu(self.conv2(x)))
        x = self.pool3(self.relu(self.conv3(x)))
        x = self.pool4(self.relu(self.conv4(x)))
        x = self.pool5(self.relu(self.conv5(x)))

        x = self.gap(x)

        x = x.view(-1, 512)

        x = self.relu(self.linear1(x))
        x = self.relu(self.linear2(x))
        x = self.linear3(x)

        return x
\end{lstlisting}

\section{Training Process}

The model is trained using the provided \texttt{train} function, which implements \ac{SGD} optimization and batch-wise backpropagation. The following code snippet outlines the training process:

\begin{lstlisting}[language=Python, caption={Training process for sound classification}]
def train(data_loader, model, loss_fn, optimizer):
    model.train()

    for batch, (sample, label, _) in enumerate(data_loader):
        sample, label = sample.to(device), label.to(device)

        # Compute prediction error
        pred = model(sample)
        loss = loss_fn(pred, label)

        # Backpropagation
        optimizer.zero_grad()
        loss.backward()
        optimizer.step()

        if batch % 100 == 0:
            loss, current = loss.item(), batch * len(sample)
            print(f"loss: {loss:>7f}  [{current:>5d}/{len(data_loader.dataset):>5d}]")

def test(data_loader, model, loss_fn):
    model.eval()

    test_loss, correct = 0, 0

    with torch.no_grad():
        for sample, label, _ in data_loader:
            sample, label = sample.to(device), label.to(device)

            pred = model(sample)
            test_loss += loss_fn(pred, label).item()

            correct += (pred.argmax(1) == label).type(
                torch.float).sum().item()
            
    test_loss /= len(data_loader)
    correct /= len(data_loader.dataset)

    print(f"Test Error: \n Accuracy: {(100*correct):>0.1f}%, Avg loss: {test_loss:>8f} \n")

epochs = 20

for epoch in range(epochs):
    print(f"Epoch {epoch + 1}\n-------------------------------")
    
    train(dataloader_train, model, loss_fn, optimizer)
    test(dataloader_test, model, loss_fn)
    print()

print("Done!")
\end{lstlisting}
    \chapter{GAN Implementation Details} \label{ann:GAN}

\section{Models}

\subsection{Generator}

The generator model is constructed as a sequential neural network. Its goal is to take a noise input and generate audio samples that resemble the desired output.

\begin{lstlisting}[language=Python, caption={Generator initialization}]
def _make_generator_model(self, input_shape=(100,)):
    model = tf.keras.Sequential()
    model.add(tf.keras.layers.Dense(
        256, use_bias=False, input_shape=input_shape))
    model.add(tf.keras.layers.Reshape((16, 16)))
    model.add(tf.keras.layers.ReLU())
    model.add(tf.keras.layers.Conv1DTranspose(
        32, 25, strides=4, padding='same'))
    model.add(tf.keras.layers.ReLU())
    model.add(tf.keras.layers.Conv1DTranspose(
        16, 25, strides=4, padding='same'))
    model.add(tf.keras.layers.ReLU())
    model.add(tf.keras.layers.Conv1DTranspose(
        8, 25, strides=4, padding='same'))
    model.add(tf.keras.layers.ReLU())
    model.add(tf.keras.layers.Conv1DTranspose(
        4, 25, strides=4, padding='same'))
    model.add(tf.keras.layers.ReLU())
    model.add(tf.keras.layers.Conv1DTranspose(
        2, 25, strides=4, padding='same'))
    model.add(tf.keras.layers.ReLU())
    model.add(tf.keras.layers.Conv1DTranspose(
        1, 25, strides=4, padding='same'))
    model.add(tf.keras.layers.Activation('tanh'))

    return model
\end{lstlisting}

The generator consists of 6 blocks, each of which consists of a transposed convolution layer followed by a \ac{ReLU} activation function (see Sections~\ref{sec:transpoed-conv} and~\ref{sec:relu}). The exception is the last block, which uses a \ac{tanh} activation function (explained in Sections~\ref{sec:tanh}) to ensure that the generated audio values fall within the range of -1 to 1.

\subsection{Discriminator}

The discriminator model is designed to distinguish between real and generated audio samples. Like the generator, it's implemented as a sequential neural network.

\begin{lstlisting}[language=Python, caption={Discriminator initialization}]
def _make_discriminator_model(self, input_shape=(INPUT_SIZE, 1)):
    model = tf.keras.Sequential()
    model.add(tf.keras.layers.Conv1D(1, 25, strides=4,
                                     input_shape=input_shape, padding='same'))
    model.add(tf.keras.layers.LeakyReLU(alpha=0.2))

    model.add(tf.keras.layers.Conv1D(2, 25, strides=4, padding='same'))
    model.add(tf.keras.layers.LeakyReLU(alpha=0.2))

    model.add(tf.keras.layers.Conv1D(4, 25, strides=4, padding='same'))
    model.add(tf.keras.layers.LeakyReLU(alpha=0.2))

    model.add(tf.keras.layers.Conv1D(8, 25, strides=4, padding='same'))
    model.add(tf.keras.layers.LeakyReLU(alpha=0.2))

    model.add(tf.keras.layers.Conv1D(16, 25, strides=4, padding='same'))
    model.add(tf.keras.layers.LeakyReLU(alpha=0.2))

    model.add(tf.keras.layers.Conv1D(32, 25, strides=4, padding='same'))
    model.add(tf.keras.layers.LeakyReLU(alpha=0.2))

    model.add(tf.keras.layers.Flatten())

    model.add(tf.keras.layers.Dense(1))

    return model
\end{lstlisting}

The discriminator consists of 6 convolutional blocks, each followed by a leaky \ac{ReLU} activation function. After these blocks, the model is flattened to be fed into a dense layer with a single output unit to provide a binary classification indicating whether the input is real or generated.

\section{Training}

The \ac{GAN} training procedure involves a series of steps to iteratively optimize the generator and discriminator networks.

\begin{lstlisting}[language=Python, caption={Training operations}]
def train(self, dataset, epochs):
    # ... (checkpoint loading operations)

    # run the epochs
    for epoch in tqdm(range(epochs)):
        print("Epoch {}/{}".format(epoch+1, epochs))

        # run the batches
        for audios in tqdm(dataset):
            self._train_step(audios)

        # ... (display and checkpoint operations)

@tf.function
def _train_step(self, audios):
    noise = tf.random.normal([BATCH_SIZE, self._fake_input_shape[0]])

    with tf.GradientTape() as gen_tape, tf.GradientTape() as disc_tape:
        generated_audios = self._generator(noise, training=True)

        real_output = self._discriminator(audios, training=True)
        fake_output = self._discriminator(generated_audios, training=True)

        gen_loss = self._generator_loss(fake_output)
        disc_loss = self._discriminator_loss(real_output, fake_output)

    gradients_of_generator = gen_tape.gradient(
        gen_loss, self._generator.trainable_variables)
    gradients_of_discriminator = disc_tape.gradient(
        disc_loss, self._discriminator.trainable_variables)

    self._generator_optimizer.apply_gradients(
        zip(gradients_of_generator, self._generator.trainable_variables))
    self._discriminator_optimizer.apply_gradients(
        zip(gradients_of_discriminator, self._discriminator.trainable_variables))

def _generator_loss(self, fake_output):
    return tf.keras.losses.BinaryCrossentropy(from_logits=True)(tf.ones_like(fake_output), fake_output)

def _discriminator_loss(self, real_output, fake_output):
    real_loss = tf.keras.losses.BinaryCrossentropy(
        from_logits=True)(tf.ones_like(real_output), real_output)
    fake_loss = tf.keras.losses.BinaryCrossentropy(
        from_logits=True)(tf.zeros_like(fake_output), fake_output)
    total_loss = real_loss + fake_loss
    return total_loss
\end{lstlisting}

For each epoch, the generator produces a set of generated audio samples using random noise. Both the real and generated audio samples are then passed through the discriminator network to calculate their respective outputs.

The generator and discriminator losses are calculated based on the discriminator outputs. The generator aims to minimize the \ac{BCE} loss associated with the output of the fake samples, thereby encouraging the discriminator to classify them as real. Conversely, the discriminator seeks to minimize the loss by discriminating between real and false samples.

The gradients of the generator and discriminator losses are computed using a gradient band, and the model parameters are updated accordingly using the Adam optimizer (see Section~\ref{sec:adam}).
    \chapter{Autoencoder Implementation Details} \label{ann:AE}

\section{Generative Model Code}
Below is the implementation of the \ac{AE} model architecture used for generating images. This code defines the architecture of the encoder and decoder components, as well as the training and testing functions.

\begin{lstlisting}[language=Python, caption={Model architecture}]
class AutoEncoder(nn.Module):
    # ... (other class operations)

    def forward(self, x):
        # passes the input through the encoder and then through the decoder
        x = x.float()
        encoded = self.encoder(x)
        decoded = self.decoder(encoded)
        return decoded

    def _build_encoder(self):
        layers = []

        layers.append(nn.Conv1d(1, 32, kernel_size=9, stride=1, padding=4))
        layers.append(nn.BatchNorm1d(32))
        layers.append(nn.Tanh())
        layers.append(nn.MaxPool1d(kernel_size=2, stride=2))

        for i in range(self.convolutional_layers - 1):
            layers.append(nn.Conv1d(2**i*32, 2**(i+1)*32,
                          kernel_size=9, stride=1, padding=4))

            layers.append(nn.BatchNorm1d(2**(i+1)*32))
            layers.append(nn.Tanh())

            layers.append(nn.MaxPool1d(kernel_size=2, stride=2))

        return nn.Sequential(*layers)

    def _build_decoder(self):
        layers = []

        for i in range(self.convolutional_layers - 2, -1, -1):
            layers.append(nn.Upsample(scale_factor=2))
            layers.append(nn.ConvTranspose1d(2**(i+1)*32, 2**i *
                          32, kernel_size=9, stride=1, padding=4))
            layers.append(nn.BatchNorm1d(2**i*32))
            layers.append(nn.Tanh())

        layers.append(nn.Upsample(scale_factor=2))
        layers.append(nn.ConvTranspose1d(
            32, 1, kernel_size=9, stride=1, padding=4))
        layers.append(nn.BatchNorm1d(1))
        layers.append(nn.Tanh())

        return nn.Sequential(*layers)
\end{lstlisting}

\section{Training Code}
The following code snippet showcases the training operations for the \ac{AE} model. It includes data loading, computation of prediction error, backpropagation, and optimization.

\begin{lstlisting}[language=Python, caption={Training operations}]
def train(data_loader, model, loss_fn, optimizer):
    model.train()

    for batch, (sample, label, _) in enumerate(data_loader):
        # normalize the audio wav sample
        sample = sample / 32768

        # ... (other input operations)

        # compute prediction error
        pred = model(sample)
        loss = loss_fn(pred, sample)

        # backpropagation
        optimizer.zero_grad()
        loss.backward()
        optimizer.step()

        # ... (printing and memory management operations)


def test(data_loader, model, loss_fn, name):
    model.eval()

    test_loss, average_mse = 0, 0
    showed = False

    with torch.no_grad():
        for sample, label, _ in data_loader:
            sample = sample / 32768

            # ... (other input operations)

            pred = model(sample)
            test_loss += loss_fn(pred, sample).item()

            average_mse += torch.mean((pred - sample) ** 2)

            # ... (input operations)

    test_loss /= len(data_loader)
    average_mse /= len(data_loader)

    return test_loss, average_mse

def run():
    # model initialization
    model = AutoEncoder(convolutional_layers=4).to(device)
    loss_fn = nn.MSELoss()
    optimizer = torch.optim.Adam(model.parameters(), lr=1e-3)

    # start training
    # ... (display and early stopping initializations)
    epoch = 0

    while True: # actual code checks for early stopping
        train(dataloader_train, model, loss_fn, optimizer)
        loss, mse = test(dataloader_test, model, loss_fn, f"{conv_layers}_{epoch}")

        # ... (early stopping settings)

        epoch += 1

    # ... (printing and storing settings)
\end{lstlisting}

\section{Results}

This section presents a comparison of an original sound, represented by a soundwave, and its recreation.

It is important to note that the recreated image is the result of the original sound passing through an \ac{AE} with the specified threshold, as detailed in the accompanying annex.

This result was achieved after only 20 epochs, and the recreated sound is nearly as good as the original.

\begin{figure}[ht]
    \centering
    \includegraphics[width=0.7\linewidth]{annexes/AE/original.png}
    \caption{Original Sample}
\end{figure}

\begin{figure}[ht]
    \centering
    \includegraphics[width=0.7\linewidth]{annexes/AE/reconstructed.png}
    \caption{Reconstructed Sample}
\end{figure}

    \chapter{Variational Autoencoder Implementation Details} \label{ann:VAE}

In this annex, code implementation of the \ac{VAE} is provided. The \ac{VAE} is a generative model that is specifically designed to learn latent representations of data and enable controlled generation. The following sections elucidate the \ac{VAE} model's structure and its training process.

\section{Model Implementation}

The subsequent Python code exemplifies the execution of the simple \ac{VAE} model.

\begin{lstlisting}[language=Python]
class VAE(nn.Module):
    # ... (class operations)

    def forward(self, x):
        x = x.float()

        y = self.encoder(x)

        mean = self.meaner(y)
        logvar = self.varer(y)

        z = self._sample(mean, logvar)

        decoded = self.decoder(z)
        return decoded, mean, logvar

    def _build_encoder(self):
        layers = []

        layers.append(nn.Conv1d(1, 32, kernel_size=9, stride=1, padding=4))
        layers.append(nn.BatchNorm1d(32))
        layers.append(nn.Tanh())
        layers.append(nn.MaxPool1d(kernel_size=2, stride=2))

        for i in range(self.convolutional_layers - 1):
            layer = nn.Conv1d(2**i*32, 2**(i+1)*32,
                              kernel_size=9, stride=1, padding=4)
            layers.append(layer)
            layers.append(nn.BatchNorm1d(2**(i+1)*32))
            layers.append(nn.Tanh())

            layers.append(nn.MaxPool1d(kernel_size=2, stride=2))

        mean_layer = nn.Sequential(
            nn.AdaptiveAvgPool1d(1),
            nn.Flatten(),
            nn.Linear(2**(self.convolutional_layers-1)*32, self.latent_dim)
        )

        var_layer = nn.Sequential(
            nn.AdaptiveAvgPool1d(1),
            nn.Flatten(),
            nn.Linear(2**(self.convolutional_layers-1)*32, self.latent_dim)
        )

        return nn.Sequential(*layers), mean_layer, var_layer

    def _build_decoder(self):
        layers = []

        layers.append(
            nn.Linear(self.latent_dim, 2**(self.convolutional_layers-1)
                      * 32 * (self.input_size // 2**self.convolutional_layers))
        )

        layers.append(
            nn.Unflatten(1, (2**(self.convolutional_layers-1)*32,
                         self.input_size // 2**self.convolutional_layers))
        )

        for i in range(self.convolutional_layers - 2, -1, -1):
            layers.append(nn.Upsample(scale_factor=2))
            layers.append(nn.ConvTranspose1d(2**(i+1)*32, 2**i *
                                             32, kernel_size=9, stride=1, padding=4))
            layers.append(nn.BatchNorm1d(2**i*32))
            layers.append(nn.Tanh())

        layers.append(nn.Upsample(scale_factor=2))
        layers.append(nn.ConvTranspose1d(
            32, 1, kernel_size=9, stride=1, padding=4))
        layers.append(nn.BatchNorm1d(1))
        layers.append(nn.Tanh())

        return nn.Sequential(*layers)

    def _sample(self, mean, logvar):
        std = torch.exp(0.5 * logvar)
        eps = torch.randn_like(std)
        return eps * std + mean

    def loss_function(self, recon_x, x, mean, logvar):
        recon_x = torch.clamp(recon_x, 0, 1) # normalize output
        x = torch.clamp(x, 0, 1) # normalize input
        BCE = F.binary_cross_entropy(recon_x, x, reduction='sum')
        KLD = -0.5 * torch.sum(1 + logvar - mean.pow(2) - logvar.exp())

        return BCE + KLD
\end{lstlisting}

The \ac{VAE} model comprises an encoder, a bottleneck layer, and a decoder. The encoder analyzes the input data to extract significant characteristics, whereas the decoder produces reconstructions from covert representations. The latent space is sampled by employing mean and variance parameters from the encoder.

\section{Training Process}

The \ac{VAE} is trained with the following code.

\begin{lstlisting}[language=Python]
def train_step(data_loader, model, optimizer):
    model.train()

    train_loss = 0

    for batch, (sample, label, _) in enumerate(data_loader):
        # ... (input preprocessing)

        recon_batch, mean, logvar = model(sample)
        loss = model.loss_function(recon_batch, sample, mean, logvar)

        optimizer.zero_grad()
        loss.backward()
        train_loss += loss.item()
        optimizer.step()

        # ... (display and memory management operations)

    train_loss /= len(data_loader.dataset)
    return train_loss


def test(data_loader, model, name):
    model.eval()

    test_loss = 0

    with torch.no_grad():
        for sample, label, _ in data_loader:
            # ... (input preprocessing)

            pred, mean, logvar = model(sample)
            loss = model.loss_function(pred, sample, mean, logvar)

            test_loss += loss

            # ... (display operations)

    test_loss /= len(data_loader)

    return test_loss

def run():
    epoch = 0

    # ... (display and early stopping initializations)

    while True: # actual code had early stopping checking
        # ... (display operations)
        train_loss = train_step(dataloader_train, model, optimizer)
        loss = test(dataloader_test, model, f"{CONV_LAYERS}_{epoch}")

        epoch += 1

    # ... (display and memory management operations)
\end{lstlisting}

The training process iteratively updates the model with data batches. In each training step, the model is optimized with a loss function that combines \ac{BCE} and \ac{KL} divergence terms. The function encourages the model to produce accurate reconstructions and learn a significant latent space.

\section{Results}

As an illustration of the \ac{VAE}'s potential, this section exhibits two images: one featuring an initial audio waveform and the other displaying its reconstructed version produced by the trained \ac{VAE}.

\begin{figure}[ht]
    \centering
    \includegraphics[width=0.7\linewidth]{figures/annexes/VAE/original.png}
    \caption{Original Sample}
\end{figure}

\begin{figure}[ht]
    \centering
    \includegraphics[width=0.7\linewidth]{figures/annexes/VAE/reconstructed.png}
    \caption{Reconstructed Sample}
\end{figure}

These images represent the best reconstruction achievable by the author, considering the hardware and time limitations.

\clearpage

    
    \chapter{GANmix Implementation Details} \label{ann:ganmix-implementation}

\section{Model Implementation}

\begin{lstlisting}[language=Python, caption={Implementation of the GANmix model.}]
class GANMix():

    # ... (initialization code)

    @classmethod
    def build(cls):
        latents_shape = (8, 128, 107)

        num_workers = torch.cuda.device_count()

        vae = models.VAE()

        netG = Generator(config.GENERATOR_INPUT_SIZE, 200, latents_shape)
        netD = Discriminator(latents_shape, 200, config.GAUSSIAN_NOISE)

        # ... (parallelization and GPU)

        criterion = BCEWithLogitsLoss()

        optimizerD = optim.Adam(netD.parameters(
        ), lr=config.LEARNING_RATE_DISCRIMINATOR, betas=(config.BETA1, 0.999))
        optimizerG = optim.Adam(netG.parameters(
        ), lr=config.LEARNING_RATE_GENERATOR, betas=(config.BETA1, 0.999))

        return cls(vae, netG, netD, optimizerG, optimizerD, criterion, num_workers)

class Generator(nn.Module):

    # ... (initialization)

    def forward(self, x):
        x = x.to(config.DEVICE)

        x = self.network(x)

        return x

    def _build_network(self):
        network = nn.Sequential(
            nn.Linear(self.input_size, self.hidden_size),
            nn.LeakyReLU(0.2),
            nn.Linear(self.hidden_size, np.prod(self.output_shape)),
        )
        return network

    def forward(self, x):
        x = x.to(config.DEVICE)

        x = self.network(x)
        x = x.view(-1, *self.output_shape)

        return x

class Discriminator(nn.Module):

    # ... (initialization)

    def forward(self, x):
        x = x + torch.randn_like(x) * self.gaussian_noise_std

        x = self.network(x)

        return x

    def _build_network(self):
        network = nn.Sequential(
            nn.Flatten(),
            nn.Linear(np.prod(self.input_shape), self.hidden_size),
            nn.LeakyReLU(0.2),
            nn.Linear(self.hidden_size, 1),
            nn.Tanh()
        )
        return network
\end{lstlisting}

The GANmix model is a \ac{GAN} that combines a \ac{VAE} to generate realistic audio from latent vectors.

\subsection{VAE}
The \ac{VAE} is a neural network that encodes an input spectrogram into a latent vector and then decodes it back into an output spectrogram.

The \ac{VAE} used in this model is AudioLDM's~\cite{liu_audioldm_2023} \ac{VAE}, which is already trained. It has an encoder and a decoder, both of which are \acp{CNN} with residue blocks. The encoder outputs two vectors: the latent vector's mean and logarithmic variance. The decoder takes a latent vector as input and outputs an image.

\subsection{Generator}
The generator is a neural network that inputs a latent vector and outputs a set of embeddings corresponding to the \ac{VAE} embedding space. The generator is trained to fool the discriminator into believing that the generated embeddings are real.

The generator used in GANmix is implemented by the class \texttt{Generator}. The generator has three attributes: \texttt{input\_size}, \texttt{hidden\_size}, and \texttt{output\_shape}. The \texttt{input\_size} is the dimension of the latent vector, which is stored in the config file and is 100 (this and all values can be seen in the appendix ~\ref{ann:ganmix-conf}. The \texttt{hidden\_size} is the dimension of the hidden layer, which in this case is 200. The \texttt{output\_shape} is the shape of the embeddings, in this case $(8, 128, 107)$.

The generator has a method called \texttt{\_build\_network()} that returns a sequential network consisting of three layers: a linear layer, a leaky \ac{ReLU} activation function, and another linear layer. The first linear layer maps the input vector to the hidden layer. The second linear layer maps the hidden layer to the output vector, which has the same size as the product of the output form.

The generator also has a \texttt{forward()} method that takes an input vector \texttt{x} and passes it through the network. It then transforms the output vector into the shape of the \ac{VAE} embedding space.

\subsection{Discriminator}

The discriminator is a neural network that takes a set of embeddings as input and outputs a value indicating how real or fake the image is. The discriminator is trained to distinguish between real embeddings from the \ac{VAE} and fake embeddings from the generator. The discriminator can also be used to evaluate how realistic the generated embeddings are by calculating their scores.

The discriminator used in this thesis is implemented using the class \texttt{Discriminator}. The discriminator has three attributes: \texttt{input\_shape}, \texttt{hidden\_size}, and \texttt{gaussian\_noise\_std}. The \texttt{input\_shape} is the shape of the input embeddings. The \texttt{hidden\_size} is the size of the hidden layer. The \texttt{gaussian\_noise\_std} is the standard deviation of the Gaussian noise added to the input image, which in this case is $0.1$, no fine-tuning was performed.

The discriminator has a method called \texttt{\_build\_network()} that returns a sequential network consisting of five layers: a flatten layer, a linear layer, a leaky \ac{ReLU} activation function, another linear layer, and a tanh activation function. The flatten layer flattens the input image into a vector. The first linear layer maps the input vector to the hidden layer. The second linear layer maps the hidden layer to the output scalar.

The discriminator also has a method called \texttt{forward()}, which takes an input image \texttt{x} and adds Gaussian noise with standard deviation \texttt{gaussian\_noise\_std} to it. Then, it passes it through the network and returns the output scalar.

\subsection{GANMix}

The GANMix class is a wrapper class that contains all of these models. Plus the loss criterion and optimizers

\section{Training Implementation}

\begin{lstlisting}[language=Python, caption={Implementation of the training loop for the GANmix.}]
# ... (initialization)

def _gen_fake_samples(generator, num_samples):
    noise = torch.randn(num_samples, config.GENERATOR_INPUT_SIZE, device=config.DEVICE)
    fake_samples = generator(noise)
    return fake_samples

def _embed_samples(vae, samples):
    embeddings = vae.encode(samples)
    embeddings = embeddings.latent_dist.mode()
    embeddings = torch.nan_to_num(embeddings, nan=0)
    return embeddings

def _train_discriminator(data, ganmix, settings):
    with autocast():
        real_data = data.to(config.DEVICE)
        batch_size = real_data.size(0)

        real_embeddings = _embed_samples(ganmix.vae, real_data)

        fake_embeddings = _gen_fake_samples(ganmix.generator, batch_size)

        ganmix.discriminator_optimizer.zero_grad()

        prediction_real = ganmix.discriminator(real_embeddings)
        prediction_fake = ganmix.discriminator(fake_embeddings.detach())

        real_label = torch.ones(batch_size, 1, device=config.DEVICE)
        fake_label = -torch.ones(batch_size, 1, device=config.DEVICE)

        loss_real = ganmix.criterion(prediction_real, real_label)
        loss_fake = ganmix.criterion(prediction_fake, fake_label)

        loss_discriminator = loss_real + loss_fake

    settings.scaler.scale(loss_discriminator).backward()
    settings.scaler.step(ganmix.discriminator_optimizer)
    settings.scaler.update()

    return loss_discriminator

def _train_generator(ganmix, settings):
    with autocast():

        fake_data = _gen_fake_samples(ganmix.generator, config.BATCH_SIZE)

        ganmix.generator_optimizer.zero_grad()

        prediction_fake = ganmix.discriminator(fake_data)

        fake_label = torch.ones(config.BATCH_SIZE, 1, device=config.DEVICE)

        loss_generator = ganmix.criterion(prediction_fake, fake_label)

    settings.scaler.scale(loss_generator).backward()
    settings.scaler.step(ganmix.generator_optimizer)
    settings.scaler.update()

    return loss_generator


# ... (output and display)


def run_train():

    # ... (initialization)

    with open(settings.stats_file_path, 'a', newline='') as csvfile:

        # ... (display options)

        for epoch in range(config.NUM_EPOCHS):
            loss_discriminator_list = []
            loss_generator_list = []

            for data, quote in tqdm(settings.dataloader):
                loss_discriminator = _train_discriminator(data, ganmix, settings)
                loss_discriminator_list.append(loss_discriminator.item())

                loss_generator = _train_generator(ganmix, settings)
                loss_generator_list.append(loss_generator.item())

            _output_epoch_results(settings.start_time, epoch, ganmix.generator, ganmix.vae, loss_discriminator_list, loss_generator_list, csv_writer, csvfile)

def main():
    run_train()
\end{lstlisting}

Training of the GANmix model is implemented using the \texttt{run\_train()} and \texttt{main()} functions. Training involves several steps: data loading, model building, loss calculation, optimization, and output display.

\subsection{Data Loading}

The data loading step is responsible for loading the dataset of audios and quotes and creating a data loader that iterates over the data batches. The data set and the data loader are stored in the \texttt{settings} object, which is an instance of the \texttt{Settings} class. The \texttt{Settings} class is defined in another module and contains various parameters and paths for the training.

The data loader is created using PyTorch's \texttt{torch.utils.data.DataLoader} class. The data loader takes the data set as an argument and returns batches of data with a specified batch size. The batch size used in this model was 8. The data loader also shuffles the data and supports multiprocessing.

\subsection{Model Building}

The model build step is responsible for creating an instance of the \texttt{GANMix} model and moving it to the \ac{GPU}, if available.

\subsection{Loss Calculation}

The loss computation step is responsible for computing the losses for the generator and the discriminator using the criterion and the predictions from the model.

The loss calculation consists of two sub-steps: generating and embedding dummy samples.

\subsubsection{Generate Dummy Samples}

The generate dummy samples substep generates dummy embeddings from random latent vectors using the generator.

This sub-step is implemented using the \texttt{\_gen\_fake\_samples()} function, which takes the generator and the number of samples to generate as arguments and returns the dummy images as a tensor.

The \texttt{\_gen\_fake\_samples()} function performs the following steps:

\begin{enumerate}
    \item Sample random latent vectors from a standard normal distribution using PyTorch's \texttt{torch.rand()} function.
    \item Generate dummy images from the latent vectors using the generator's \texttt{forward()} method.
\end{enumerate}

\subsubsection{Embedding Samples}

The embedding samples sub-step is responsible for embedding real spectrograms into latent vectors using the \ac{VAE}'s encoder.

This sub-step is implemented using the \texttt{\_embed\_samples()} function, which takes the \ac{VAE} and the samples to embed as arguments and returns the latent vectors as a tensor.

The \texttt{\_embed\_samples()} function performs the following steps:

\begin{enumerate}
    \item Encodes the samples using the \texttt{encode()} method of the \ac{VAE}.
    \item Extract the mode of the latent vector from the \texttt{latent\_dist} attribute.
\end{enumerate}

The loss computation step also consists of two main steps: training the discriminator and training the generator.

\subsubsection{Discriminator Training}

The discriminator training step is responsible for updating the discriminator parameters using the discriminator optimizer and the discriminator loss. The discriminator loss is calculated by comparing the discriminator predictions for the real and fake emails with the real and fake labels.

The discriminator training step is implemented using the \texttt{\_train\_discriminator()} function, which takes the data, the ganmix model, and the settings object as arguments and returns the discriminator loss as a scalar.

The \texttt{\_train\_discriminator()} function performs the following steps:

\begin{enumerate}
    \item Move the real data to the \ac{GPU} using the \texttt{to()} method.
    \item Get the batch size from the real data using the \texttt{size()} method.
    \item Embed the real data into latent vectors using the \texttt{\_embed\_samples()} function.
    \item Generate fake embeddings from random latent vectors using the \texttt{\_gen\_fake\_samples()} function.
    \item Compute the discriminator predictions for the real and fake embeddings using the discriminator's \texttt{forward()} method.
    \item Compute the discriminator losses for the real and fake predictions using the criterion's \texttt{forward()} method.
    \item Compute the total discriminator loss by adding the real and fake losses.
\end{enumerate}

\subsubsection{Training the Generator}

The generator training step updates the generator parameters using the generator optimizer and the generator loss. The generator loss is calculated by comparing the discriminator predictions for the fake images with the real labels.

The generator training step is implemented using the \texttt{\_train\_generator()} function, which takes the ganmix model and the settings object as arguments and returns the generator loss.

The \texttt{\_train\_generator()} function performs the following steps:

\begin{enumerate}
    \item Generate fake data from random latent vectors using the \texttt{\_gen\_fake\_samples()} function.
    \item Compute discriminator predictions for the fake data using the discriminator's \texttt{forward()} method.
    \item Generate the real labels using PyTorch's \texttt{torch.ones()} function.
    \item Compute the generator loss by comparing the fake predictions and the real labels using the criterion's \texttt{forward()} method.
\end{enumerate}

\subsection{Training Loop}


For the specified number of epochs, the program analyzes the entire data set, first training the discriminator and then training the generator. The corresponding losses for both are noted and displayed by the program.
    \chapter{GANmix Model Configuration and Parameters}
\label{ann:ganmix-conf}

This annex contains the details of the final GANmix model. The GANmix model is a \ac{GAN} that uses fully connected neural networks for both the generator and the discriminator, and leverages a pretrained \ac{VAE} for generating and discriminating audio embeddings.

\begin{table}[h]
\centering
\caption{GANmix model parameters}
\label{tab:ganmix-params}
\begin{tabular}{|l|l|}
\hline
Parameter & Value \\ \hline
Number of trainable parameters (generator) & 22043368 \\ \hline
Number of trainable parameters (discriminator) & 21914001 \\ \hline
Total number of trainable parameters & 43957369 \\ \hline
Total number of parameters & 99334242 \\ \hline
Size of the noise vector ($NZ$) & 100 \\ \hline
Size of the hidden layer ($NH$) & 200 \\ \hline
Generator learning rate & 0.001 \\ \hline
Discriminator learning rate & 0.0001 \\ \hline
Generator scheduler step size & 10 \\ \hline
Generator scheduler gamma & 0.1 \\ \hline
Discriminator scheduler step size & 10 \\ \hline
Discriminator scheduler gamma & 0.1 \\ \hline
Batch size & 8 \\ \hline
\end{tabular}
\end{table}

\begin{table}[h]
\centering
\caption{AudioLDM's \ac{VAE} model encodings}
\label{tab:ganmix-encodings}
\begin{tabular}{|l|l|}
\hline
Encoding dimension & Value \\ \hline
Number of dimensions ($ED$) & 8 \\ \hline
Width ($EW$) & 128 \\ \hline
Height ($EH$) & 107 \\ \hline
\end{tabular}
\end{table}

\begin{lstlisting}[language=Python, caption=Generator summary, label=lst:gen-summary]
----------------------------------------------------------------
        Layer (type)               Output Shape         Param #
================================================================
            Linear-1                   [8, 200]          20,200
         LeakyReLU-2                   [8, 200]               0
            Linear-3                [8, 109568]      22,023,168
================================================================
Total params: 22,043,368
Trainable params: 22,043,368
Non-trainable params: 0
----------------------------------------------------------------
Input size (MB): 0.00
Forward/backward pass size (MB): 6.71
Params size (MB): 84.09
Estimated Total Size (MB): 90.80
----------------------------------------------------------------

\end{lstlisting}

\begin{lstlisting}[language=Python, caption=Discriminator summary, label=lst:dis-summary]
----------------------------------------------------------------
        Layer (type)               Output Shape         Param #
================================================================
           Flatten-1                [8, 109568]               0
            Linear-2                   [8, 200]      21,913,800
         LeakyReLU-3                   [8, 200]               0
            Linear-4                     [8, 1]             201
              Tanh-5                     [8, 1]               0
================================================================
Total params: 21,914,001
Trainable params: 21,914,001
Non-trainable params: 0
----------------------------------------------------------------
Input size (MB): 3.34
Forward/backward pass size (MB): 6.71
Params size (MB): 83.60
Estimated Total Size (MB): 93.65
----------------------------------------------------------------

\end{lstlisting}

This annex provides the configuration and parameters of the GANmix model that was used to generate audio embeddings from random noise and discriminate them from real audio embeddings.

    
    \chapter{VAMOS} \label{ann:vamos}

Contained in this appendix are important insights into the architectural foundations of the VAMOS model, along with three key components that are central to its effectiveness. These components include the \textit{Text Encoder}, which uses BERT's encoder for transforming textual data, the \textit{ResNet (Audio Encoder)} designed for audio processing tasks, and the \textit{CLAP} model for creating joint audio-text embeddings. Furthermore, this text presents an implementation and analysis of the \textit{U-Net} architecture, which is known for its superior segmentation capabilities. All components of the model are accompanied by their code and academic explanation to highlight their importance in the overall framework.

\section{Text Encoder}

\begin{lstlisting}[language=Python, caption={Text encoding function utilizing BERT tokenizer and encoder for semantic representation extraction.}]
def encode(text):
    # Pass text through the BERT tokenizer
    input_ids = tokenizer(text, add_special_tokens=True)["input_ids"]
    input_tensor = torch.tensor([input_ids])

    # Pass the input tensor through the BERT encoder
    with torch.no_grad():
        encoded_output = self.model(input_tensor)[0]
        flatten_output = torch.flatten(encoded_output, start_dim=1)

    return flatten_output
\end{lstlisting}

The provided code excerpt describes the \texttt{encode} function, which plays an essential role in the developed model. This function utilizes a BERT model's encoder to convert input textual data into a structured numerical representation that encapsulates its semantic essence.

The function begins by calling a BERT tokenizer that translates human-readable text into an ordered sequence of discrete numerical identifiers. This conversion aligns with the established protocol for facilitating computational analysis and interpretation.

Afterward, the input tensor passes through the encoder module of the BERT model, which captures contextual dependencies within the text data. This process fosters a dynamic and enriched representation at the token level.

\section{ResNet (Audio Encoder)}

\begin{lstlisting}[language=Python, caption={Residual Network (ResNet) architecture for audio processing with custom residual block implementation.}]
class ResNet(nn.Module):
  
    # ... (other initialization code)

    def build_resnet(self):
        if self.useBottleneck:
            filters = [64, 256, 512, 1024, 2048]
        else:
            filters = [64, 64, 128, 256, 512]

        self.layer1 = nn.Sequential()
        self.layer1.add_module('conv2_1', resblock(
            filters[0], filters[1], downsample=False))
        for i in range(1, self.repeat[0]):
            self.layer1.add_module('conv2_%d' % (
                i+1,), resblock(filters[1], filters[1], downsample=False))

        self.layer2 = nn.Sequential()
        self.layer2.add_module('conv3_1', resblock(
            filters[1], filters[2], downsample=True))
        for i in range(1, self.repeat[1]):
            self.layer2.add_module('conv3_%d' % (
                i+1,), resblock(filters[2], filters[2], downsample=False))

        self.layer3 = nn.Sequential()
        self.layer3.add_module('conv4_1', resblock(
            filters[2], filters[3], downsample=True))
        for i in range(1, self.repeat[2]):
            self.layer3.add_module('conv2_%d' % (
                i+1,), resblock(filters[3], filters[3], downsample=False))

        self.layer4 = nn.Sequential()
        self.layer4.add_module('conv5_1', resblock(
            filters[3], filters[4], downsample=True))
        for i in range(1, self.repeat[3]):
            self.layer4.add_module('conv3_%d' % (
                i+1,), resblock(filters[4], filters[4], downsample=False))

        self.gap = torch.nn.AdaptiveAvgPool2d(1)
        self.fc = torch.nn.Linear(filters[4], outputs)

    def forward(self, input):
        input = self.layer0(input)
        input = self.layer1(input)
        input = self.layer2(input)
        input = self.layer3(input)
        input = self.layer4(input).detach()
        input = self.gap(input)
        input = torch.flatten(input, start_dim=1)
        input = self.fc(input)

        return input
  
class ResBlock(nn.Module):

    def __init__(self, in_channels, out_channels, downsample):
  
        # ... (other initialization code)

        # Define convolutional layers and batch normalization
        if downsample:
            # If downsample is True
            # apply convolution that reduces the size and apply a shortcut
            self.conv1 = nn.Conv2d(
                in_channels, out_channels, kernel_size=3, stride=2, padding=1)
            self.shortcut = nn.Sequential(
                nn.Conv2d(in_channels, out_channels, kernel_size=1, stride=2),
                nn.BatchNorm2d(out_channels)
            )
        else:
            # If downsample is False, apply convolution with stride 1 and no
            # shortcut path
            self.conv1 = nn.Conv2d(
                in_channels, out_channels, kernel_size=3, stride=1, padding=1)
            self.shortcut = nn.Identity()

        # Define additional convolutional layers and batch normalization
        self.conv2 = nn.Conv2d(out_channels, out_channels,
                               kernel_size=3, stride=1, padding=1)
        self.bn1 = nn.BatchNorm2d(out_channels)
        self.bn2 = nn.BatchNorm2d(out_channels)

    def forward(self, input_tensor):
        # Apply the shortcut path to the input
        shortcut = self.shortcut(input_tensor)
        shortcut = shortcut.detach()

        # Apply the first convolutional layer and ReLU activation, followed by
        # batch normalization
        x = nn.ReLU()(self.bn1(self.conv1(input_tensor)))
        x = x.detach()

        # Apply the second convolutional layer and ReLU activation, followed by
        # batch normalization
        x = nn.ReLU()(self.bn2(self.conv2(x)))
        x = x.detach()

        # Add the shortcut to the output of the second convolutional layer
        output_tensor = x + shortcut

        # Apply ReLU activation to the output and return it
        return nn.ReLU()(output_tensor)
\end{lstlisting}

The provided code segment introduces a customized neural network architecture based on the ResNet framework, specifically designed for audio processing. The architecture includes a variety of key components, including the use of residual blocks that are critical to constructing the ResNet layers. The explanation then provides a concise overview of the overall structure and operation of the code. 

At the core of this demonstration is the \texttt{ResNet} class. This class includes methods for assembling the architecture and managing data flow through the network.  This class includes methods for assembling the architecture and managing data flow through the network. A key method is \texttt{build\_resnet}, which creates the network by stacking residual blocks. This assembly follows the architectural configuration principles outlined by the specified guidelines.

Additionally, the \texttt{ResBlock} class serves as a fundamental construct that houses a single residual block- revered for its role as a building block within the larger ResNet architecture. Within this enclosed domain, the architectural formulation of the residual block is contextualized.

\section{CLAP}

\begin{lstlisting}[language=Python, caption={CLAP model architecture for joint audio-text embeddings with pairwise cosine similarity-based alignment.}]
class Clap(nn.Module):
    def __init__(self,
                 audio_feature_dim,
                 text_feature_dim,
                 shared_embedding_dim
                 ):
        # ... (other initialization code)
  
        # Initialize the weights that will connect the input audio and text
        # features to the shared embedding space.
        self._audio_projection = torch.nn.Linear(
            audio_feature_dim, shared_embedding_dim)
        self._text_projection = torch.nn.Linear(
            text_feature_dim, shared_embedding_dim)

        # Initialize the temperature parameter used for scaling the pairwise
        # cosine similarities between image and text embeddings.
        self._learned_temperature = torch.nn.Parameter(torch.tensor([1.0]))

    def encode_audio(self, audio_features):
        # Project the audio features into the shared embedding space and
        # normalize the resulting embedding vectors.
        audio_embeddings = F.normalize(
            self._audio_projection(audio_features), p=2, dim=1)
        return audio_embeddings

    def encode_text(self, text_features):
        # Project the text features into the shared embedding space and
        # normalize the resulting embedding vectors.
        text_embeddings = F.normalize(
            self._text_projection(text_features), p=2, dim=1)
        return text_embeddings

    def forward(self, audio_features, text_features):
        # Encode the audio and text features into their respective embedding
        # spaces.
        audio_embeddings = self.encode_audio(audio_features)
        text_embeddings = self.encode_text(text_features)

        # Compute the pairwise cosine similarities between the image and text
        # embeddings, scaled by the learned temperature parameter.
        pairwise_similarities = torch.matmul(
            audio_embeddings, text_embeddings.T) * torch.exp(
            self._learned_temperature)

        # Compute the symmetric cross-entropy loss between the predicted
        # pairwise similarities and the true pairwise similarities.
        labels = torch.arange(audio_features.size(0))
        loss_i = F.cross_entropy(
            pairwise_similarities, labels, reduction='mean')
        loss_t = F.cross_entropy(
            pairwise_similarities.T, labels, reduction='mean')
        loss = (loss_i + loss_t) / 2

        return loss
\end{lstlisting}

The presented code introduces a class called \texttt{Clap}. It is a neural network model designed to acquire shared embeddings for both audio and text input data. The model aims to align audio and text features in a shared embedding space by utilizing pairwise cosine similarities. 

The essence of the code is explained in further detail below. The initialization of the model includes parameterizing the primary dimensions: audio and text feature dimensions, coupled with the dimensionality of the embedding space for their alignment.

The model's architecture incorporates linear projection layers for audio and text input streams, as well as a temperature parameter.  The learned temperature parameter is useful in calibrating pairwise cosine similarities by incorporating a scaling factor to the magnitudes.

The \texttt{forward} method represents the procedural logic for the model's forward pass. This includes incorporating audio and text input features into their respective embedding spaces, accomplished through the use of the \texttt{encode\_audio} and \texttt{encode\_text} methods. An important step is the subsequent calculation of pairwise cosine similarities using matrix multiplication, which reflects the underlying relationships between audio and text embeddings. The use of the acquired temperature parameter adjusts the importance of these pairwise similarities. Based on this foundation, the algorithm's optimization is enhanced by calculating a symmetric cross-entropy loss that takes into account audio-to-text as well as text-to-audio relationships. The final loss metric directing the model's training trajectory is achieved by averaging these dual losses.

\section{U-Net}

\begin{lstlisting}[language=Python, caption={U-Net architecture for semantic segmentation}]
class UNet(nn.Module):

    # ... (other initialization code)

    def forward(self, x):
        x = x.to(self.device)

        # implements the forward pass with concatenations
        skip_connections = []

        # contracting path
        for block in self._contracting_path:
            # create a sequential block from the list of layers
            net = nn.Sequential(*block).to(self.device)

            # apply
            x = net(x)

            # save the skip connection
            skip_connections.append(x)

        # bottleneck
        x = self._bottleneck(x)

        # expanding path
        for layer_idx in range(self.depth - 1):
            # the first layer is a transposed convolutional layer
            transposed_conv = self._expanding_path_layers[layer_idx * 2].to(
                self.device)
            x = transposed_conv(x)

            # concatenate the skip connection
            skip_connection = skip_connections.pop()
            # make sure the shapes match
            if x.shape != skip_connection.shape:
                # resize the skip connection
                skip_connection = nn.functional.interpolate(skip_connection,
                                                            size=x.shape[2:],
                                                            mode='nearest')

            # concatenate the skip connection
            x = torch.cat((x, skip_connection), dim=1)

            # the second layer is a sequential block of convolutional layers
            conv_block = self._expanding_path_layers[layer_idx * 2 + 1].to(
                self.device)
            x = conv_block(x)

        # final convolutional layer
        x = self._expanding_path_layers[-1].to(self.device)(x)

        return x

    def _make_contracting_path(self):
        """
        Create the contracting path of the U-Net.
        """
        layers = []

        # configs the used convolutional layers
        in_channels = self.in_channels
        out_channels = 64

        # create a convolutional block for each number in depth
        for _ in range(self.depth - 1):
            # create a convolutional block
            block = []

            # create the number of convolutional layers specified by
            # conv_layers_per_block
            for _ in range(self.conv_layers_per_block):
                block.append(nn.Conv2d(in_channels=in_channels,
                             out_channels=out_channels, kernel_size=3, padding="same"))
                # batch normalization
                block.append(nn.BatchNorm2d(out_channels))
                # ReLU activation
                block.append(nn.ReLU())

                # update the in_channels
                in_channels = out_channels

            # add a max pooling layer
            block.append(nn.MaxPool2d(kernel_size=2))

            # add the block to the layers
            layers.append(block)

            # double the number of channels
            out_channels *= 2

        # create a network from the layers
        return layers

    def _make_bottleneck(self):
        """
        Create the bottleneck of the U-Net.
        """
        # build the bottleneck
        layers = []

        # config the bottleneck channels
        in_channels = 64 * (2 ** (self.depth - 2))
        out_channels = in_channels * 2

        # create the number of convolutional layers specified by
        # conv_layers_per_block
        for _ in range(self.conv_layers_per_block):
            layers.append(nn.Conv2d(in_channels=in_channels,
                                    out_channels=out_channels,
                                    kernel_size=3,
                                    padding="same"))
            # batch normalization
            layers.append(nn.BatchNorm2d(out_channels))
            # ReLU activation
            layers.append(nn.ReLU())

            # update the in_channels
            in_channels = out_channels

        # add all the layers in block in the layers list
        return nn.Sequential(*layers).to(self.device)

    def _make_expanding_path(self):
        """
        Create the expanding path of the U-Net.
        This returns a list of layers, which will be used in the forward pass.
        Some layers are transposed convolutional layers, others are sequential
            blocks of convolutional layers.
        """
        layers = []

        # configs the used convolutional layers
        # the number of in channels is the number of out channels from the last
        # block in the contracting path
        in_channels = 64 * (2 ** (self.depth - 1))
        out_channels = in_channels // 2

        # create a convolutional block for each number in depth
        for _ in range(self.depth - 1):
            # add an up conv 2x2
            layers.append(nn.ConvTranspose2d(in_channels=in_channels,
                                             out_channels=out_channels,
                                             kernel_size=2,
                                             stride=2))

            block = []

            # create the number of convolutional layers specified by
            # conv_layers_per_block
            for _ in range(self.conv_layers_per_block):
                block.append(nn.Conv2d(in_channels=in_channels,
                             out_channels=out_channels, kernel_size=3, padding="same"))
                # batch normalization
                block.append(nn.BatchNorm2d(out_channels))
                # ReLU activation
                block.append(nn.ReLU())

                # update the in_channels
                in_channels = out_channels

            # add the block to the layers
            layers.append(nn.Sequential(*block))

            # double the number of channels
            out_channels //= 2

        # final convolutional layer
        final_layer = []
        final_layer.append(nn.Conv2d(in_channels=in_channels,
                                     out_channels=self.out_channels,
                                     kernel_size=1,
                                     padding="same"))

        final_layer.append(nn.Softmax(dim=1))

        layers.append(nn.Sequential(*final_layer))

        # create a network from the layers
        return layers
\end{lstlisting}

The code provides insight into the U-Net architecture, renowned for its effectiveness in segmentation tasks. It includes a class called \texttt{UNet}, which embodies the architectural blueprint of the U-Net model. Its structure is composed of three primary components - the contracting path, the bottleneck layer, and the expanding path - which are crucial to the model's implementation. The forward pass follows the characteristic U-shaped configuration.  This architectural layout is used to create the favorable integration of skip connections, which enhances the capacity for precise segmentation.
    
    \chapter{GANmix Results Table} \label{ann:ganmix-results}

This appendix presents the results of experiments conducted with the GANmix model.

The experiments were designed to evaluate the performance of the GANmix model on two datasets: Audio MNIST and Clotho. The experiments also varied the number of parameters, the loss function, the regularization scheme, the dataset, and the optimizer used to train the GANmix model. The results are summarized in two tables: Table~\ref{tab:ganmix-results-1} shows the experimental settings and~\ref{tab:ganmix-results-2} shows the corresponding losses for the generator, the discriminator, and the total loss. Losses are calculated using \ac{BCE}.

\begin{table}[ht]
\caption{Experimental GANmix results (Part 1).}
\label{tab:ganmix-results-1}
\begin{tabularx}{\textwidth}{|l|X|X|X|X|X|}
\hline
\textbf{Exp.} & \textbf{\#Params}                           & \textbf{Loss}     & \textbf{Regularization} & \textbf{Dataset} & \textbf{Optimizer} \\ \hline
\textbf{1}    & $\sim$4M even                               & BCE               & FALSE                   & Audio MNIST      & Adam               \\ \hline
\textbf{2}    & $\sim$4M even                               & BCE               & FALSE                   & Audio MNIST      & Adam               \\ \hline
\textbf{3}    & $\sim$2M discriminator, $\sim$25M generator & BCE               & FALSE                   & Audio MNIST      & Adam               \\ \hline
\textbf{4}    & $\sim$2M discriminator, $\sim$25M generator & BCE               & FALSE                   & Audio MNIST      & SGD                \\ \hline
\textbf{5}    & $\sim$2M discriminator, $\sim$25M generator & BCE + Elastic Net & TRUE                    & Audio MNIST      & Adam               \\ \hline
\textbf{6}    & $\sim$4M even                               & BCE + Elastic Net & TRUE                    & Clotho           & Adam               \\ \hline
\textbf{7}    & $\sim$50M even                              & BCE + Elastic Net & TRUE                    & Clotho           & Adam               \\ \hline
\textbf{8}    & $\sim$50M even                              & BCE               & FALSE                   & Clotho           & Adam               \\ \hline
\textbf{9}    & $\sim$50M even                              & BCE               & TRUE                    & Clotho           & Adam               \\ \hline
\textbf{10}   & $\sim$50M even (linear)                             & BCE               & TRUE                    & Clotho           & Adam               \\ \hline
\end{tabularx}
\end{table}

\begin{table}[ht]
\caption{Experimental GANmix results (Part 2).}
\label{tab:ganmix-results-2}
\begin{tabularx}{\textwidth}{|l|X|X|X|X|X|X|}
\hline
\textbf{Exp.} & \textbf{HW} & \textbf{LR Generator} & \textbf{LR Discriminator} & \textbf{Gen Loss} & \textbf{Dis Loss} & \textbf{Total Loss} \\ \hline
\textbf{1}    & Kaggle      & 1.00E-04              & 1.00E-04                  & 0.487             & 1.440             & 1.927               \\ \hline
\textbf{2}    & Kaggle      & 1.00E-02              & 1.00E-02                  & 0.545             & 1.412             & 1.957               \\ \hline
\textbf{3}    & Kaggle      & 1.00E-03              & 1.00E-04                  & 0.351             & 1.686             & 2.037               \\ \hline
\textbf{4}    & LIACC 1     & 1.00E-03              & 1.00E-04                  & 0.516             & 1.429             & 1.945               \\ \hline
\textbf{5}    & LIACC 1     & 1.00E-03              & 1.00E-04                  & 3.362             & 1.428             & 4.791               \\ \hline
\textbf{6}    & LIACC 2     & 1.00E-03              & 1.00E-04                  & 0.587             & 1.458             & 2.045               \\ \hline
\textbf{7}    & LIACC 2     & 1.00E-03              & 1.00E-04                  & 1.738             & 1.439             & 3.177               \\ \hline
\textbf{8}    & LIACC 2     & 1.00E-03              & 1.00E-04                  & 0.693             & 1.139             & 1.832               \\ \hline
\textbf{9}    & LIACC 2     & 1.00E-03              & 1.00E-04                  & 0.693             & 1.007             & 1.700               \\ \hline
\textbf{10}   & LIACC 2     & 1.00E-03              & 1.00E-04                  & 6.987                  & -0.562                  & 6.425               \\ \hline
\end{tabularx}
\end{table}
\end{appendices}

%% Index
%% Uncomment next command if index is required
%% don't forget to run ``makeindex thesis'' command
% \PrintIndex

\end{document}