\usepackage{tikz}
\usetikzlibrary{backgrounds}
\usetikzlibrary{arrows}
\usetikzlibrary{shapes,shapes.geometric,shapes.misc}
\usetikzlibrary{fadings}
\usetikzlibrary{3d,decorations.text,shapes.arrows,positioning,fit}

% this style is applied by default to any tikzpicture included via \tikzfig
\tikzstyle{tikzfig}=[baseline=-0.25em,scale=0.5]

% standard layers used in .tikz files
\pgfdeclarelayer{edgelayer}
\pgfdeclarelayer{nodelayer}
\pgfsetlayers{background,edgelayer,nodelayer,main}

% style for blank nodes
\tikzstyle{none}=[inner sep=0mm]

% include a .tikz file
\newcommand{\tikzfig}[1]{%
{\tikzstyle{every picture}=[tikzfig]
\IfFileExists{#1.tikz}
  {\input{#1.tikz}}
  {%
    \IfFileExists{./figures/#1.tikz}
      {\input{./figures/#1.tikz}}
      {\tikz[baseline=-0.5em]{\node[draw=red,font=\color{red},fill=red!10!white] {\textit{#1}};}}%
  }}%
}

% the same as \tikzfig, but in a {center} environment
\newcommand{\ctikzfig}[1]{%
\begin{center}\rm
  \tikzfig{#1}
\end{center}}

% fix strange self-loops, which are PGF/TikZ default
\tikzstyle{every loop}=[]

% TiKZ style file generated by TikZiT. You may edit this file manually,
% but some things (e.g. comments) may be overwritten. To be readable in
% TikZiT, the only non-comment lines must be of the form:
% \tikzstyle{NAME}=[PROPERTY LIST]

% Node styles
\tikzstyle{input}=[draw={rgb,255: red,8; green,111; blue,175}, ultra thick, shape=circle, minimum size=1cm]
\tikzstyle{neuron}=[ultra thick, shape=circle, minimum size=1cm, draw={rgb,255: red,7; green,183; blue,133}]
\tikzstyle{output}=[draw={rgb,255: red,15; green,171; blue,255}, ultra thick, shape=circle, minimum size=1cm]
\tikzstyle{etc}=[fill=black, draw=none, shape=circle, scale=0.2, text height=0.333cm]
\tikzstyle{main rect}=[fill={rgb,255: red,197; green,255; blue,190}, draw=none, shape=rectangle]
\tikzstyle{blue square}=[fill={rgb,255: red,152; green,197; blue,255}, draw={rgb,255: red,84; green,110; blue,141}, shape=rectangle, minimum width=1cm, minimum height=1cm]
\tikzstyle{paint gray}=[fill={rgb,255: red,7; green,183; blue,133}, fill opacity=0.1, draw=black]
\tikzstyle{transformer rect}=[fill={rgb,255: red,244; green,244; blue,244}, draw=none, shape=rectangle, minimum width=3cm, minimum height=6cm, rounded corners=10]

% Edge styles
\tikzstyle{->input}=[draw={rgb,255: red,8; green,111; blue,175}, ->, thick]
\tikzstyle{input->neuron}=[draw={rgb,255: red,8; green,111; blue,175}, ->, thick]
\tikzstyle{neuron->neuron}=[draw={rgb,255: red,7; green,183; blue,133}, ->, thick]
\tikzstyle{neuron->output}=[draw={rgb,255: red,15; green,171; blue,255}, ->, thick]
\tikzstyle{new edge style 0}=[-, draw={rgb,255: red,114; green,67; blue,255}]
\tikzstyle{arrow}=[->, thin]
\tikzstyle{neuron-neuron}=[-, draw={rgb,255: red,7; green,183; blue,133}, thick]
\tikzstyle{new edge style 1}=[-, fill={rgb,255: red,213; green,213; blue,213}, draw=none]



% For the CNN

\tikzset{pics/fake box/.style args={% #1=color, #2=x dimension, #3=y dimension, #4=z dimension
#1 with dimensions #2 and #3 and #4}{
code={
\draw[gray,ultra thin,fill=#1]  (0,0,0) coordinate(-front-bottom-left) to
++ (0,#3,0) coordinate(-front-top-right) --++
(#2,0,0) coordinate(-front-top-right) --++ (0,-#3,0) 
coordinate(-front-bottom-right) -- cycle;
\draw[gray,ultra thin,fill=#1] (0,#3,0)  --++ 
 (0,0,#4) coordinate(-back-top-left) --++ (#2,0,0) 
 coordinate(-back-top-right) --++ (0,0,-#4)  -- cycle;
\draw[gray,ultra thin,fill=#1!80!black] (#2,0,0) --++ (0,0,#4) coordinate(-back-bottom-right)
--++ (0,#3,0) --++ (0,0,-#4) -- cycle;
\path[gray,decorate,decoration={text effects along path,text={CNN}}] (#2/2,{2+(#3-2)/2},0) -- (#2/2,0,0);
}
}}
% from https://tex.stackexchange.com/a/52856/121799
\tikzset{circle dotted/.style={dash pattern=on .05mm off 2mm,
                                         line cap=round}}
