\section{Motivation} \label{sec:motivation}

In recent years, there has been a significant increase in the use of \ac{ML} techniques for audio processing tasks, such as sound synthesis, audio restoration, and speech recognition. The ability of \ac{ML} algorithms to learn and extract complex patterns from large datasets has shown promising results in improving the quality and efficiency of audio processing tasks.

Furthermore, integrating \ac{ML} techniques in sound generation technologies can revolutionize how one creates and experiences sound. It can provide new avenues for artists and musicians to explore their creativity and produce unique and innovative audio content. It can also offer new possibilities for sound design in various industries, such as film, gaming, and virtual reality.

The current need for studies in sound generation technologies highlights the need for further research and development. This dissertation endeavors to establish itself as a significant study in this field. It offers high-quality resources to researchers and developers to investigate the potential and limitations of \ac{ML} techniques for sound synthesis.

In today's world, digital technologies are reshaping our relationship with music and sound by enabling innovative capabilities \cite{tahiroglu_-terity_2020}. This study strives to investigate and broaden this impact by offering a novel tool that bolsters human potential in sound creation. Additionally, the findings can be a valuable resource for audio processing researchers, developers, and practitioners. The research findings offer guidance for future research and development endeavors concerning sound generation technologies and provide valuable insights into the most effective practices and techniques for utilizing \ac{ML} in audio processing.

Overall, this study's significance and potential impact make it a worthwhile and valuable contribution to the field of audio processing and machine learning.