\section{Objectives} \label{sec:objectives}

To address the core motivations behind this research, this dissertation aims to undertake a comprehensive study of \ac{DL} and, more specifically, generative deep learning models in the context of sound synthesis.

The goals include conducting a comprehensive survey of existing \ac{DL} architectures for audio generation while analyzing their strengths and limitations. In addition, novel approaches will be proposed and developed to further advance the field.

By pursuing these revised objectives, this research aims to provide valuable insights into state-of-the-art in sound synthesis using \ac{DL} methods. Furthermore, it aims to provide practical guidance for future advances in creating high-quality audio outputs based on textual inputs.

In order to accomplish this implementation, some specific goals are set:

\begin{enumerate}
	\item Make a study of the current state-of-the-art deep learning architectures, focusing on generative ones.
	\item Examine prior algorithms that can process sound for augmentation, feature extraction, or other purposes.
	\item Make a study of the current state-of-the-art architectures used to develop sounds artificially.
	\item Develop end-to-end systems that can synthesize sound from any given text input, while accounting for hardware constraints and ensuring reliable performance.
	\item Evaluate the systems' ability to generate a sound from the given textual input accurately.
\end{enumerate}