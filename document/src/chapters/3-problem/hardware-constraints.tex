\section{Scope, Limitations, and Technical Constraints} \label{sec:limitations}

This study aims to advance the capabilities of audio generation from textual input, at the intersection of audio synthesis and generative \ac{AI}. This research focuses on developing and refining end-to-end generative models that seamlessly translate textual prompts into immersive soundscapes, in the landscape of \ac{AI}-driven audio synthesis.

\subsection{Technical Constraints}

This research in algorithmic audio generation is shaped by a series of pragmatic technical constraints that influence our study's course. These constraints include hardware and computational considerations that are intrinsic to the development and training of sophisticated AI models for audio synthesis.

Contemporary AI research requires significant resource allocation, profoundly impacting our efforts. Studies, such as the development of DALL-E 2 (see Section~\ref{sec:dall-e-2}), vividly illustrate this resource-intensive trajectory.

The training process for this model requires an investment of 100,000 to 200,000 GPU hours, based on the architecture presented in Annex C of the paper. Assuming V100 GPUs are utilized at \$3 per hour (in AWS~\cite{amazon_web_services_inc_amazon_nodate}), the minimum financial commitment amounts to \$300,000. This significant investment highlights the need for substantial resources in the ongoing pursuit of AI advancements. This substantial figure reflects not only the computational demands of AI but also the economic dimensions that drive progress within these domains.

When focusing on audio synthesis, a parallel relationship becomes evident. Generating audio snippets from textual cues poses similar challenges to those encountered in computer vision. Effective audio synthesis relies on robust generative \ac{AI} models for which the training process demands a substantial share of computational resources. These computational resources are not available to individual scientists working alone. As larger models tend to perform better in the generative domain, it is increasingly challenging for scientists and developers to compete against big tech companies.

\subsection{Ethical Considerations}

Ethical considerations play a crucial role in the domain of audio synthesis driven by generative \ac{AI} models. As the possibilities for creative expression continue to grow, it is important to address the ethical aspects that come with the transformative potential of these technologies. This discussion explores the potential ethical implications and highlights the utmost significance of responsible deployment in the field of generative audio synthesis.

One major concern arises in the potential creation of harmful or deceitful content. Generative models' ability to create various audio outputs from text inputs presents opportunities for innovation but also poses inherent risks of producing malicious or deceitful content. The ethical tightrope that creators must balance is underscored by the dual nature of generative prowess.

For example, if an ill-intentioned person utilizes generative audio synthesis to create authentic voice recordings of people, it may result in spreading false information or fabricating forged audio evidence. The remarkable precision of generative models in imitating voices could be misused to commit identity fraud, cause disharmony, or construct digital impersonations that deceive and manipulate unsuspecting listeners.

Moreover, ethical concerns apply to the various sources of textual inputs. Textual clues extracted from public texts, personal submissions, or online repositories are used as raw material for generative processes. These sources require rigorous methods to refine the generative process itself, ensuring that audio outputs are true and precise. Moreover, the responsibility also includes careful curation of data inputs to prevent the perpetuation of biases or prejudiced content. The handling of text inputs may perpetuate ethical implications based on cultural, social, or ideological connotations, which calls for robust mechanisms to prevent the unintended amplification of unethical content.

In studies like this, the researcher does not collect the dataset, but they are responsible for ensuring that the dataset used is unbiased and free of prejudices.

Considering these ethical dimensions, it is clear that the use of generative audio synthesis carries significant ethical responsibilities. While one explores this unexplored territory, it is essential that ethical considerations serve as steadfast sentinels, guiding the responsible use of generative \ac{AI} models in audio synthesis.