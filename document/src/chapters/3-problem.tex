\chapter{The Synthesis Problem}\label{chap:problem}

\minitoc

This chapter explores the intricate domain of sound synthesis through advanced generative \ac{AI} models that operate on textual input. The exploration seeks to unravel the complexities and nuances inherent in the fusion of \ac{AI} and sound engineering, resulting in exceptional fidelity and realism in auditory creations.

Sound synthesis, a venerable discipline in the aural arts, stands at the intersection of traditional craftsmanship and contemporary technology. The digital age is ushering in a profound transformation in which the marriage of textual prompts and generative \ac{AI} creates a symphony of possibilities for creating immersive auditory experiences.

Generative \ac{AI} orchestrates a paradigm shift in the landscape of sound synthesis. This chapter seeks to illuminate the nascent symphony of generative \ac{AI}'s role in creating intricate soundscapes, shedding light on its potential to revolutionize content creation, music composition, and multimedia production.

This chapter delves into the synthesis problem through several key sections. The first Section (\ref{sec:problem-definition}) defines the core challenges. Then, Section~\ref{sec:significance} highlights the broader implications. Section~\ref{sec:prob-datasets} examines the underlying data. Section~\ref{sec:limitations} outlines the limitations of the research. Model fine-tuning is explored in Section~\ref{sec:parametric-control}, while architectural subtleties are revealed in Section~\ref{sec:model}.


\section{Problem Definition} \label{sec:problem-definition}

The rapid advancement of artificial intelligence and machine learning techniques has paved the way for transformative developments in content creation across multiple disciplines. However, a significant gap remains in the field of audio synthesis, especially in the context of generative \ac{AI} models. The challenge at hand is to bridge this gap and enable the transformation of textual input into high-fidelity and immersive soundscapes, thereby creating a new form of creative expression.

Recently, the landscape of audio generation has been undergoing a paradigm shift, catalyzed by significant investments and efforts from prominent technology conglomerates. As of 2023, major industry players are placing significant bets on the potential of \ac{AI}-driven audio synthesis. This noteworthy development underscores the growing recognition of the transformative impact that sophisticated audio generation can have across sectors, including entertainment, education, virtual reality, and more.

\subsection{Gap in the Literature}

The landscape of audio synthesis through generative \ac{AI} models presents a spectrum of challenges and limitations that impact various applications, including content creation, sound design, music production, and interactive media. These challenges include computational constraints, model architectural intricacies, quality/realism trade-offs, data-related hurdles, interpretability issues, and scalability concerns. Addressing these limitations has the potential to significantly improve the quality and utility of audio synthesis methods.

Significant hurdles in the use of generative \ac{AI} models for audio synthesis are the computational cost and the training time. PixelCNN Decoders, Jukebox, and Hi-Fi GAN, which exemplify these models, require extensive computational resources and long training times. These limitations render these models inaccessible to researchers and developers and limit their applicability in real-world situations where efficient generation is critical. Developing more efficient training algorithms or model architectures could democratize access and broaden the reach of audio synthesis methods by overcoming these challenges.

Producing \textbf{high-quality and realistic audio} synthesis is a complex task. The lack of ability to capture minute details, as seen in models such as WaveNet and MelGAN, causes audio outputs to be deficient in delicate nuances and textures. Limitations such as mode collapse, GANSynth, and Jukebox can cause a lack of variety, leading to repetitive and monotonous audio outputs. Additionally, maintaining coherence throughout extended audio sequences - as observed in PixelCNN, WaveNet, Jukebox, and Riffusion - is a challenge that can affect the final output.

The versatility of synthesis models is impacted by data efficiency and generalization capabilities. Models such as DALL-E, VALL-E, and MusicLM have suboptimal data efficiency, which requires large datasets for effective training. It is crucial to address the challenges in handling unusual concepts, as observed in \ac{GLIDE}, to broaden the creative potential of these models. Moreover, the practical utility and user-friendliness of models such as DALL-E and VALL-E are constrained due to their limited control and interpretability.

The challenges and limitations within audio synthesis through generative AI models align with the key objectives of this thesis. The aim to develop advanced systems that handle sound, classification, and end-to-end generation while accounting for hardware limitations aligns with the need to overcome computational constraints highlighted in the literature. The thesis aims to address the computational hurdles that hinder the accessibility and practicality of audio synthesis models by devising more efficient training algorithms or model architectures. This directly contributes to the broader objective of providing valuable contributions to deep learning and its applications.

The goal to create end-to-end systems that generate sound from any textual input is intricately connected to the challenges of quality, realism, and coherence in audio synthesis. The thesis aims to bridge the gap between textual input and high-fidelity, immersive soundscapes by exploring the intricacies of model architectural design and refining the generation process. Evaluating the accuracy of generated audio aligns with the broader pursuit of overcoming limitations in capturing fine-grained details, mode collapse, and long-term coherence. This pursuit ultimately enhances the authenticity and richness of the synthesized audio.

The ambitious mission to contribute to \ac{DL} and its applications aligns well with the goal of addressing data-related issues, such as biases and limitations arising from training data. This thesis aims to enhance the adaptability of generative \ac{AI} models by curating diverse and representative datasets and implementing techniques for bias detection and mitigation, to make them more responsive to various tasks and domains. This alignment highlights the research's importance in advancing audio synthesis and establishing a wider framework for responsible and inclusive development of \ac{AI}.

To summarize, the literature's identified gap, marked by challenges ranging from computational constraints and quality/realism to data efficiency and interpretability issues, is intricately linked with this thesis's objectives. Advancing end-to-end audio synthesis systems while addressing these challenges not only deepens the field's scientific comprehension but also profoundly aligns with the broader goals of your research endeavor.

\subsection{Formal Problem Definition}

The goal of this study is to address the synthesis of audio based on textual prompts. The audio is not limited to a specific domain, such as speech or music. The audio produced by the model must be dependent on the inserted textual input. The generated audio must be produced in real-time, with minimal latency after the textual prompt is inserted into the model. This study considers a model that resolves this issue and refers to it as an end-to-end model. Furthermore, it is considered that the terms end-to-end and text-to-sound are interchangeable in this context.

These models are trained using sounds labeled by humans from openly available datasets. Furthermore, the evaluation involves comparing the generated sounds with their corresponding real sounds.

To formally define this problem, the following elements must be specified:

\begin{itemize}
    \item input data
    \item output data
    \item objective function
    \item optimization algorithm
    \item evaluation metrics
\end{itemize}

\subsubsection{Input Data}

The synthesis of audio from textual prompts introduces a captivating interplay between audio samples and textual labels, infusing the model's learning process with a symphony of complexity and richness. However, synthesizing audio from textual prompts entails several challenges, especially when dealing with the diverse formats present in sound samples and text labels. As these formats converge, they add layers of intricacy that require innovative solutions and thoughtful consideration.

In the field of sound samples, dealing with diverse formats manifests itself in a medley of sonic expressions, ranging from musical melodies and spoken dialogues to environmental noises and abstract compositions. These formats encompass a spectrum of acoustic textures and temporal dynamics, demanding a unique processing and analysis approach for each. The model must navigate through this sonic kaleidoscope, effectively capturing and learning from the essence of each format without resorting to a one-size-fits-all approach.

Text labels also introduce a variety of challenges as they take various forms from different sources. The range of text labels spans from brief categorical tags that capture high-level features to verbose textual descriptions that explore intricate details. This variation poses the model with the task of decoding and utilizing the diverse meanings, tones, and contexts embedded within these labels. Achieving balance between the weight of these different formats while incorporating them into the generative fabric poses its own challenge.

The intersection of different audio formats and textual labeling structures presents a challenge for the model to generate coherent and harmonious audio outputs. Each format has its implicit dimensions, nuances, and potential biases associated with it. Ensuring that the model avoids being biased towards a particular format or ignores others is a challenge that needs precise calibration. Additionally, the model must handle the complex task of cross-modal learning, deciphering the interplay between sound and text to produce outputs that match seamlessly with textual prompts.

When embarking on the complex process of audio synthesis from text, it is important to acknowledge the intricacies that arise from combining sound samples and text labels. The difficulties that come with handling various formats serve as milestones of progress and discovery. These challenges put the model's true capabilities to the test as it navigates through the maze of formats, becoming a creative force capable of orchestrating sonic symphonies that blend seamlessly with the tapestry of human expression.

\subsubsection{Output Data}

The output produced by these models is in a particular audio format, resulting from the complex interplay between textual cues and sonic expression. An essential factor in this synthesis is real-time generation, which is critical for both user experience and practical applications.

Real-time generation has the potential to provide immediate results, smoothly converting textual input into audio output. In practical contexts, such as interactive media, virtual environments, or assistive technologies, the capability to respond promptly to user inputs with sonic output enhances the immersive quality of the experience.

The pursuit of real-time generation comes with its share of difficulties and trade-offs. The requirement for swift responsiveness can create limitations that may affect the precision and complexity of the produced audio. The delicate balance between speed and sonic complexity requires careful consideration.

Real-time generation plays a critical role in generative audio synthesis. Achieving real-time outputs requires a balance between immediacy and auditory finesse, guided by studies on human perception. During this pursuit, the generative model attunes itself not only to the cadence of textual cues, but also to the rhythm of human interaction and appreciation.

\subsubsection{Objective Function}

The development of an objective function is a crucial step in this work, where challenges, strategies, and insights come together to shape the core of creative synthesis.

The purpose of generative models is to uncover and encapsulate the complex distributions that form the basis of the data. This journey of comprehension accompanies the aspiration to create new data, which evokes familiarity with the original while testing the limits of artistic innovation. This essence, charming as it may be, emerges within a paradox that challenges us to bridge the gap between accuracy and realism by channeling the learned distributions into outputs that reflect authenticity.

The challenge in defining an objective function lies in extracting the core components of accuracy and realism, and quantifying them into a measurable metric. One approach that aligns with the generative pursuit involves reducing the divergence between the original and synthesized data.

Within the realm of generative literature, numerous strategies intertwine to inspire the creation of objective functions. Approaches presented in Section~\ref{sec:evaluation} should be taken into account.

The objective function emerges as a guiding principle in this interplay of creativity and computation. The objective function embodies the urge for realism and accuracy, reflecting the very essence of generative models in its mathematical grasp. The definition of the objective function is intertwined with the fabric of the generative narrative, uniting aspirations and strategies into a symphony that echoes across the domains of imagination and reality.

\subsubsection{Optimization Algorithm}

The optimization algorithm aims to shape the generative landscape based on the objectives defined within the objective function.

Selecting an optimization algorithm is a crucial decision in this field that needs careful consideration. Each generative model has its own intrinsic characteristics, strengths, and idiosyncrasies, and it is these traits that determine the choice of optimization algorithm. The algorithm guides the process towards optimal solutions and shapes the generative narrative with each iterative step.

Selecting an optimization algorithm is an arbitrated process where the properties of the generative model are tailored to the unique contours of problem landscape. It entails a delicate calibration to adapt the nuances of established algorithms, harmonizing them with the demands of the generative journey. Fine-tuning hyperparameters, sculpting convergence criteria, and managing trade-offs are integral facets of optimization.

The optimization landscape within the explored generative models consists of various algorithms that align with the specific requirements of each model.

Exploring optimization algorithms reveals a landscape that mirrors the fusion of disciplines within generative fields. Gaining insights from seminal works across various optimization methodologies, the optimization algorithm acts as a creative tool, ready to enhance the generative canvas with precision and convergence.

\subsubsection{Evaluation Metrics}

Evaluation metrics are necessary to guide exploration into the realm of generative models.

However, evaluating generative models involves navigating uncharted territories, especially when considering the intricate realm of subjective attributes. The concepts of audio realism, coherence, and emotive resonance are subjective, making it challenging to measure them objectively. Achieving sound realism requires balancing the timbral fidelity and temporal authenticity, which creates a challenge requiring innovative solutions. Similarly, coherence in generative outputs, determined by the harmony among sequential elements, necessitates metrics that can reveal the intricate patterns.

Specific evaluation metrics emerge in this pursuit as guiding stars, each one poised to illuminate a unique facet of generative prowess. Perceptual audio quality metrics, which are rooted in the perceptual space of human hearing, transcend the realm of raw signal processing. They reflect the intricate tapestry of auditory perception. Subjective metrics such as \ac{MOS} are used. Research delves into the visceral nature of human judgment and provides a medium on which subjective impressions are recorded.

Among the many metrics, their relevance, appropriateness, and coherence with the generative narrative are crucial. Metrics anchor the ephemeral expanse of creativity to the solid ground of quantitative analysis.
\section{Application of Synthesizing Soundscapes with Generative AI} \label{sec:significance}

The development of \ac{AI} technologies has enabled significant progress in sound synthesis, enabling the creation of sounds based on textual input. This technology has potential applications in various fields, including music production, film and game sound design, and therapeutic soundscapes. In this response, this Section explores possible applications for the \ac{AI} models described in this document.

One potential application of such \ac{AI} models is in the field of music composition. By generating sounds from textual descriptions, the model could assist composers in generating novel and unique sounds that match a piece's intended mood or atmosphere. For example, a composer might input the phrase ``eerie forest at night'', and the \ac{AI} model could generate a soundscape incorporating the sounds of rustling leaves, distant animal calls, and other eerie sounds one might associate with a forest at night. This technology could help composers to create soundscapes more efficiently that match their creative vision, saving time and increasing their overall output. Besides, the real-time inference characteristic of this model may help a live performer blend the timbres of instruments with, for instance, natural sounds, creating soundscapes on the fly \cite{huzaifah_deep_2021}.

Another possible application of an \ac{AI} model that generates sound from text is in the field of film and game sound design. Sound design plays a crucial role in creating immersive and engaging experiences in film and games, and the ability to generate custom soundscapes from textual input could enhance the creative potential of sound designers. For example, a sound designer might input the phrase ``a bustling city street'', and the \ac{AI} model could generate a soundscape that includes the sounds of car horns, people talking, footsteps, and other city noises. Alternatively, they may want to generate sound cues such as explosions \cite{huzaifah_deep_2021}, and a simple query would do that. This technology could help sound designers create more realistic and immersive soundscapes, improving the overall quality of the final product. 

A panoply of sounds is usually taken from expensive libraries in TV and film. One might imagine an infinite library with \ac{ML} models, and one should imagine how that will enhance the power of sound producers for such endeavors, especially in indie and low-budget productions.
\section{Datasets} \label{sec:prob-datasets}

For the proposed problem, ideal datasets would have entries of sound samples under multiple conditions with labels written in a \ac{NLP} form.

With multiple conditions, one may think, for instance, of multiple sounds of a dog barking. An ideal dataset would have a dog barking in different places, under different atmospheric conditions, and with other possible variants. For example, this ideal dataset would have ``A dog barking in a city'', ``A dog barking in the countryside where it is raining'', and others.

Multiple sounds over or followed by each other would exist in the ideal dataset. For example, ``A dog barking followed by a truck honking'' or ``A traffic jam while a woman sings'' would be examples of the entries in the perfect dataset.

The perfect dataset would also have characteristics of the sounds themselves. For instance, not only ``A dog barking'', but rather ``A dog barking aggressively'', or ``A dog barking with joy'' would be examples of entries in the perfect dataset.

These datasets do not exist in the real world. Thus, some measures must be taken. This discussion is proposed in Section~\ref{sec:sol-datasets}.
\section{Scope, Limitations, and Technical Constraints} \label{sec:limitations}

This study aims to advance the capabilities of audio generation from textual input, at the intersection of audio synthesis and generative \ac{AI}. This research focuses on developing and refining end-to-end generative models that seamlessly translate textual prompts into immersive soundscapes, in the landscape of \ac{AI}-driven audio synthesis.

\subsection{Technical Constraints}

This research in algorithmic audio generation is shaped by a series of pragmatic technical constraints that influence our study's course. These constraints include hardware and computational considerations that are intrinsic to the development and training of sophisticated AI models for audio synthesis.

Contemporary AI research requires significant resource allocation, profoundly impacting our efforts. Studies, such as the development of DALL-E 2 (see Section~\ref{sec:dall-e-2}), vividly illustrate this resource-intensive trajectory.

The training process for this model requires an investment of 100,000 to 200,000 GPU hours, based on the architecture presented in Annex C of the paper. Assuming V100 GPUs are utilized at \$3 per hour (in AWS~\cite{amazon_web_services_inc_amazon_nodate}), the minimum financial commitment amounts to \$300,000. This significant investment highlights the need for substantial resources in the ongoing pursuit of AI advancements. This substantial figure reflects not only the computational demands of AI but also the economic dimensions that drive progress within these domains.

When focusing on audio synthesis, a parallel relationship becomes evident. Generating audio snippets from textual cues poses similar challenges to those encountered in computer vision. Effective audio synthesis relies on robust generative \ac{AI} models for which the training process demands a substantial share of computational resources. These computational resources are not available to individual scientists working alone. As larger models tend to perform better in the generative domain, it is increasingly challenging for scientists and developers to compete against big tech companies.

\subsection{Ethical Considerations}

Ethical considerations play a crucial role in the domain of audio synthesis driven by generative \ac{AI} models. As the possibilities for creative expression continue to grow, it is important to address the ethical aspects that come with the transformative potential of these technologies. This discussion explores the potential ethical implications and highlights the utmost significance of responsible deployment in the field of generative audio synthesis.

One major concern arises in the potential creation of harmful or deceitful content. Generative models' ability to create various audio outputs from text inputs presents opportunities for innovation but also poses inherent risks of producing malicious or deceitful content. The ethical tightrope that creators must balance is underscored by the dual nature of generative prowess.

For example, if an ill-intentioned person utilizes generative audio synthesis to create authentic voice recordings of people, it may result in spreading false information or fabricating forged audio evidence. The remarkable precision of generative models in imitating voices could be misused to commit identity fraud, cause disharmony, or construct digital impersonations that deceive and manipulate unsuspecting listeners.

Moreover, ethical concerns apply to the various sources of textual inputs. Textual clues extracted from public texts, personal submissions, or online repositories are used as raw material for generative processes. These sources require rigorous methods to refine the generative process itself, ensuring that audio outputs are true and precise. Moreover, the responsibility also includes careful curation of data inputs to prevent the perpetuation of biases or prejudiced content. The handling of text inputs may perpetuate ethical implications based on cultural, social, or ideological connotations, which calls for robust mechanisms to prevent the unintended amplification of unethical content.

In studies like this, the researcher does not collect the dataset, but they are responsible for ensuring that the dataset used is unbiased and free of prejudices.

Considering these ethical dimensions, it is clear that the use of generative audio synthesis carries significant ethical responsibilities. While one explores this unexplored territory, it is essential that ethical considerations serve as steadfast sentinels, guiding the responsible use of generative \ac{AI} models in audio synthesis.
\section{Parametric Control} \label{sec:parametric-control}

If one creates a generator model that generates sounds without any input, one would only generate random sounds without any meaning behind them. Without conditioning, WaveNet generates ``babble'' \cite{huzaifah_deep_2021}.

An end-to-end model must be able to take textual input and generate a sound from it. For this, the textual input must be somehow passed to the model during training and inference.

This is called parametric control, sound modeling, or model conditioning. ``Sound modelling is [...] developing algorithms that generate sound under parametric control'' \cite{huzaifah_deep_2021}. This is the ability to manipulate and control the characteristics of the generated audio by adjusting the model parameters. For example, a generative model for musical audio synthesis may allow the user to control the pitch, timbre, or duration of the generated audio by adjusting the specific parameters of the model. The goal of parametric control in generative audio models is to allow the user to fine-tune the characteristics of the generated audio and achieve the desired results. Parametric control is essential in this work because the output must be tailored specifically for the user's input prompt.

This is a crucial job for these types of models and also one of the toughest obstacles in creating digital audio \cite{huzaifah_deep_2021}.

Luckily, this is a solved problem in the data generation realm. Given the quick advances in generative deep learning technologies, every generator production model relies on model conditioning. For instance, transformers receive an input text vector natively. Also, \acp{GAN} can be conditioned on a specific range of inputs. These examples illustrate that it is more than possible to incorporate this kind of technology in an end-to-end model for sound generation.

Modern models for multimodal learning, which involve processing information from different media types such as images, sound, and text, have been based on a variation of \acp{VAE} as seen in sections \ref{sec:data-generators} and \ref{sec:related-work}. One approach to conditioning \acp{VAE} on text involves training different media types, such as images and text, to share a common latent space. This is achieved by optimizing a joint objective function that balances the reconstruction loss of each modality with the alignment of the respective latent spaces.

The text input is first encoded into a latent representation using an encoder network during inference. The image decoder network then decodes this latent representation to generate the output image that corresponds to the given text input.

This is lightly debated in Section~\ref{sec:model}, and more deeply argued in Section~\ref{sec:sol-models}.
\section{Model Requirements and Design} \label{sec:model}

A model that solves the end-to-end problem aims to transform a textual prompt into a sound.

A model to solve this problem must be able to extract a representation of latent features of the given prompt. Then, it must condition a given generator on this representation. This generator must create the sound itself or create a lower-level sound representation. If the latter occurs, another piece must transform this representation into a raw sound, a vocoder.

The generator model needs to be trained with the text embeddings first. So, let us follow the example of training for a single sample: a pair of a sound and the respective natural language label. First, one must translate the sound into a lower-level feature representation. Second, one must have a model that translates the label into a latent features representation, this is, into a text embedding vector. Then, a model that generates a sound representation must be conditioned on these embeddings. Finally, the results of this model must be compared with the lower-level representation of the training sample. This process would train the generator of sound representations. A second process of training the vocoder could be done separately. Other architectures are possible. This simple architectural example would scale well because the two most difficult parts, the generator, and the vocoder, can be trained in parallel and do not depend on one another. Further discussion on this topic is presented in section \ref{sec:sol-models}.

For this architecture to be successful, at least three models need to be trained: the text embedding, the lower-level sound generation, and the vocoder. Vocoders already exist and are a field of relatively intense development, as shown in \ref{sec:vocoders}. Also, text embedding can be already done very well, as \ref{sec:text-embedding} shows. This simplifies the problem as, theoretically, only the generator needs to be developed and trained. The possibility of focusing on this generator is an asset because it eases the workload. However, for best results, one should develop all the models or at least fine-tune them to the specific data that one is working with.