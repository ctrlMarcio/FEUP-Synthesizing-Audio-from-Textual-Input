\section{Research Plan} \label{sec:work-plan}

% The work plan refers to the specific tasks and activities that will be carried out in order to execute the approach

% The work plan, on the other hand, is a more detailed plan that outlines the specific tasks and activities that will be carried out in order to implement the approach. This might include tasks such as data collection, data analysis, and the development of any necessary software or tools. The work plan should also include a timeline for completing these tasks and any milestones that need to be achieved along the way.

% the work plan should be described in more detail in the later stages of the project, as the research progresses and the specific tasks and activities become clearer.

\begin{figure}
    \centering
    \begin{ganttchart}[
            time slot format=isodate,
            expand chart=\textwidth,
            title/.style={draw=none, fill=none},
            bar label node/.append style={align=right},
        ]{2022-09-01}{2023-08-31}
        \gantttitlecalendar{year, month} \\
        
        \ganttbar[bar/.append style={fill=input, draw=none}]{State-of-the-Art}{2022-09-01}{2023-02-15} \\
                    
        \ganttbar[bar/.append style={fill=output, draw=none}]{Introduction}{2022-10-01}{2022-12-31} \\
                    
        \ganttbar[bar/.append style={fill=input, draw=none}]{Gather of Dataset}{2022-11-01}{2023-05-31} \\
        
        \ganttbar[bar/.append style={fill=output, draw=none}]{Defining the Problem}{2022-11-15}{2022-12-15} \\
        
        \ganttbar[bar/.append style={fill=neuron, draw=none}]{Basic Classification}{2023-03-01}{2023-03-31} \\
                    
        \ganttlinkedbar[bar/.append style={fill=neuron, draw=none}]{Basic Generation}{2023-04-01}{2023-05-31} \\
                    
        \ganttlinkedbar[bar/.append style={fill=neuron, draw=none}]{Apply Large Datasets}{2023-06-01}{2023-06-30} \\

        \ganttlinkedbar[bar/.append style={fill=neuron, draw=none}]{Collect Results}{2023-07-01}{2023-08-14} \\

        \ganttbar[bar/.append style={fill=output, draw=none}]{Write Survey Paper}{2023-07-01}{2023-07-31} \\
                    
        \ganttbar[bar/.append style={fill=output, draw=none}]{Dissertation Writing}{2023-05-01}{2023-08-31}
    \end{ganttchart}
    \caption[Work Plan Gantt Chart]{\textbf{Work Plan Gantt Chart} where the dark blue represents researching, light blue represents writing, and green represents developing.}
    \label{fig:gantt}
\end{figure}

This section presents a brief overview of the research tasks, complete with their corresponding timelines and milestones. One should be aware that as the activities and tasks become more defined, the research plan may be altered and enhanced over time.

The primary phase of this research entails conducting an exhaustive examination of existing generative \ac{AI} models for audio. This task requires acquiring a fundamental understanding of the topic, exploring the history of \ac{AI}, and gaining basic knowledge of sound. Furthermore, an extensive study was conducted on the latest advancements in generative models, specifically for audio. The findings of this analysis are presented in the Chapter~\ref{chap:sota}. The interim checkpoint in February 2023, ending of the Dissertation Planning course, mandated the completion of this task.

After conducting a thorough literature review, the next step is to craft the introduction section of the thesis, which involves understanding the research objectives and scope and providing a comprehensive overview. The introduction sets the context, motivation, and objectives of the study. The deadline for completing the introduction is February 2023. 

To aid in the experimentation and development of generative \ac{AI} models for audio, a comprehensive exploration and analysis of prevalent audio datasets was conducted. This included identifying relevant datasets that align with the research objectives and gaining permission and access to these datasets.

Following the assembly of the datasets, the subsequent step involves outlining the problem to be addressed in the thesis. This involves clearly stating the research questions and goals that will direct the development and assessment of these models.

In anticipation of future assignments, a rudimentary dataset to generate a straightforward classification model was utilized. This entailed constructing and training the model, evaluating its effectiveness, and recording the outcomes.

Having established the classification model, the author proceeded to build a foundational audio generation model. This task involved implementing and training the model on the same basic dataset, evaluating its performance, and documenting the findings.

To improve the developed generative models, significant datasets were employed. The models were modified and updated to account for the limitations and intricacies of these larger datasets. The model's performance was assessed, and the outcomes were analyzed and documented.

During this phase, a range of extensive experiments were conducted to evaluate the effectiveness of the developed models. Different techniques were examined, and the models were improved based on the findings. The outcomes of these experiments were carefully examined and recorded.

Subsequently, after concluding the research and experimentation phases, the focus shifted to drafting the thesis. This involved consolidating the research results, analyzing their significance, and producing an organized paper that neutrally presents the research methodology, findings, and outcomes.

In addition to the dissertation, a survey paper was composed that outlines the latest developments in generative \ac{AI} models for soundscapes. The paper provides an extensive synopsis of advancements in the domain and contributes significantly to the scholarly audience.

It is important to note that the deadline for completing these tasks has been extended until September 1st due to the challenges in obtaining permission to access the \ac{LIACC} servers. As the research progressed, the schedule was reviewed and updated to ensure accuracy and importance.