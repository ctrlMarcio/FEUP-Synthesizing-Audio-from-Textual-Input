\subsection{Exploratory Experiments} \label{sec:findings}

In the development of generative \ac{AI} models for audio synthesis, it is crucial to conduct extensive experiments that explore various aspects of model architecture, training processes, datasets, and evaluation metrics. This section presents a series of experiments that provide a valuable learning phase before developing the primary model, GANmix. The experiments aim to gain insight into the performance, limitations, and potential of different generative audio models. The author analyzes empirical findings to inform the development of GANmix, aiming for a robust and efficient solution.

The initial experiment focuses on a classification model to uncover the fundamental principles of sound. Training the model on labeled audio data, the author aims to grasp the connections among different audio features and their corresponding categories. This experiment lays the foundation for further research on audio representation.

Based on the results of the classification experiment, the author proceeds to develop a \ac{GAN} for audio synthesis. The aim of this study is to assess the effectiveness of \acp{GAN} in generating high-quality and realistic audio samples.

Furthermore, the author explores the efficacy of \acp{AE} and \acp{VAE} in audio generation. The experiments investigate the reconstruction and generation capabilities of these models, respectively.

The purpose of these experiments is to achieve a comprehensive understanding of both audio and different generative models for audio synthesis. These empirical findings provide a crucial basis for developing GANmix, ensuring a robust and effective solution for audio generation.

\subsubsection{Preliminary Classification} \label{sec:classification-model}

To develop a comprehensive understanding of sound and its representation, a preliminary classification model was developed as a starting point.

The Audio MNIST dataset (see Section~\ref{sec:dataset-amnist}) was selected due to its simplicity and appropriateness. The dataset consists of spoken digit recordings divided into ten categories (numbers from 0 to 9). This classification was developed model using both TensorFlow and PyTorch frameworks (see Section~\ref{sec:dl-frameworks}).

This model's central premise is the possibility of achieving satisfactory outcomes using a simple \ac{CNN}. The aim is to surpass 90\% accuracy.

Conventional \acp{CNN} usually have convolutional and pooling layers (see Section~\ref{sec:CNN}), which are versatile and can handle inputs of different dimensions. However, utilizing these layers with inputs of varying sizes leads to different output dimensions. It's essential to conduct necessary pre-processing steps to impose uniform dimensions for all input data to obtain a more accurate assessment of the network's outputs.

To reduce variations in sample size, a combination of techniques, such as padding and random cropping, was used. To address padding, an approach called edge padding was adopted. This technique includes extending the beginning and end of smaller audio samples to match a fixed size. The main objective of edge padding is to preserve the audio content's integrity and ensure consistency in the dataset.

In contrast, the random crop technique has a unique operation within the confines of this study because the dataset mainly includes recordings of spoken digits that possess inherent characteristics. The spoken digit samples' nature is leveraged to systematically select discrete segments from the audio samples using random crop. This process produces 1.5-second segments.

The classification model is based on an architecture of \ac{CNN}, with a sequence of five blocks, its general architecture is present in Figure~\ref{fig:cnn}. Each block included a convolutional layer followed by max pooling that systematically extracted hierarchical features. Finally, the network ended with an average pooling layer, which connects to three linear layers for final classification. The output layer, consisting of ten units, corresponds to the ten distinct classes in the dataset.

Average pooling is a down-sampling technique that partitions the input feature map into distinct regions and computes the average value for each partition, resulting in a down-sampled feature map that retains essential information about the input while decreasing spatial dimensions. As a result, the pooled feature map preserves essential information about the input while reducing the spatial dimensions. This technique helps control parameter count and reduce overfitting, thereby improving the generalization ability of the model.

In the present classification model, the average pooling layer effectively reduces dimensionality. The pooled features maintain vital high-level information gathered from previous convolutional layers, allowing the ensuing linear layers to concentrate on extracting more sophisticated semantic features for classification purposes. The decision to use this architecture was purposefully made to make the model lighter. Flattening the convolutional output would result in a higher number of parameters.

Model training involved using \ac{SGD} optimization and Cross-Entropy loss. The training process lasted 20 epochs and involved batch-wise backpropagation to update parameters. It is noteworthy that the training occurred on Tesla P100 with 8GB.

The code for the model and the training loop can be see in Annex~\ref{ann:classification}.
\subsubsection{Audio Generation with GAN} \label{sec:sol-gan}

This section presents the methodology and results of applying a \ac{GAN} model to generate audio samples from raw sound waves without spectrograms.

A \ac{GAN} model, as explained in Section \ref{sec:gan}, consists of two neural networks: the generator and the discriminator. The generator creates new data samples that mimic the input data distribution while the discriminator evaluates how realistic the created data samples are. The goal is to train the generator to create samples that are indistinguishable from the real data, according to the discriminator.

However, training \ac{GAN} models poses significant computational cost and time challenges. This is because a \ac{GAN} model involves two neural networks that must be trained simultaneously. The generator creates new data samples, and the discriminator evaluates their realism. Therefore, each network has to wait for the output of the other network to update its parameters. This process can be time-consuming, especially for larger datasets or more complex architectures. Moreover, since \ac{GAN} models are computationally intensive, they require powerful \acp{GPU} and can take days or weeks to train.

In this section, raw sound waves were used as input data instead of spectrograms. This was done to eliminate the need for a vocoder (see Section~\ref{sec:vocoders}). The audio input size of the Audio MNIST dataset varied, so the first step was to pad the audio samples with zeros to a fixed size that could accommodate all samples. This was achieved by applying the same preprocessing techniques applied in Section~\ref{sec:classification-model}.

A normalization process was applied to the audio data to confine the audio signal values within a well-defined range, mitigating the potential occurrence of gradient explosion. This facilitates the convergence of the learning process for the model.

The TensorFlow framework was utilized to implement the \ac{GAN} model.

The generator network architecture consists of a series of layered operations that include dense, reshape, and transposed convolutional transformations. The layers are comprehensively depicted in Annex~\ref{ann:GAN}.

The input layer takes a noise vector ($z$) of size 100 as input and passes it through a dense layer with 256 units. Then, six transposed convolution layers are applied with a kernel of 25 and stride set to 4. The number of filters goes from 32 to 1, generating an output with only one channel. Each transposed convolution layer is followed by a ReLU activation function (see Section~\ref{sec:activation}), except for the last one, which uses a tanh activation function to bound the last values between -1 and 1, emulating a normalized raw sound. The output of the last layer is a $16384 \times 1$ vector, which represents the generated audio sample ($G(z)$).

The generator network can be formally defined as follows:

\begin{equation}
\begin{split}
G(z) &= \text{tanh}(\text{Conv1DTranspose}_6(\text{ReLU}(\text{Conv1DTranspose}_5 (\cdots \\
&\quad \text{ReLU}(\text{Conv1DTranspose}_1(\text{Reshape}(\text{Dense}(z))))))))
\end{split}
\end{equation}

where $\text{Dense}$, $\text{Reshape}$, $\text{Conv1DTranspose}$, $\text{ReLU}$, and $\text{tanh}$ denote the corresponding operations with their parameters omitted for brevity.

The discriminator network is composed of several layers of convolution and dense operations. These can also be seen in Annex~\ref{ann:GAN}

The input layer takes either a real audio sample ($x$) or a fake audio sample ($G(z)$) of size $16384 \times 1$ as input. Its architecture mirrors that of the generator. It passes through six convolution layers with a kernel of 25 and a stride of 4, starting from one filter until 32. Each convolution layer is followed by a leaky ReLU activation function with an alpha parameter of 0.2 to avoid the dying ReLU problem (see Section~\ref{sec:activation}). The output of the last layer is flattened into a 32-dimensional vector. Then, a dense layer with one unit is applied to produce a scalar value, which represents the probability ($D(x)$ or $D(G(z))$) of the input being real.

The discriminator network can be formally defined as follows:

\begin{equation}
D(x) = \text{Dense}(\text{Flatten}(\text{Conv1D}_6(\text{LeakyReLU}(\text{Conv1D}_5(\cdots \text{LeakyReLU}(\text{Conv1D}_1(x)))))))
\end{equation}

where $\text{Conv1D}$, $\text{Flatten}$, $\text{Dense}$, and $\text{LeakyReLU}$ denote the corresponding operations with their parameters omitted for brevity.

The training process alternates between updating the parameters of the generator and the discriminator using gradient descent. Both networks use the \ac{BCE} loss function, which is a variation of the cross-entropy loss function explained in Section~\ref{sec:cross-entropy}. Cross-entropy measures how accurately the networks predict the input labels. \Ac{BCE} is designed for discriminating between two categories — real and fake samples, in this case. The generator aims to reduce \ac{BCE} loss by encouraging the discriminator to assign high scores to fake samples. On the other hand, the discriminator aims to minimize \ac{BCE} by producing low output values for fake samples and high output values for real samples. The interaction between generator and discriminator, which is guided by binary cross-entropy, helps in the training and enhancement of the generative model.

The training algorithm can be summarized as presented in Algorithm~\ref{alg:gan}.

\begin{algorithm}
\caption{\ac{GAN} Training Algorithm}
\label{alg:gan}
\begin{algorithmic}[1]
\State Initialize generator ($G$) and discriminator ($D$) parameters randomly
\State Set number of epochs ($E$) and batch size ($B$)
\For{$e = 1, \dots, E$}
\State Shuffle the real audio samples ($X$)
\For{$b = 1, \dots, \frac{|X|}{B}$}
\State Sample a batch of noise vectors ($Z$) from a normal distribution
\State Generate a batch of fake audio samples ($G(Z)$) using $G$
\State Compute the discriminator outputs for real ($D(X)$) and fake ($D(G(Z))$) samples using $D$
\State Compute the generator loss ($L_G$) using BCE and $D(G(Z))$
\State Compute the discriminator loss ($L_D$) using BCE and $D(X)$ and $D(G(Z))$
\State Update $G$ parameters by descending the gradients of $L_G$
\State Update $D$ parameters by descending the gradients of $L_D$
\EndFor
\State Generate and save a sample audio using $G$
\EndFor
\end{algorithmic}
\end{algorithm}

%%

The implementation code for this architecture is presented in Appendix \ref{ann:GAN}.

The \ac{GAN} model was trained on a small dataset of audio samples for 20 epochs. This was done to reduce training time since it was developed on a personal computer. Despite limited training time, results were promising. Generated audio samples were based on random noise but resembled real ones in terms of wave amplitude.

These results demonstrated that it was possible to generate realistic audio samples using raw sound waves as input data.

However, developing this model was challenging. The author used TensorFlow but faced difficulties with code complexity and memory usage. This led to frustration as the model needed scaling up.

To overcome this challenge, the author had to learn PyTorch, known for simplicity and flexibility, as seen in section \ref{sec:pytorch}. From now on, assume every implementation was done in PyTorch.
\subsubsection{Simple Autoencoder for Audio Data Compression}

This section outlines the procedure for building a basic \ac{AE} (refer to Section~\ref{sec:autoencoders}), which is an essential step in making more complex versions like the \ac{VAE} (refer to Section~\ref{sec:vae}) in future research. The PyTorch-based implementation of the \ac{AE} is available in the given code repository (refer to Annex~\ref{ann:AE}).

The same preprocessing applied to previous models (padding and random cropping) is applied in this one.

The architecture of the \ac{AE} is reminiscent of a U-Net (see Section~\ref{sec:u-net}), incorporating four types of layers: convolutional, max pooling, upsampling, and transposed convolutional layers (see Section~\ref{sec:conv-layers}). The activation function used in convolutional and transposed convolutional layers is the \ac{tanh} function (see Section~\ref{sec:activation}). The \ac{tanh} activation function ensures that the output of the \ac{AE} remains within the range of -1 to 1, which is crucial for the subsequent denormalization process. Code for defining and training the model, as well as a result example can be seen in Annex~\ref{ann:AE}, an abstract representation of this architecture is in Figure~\ref{fig:autoencoder}.

The encoder is constructed as a sequence of convolutional layers, followed by batch normalization, activation functions, and max-pooling operations. The input to the encoder is a 1D audio waveform with a single channel. The first convolutional layer has 32 filters, a kernel size of 9, a stride of 1, and a padding of 4. It is followed by batch normalization and the \ac{tanh} activation function. A max-pooling layer with kernel size 2 and stride 2 is then applied. 

There were four convolutional layers. For each one, the number of filters is doubled from the previous layer, and the exact configuration of convolution, batch normalization, activation, and max-pooling operations is repeated. The kernel size, stride, and padding remain consistent for all convolutional layers.

The decoder is constructed as a sequence of upsampling (by a factor of 2), transposed convolutional layers, batch normalization, and activation functions. The decoder's architecture is symmetric to the encoder, with the number of filters halving at each layer until reaching the original number of channels (1). The transposed convolutional layers have the same kernel size, stride, and padding as the corresponding encoder convolutional layers. The last layer of the decoder applies batch normalization, a \ac{tanh} activation function, and produces the reconstructed audio waveform.

The \ac{ReLU} activation function was tested during experimentation, but the results proved unsatisfactory, as every sound frame was above 0. Consequently, the decision was made to continue using the \ac{tanh} activation function for the simple \ac{AE}.

A series of steps is carried out during the training loop to train the model. First, each batch of audio samples undergoes a normalization process to ensure consistent data representation. The normalized data is then fed through the model, resulting in two crucial outputs: an encoded representation of the audio and a reconstructed audio waveform.

To assess the quality of the reconstructed audio, a loss function is employed, which quantifies the dissimilarity between the reconstructed waveform and the original input using \ac{MSE} (see Section~\ref{sec:mse}). This loss value serves as a measure of how well the \ac{AE} can capture and reproduce the essential characteristics of the audio data.

The calculated backpropagation gradients are subsequently used to update the model's parameters using the Adam optimizer (see Section~\ref{sec:adam}), iteratively refining the \ac{AE}'s ability to encode and decode the audio samples.

The training process follows an iterative approach, where the model is trained for multiple epochs. The training data is iterated over in each epoch, and the model's parameters are updated based on the computed gradients. The best model (with the lowest loss) obtained during training is saved for future use.

By constructing this simple \ac{AE}, valuable insights into the underlying mechanisms of \ac{AE}s are gained, which is instrumental in developing more sophisticated techniques. Moreover, the provided code implementation in PyTorch (refer to Annex~\ref{ann:AE}) facilitates a deeper understanding and exploration of the \ac{AE} architecture, enabling improvements and extensions to audio data compression.


\subsubsection{Simple Variational Autoencoder} \label{sec:training-vae}

This section focuses on the development and experimentation of the \ac{VAE}, a generative model that, as explained in Section~\ref{sec:vae}, with a representation in Figure~\ref{fig:vae}, aims to learn latent representations of data while enabling controlled generation. The \ac{VAE} was developed as a means of further exploring generative techniques, expanding on the groundwork established by earlier models and adjusting it to suit the specific characteristics of audio data.

\paragraph{Initial implementation}

The developed \ac{VAE} utilized the identical dataset as the previous models, covering unprocessed audio waveforms. Comparable preprocessing transformations were performed, such as random cropping and padding.

Three primary components constituted the \ac{VAE}'s structure: an encoder, a bottleneck layer, and a decoder. The encoder utilized a variable number of convolutional layers, but for practical reasons, the number was restricted to two. Hardware restrictions played a vital role in imposing this constraint since more layers would result in excessively extended convergence times. Despite its limitations, the bottleneck layer, consisting of two separate linear layers for mean and variance, effectively encapsulated latent features.

The decoder mirrored the encoder's structure, using deconvolutional layers instead of convolutional ones and upsampling instead of max-pooling. This architecture ensured the generation of audio samples preserving the input data's characteristics.

The training process utilized a loss function that combined \ac{BCE} and \ac{KL} divergence, which is in line with standard \ac{VAE} practice. The training emphasized the reconstruction of input samples and learning the latent space, similar to the previous\ac{AE} model.

\paragraph{Fine-Tuning Given Resource Constraints}

As the project progressed, resource limitations played a significant role in shaping the trajectory of the model. Achieving optimal performance of the \ac{VAE} within these constraints became a major challenge. To overcome this challenge, the author underwent fine-tuning of multiple elements such as convolutional layer count and bottle neck layers dimension size. However, the computational demands of these fine-tuning steps proved to be prohibitive, leading the author to explore a different approach.

Focusing on the learning rate, a critical parameter that affects training speed and convergence, was crucial. A structured exploration of learning rates was conducted, considering a range of values spanning several orders of magnitude. This approach facilitated meaningful experimentation within a limited timeframe. The exploration of learning rates uncovered the trade-offs between excessively small and overly large values. Values in the range of $1 \times 10^{-4}$ to $1 \times 10^{-3}$ have shown promising results, indicating a need for further investigation.

The process of training and fine-tuning the \ac{VAE} has highlighted the significant impact of computational resources on model development. Balancing model complexity with resource limitations has proven to be a complex task, resulting in innovative approaches to optimizing training while retaining high-quality generative capabilities.

After training for just five epochs, the loss amounted to $124261.2969$. A visual representation of the reconstructed sample can be found in the annex~\ref{ann:VAE}.