\chapter{Introduction} \label{chap:intro}

\section{Context} \label{sec:context}

% TODO
The current developments of tools are advancing such existing models, further lead to modeling new sounds, providing possibilities to generate audio efficiently and faster but at the same time interesting results from the trained data set of samples in audio domain. \cite{tahiroglu_-terity_2020}

% TODO
Deep learning with audio shifts the focus to next level of real-time synthesis of sound by creating completely new-sounding sounds. \cite{tahiroglu_-terity_2020}

\section{Motivation} \label{sec:motivation}

% TODO
Today, digital technologies and advanced computational features, e.g. deep learning and artificial intelligence (AI) tools are shaping our relationships with music as well as enabling new possibilities of utilising new musical instruments and interfaces. \cite{tahiroglu_-terity_2020}

\section{Objectives} \label{sec:objectives}


\section{Dissertation Structure} \label{sec:struct}

Para além da introdução, esta dissertação contém mais x capítulos.
No Capítulo~\ref{chap:sota}, é descrito o estado da arte e são
apresentados trabalhos relacionados. 
No Capítulo~\ref{chap:chap3}, ipsum dolor sit amet, consectetuer
adipiscing elit.
No Capítulo~\ref{chap:chap4} praesent sit amet sem. 
No Capítulo~\ref{chap:concl} posuere, ante non tristique
consectetuer, dui elit scelerisque augue, eu vehicula nibh nisi ac
est. 
